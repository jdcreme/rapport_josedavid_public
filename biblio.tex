% This file was created with JabRef 2.6.
% Encoding: UTF8

@ARTICLE{defects_2007_09,
  author = {Brahim Akdim and Tapas Kar and Xiaofeng Duan and Ruth Pachter},
  title = {Density functional theory calculations of ozone adsorption on sidewall
	single-wall carbon nanotubes with Stone-Wales defects},
  journal = {Chemical Physics Letters},
  year = {2007},
  volume = {445},
  pages = {281 - 287},
  number = {4?6},
  abstract = {In this study, we employed density functional theory to investigate
	the adsorption mechanisms of O3 on the sidewalls of C(5,5), C(8,8),
	and C(10,0) single-wall carbon nanotubes (SWCNTs), also having Stone-Wales
	(SW) defects with different orientations. An understanding of the
	adsorption of O3 on SWCNT sidewalls with SW defects was obtained
	by examining local structural changes, described by the pyramidalization
	angle, while in investigating the electronic structure of ozonized
	SWCNT, the results were found to be consistent with experimental
	observations in some cases.},
  doi = {10.1016/j.cplett.2007.08.001},
  file = {defects_2007_09.pdf:defects_2007_09.pdf:PDF},
  issn = {0009-2614},
  owner = {vijay},
  timestamp = {2012.06.08},
  url = {http://www.sciencedirect.com/science/article/pii/S0009261407010615}
}

@ARTICLE{defects_2009_07,
  author = {Al-Aqtash, Nabil and Vasiliev, Igor},
  title = {Ab Initio Study of Carboxylated Graphene},
  journal = {The Journal of Physical Chemistry C},
  year = {2009},
  volume = {113},
  pages = {12970-12975},
  number = {30},
  doi = {10.1021/jp902280f},
  eprint = {http://pubs.acs.org/doi/pdf/10.1021/jp902280f},
  file = {defects_2009_07.pdf:defects_2009_07.pdf:PDF},
  owner = {vijay},
  timestamp = {2012.06.08},
  url = {http://pubs.acs.org/doi/abs/10.1021/jp902280f}
}

@ARTICLE{defects_2010_07,
  author = {Allen, Matthew J. and Tung, Vincent C. and Kaner, Richard B.},
  title = {Honeycomb Carbon: A Review of Graphene},
  journal = {Chemical Reviews},
  year = {2010},
  volume = {110},
  pages = {132-145},
  number = {1},
  note = {PMID: 19610631},
  doi = {10.1021/cr900070d},
  eprint = {http://pubs.acs.org/doi/pdf/10.1021/cr900070d},
  file = {defects_2010_07.pdf:defects_2010_07.pdf:PDF},
  owner = {vijay},
  timestamp = {2012.06.08},
  url = {http://pubs.acs.org/doi/abs/10.1021/cr900070d}
}

@ARTICLE{defects_2006_02,
  author = {Jan Andzelm and Niranjan Govind and Amitesh Maiti},
  title = {Nanotube-based gas sensors ? Role of structural defects},
  journal = {Chemical Physics Letters},
  year = {2006},
  volume = {421},
  pages = {58 - 62},
  number = {1?3},
  abstract = {Existing theoretical literature suggests that defect-free, pristine
	carbon nanotubes (CNTs) interact weakly with many gas molecules like
	H2O, CO, NH3, H2, and so on. The case of NH3 is particularly intriguing,
	because this is in disagreement with experimentally observed changes
	in electrical conductance of CNTs upon exposure to these gases. In
	order to explain such discrepancy, we have carried out density functional
	theory investigations of the role of common atomistic defects in
	CNT (Stone?Wales, monovacancy, and interstitial) on the chemisorption
	of NH3. Computed binding energies, charge transfer, dissociation
	barriers, and vibrational modes are compared with existing experimental
	results on electrical conductance, thermal desorption and infrared
	spectroscopy.},
  doi = {10.1016/j.cplett.2005.12.099},
  file = {defects_2006_02.pdf:defects_2006_02.pdf:PDF},
  issn = {0009-2614},
  owner = {vijay},
  timestamp = {2012.06.08},
  url = {http://www.sciencedirect.com/science/article/pii/S0009261406000327}
}

@ARTICLE{hydrogenation1,
  author = {Veronica Barone and Jochen Heyd and Gustavo E. Scuseria},
  title = {Interaction of atomic hydrogen with single-walled carbon nanotubes:
	A density functional theory study},
  journal = {The Journal of Chemical Physics},
  year = {2004},
  volume = {120},
  pages = {7169-7173},
  number = {15},
  doi = {10.1063/1.1668635},
  file = {hydrogenation1.pdf:hydrogenation1.pdf:PDF},
  keywords = {hydrogen neutral atoms; carbon nanotubes; density functional theory;
	binding energy; potential energy surfaces; atom-atom collisions;
	GO calculations},
  owner = {vijay},
  publisher = {AIP},
  timestamp = {2012.06.11},
  url = {http://link.aip.org/link/?JCP/120/7169/1}
}

@ARTICLE{dft_polari_nntb,
  author = {Brothers, Edward N. and Kudin, Konstantin N. and Scuseria, Gustavo
	E. and Bauschlicher, Charles W.},
  title = {Transverse polarizabilities of carbon nanotubes: A Hartree-Fock and
	density functional study},
  journal = {Phys. Rev. B},
  year = {2005},
  volume = {72},
  pages = {033402},
  month = {Jul},
  doi = {10.1103/PhysRevB.72.033402},
  file = {dft_polari_nntb.pdf:dft_polari_nntb.pdf:PDF},
  issue = {3},
  numpages = {4},
  publisher = {American Physical Society},
  url = {http://link.aps.org/doi/10.1103/PhysRevB.72.033402}
}

@ARTICLE{coaxial_nntb,
  author = {Edward N. Brothers and Gustavo E. Scuseria and Konstantin N. Kudin},
  title = {Coaxial carbon nanotubes as shielded nanowires},
  journal = {The Journal of Chemical Physics},
  year = {2006},
  volume = {124},
  pages = {041101},
  number = {4},
  eid = {041101},
  doi = {10.1063/1.2149850},
  file = {coaxial_nntb.pdf:coaxial_nntb.pdf:PDF},
  keywords = {carbon nanotubes; nanowires; nanoelectronics; wires (electric); polarisability;
	density functional theory},
  numpages = {3},
  owner = {vijay},
  publisher = {AIP},
  timestamp = {2012.06.04},
  url = {http://link.aip.org/link/?JCP/124/041101/1}
}

@ARTICLE{dft_polari_nntb2,
  author = {Brothers, Edward N. and Scuseria, Gustavo E. and Kudin, Konstantin
	N.},
  title = {Longitudinal Polarizability of Carbon Nanotubes},
  journal = {The Journal of Physical Chemistry B},
  year = {2006},
  volume = {110},
  pages = {12860-12864},
  number = {26},
  abstract = { The longitudinal polarizabilities of carbon nanotubes are determined
	using first principles density functional theory. These results demonstrate
	that the polarizability per atom of a nanotube in the axial direction
	is primarily determined by the band gap. In fact, polarizability
	per atom versus inverse band gap yields a linear trend for all nanotubes
	and methods utilized in this study, creating a universal relationship
	for longitudinal polarizability. This can be explained by examining
	the terms in the sum over states equation used to determine polarizability
	and noting that the vast majority of the polarizability arises from
	a few elements near the band gap. This universal trend is then used
	with experimentally determined band gaps to predict the experimental
	polarizability of carbon nanotubes. },
  doi = {10.1021/jp0603839},
  eprint = {http://pubs.acs.org/doi/pdf/10.1021/jp0603839},
  file = {dft_polari_nntb2.pdf:dft_polari_nntb2.pdf:PDF},
  owner = {vijay},
  timestamp = {2012.06.04},
  url = {http://pubs.acs.org/doi/abs/10.1021/jp0603839}
}

@ARTICLE{hydrogenation3,
  author = {Marko Burghard},
  title = {Electronic and vibrational properties of chemically modified single-wall
	carbon nanotubes},
  journal = {Surface Science Reports},
  year = {2005},
  volume = {58},
  pages = {1 - 109},
  number = {1–4},
  abstract = {Recent experimental work on the chemical functionalization of single-wall
	carbon nanotubes and its implication on the tubes’ electronic and
	vibrational properties are surveyed. The dependence of chemical reactivity
	on the physical and electronic structure of the tubes is discussed.
	Moreover, experimentally observed changes of the electronic and photonic
	structure are compared to theoretical predictions for various non-covalent
	and covalent functionalization schemes.},
  doi = {10.1016/j.surfrep.2005.07.001},
  file = {hydrogenation3.html:hydrogenation3.html:URL;hydrogenation3.pdf:hydrogenation3.pdf:PDF},
  issn = {0167-5729},
  keywords = {Carbon nanotubes},
  owner = {vijay},
  timestamp = {2012.06.11},
  url = {http://www.sciencedirect.com/science/article/pii/S0167572905000452}
}

@ARTICLE{mcscf_polari,
  author = {Roberto Cammi and Luca Frediani and Benedetta Mennucci and Kenneth
	Ruud},
  title = {Multiconfigurational self-consistent field linear response for the
	polarizable continuum model: Theory and application to ground and
	excited-state polarizabilities of para-nitroaniline in solution},
  journal = {The Journal of Chemical Physics},
  year = {2003},
  volume = {119},
  pages = {5818-5827},
  number = {12},
  doi = {10.1063/1.1603728},
  file = {mcscf_polari.pdf:mcscf_polari.pdf:PDF},
  keywords = {Franck-Condon factors; polarisability; SCF calculations; solvation;
	solvent effects; excited states; integral equations},
  owner = {vijay},
  publisher = {AIP},
  timestamp = {2012.06.04},
  url = {http://link.aip.org/link/?JCP/119/5818/1}
}

@ARTICLE{defects_2006_03,
  author = {A. C. M. Carvalho and M. C. dos Santos},
  title = {Nitrogen-substituted nanotubes and nanojunctions: Conformation and
	electronic properties},
  journal = {Journal of Applied Physics},
  year = {2006},
  volume = {100},
  pages = {084305},
  number = {8},
  eid = {084305},
  doi = {10.1063/1.2357646},
  file = {defects_2006_03.pdf:defects_2006_03.pdf:PDF},
  keywords = {nitrogen; carbon nanotubes; electronic density of states; density
	functional theory; conduction bands; bonds (chemical)},
  numpages = {5},
  owner = {vijay},
  publisher = {AIP},
  timestamp = {2012.06.08},
  url = {http://link.aip.org/link/?JAP/100/084305/1}
}

@ARTICLE{cclr1,
  author = {Ove Christiansen and Asger Halkier and Henrik Koch and Poul Jorgensen
	and Trygve Helgaker},
  title = {Integral-direct coupled cluster calculations of frequency-dependent
	polarizabilities, transition probabilities and excited-state properties},
  journal = {The Journal of Chemical Physics},
  year = {1998},
  volume = {108},
  pages = {2801-2816},
  number = {7},
  doi = {10.1063/1.475671},
  file = {cclr1.pdf:cclr1.pdf:PDF},
  keywords = {organic compounds; coupled cluster calculations; polarisability; radiative
	lifetimes; excited states},
  owner = {vijay},
  publisher = {AIP},
  timestamp = {2012.06.06},
  url = {http://link.aip.org/link/?JCP/108/2801/1}
}

@ARTICLE{cclr3,
  author = {Christiansen, Ove and Jørgensen, Poul and Hättig, Christof},
  title = {Response functions from Fourier component variational perturbation
	theory applied to a time-averaged quasienergy},
  journal = {International Journal of Quantum Chemistry},
  year = {1998},
  volume = {68},
  pages = {1--52},
  number = {1},
  abstract = {It is demonstrated that frequency-dependent response functions can
	conveniently be derived from the time-averaged quasienergy. The variational
	criteria for the quasienergy determines the time-evolution of the
	wave-function parameters and the time-averaged time-dependent Hellmann–Feynman
	theorem allows an identification of response functions as derivatives
	of the quasienergy. The quasienergy therefore plays the same role
	as the usual energy in time-independent theory, and the same techniques
	can be used to obtain computationally tractable expressions for response
	properties, as for energy derivatives in time-independent theory.
	This includes the use of the variational Lagrangian technique for
	obtaining expressions for molecular properties in accord with the
	2n+1 and 2n+2 rules. The derivation of frequency-dependent response
	properties becomes a simple extension of variational perturbation
	theory to a Fourier component variational perturbation theory. The
	generality and simplicity of this approach are illustrated by derivation
	of linear and higher-order response functions for both exact and
	approximate wave functions and for both variational and nonvariational
	wave functions. Examples of approximate models discussed in this
	article are coupled-cluster, self-consistent field, and second-order
	Møller–Plesset perturbation theory. A discussion of symmetry properties
	of the response functions and their relation to molecular properties
	is also given, with special attention to the calculation of transition-
	and excited-state properties. © 1998 John Wiley & Sons, Inc. Int
	J Quant Chem 68: 1–52, 1998},
  doi = {10.1002/(SICI)1097-461X(1998)68:1<1::AID-QUA1>3.0.CO;2-Z},
  file = {cclr3.pdf:cclr3.pdf:PDF},
  issn = {1097-461X},
  owner = {vijay},
  publisher = {John Wiley \& Sons, Inc.},
  timestamp = {2012.06.06},
  url = {http://dx.doi.org/10.1002/(SICI)1097-461X(1998)68:1<1::AID-QUA1>3.0.CO;2-Z}
}

@ARTICLE{CC3_response,
  author = {Ove Christiansen and Henrik Koch and Poul Jorgensen},
  title = {Response functions in the CC3 iterative triple excitation model},
  journal = {The Journal of Chemical Physics},
  year = {1995},
  volume = {103},
  pages = {7429-7441},
  number = {17},
  doi = {10.1063/1.470315},
  file = {CC3_response.pdf:CC3_response.pdf:PDF},
  keywords = {ELECTRON CORRELATION; EXCITATION; ITERATIVE METHODS; MATHEMATICAL
	MODELS; QUANTUM CHEMISTRY; RESPONSE FUNCTIONS},
  owner = {vijay},
  publisher = {AIP},
  timestamp = {2012.06.08},
  url = {http://link.aip.org/link/?JCP/103/7429/1}
}

@ARTICLE{dft_polari_2007,
  author = {Sonia Coriani and Stinne Host and Branislav Jansik and Lea Thogersen
	and Jeppe Olsen and Poul Jorgensen and Simen Reine and Filip Pawlowski
	and Trygve Helgaker and Pawel Salek},
  title = {Linear-scaling implementation of molecular response theory in self-consistent
	field electronic-structure theory},
  journal = {The Journal of Chemical Physics},
  year = {2007},
  volume = {126},
  pages = {154108},
  number = {15},
  eid = {154108},
  doi = {10.1063/1.2715568},
  file = {dft_polari_2007.pdf:dft_polari_2007.pdf:PDF},
  keywords = {organic compounds; SCF calculations; HF calculations; orbital calculations;
	molecular electronic states; iterative methods; matrix algebra; polarisability;
	molecular configurations},
  numpages = {11},
  owner = {vijay},
  publisher = {AIP},
  timestamp = {2012.06.04},
  url = {http://link.aip.org/link/?JCP/126/154108/1}
}

@ARTICLE{vacancy3,
  author = {Czerw, R. and Terrones, M. and Charlier, J.-C. and Blase, X. and
	Foley, B. and Kamalakaran, R. and Grobert, N. and Terrones, H. and
	Tekleab, D. and Ajayan, P. M. and Blau, W. and Rühle, M. and Carroll,
	D. L.},
  title = {Identification of Electron Donor States in N-Doped Carbon Nanotubes},
  journal = {Nano Letters},
  year = {2001},
  volume = {1},
  pages = {457-460},
  number = {9},
  abstract = { Nitrogen-doped carbon nanotubes have been synthesized using pyrolysis
	and characterized by scanning tunneling spectroscopy and transmission
	electron microscopy. The doped nanotubes are all metallic and exhibit
	strong electron donor states near the Fermi level. Using tight-binding
	and ab initio calculations, we observe that pyridine-like N structures
	are responsible for the metallic behavior and the prominent features
	near the Fermi level. These electron rich structures are the first
	example of n-type nanotubes, which could pave the way to real molecular
	heterojunction devices. },
  doi = {10.1021/nl015549q},
  eprint = {http://pubs.acs.org/doi/pdf/10.1021/nl015549q},
  file = {vacancy3.pdf:vacancy3.pdf:PDF},
  owner = {vijay},
  timestamp = {2012.06.11},
  url = {http://pubs.acs.org/doi/abs/10.1021/nl015549q}
}

@ARTICLE{dft,
  author = {De Proft, Frank and Geerlings, Paul},
  title = {Conceptual and Computational DFT in the Study of Aromaticity},
  journal = {Chemical Reviews},
  year = {2001},
  volume = {101},
  pages = {1451-1464},
  number = {5},
  doi = {10.1021/cr9903205},
  eprint = {http://pubs.acs.org/doi/pdf/10.1021/cr9903205},
  file = {dft.pdf:dft.pdf:PDF;dft_polar_2000.pdf:dft_polar_2000.pdf:PDF;dft_polari_2002.pdf:dft_polari_2002.pdf:PDF;dft_polari_2007.pdf:dft_polari_2007.pdf:PDF;dft_polari_nntb.pdf:dft_polari_nntb.pdf:PDF;dft_polari_2006.pdf:dft_polari_2006.pdf:PDF;dft_response.pdf:dft_response.pdf:PDF;dft_polari_nntb2.pdf:dft_polari_nntb2.pdf:PDF},
  owner = {vijay},
  timestamp = {2012.06.15},
  url = {http://pubs.acs.org/doi/abs/10.1021/cr9903205}
}

@ARTICLE{bipartite,
  author = {Fern\'andez-Rossier, J. and Palacios, J. J.},
  title = {Magnetism in Graphene Nanoislands},
  journal = {Phys. Rev. Lett.},
  year = {2007},
  volume = {99},
  pages = {177204},
  month = {Oct},
  doi = {10.1103/PhysRevLett.99.177204},
  file = {bipartite.pdf:bipartite.pdf:PDF},
  issue = {17},
  numpages = {4},
  publisher = {American Physical Society},
  url = {http://link.aps.org/doi/10.1103/PhysRevLett.99.177204}
}

@ARTICLE{CC2-MP2_RI_polarizability,
  author = {Daniel H. Friese and Nina O. C. Winter and Patrick Balzerowski and
	Raffael Schwan and Christof Hattig},
  title = {Large scale polarizability calculations using the approximate coupled
	cluster model CC2 and MP2 combined with the resolution-of-the-identity
	approximation},
  journal = {The Journal of Chemical Physics},
  year = {2012},
  volume = {136},
  pages = {174106},
  number = {17},
  eid = {174106},
  doi = {10.1063/1.4704788},
  file = {CC2-MP2_RI_polarizability.pdf:CC2-MP2_RI_polarizability.pdf:PDF},
  keywords = {coupled cluster calculations; fullerenes; Laplace transforms; organic
	compounds; perturbation theory; polarisability},
  numpages = {14},
  owner = {vijay},
  publisher = {AIP},
  timestamp = {2012.06.08},
  url = {http://link.aip.org/link/?JCP/136/174106/1}
}

@ARTICLE{dft_response,
  author = {Filipp Furche},
  title = {On the density matrix based approach to time-dependent density functional
	response theory},
  journal = {The Journal of Chemical Physics},
  year = {2001},
  volume = {114},
  pages = {5982-5992},
  number = {14},
  doi = {10.1063/1.1353585},
  file = {dft_response.pdf:dft_response.pdf:PDF},
  keywords = {density functional theory; matrix algebra; variational techniques},
  owner = {vijay},
  publisher = {AIP},
  timestamp = {2012.06.04},
  url = {http://link.aip.org/link/?JCP/114/5982/1}
}

@ARTICLE{defects_2007_02,
  author = {Gayathri, V. and Geetha, R.},
  title = {Hydrogen adsorption in defected carbon nanotubes},
  journal = {Adsorption},
  year = {2007},
  volume = {13},
  pages = {53-59},
  note = {10.1007/s10450-007-9002-z},
  abstract = {Recently there has been lot of interest in the development of hydrogen
	storage in various systems for the large-scale application of fuel
	cells, mobiles and for automotive uses. Hectic materials research
	is going on throughout the world with various adsorption mechanisms
	to increase the storage capacity. It was observed that physisorption
	proves to be an effective way for this purpose. Some of the materials
	in this race include graphite, zeolite, carbon fibers and nanotubes.
	Among all these, the versatile material carbon nanotube (CNT) has
	a number of favorable points like porous nature, high surface area,
	hollowness, high stability and light weight, which facilitate the
	hydrogen adsorption in both outer and inner portions. In this work
	we have considered armchair (5,5), zig zag (10,0) and chiral tubes
	(8,2) and (6,4) with and without structural defects to study the
	physisorption of hydrogen on the surface of carbon nanotubes using
	DFT calculations. For two different H 2 configurations, adsorption
	binding energies are estimated both for defect free and defected
	carbon nanotubes. We could observe larger adsorption energies for
	the configuration in which the hydrogen molecular axis perpendicular
	to the hexagonal carbon ring than for parallel to C?C bond configuration
	corresponding to the defect free nanotubes. For defected tubes the
	adsorption energies are calculated for various configurations such
	as molecular axis perpendicular to a defect site octagon and parallel
	to C?C bond of octagon and another case where the axis perpendicular
	to hexagon in defected tube. The adsorption binding energy values
	are compared with defect free case. The results are discussed in
	detail for hydrogen storage applications.},
  affiliation = {Thiagarajar College of Engineering Department of Physics Madurai 625
	015 India Madurai 625 015 India},
  file = {defects_2007_02.pdf:defects_2007_02.pdf:PDF},
  issn = {0929-5607},
  issue = {1},
  keyword = {Chemistry and Materials Science},
  owner = {vijay},
  publisher = {Springer Netherlands},
  timestamp = {2012.06.08},
  url = {http://dx.doi.org/10.1007/s10450-007-9002-z}
}

@ARTICLE{tddft_polari_2000,
  author = {van Gisbergen, S. J. A. and Fonseca Guerra, C. and Baerends, E. J.},
  title = {Towards excitation energies and (hyper)polarizability calculations
	of large molecules. Application of parallelization and linear scaling
	techniques to time-dependent density functional response theory},
  journal = {Journal of Computational Chemistry},
  year = {2000},
  volume = {21},
  pages = {1511--1523},
  number = {16},
  abstract = {We document recent improvements in the efficiency of our implementation
	in the Amsterdam Density Functional program (ADF) of the response
	equations in time-dependent density functional theory (TDDFT). Applications
	to quasi one-dimensional polyene chains and to three-dimensional
	water clusters show that, using our all-electron atomic orbital (AO)-based
	implementation, calculations of excitation energies and (hyper)polarizabilities
	on molecules with several hundred atoms and several thousand basis
	functions are now feasible, even on (a small cluster of) personal
	computers. The matrix elements, which are required in TDDFT, are
	calculated on an AO basis and the same linear scaling techniques
	as used in ADF for the iterative solution of the Kohn–Sham (KS)
	equations are applied to the determination of these matrix elements.
	Near linear scaling is demonstrated for this part of the calculation,
	which used to be the time-determining step. Transformations from
	the AO basis to the KS orbital basis and back exhibit N3 scaling,
	but due to a very small prefactor this N3 scaling is still of little
	importance for currently accessible system sizes. The main CPU bottleneck
	in our current implementation is the multipolar part of the Coulomb
	potential, scaling quadratically with the system size. It is shown
	that the parallelization of our code leads to further significant
	reductions in execution times, with a measured speed-up of 70 on
	90 processors for both the SCF and the TDDFT parts of the code. This
	brings high-level calculations on excitation energies and dynamic
	(hyper)polarizabilities of large molecules within reach. © 2000
	John Wiley & Sons, Inc. J Comput Chem 21: 1511–1523, 2000},
  doi = {10.1002/1096-987X(200012)21:16<1511::AID-JCC8>3.0.CO;2-C},
  file = {tddft_polari_2000.pdf:tddft_polari_2000.pdf:PDF},
  issn = {1096-987X},
  keywords = {linear scaling methods, parallelization, (time-dependent) density
	functional theory, large molecules},
  owner = {vijay},
  publisher = {John Wiley \& Sons, Inc.},
  timestamp = {2012.06.04},
  url = {http://dx.doi.org/10.1002/1096-987X(200012)21:16<1511::AID-JCC8>3.0.CO;2-C}
}

@ARTICLE{defects_2007_10,
  author = {Gou, G. Y. and Pan, B. C. and Shi, L.},
  title = {Theoretical study of size-dependent properties of BN nanotubes with
	intrinsic defects},
  journal = {Phys. Rev. B},
  year = {2007},
  volume = {76},
  pages = {155414},
  month = {Oct},
  doi = {10.1103/PhysRevB.76.155414},
  file = {defects_2007_10.pdf:defects_2007_10.pdf:PDF},
  issue = {15},
  numpages = {6},
  owner = {vijay},
  publisher = {American Physical Society},
  timestamp = {2012.06.08},
  url = {http://link.aps.org/doi/10.1103/PhysRevB.76.155414}
}

@ARTICLE{ccsd_linearresponse,
  author = {Jeff R. Hammond and Karol Kowalski and Wibe A. deJong},
  title = {Dynamic polarizabilities of polyaromatic hydrocarbons using coupled-cluster
	linear response theory},
  journal = {The Journal of Chemical Physics},
  year = {2007},
  volume = {127},
  pages = {144105},
  number = {14},
  eid = {144105},
  doi = {10.1063/1.2772853},
  file = {ccsd_linearresponse.pdf:ccsd_linearresponse.pdf:PDF},
  keywords = {bond lengths; coupled cluster calculations; density functional theory;
	HF calculations; molecular configurations; organic compounds; polarisability},
  numpages = {9},
  owner = {vijay},
  publisher = {AIP},
  timestamp = {2012.06.04},
  url = {http://link.aip.org/link/?JCP/127/144105/1}
}

@ARTICLE{zvector,
  author = {Nicholas C. Handy and Henry F. Schaefer III},
  title = {On the evaluation of analytic energy derivatives for correlated wave
	functions},
  journal = {The Journal of Chemical Physics},
  year = {1984},
  volume = {81},
  pages = {5031-5033},
  number = {11},
  doi = {10.1063/1.447489},
  file = {zvector.pdf:zvector.pdf:PDF},
  keywords = {WAVE FUNCTIONS; CONFIGURATION INTERACTION; ELECTRONIC STRUCTURE; PERTURBATION
	THEORY; MOLECULAR MODELS},
  owner = {vijay},
  publisher = {AIP},
  timestamp = {2012.06.07},
  url = {http://link.aip.org/link/?JCP/81/5031/1}
}

@ARTICLE{Summary_of_CC_response,
  author = {Christof Hattig and Ove Christiansen and Poul Jorgensen},
  title = {Multiphoton transition moments and absorption cross sections in coupled
	cluster response theory employing variational transition moment functionals},
  journal = {The Journal of Chemical Physics},
  year = {1998},
  volume = {108},
  pages = {8331-8354},
  number = {20},
  doi = {10.1063/1.476261},
  file = {Summary_of_CC_response.pdf:Summary_of_CC_response.pdf:PDF},
  keywords = {transition moments; perturbation theory; variational techniques; coupled
	cluster calculations; multiphoton processes; ground states},
  owner = {vijay},
  publisher = {AIP},
  timestamp = {2012.06.08},
  url = {http://link.aip.org/link/?JCP/108/8331/1}
}

@ARTICLE{cclr2,
  author = {Christof Hattig and Ove Christiansen and Poul Jorgensen},
  title = {Cauchy moments and dispersion coefficients using coupled cluster
	linear response theory},
  journal = {The Journal of Chemical Physics},
  year = {1997},
  volume = {107},
  pages = {10592-10598},
  number = {24},
  doi = {10.1063/1.474223},
  file = {cclr2.pdf:cclr2.pdf:PDF},
  keywords = {coupled cluster calculations; neon},
  owner = {vijay},
  publisher = {AIP},
  timestamp = {2012.06.06},
  url = {http://link.aip.org/link/?JCP/107/10592/1}
}

@ARTICLE{lagrange,
  author = {Helgaker, Trygve and J{\"o}rgensen, Poul},
  title = {Configuration-interaction energy derivatives in a fully variational
	formulation},
  journal = {Theoretical Chemistry Accounts: Theory, Computation, and Modeling
	(Theoretica Chimica Acta)},
  year = {1989},
  volume = {75},
  pages = {111-127},
  note = {10.1007/BF00527713},
  abstract = {A configuration-interaction energy function (Lagrange) which is variational
	in all variables, including the orbital rotational parameters, is
	constructed. When this Lagrangian is used for obtaining configuration-interaction
	derivatives, all the important simplifications which occur for derivatives
	of variational wave functions carry over in a straightforward way.
	In particular, the state and orbital rotational response parameters
	obey the 2 n +1 rule and the Lagrange multipliers obey the somewhat
	stronger 2 n +2 rule. The simplifications which are normally obtained
	by invoking the Handy-Schaefer technique are automatically incorporated
	to all orders. Simple expressions for energy derivatives up to third
	order are presented. The relationship between the numerical errors
	in the variational parameters and the errors in the calculated energy
	derivatives is discussed.},
  file = {lagrange.pdf:lagrange.pdf:PDF},
  issn = {1432-881X},
  issue = {2},
  keyword = {Chemistry and Materials Science},
  owner = {vijay},
  publisher = {Springer Berlin / Heidelberg},
  timestamp = {2012.06.07},
  url = {http://dx.doi.org/10.1007/BF00527713}
}

@ARTICLE{dft_polari_2006,
  author = {Artur F. Izmaylov and Edward N. Brothers and Gustavo E. Scuseria},
  title = {Linear-scaling calculation of static and dynamic polarizabilities
	in Hartree-Fock and density functional theory for periodic systems},
  journal = {The Journal of Chemical Physics},
  year = {2006},
  volume = {125},
  pages = {224105},
  number = {22},
  eid = {224105},
  doi = {10.1063/1.2404667},
  file = {coaxial_nntb.pdf:coaxial_nntb.pdf:PDF;dft_polari_2006.pdf:dft_polari_2006.pdf:PDF},
  keywords = {HF calculations; density functional theory; GO calculations; polarisability},
  numpages = {9},
  owner = {vijay},
  publisher = {AIP},
  timestamp = {2012.06.04},
  url = {http://link.aip.org/link/?JCP/125/224105/1}
}

@BOOK{frankjensen,
  title = {Introduction to Computational Chemistry},
  publisher = {john wiley \& sons, ltd.},
  year = {2000},
  editor = {john wiley \& sons, ltd.},
  author = {Frank Jensen},
  file = {frankjensen.pdf:frankjensen.pdf:PDF},
  owner = {vijay},
  timestamp = {2012.06.07}
}

@ARTICLE{defects_2008_10,
  author = {Jia, Gui-Xiao and Li, Jun-Qian and Chen, Lin-Gang and Li, Yi and
	Ding, Kai-Ning and Zhang, Yong-Fan},
  title = {A reasonable criterion of reactivities at the defective region of
	single-walled carbon nanotubes},
  journal = {International Journal of Quantum Chemistry},
  year = {2009},
  volume = {109},
  pages = {668--678},
  number = {4},
  abstract = {Defect directional curvature KD-def was proposed as a reasonable criterion
	for the reactivities and adduct structures at the defective region
	of single-walled carbon nanotubes (SWCNTs). B3LYP/6-31G* calculations
	for the [2 + 1] and [1 + 1] additions of a series of 11-layer (n,
	n) (n = 4?8) and six-layer (10,0) SWCNTs with Stone-Wales defects
	showed that the KD-def or its mean KM-def was a good index to judge
	the adduct structures and the reactivities. Adducts of the [2 + 1]
	additions were divided into two types: one was the adduct with the
	C-X-C configuration and corresponding to the large KD-def and the
	large binding energy, and another was the adduct with the closed
	?3MR structure and corresponding to the small KD-def and the small
	binding energy. It must be pointed out that, besides mainly relying
	on the KD-def, the adduct structures and the reactivities of the
	[2 + 1] additions had been weakly affected by topologic structures.
	The calculated results for the [1 + 1] additions of the 11-layer
	(5,5) SWCNT with defect A revealed that the binding energies monotonously
	increased with the KM-def. © 2008 Wiley Periodicals, Inc. Int J Quantum
	Chem, 2009},
  doi = {10.1002/qua.21859},
  file = {defects_2008_10.pdf:defects_2008_10.pdf:PDF},
  issn = {1097-461X},
  keywords = {SWCNTs, Stone-Wales, directional curvature, defect},
  owner = {vijay},
  publisher = {Wiley Subscription Services, Inc., A Wiley Company},
  timestamp = {2012.06.08},
  url = {http://dx.doi.org/10.1002/qua.21859}
}

@ARTICLE{CCLR_theory,
  author = {Henrik Koch and Poul Jorgensen},
  title = {Coupled cluster response functions},
  journal = {The Journal of Chemical Physics},
  year = {1990},
  volume = {93},
  pages = {3333-3344},
  number = {5},
  doi = {10.1063/1.458814},
  file = {CCLR_theory.pdf:CCLR_theory.pdf:PDF},
  keywords = {RESPONSE FUNCTIONS; TRANSITION PROBABILITIES; QUANTUM CHEMISTRY; EXCITED
	STATES; ELECTRONIC STRUCTURE; MOLECULES},
  owner = {vijay},
  publisher = {AIP},
  timestamp = {2012.06.08},
  url = {http://link.aip.org/link/?JCP/93/3333/1}
}

@ARTICLE{cholesky_cc2,
  author = {Henrik Koch and Alfredo Sanchez de Meras and Thomas Bondo Pedersen},
  title = {Reduced scaling in electronic structure calculations using Cholesky
	decompositions},
  journal = {The Journal of Chemical Physics},
  year = {2003},
  volume = {118},
  pages = {9481-9484},
  number = {21},
  doi = {10.1063/1.1578621},
  file = {cholesky_cc2.pdf:cholesky_cc2.pdf:PDF},
  keywords = {molecular electronic states; quantum chemistry},
  owner = {vijay},
  publisher = {AIP},
  timestamp = {2012.06.05},
  url = {http://link.aip.org/link/?JCP/118/9481/1}
}

@ARTICLE{hydrogenation4,
  author = {W. Kolos and L. Wolniewicz},
  title = {Improved Theoretical Ground-State Energy of the Hydrogen Molecule},
  journal = {The Journal of Chemical Physics},
  year = {1968},
  volume = {49},
  pages = {404-410},
  number = {1},
  doi = {10.1063/1.1669836},
  file = {hydrogenation4.pdf:hydrogenation4.pdf:PDF},
  owner = {vijay},
  publisher = {AIP},
  timestamp = {2012.06.12},
  url = {http://link.aip.org/link/?JCP/49/404/1}
}

@ARTICLE{vacancy2,
  author = {Li, Yafei and Zhou, Zhen and Shen, Panwen and Chen, Zhongfang},
  title = {Spin Gapless Semiconductor−Metal−Half-Metal Properties in Nitrogen-Doped
	Zigzag Graphene Nanoribbons},
  journal = {ACS Nano},
  year = {2009},
  volume = {3},
  pages = {1952-1958},
  number = {7},
  note = {PMID: 19555066},
  abstract = { The geometries, formation energies, and electronic and magnetic properties
	of N-doping defects, including single atom substitution and pyridine-
	and pyrrole-like substructures in zigzag graphene nanoribbons (ZGNRs),
	were investigated by means of spin-unrestricted density functional
	theory computations. The edge carbon atoms are more easily substituted
	with N atoms, and three-nitrogen vacancy (3NV) defect and four-nitrogen
	divacancy (4ND) defect also prefer the ribbon edge. Single N atom
	substitution and pyridine- and pyrrole-like N-doping defects can
	all break the degeneracy of the spin polarization of pristine ZGNRs.
	One single N atom substitution makes the antiferromagnetic semiconducting
	ZGNRs into spin gapless semiconductors, while double edge substitution
	transforms N-doped graphenes into metals. Pyridine- and pyrrole-like
	N-doping defects make ZGNRs into half-metals or spin gapless semiconductors.
	These results suggest the potential applications of N-doped ZGNRs
	in nanoelectronics. },
  doi = {10.1021/nn9003428},
  eprint = {http://pubs.acs.org/doi/pdf/10.1021/nn9003428},
  file = {vacancy2.pdf:vacancy2.pdf:PDF},
  owner = {vijay},
  timestamp = {2012.06.11},
  url = {http://pubs.acs.org/doi/abs/10.1021/nn9003428}
}

@ARTICLE{defects_2006_11,
  author = {Giannis Mpourmpakis and George Froudakis},
  title = {Why alkali metals preferably bind on structural defects of carbon
	nanotubes: A theoretical study by first principles},
  journal = {The Journal of Chemical Physics},
  year = {2006},
  volume = {125},
  pages = {204707},
  number = {20},
  eid = {204707},
  doi = {10.1063/1.2397679},
  file = {defects_2006_11.pdf:defects_2006_11.pdf:PDF},
  keywords = {carbon nanotubes; sodium; defect states; ab initio calculations; binding
	energy; localised states},
  numpages = {5},
  owner = {vijay},
  publisher = {AIP},
  timestamp = {2012.06.08},
  url = {http://link.aip.org/link/?JCP/125/204707/1}
}

@ARTICLE{defects_2009_10,
  author = {Okada, Susumu},
  title = {Atomic configurations and energetics of vacancies in hexagonal boron
	nitride: First-principles total-energy calculations},
  journal = {Phys. Rev. B},
  year = {2009},
  volume = {80},
  pages = {161404},
  month = {Oct},
  doi = {10.1103/PhysRevB.80.161404},
  file = {defects_2009_10.pdf:defects_2009_10.pdf:PDF},
  issue = {16},
  numpages = {4},
  owner = {vijay},
  publisher = {American Physical Society},
  timestamp = {2012.06.08},
  url = {http://link.aps.org/doi/10.1103/PhysRevB.80.161404}
}

@ARTICLE{dft_polari_2002,
  author = {SALEK Pawet and VAHTRAS Olav and HELGAKER Trygve and AGREN Hans},
  title = {Density-functional theory of linear and nonlinear time-dependent
	molecular properties},
  journal = {The Journal of chemical physics},
  year = {2002},
  volume = {117},
  pages = {9630--9645},
  number = {21},
  note = {eng},
  editor = {American Institute of Physics},
  file = {dft_polari_2002.pdf:dft_polari_2002.pdf:PDF},
  issn = {0021-9606},
  owner = {vijay},
  timestamp = {2012.06.04},
  url = {http://www.refdoc.fr/Detailnotice?idarticle=10139191}
}

@ARTICLE{linearresponse,
  author = {Thomas Bondo Pedersen and Alfredo M. J. Sanchez de Meras and Henrik
	Koch},
  title = {Polarizability and optical rotation calculated from the approximate
	coupled cluster singles and doubles CC2 linear response theory using
	Cholesky decompositions},
  journal = {The Journal of Chemical Physics},
  year = {2004},
  volume = {120},
  pages = {8887-8897},
  number = {19},
  doi = {10.1063/1.1705575},
  file = {linearresponse.pdf:linearresponse.pdf:PDF},
  keywords = {optical rotation; coupled cluster calculations; organic compounds;
	polarisability; fullerene compounds},
  owner = {vijay},
  publisher = {AIP},
  timestamp = {2012.06.05},
  url = {http://link.aip.org/link/?JCP/120/8887/1}
}

@ARTICLE{defects_2009_02,
  author = {Rigo, V. A. and Martins, T. B. and da Silva, Antonio J. R. and Fazzio,
	A. and Miwa, R. H.},
  title = {Electronic, structural, and transport properties of Ni-doped graphene
	nanoribbons},
  journal = {Phys. Rev. B},
  year = {2009},
  volume = {79},
  pages = {075435},
  month = {Feb},
  doi = {10.1103/PhysRevB.79.075435},
  file = {defects_2009_02.pdf:defects_2009_02.pdf:PDF},
  issue = {7},
  numpages = {9},
  owner = {vijay},
  publisher = {American Physical Society},
  timestamp = {2012.06.08},
  url = {http://link.aps.org/doi/10.1103/PhysRevB.79.075435}
}

@ARTICLE{ccsd_linearresponse2,
  author = {Nicholas J. Russ and T. Daniel Crawford},
  title = {Local correlation in coupled cluster calculations of molecular response
	properties},
  journal = {Chemical Physics Letters},
  year = {2004},
  volume = {400},
  pages = {104 - 111},
  number = {1–3},
  abstract = {We have extended the local coupled cluster approach of Pulay and Saebø,
	which has seen great success in the computation of ground-state energies,
	to molecular response properties such as dipole polarizabilities.
	This scheme uses an atom-based coupled-perturbed Hartree–Fock breakdown
	of the desired property to expand the usual ground-state orbital
	domains. Benchmark tests of the static polarizabilities of helium
	chains, linear alkanes, and non-saturated systems up to N-acetylglycine
	indicate that the method can reproduce untruncated coupled cluster
	properties to within 1% given appropriately chosen cutoffs, even
	without including orbital relaxation in the method. The method requires
	increased computational demands, but crossover points between non-local
	and local approaches are still well within reach of production-level
	implementations.},
  doi = {10.1016/j.cplett.2004.10.083},
  file = {ccsd_linearresponse2.pdf:ccsd_linearresponse2.pdf:PDF},
  issn = {0009-2614},
  owner = {vijay},
  timestamp = {2012.06.04},
  url = {http://www.sciencedirect.com/science/article/pii/S000926140401680X}
}

@ARTICLE{dft_polar_2000,
  author = {P. R. T. Schipper and O. V. Gritsenko and S. J. A. van Gisbergen
	and E. J. Baerends},
  title = {Molecular calculations of excitation energies and (hyper)polarizabilities
	with a statistical average of orbital model exchange-correlation
	potentials},
  journal = {The Journal of Chemical Physics},
  year = {2000},
  volume = {112},
  pages = {1344-1352},
  number = {3},
  doi = {10.1063/1.480688},
  file = {dft_polar_2000.pdf:dft_polar_2000.pdf:PDF},
  keywords = {density functional theory; orbital calculations; polarisability},
  owner = {vijay},
  publisher = {AIP},
  timestamp = {2012.06.04},
  url = {http://link.aip.org/link/?JCP/112/1344/1}
}

@ARTICLE{vacancy1,
  author = {Srivastava, Deepak and Menon, Madhu and Daraio, C. and Jin, S. and
	Sadanadan, Bindu and Rao, Apparao M.},
  title = {Vacancy-mediated mechanism of nitrogen substitution in carbon nanotubes},
  journal = {Phys. Rev. B},
  year = {2004},
  volume = {69},
  pages = {153414},
  month = {Apr},
  doi = {10.1103/PhysRevB.69.153414},
  file = {vacancy1.pdf:vacancy1.pdf:PDF},
  issue = {15},
  numpages = {4},
  publisher = {American Physical Society},
  url = {http://link.aps.org/doi/10.1103/PhysRevB.69.153414}
}

@ARTICLE{defects_2010_10,
  author = {Zi-Rong Tang},
  title = {The adsorption of methanol at the defective site of single-walled
	carbon nanotube},
  journal = {Physica B: Condensed Matter},
  year = {2010},
  volume = {405},
  pages = {770 - 773},
  number = {2},
  abstract = {The adsorption of methanol on the perfect and defective single-walled
	carbon nanotubes (SWCNTs) has been investigated using effective cluster
	models in conjunction with density functional theory. It has been
	found that methanol is adsorbed very weakly on the sidewall of perfect
	SWCNT, which is in agreement with experiment observation. In contrast,
	it is quite interesting to find that methanol is not only strongly
	chemisorbed at the zigzag edge site of defective SWCNT, but also
	the O?H bond of methanol is completely dissociated. This suggests
	that the zigzag edge of SWCNT can be the active site for adsorption
	and activation of methanol. However, the adsorption of methanol at
	the armchair edge of SWCNT is rather weak, hence suggesting the crucial
	effect of local edge carbon atoms arrangement on the adsorption behavior
	of methanol on carbon nanotubes.},
  doi = {10.1016/j.physb.2009.09.103},
  file = {defects_2010_10.pdf:defects_2010_10.pdf:PDF},
  issn = {0921-4526},
  keywords = {The surface of carbon nanotube},
  owner = {vijay},
  timestamp = {2012.06.08},
  url = {http://www.sciencedirect.com/science/article/pii/S0921452609012125}
}

@ARTICLE{defects_2009_01,
  author = {Tsetseris, Leonidas and Pantelides, Sokrates T.},
  title = {Adsorbate-Induced Defect Formation and Annihilation on Graphene and
	Single-Walled Carbon Nanotubes},
  journal = {The Journal of Physical Chemistry B},
  year = {2009},
  volume = {113},
  pages = {941-944},
  number = {4},
  abstract = { We used density functional theory calculations to probe the chemical
	reactivity of graphene and single-wall carbon nanotubes (CNTs) toward
	the small molecules O2, H2, N2, C2H2, CO, and CO2. We found that
	there is a threshold CNT size below which C2H2 and CO, typical feedstock
	precursors for CNT growth, become trapped in decorated hillock-like
	defects on the side walls of CNTs. We also found that O2, H2, and
	CO2 can etch isolated C adatoms and C adatom pairs. These processes
	play a role not only in the growth of CNTs, but also in the postgrowth
	evolution of defects on CNTs through exposure to typical ambient
	gases. },
  doi = {10.1021/jp809228p},
  eprint = {http://pubs.acs.org/doi/pdf/10.1021/jp809228p},
  file = {defects_2009_01.pdf:defects_2009_01.pdf:PDF},
  owner = {vijay},
  timestamp = {2012.06.08},
  url = {http://pubs.acs.org/doi/abs/10.1021/jp809228p}
}

@ARTICLE{hydrogenation2,
  author = {Yang, Frances H. and Lachawiec, Anthony J. and Yang, Ralph T.},
  title = {Adsorption of Spillover Hydrogen Atoms on Single-Wall Carbon Nanotubes},
  journal = {The Journal of Physical Chemistry B},
  year = {2006},
  volume = {110},
  pages = {6236-6244},
  number = {12},
  abstract = { Spillover of hydrogen on nanostructured carbons is a phenomenon that
	is critical to understand in order to produce efficient hydrogen
	storage adsorbents for fuel cell applications. The spillover and
	interaction of atomic hydrogen with single-walled carbon nanotubes
	(SWNTs) is the focus of this combined theoretical and experimental
	work. To understand the spillover mechanism, very low occupancies
	(i.e., 1 and 2 H atoms adsorbed) on (5,0), (7,0), (9,0) zigzag (semiconducting)
	SWNTs and a (5,5) armchair (metallic) SWNT, with corresponding diameters
	of 3.9, 5.5, 7.0, and 6.8 Å, were investigated. The adsorption binding
	energy of H atoms depends on H occupancy, tube diameter, and helicity
	(or chirality), as well as endohedral (interior) vs exohedral (exterior)
	binding. Exohedral binding energies are substantially higher than
	endohedral binding energies due to easier sp2−sp3 transition in
	hybridization of carbon on exterior walls upon binding. A binding
	energy as low as −8.9 kcal/mol is obtained for 2H atoms on the
	exterior wall of a (5, 0) SWNT. The binding energies of H atoms on
	the metallic SWNT are significantly weaker (about 23 kcal/mol weaker)
	than that on the semiconductor SWNT, for both endohedral and exohedral
	adsorption. The binding energy is generally higher on SWNTs of larger
	diameters, while its dependence on H occupancy is relatively weak
	except at very low occupancies. Experimental results at 298 K and
	for pressures up to 10 MPa with a carbon-bridged composite material
	containing SWNTs demonstrate the presence of multiple adsorption
	sites based on desorption hysteresis for the spiltover H on SWNTs,
	and the experimental results were in qualitative agreement with the
	molecular orbital calculation results. },
  doi = {10.1021/jp056461u},
  eprint = {http://pubs.acs.org/doi/pdf/10.1021/jp056461u},
  file = {hydrogenation2.pdf:hydrogenation2.pdf:PDF},
  owner = {vijay},
  timestamp = {2012.06.11},
  url = {http://pubs.acs.org/doi/abs/10.1021/jp056461u}
}

@ARTICLE{defects_2008_04,
  author = {Yuanfeng Ye and Milin Zhang and Jianwei Zhao and Hongmei Liu and
	Nan Wang},
  title = {Ab initio study on the structural, energetic and electronic features
	of the asymmetric armchair SWCNT junctions},
  journal = {Journal of Molecular Structure: THEOCHEM},
  year = {2008},
  volume = {861},
  pages = {79 - 84},
  number = {1?3},
  abstract = {The structural, energetic and electronic features of asymmetric armchair
	single-walled carbon nanotube (SWCNT) junctions have been studied
	by ab initio calculations at the B3LYP/6-31G?//HF/3-21G? levels.
	The junctions are composed of two SWCNTs with different radius, which
	are connected by a set of 5-membered and 7-membered carbon rings.
	The results show that the metallic?metallic junction is more energetically
	favorable if the junction is formed with a hexagon inserted between
	the pentagon?heptagon (5/7) pair defects in the armchair nanotube.
	The shift of the spatial distribution of HOMO and LUMO shows that
	the asymmetric electronic structure of the junction could be used
	as a molecular rectifier.},
  doi = {10.1016/j.theochem.2008.04.023},
  file = {defects_2008_04.pdf:defects_2008_04.pdf:PDF},
  issn = {0166-1280},
  keywords = {Carbon nanotubes},
  owner = {vijay},
  timestamp = {2012.06.08},
  url = {http://www.sciencedirect.com/science/article/pii/S0166128008002388}
}

@ARTICLE{ammon_spin-1_2000,
  author = {Ammon, Beat and Imada, Masatoshi},
  title = {Spin-1 Chain Doped with Mobile S=1/2 Fermions},
  journal = {Phys. Rev. Lett.},
  year = {2000},
  volume = {85},
  pages = {1056--1059},
  number = {5},
  month = jul,
  abstract = {We investigate the doping of a two-orbital chain with mobile S=1/2
	fermions as a valid model for {Y2−xCaxBaNiO5.} The S=1 spins are
	stabilized by strong, ferromagnetic Hund's rule couplings. We calculate
	correlation functions and thermodynamic quantities by density matrix
	renormalization group methods and find a new hierarchy of energy
	scales in the spin sector upon doping. Gapless spin excitations are
	generated at a lower energy scale by interactions among itinerant
	polarons created by each hole and coexist with the larger scale of
	the gapful spin-liquid background of the S=1 chain accompanied by
	a finite string order parameter.},
  doi = {10.1103/PhysRevLett.85.1056},
  owner = {vijay},
  timestamp = {2014.07.18},
  url = {http://link.aps.org/doi/10.1103/PhysRevLett.85.1056},
  urldate = {2014-06-06}
}

@ARTICLE{aandh,
  author = {Anderson, P. W. and Hasegawa, H.},
  title = {Considerations on Double Exchange},
  journal = {Phys. Rev.},
  year = {1955},
  volume = {100},
  pages = {675--681},
  number = {2},
  month = oct,
  abstract = {Zener has suggested a type of interaction between the spins of magnetic
	ions which he named "double exchange." This occurs indirectly by
	means of spin coupling to mobile electrons which travel from one
	ion to the next. We have calculated this interaction for a pair of
	ions with general spin S and with general transfer integral, b, and
	internal exchange integral J.},
  doi = {10.1103/PhysRev.100.675},
  file = {APS Snapshot:files/367/Anderson and Hasegawa - 1955 - Considerations on Double Exchange.html:text/html;Full Text PDF:files/340/Anderson and Hasegawa - 1955 - Considerations on Double Exchange.pdf:application/pdf},
  owner = {vijay},
  timestamp = {2014.11.27},
  url = {http://link.aps.org/doi/10.1103/PhysRev.100.675},
  urldate = {2014-07-18}
}

@ARTICLE{anderson1,
  author = {Anderson, P. W. and Hasegawa, H.},
  title = {Considerations on Double Exchange},
  journal = {Phys. Rev.},
  year = {1955},
  volume = {100},
  pages = {675--681},
  month = {Oct},
  doi = {10.1103/PhysRev.100.675},
  issue = {2},
  numpages = {0},
  publisher = {American Physical Society},
  url = {http://link.aps.org/doi/10.1103/PhysRev.100.675}
}

@ARTICLE{gp1,
author = {Papaefthymiou, V. and Girerd, J. J. and Moura, I. and Moura, J. J. G. and Muenck, E.},
title = {Moessbauer study of D. gigas ferredoxin II and spin-coupling model for Fe3S4 cluster with valence delocalization},
journal = {J. Am. Chem. Soc.},
volume = {109},
number = {15},
pages = {4703-4710},
year = {1987},
}

@ARTICLE{gp2,
author = {J.-J. Girerd,V. Papaefthymiou,K. K. Surerus and E. Munck},
title = {Double exchange in iron-sulfur clusters and a proposed spin-dependent transfer mechanism},
journal = {Pure Appl. Chem.},
volume = {61},
number = {5},
pages = {805},
year = {1989},
}

@ARTICLE{PhysRevB.74.014432,
  author = {Bastardis, Roland and Guih{\'e}ry, Nathalie and de Graaf, Coen},
  title = {Ab initio},
  journal = {Phys. Rev. B},
  year = {2006},
  volume = {74},
  pages = {014432},
  month = {Jul},
  doi = {10.1103/PhysRevB.74.014432},
  issue = {1},
  numpages = {10},
  publisher = {American Physical Society},
  url = {http://link.aps.org/doi/10.1103/PhysRevB.74.014432}
}

@ARTICLE{batista_spin_1998-1,
  author = {Batista, C. D. and Aligia, A. A. and Eroles, J.},
  title = {Spin dynamics of hole-doped Y2BaNiO5},
  journal = {{EPL}},
  year = {1998},
  volume = {43},
  pages = {71},
  number = {1},
  month = jul,
  abstract = {Starting from a multiband Hamiltonian containing the relevant Ni and
	O orbitals, we derive an effective Hamiltonian Heff for the low-energy
	physics of doped Y2BaNiO5. For hole doping, Heff describes O fermions
	interacting with S = 1 Ni spins in a chain, and cannot be further
	reduced to a simple one-band model. Using numerical techniques, we
	obtain a dynamical spin structure factor with weight inside the Haldane
	gap. The nature of these low-energy excitations is identified and
	the emerging physical picture is consistent with most of the experimental
	information on Y2 − {xCaxBaNiO}5.},
  doi = {10.1209/epl/i1998-00321-x},
  file = {Full Text PDF:files/204/Batista et al. - 1998 - Spin dynamics of hole-doped Y2BaNiO5.pdf:application/pdf;Snapshot:files/184/Batista et al. - 1998 - Spin dynamics of hole-doped Y2BaNiO5.html:text/html},
  issn = {0295-5075},
  language = {en},
  owner = {vijay},
  timestamp = {2014.07.18},
  url = {http://iopscience.iop.org/0295-5075/43/1/071},
  urldate = {2014-06-07}
}

@ARTICLE{batlogg_haldane_1994,
  author = {Batlogg, B. and Cheong, S-W. and Rupp Jr., L. W.},
  title = {Haldane spin state in Y2Ba(Ni, Zn or Mg)O5},
  journal = {Physica B},
  year = {1994},
  volume = {194--196, Part 1},
  pages = {173--174},
  month = feb,
  __markedentry = {[vijay:1]},
  abstract = {We have identified Y2BaNiO5 as a new Haldane state chain compound
	with a magnetic excitation gap Δmin/{kB} of 100±5K and Δmin/{\textbar}J{\textbar}≈0.3.
	Single crystal results of χ(T→0) reveal a splitting of the excited
	magnetic state. The S=1 chains have been severed in a controlled
	way by the substitution of Zn or Mg for Ni, and the resulting modifications
	of χ(T) are presented.},
  doi = {10.1016/0921-4526(94)90416-2},
  file = {ScienceDirect Full Text PDF:files/220/Batlogg et al. - 1994 - Haldane spin state in Y2Ba(Ni, Zn or Mg)O5.pdf:application/pdf;ScienceDirect Snapshot:files/199/Batlogg et al. - 1994 - Haldane spin state in Y2Ba(Ni, Zn or Mg)O5.html:text/html},
  issn = {0921-4526},
  owner = {vijay},
  timestamp = {2014.07.18},
  url = {http://www.sciencedirect.com/science/article/pii/0921452694904162},
  urldate = {2014-06-07}
}

@ARTICLE{Causa98p2,
  author = {M. T. Causa and M. Tovar and A. Caneiro and F. Prado and G. Ibanez
	and C. A. Ramos and A. Butera and B. Alascio},
  title = {High-temperature spin dynamics in CMR manganites: ESR and magnetization},
  journal = {Phys. Rev.},
  year = {1998},
  volume = {58},
  pages = {2 - 4},
  number = {6},
  __markedentry = {[vijay:1]},
  file = {:/home/vijay/Documents/references_galore/cmr_1.pdf::},
  owner = {vijay},
  timestamp = {2014.07.18}
}

@article{cmrTc,
  title = {Colossal magnetoresistance behavior and ESR studies of ${\mathrm{La}}_{1-x}{\mathrm{Te}}_{x}{\mathrm{MnO}}_{3}$ $(0.04&lt;~x&lt;~0.2)$},
  author = {Tan, G. T. and Dai, S. and Duan, P. and Zhou, Y. L. and Lu, H. B. and Chen, Z. H.},
  journal = {Phys. Rev. B},
  volume = {68},
  issue = {1},
  pages = {014426},
  numpages = {5},
  year = {2003},
  month = {Jul},
  publisher = {American Physical Society},
  doi = {10.1103/PhysRevB.68.014426},
  url = {http://link.aps.org/doi/10.1103/PhysRevB.68.014426}
}

@ARTICLE{dagotto_colossal_2001,
  author = {Dagotto, Elbio and Hotta, Takashi and Moreo, Adriana},
  title = {Colossal magnetoresistant materials: the key role of phase separation},
  journal = {Phys. Rep.},
  year = {2001},
  volume = {344},
  pages = {1--153},
  number = {1–3},
  month = apr,
  abstract = {The study of the manganese oxides, widely known as manganites, that
	exhibit the “colossal” magnetoresistance effect is among the
	main areas of research within the area of strongly correlated electrons.
	After considerable theoretical effort in recent years, mainly guided
	by computational and mean-field studies of realistic models, considerable
	progress has been achieved in understanding the curious properties
	of these compounds. These recent studies suggest that the ground
	states of manganite models tend to be intrinsically inhomogeneous
	due to the presence of strong tendencies toward phase separation,
	typically involving ferromagnetic metallic and antiferromagnetic
	charge and orbital ordered insulating domains. Calculations of the
	resistivity versus temperature using mixed states lead to a good
	agreement with experiments. The mixed-phase tendencies have two origins:
	(i) electronic phase separation between phases with different densities
	that lead to nanometer scale coexisting clusters, and (ii) disorder-induced
	phase separation with percolative characteristics between equal-density
	phases, driven by disorder near first-order metal–insulator transitions.
	The coexisting clusters in the latter can be as large as a micrometer
	in size. It is argued that a large variety of experiments reviewed
	in detail here contain results compatible with the theoretical predictions.
	The main phenomenology of mixed-phase states appears to be independent
	of the fine details of the model employed, since the microscopic
	origin of the competing phases does not influence the results at
	the phenomenological level. However, it is quite important to clarify
	the electronic properties of the various manganite phases based on
	microscopic Hamiltonians, including strong electron–phonon {Jahn–Teller}
	and/or Coulomb interactions. Thus, several issues are discussed here
	from the microscopic viewpoint as well, including the phase diagrams
	of manganite models, the stabilization of the charge/orbital/spin
	ordered half-doped correlated electronics ({CE)-states}, the importance
	of the naively small Heisenberg coupling among localized spins, the
	setup of accurate mean-field approximations, the existence of a new
	temperature scale T∗ where clusters start forming above the Curie
	temperature, the presence of stripes in the system, and many others.
	However, much work remains to be carried out, and a list of open
	questions is included here. It is also argued that the mixed-phase
	phenomenology of manganites may appear in a large variety of compounds
	as well, including ruthenates, diluted magnetic semiconductors, and
	others. It is concluded that manganites reveal such a wide variety
	of interesting physical phenomena that their detailed study is quite
	important for progress in the field of correlated electrons.},
  doi = {10.1016/S0370-1573(00)00121-6},
  file = {ScienceDirect Full Text PDF:files/33/Dagotto et al. - 2001 - Colossal magnetoresistant materials the key role .pdf:application/pdf;ScienceDirect Snapshot:files/36/Dagotto et al. - 2001 - Colossal magnetoresistant materials the key role .html:text/html},
  issn = {0370-1573},
  keywords = {Colossal magnetoresistance, Computational physics, Inhomogeneities,
	Manganites, Phase separation},
  owner = {vijay},
  shorttitle = {Colossal magnetoresistant materials},
  timestamp = {2014.07.18},
  url = {http://www.sciencedirect.com/science/article/pii/S0370157300001216},
  urldate = {2014-05-21}
}

@ARTICLE{dagotto_spin_1996,
  author = {Dagotto, E. and Riera, J. and Sandvik, A. and Moreo, A.},
  title = {Spin Dynamics of Hole Doped Y2-{xCaxBaNiO}5},
  journal = {Phys. Rev. Lett.},
  year = {1996},
  volume = {76},
  pages = {1731--1734},
  number = {10},
  month = mar,
  abstract = {We propose an electronic model for the recently discovered hole doped
	compound Y2−{xCaxBaNiO}5. From a multiband Hamiltonian with oxygen
	and nickel orbitals, a one band model is discussed. Holes are described
	using Zhang-Rice-like S=12 states at the nickels propagating on a
	S=1 spin chain. Using numerical techniques to calculate the dynamical
	spin structure factor S(q,ω) in a realistic regime of couplings,
	spectral weight in the Haldane gap is observed in agreement with
	neutron scattering data. The case of static defects relevant for
	Zn-doped chains is also discussed. Ferromagnetic states at high hole
	mobility are favored in our model, contrary to what occurs in the
	1D t-J model.},
  doi = {10.1103/PhysRevLett.76.1731},
  file = {Full Text PDF:files/321/Dagotto et al. - 1996 - Spin Dynamics of Hole Doped Y2-xCaxBaNiO5.pdf:application/pdf;APS Snapshot:files/322/PhysRevLett.76.html:text/html},
  owner = {vijay},
  timestamp = {2014.07.18},
  url = {http://link.aps.org/doi/10.1103/PhysRevLett.76.1731},
  urldate = {2014-07-17}
}

@ARTICLE{darriet_compound_1993,
  author = {Darriet, J. and Regnault, L. P.},
  title = {The compound Y2BaNiO5: A new example of a haldane gap in A S = 1
	magnetic chain},
  journal = {Solid State Commun.},
  year = {1993},
  volume = {86},
  pages = {409--412},
  number = {7},
  month = may,
  __markedentry = {[vijay:5]},
  abstract = {The crystal structure of Y2BaNiO5 is characterized by the existence
	of isolated linear magnetic chains. Nickel is located within compressed
	oxygen octahedra sharing two opposite corners to form ({NiO}5)n chains
	along the a-axis separated by Y and Ba atoms. Within a large temperature
	range, the magnetic susceptibility can be fitted using a one dimensional
	S = 1 Heisenberg model. The best fit is obtained for J/k ≈ −285
	K (Tmax ≈ 410 K). The three dimensional long range magnetic ordering
	has not been observed by neutron diffraction down to 1.8 K, implying
	an inter-intrachain coupling ratio J'/J ⪡ 10−2. Inelastic neutron
	scattering experiments have given evidences for a singlet ground
	state and two gaps in the excitation spectrum at energies Δxy ≈
	8.5 {meV} and Δz ≈ 16 {meV}. Our experimental data are interpreted
	quantitatively within the framework of the Haldane conjecture for
	the S = 1 antiferromagnetic chain.},
  doi = {10.1016/0038-1098(93)90455-V},
  file = {ScienceDirect Full Text PDF:files/186/Darriet and Regnault - 1993 - The compound Y2BaNiO5 A new example of a haldane .pdf:application/pdf;ScienceDirect Snapshot:files/219/Darriet and Regnault - 1993 - The compound Y2BaNiO5 A new example of a haldane .html:text/html},
  issn = {0038-1098},
  owner = {vijay},
  shorttitle = {The compound Y2BaNiO5},
  timestamp = {2014.07.18},
  url = {http://www.sciencedirect.com/science/article/pii/003810989390455V},
  urldate = {2014-06-07}
}

@ARTICLE{nathalie1,
  author = {Guihery, Nathalie and Malrieu, Jean Paul},
  title = {The double exchange mechanism revisited: An ab initio study of the
	[Ni2(napy)4Br2]+ complex},
  journal = {J. Chem. Phys},
  year = {2003},
  volume = {119},
  pages = {8956-8965},
  number = {17},
  doi = {http://dx.doi.org/10.1063/1.1614249},
  url = {http://scitation.aip.org/content/aip/journal/jcp/119/17/10.1063/1.1614249}
}

@ARTICLE{ito_charge_2001,
  author = {Ito, T. and Yamaguchi, H. and Oka, K. and Kojima, K. M. and Eisaki,
	H. and Uchida, S.},
  title = {Charge dynamics of doped holes in one-dimensional S=1 Haldane-gap
	system Y2-{xCaxBaNiO5}},
  journal = {Phys. Rev. B},
  year = {2001},
  volume = {64},
  pages = {060401},
  number = {6},
  month = jun,
  abstract = {The evolution of the anisotropic optical spectra of Y2-{xCaxBaNiO5}
	with x demonstrates that holes are doped into the parent insulator
	and that the dynamics of the holes is one dimensional. In the conductivity
	for E‖ chain a single peak develops with doping, centered at E∼1
	{eV} and within the charge-transfer gap of the parent insulator.
	The peak energy is almost unchanged and its spectral weight increases
	in proportion to x. This feature indicates that the doped holes form
	local bound states rather than a charge ordered state as is realized
	in the two-dimensional La2-{xSrxNiO4+δ.}},
  doi = {10.1103/PhysRevB.64.060401},
  file = {Full Text PDF:files/46/Ito et al. - 2001 - Charge dynamics of doped holes in one-dimensional .pdf:application/pdf;APS Snapshot:files/42/Ito et al. - 2001 - Charge dynamics of doped holes in one-dimensional .html:text/html},
  owner = {vijay},
  timestamp = {2014.07.18},
  url = {http://link.aps.org/doi/10.1103/PhysRevB.64.060401},
  urldate = {2014-05-16}
}

@ARTICLE{Ito00p6,
  author = {T. Ito and H. Yamaguchi and K. Oka and K. M. Kojima and H. Eisaki
	and S. Uchida},
  title = {Charge dynamics of doped holes in one-dimensional S=1 Haldane-gap
	system Y2-xCaxBaNiO5},
  journal = {Phys. Rev.},
  year = {2000},
  volume = {64},
  pages = {6 - 9},
  number = {75},
  __markedentry = {[vijay:5]},
  doi = {10.1103/PhysRevB.64.060401},
  file = {/home/vijay/Documents/references_galore/nickelates/1D_2D_comparison.pdf:/home/vijay/Documents/references_galore/nickelates/1D_2D_comparison.pdf:PDF},
  owner = {vijay},
  timestamp = {2014.07.18}
}

@ARTICLE{izyumov_double_2001,
  author = {Izyumov, Yurii A. and Skryabin, Yu N.},
  title = {Double exchange model and the unique properties of the manganites},
  journal = {Phys.-Usp.},
  year = {2001},
  volume = {44},
  pages = {109},
  number = {2},
  month = feb,
  abstract = {In this review the double exchange ({DE)} model forming a basis for
	the description of the physics of colossal magnetoresistance manganites
	is discussed. For a limiting case of exchange interaction which is
	large compared with the band width, the effective Hamiltonian of
	the {DE} model is derived from that of the sd-exchange model. Since
	this Hamiltonian is very complicated, the dynamical mean field approximation,
	successful for other strongly correlated systems, is found to be
	more suitable for describing the model of interest. Two simplified
	versions of the {DE} model, both capable of accounting for a wide
	range of physical properties, are proposed — one using classical
	localized spins and the other involving quantum spins but no transverse
	spin fluctuations. A temperature–electron concentration phase diagram
	for a system with consideration for the domain of phase separation
	is constructed, whose basic features are shown to be in qualitative
	agreement with experimental data for the manganites, as also are
	the temperature and electron concentration dependences of their electrical
	resistivity, magnetization, and spectral characteristics. At the
	quantitative level, introducing additional electron–lattice interaction
	yields a good agreement. A number of yet unresolved problems in the
	physics of manganites, including the mechanism of temperature- or
	doping-induced metal–insulator phase transition and the nature
	of charge ordering, are also discussed. By comparing predictions
	made by computing approach with the experimental data, the adequacy
	of the {DE} model is assessed and its drawbacks are analyzed. Numerous
	recent theoretical studies of the unique properties of this broad
	class of strongly correlated systems are summarized in this review.},
  doi = {10.1070/PU2001v044n02ABEH000840},
  file = {Full Text PDF:files/56/Izyumov and Skryabin - 2001 - Double exchange model and the unique properties of.pdf:application/pdf;Snapshot:files/43/Izyumov and Skryabin - 2001 - Double exchange model and the unique properties of.html:text/html},
  issn = {1063-7869},
  language = {en},
  owner = {vijay},
  timestamp = {2014.11.25},
  url = {http://iopscience.iop.org/1063-7869/44/2/R01},
  urldate = {2014-05-20}
}

@ARTICLE{jin_colossal_1999,
  author = {Jin, Sungho},
  title = {Colossal magnetoresistance in La-Ca-Mn-O},
  journal = {Metals and Materials},
  year = {1999},
  volume = {5},
  pages = {533--537},
  number = {6},
  month = dec,
  __markedentry = {[vijay:4]},
  abstract = {When an external magnetic field is applied to the colossal magnetoresistance
	({CMR)} materials such as La-Ca-Mn-O, a very large change in electrical
	conductivity by several orders of magnitude is obtained. The magnetoresistance
	is strongly temperature-dependent, and exhibits a sharp peak below
	room temperature, which can be shifted by adjusting the composition
	or processing parameters. The control of lattice geometry or strain,
	e.g., by chemical substitution, epitaxial growth or post-deposition
	anneal of thin films appears to be crucial in obtaining the {CMR}
	properties. The orders of magnitude change in electrical resistivity
	could be useful for various magnetic and electric device applications.},
  doi = {10.1007/BF03026300},
  file = {Full Text:files/49/Jin - 1999 - Colossal magnetoresistance in La-Ca-Mn-O.pdf:application/pdf;Snapshot:files/49/Jin - 1999 - Colossal magnetoresistance in La-Ca-Mn-O.html:text/html},
  issn = {1598-9623},
  keywords = {Characterization and Evaluation of Materials, Continuum Mechanics
	and Mechanics of Materials, electrical conductivity, Engineering
	Thermodynamics, Heat and Mass Transfer, Magnetism, Magnetic Materials,
	magnetoresistance, Metallic Materials, Operating Procedures, Materials
	Treatment, thin films},
  language = {en},
  owner = {vijay},
  timestamp = {2014.05.24},
  url = {http://link.springer.com/article/10.1007/BF03026300},
  urldate = {2014-05-24}
}

@ARTICLE{Jonker1953120,
  author = {G.H. Jonker and J.H. Van Santen},
  title = {Magnetic compounds wtth perovskite structure III. ferromagnetic compounds
	of cobalt },
  journal = {Physica },
  year = {1953},
  volume = {19},
  pages = {120 - 130},
  number = {1?12},
  abstract = {Synopsis Polycrystalline mixed crystals (La, Sr) CoO3 have been prepared.
	Perovskite structure is found for all compositions. Ferromagnetism
	is observed for intermediate Sr concentrations. Curves are given
	for the saturation magnetizations, the paramagnetic Curie temperatures,
	and the effective paramagnetic moments as a function of composition.
	It is suggested that the ferromagnetism observed is caused essentially
	by a positive Co3+ ? Co4+ interaction. The sign of the exchange interaction
	is discussed in connection with the theories of Anderson and Polder,
	and of Zener. },
  doi = {http://dx.doi.org/10.1016/S0031-8914(53)80011-X},
  issn = {0031-8914},
  url = {http://www.sciencedirect.com/science/article/pii/S003189145380011X}
}

@ARTICLE{Jonker1950337,
  author = {G.H. Jonker and J.H. Van Santen},
  title = {Ferromagnetic compounds of manganese with perovskite structure },
  journal = {Physica },
  year = {1950},
  volume = {16},
  pages = {337 - 349},
  number = {3},
  abstract = {Various manganites of the general formula La3+Mn3+O32?-Me2+Mn4+O32?
	have been prepared in the form of polycrystalline products. Perovskite
	structures were found, i.a. for all mixed crystals LaMnO3?CaMnO3,
	for LaMnO3?SrMnO3 containing up to 70% SrMnO3, and for LaMnO3?BaMnO3
	containing less than 50% BaMnO3. The mixed crystals with perovskite
	structure are ferromagnetic. Curves for the Curie temperature versus
	composition and saturation versus composition are given for LaMnO3?CaMnO3,
	LaMnO3?SrMnO3, and LaMnO3?BaMnO3. Both types of curves show maxima
	between 25 and 40% Me2+Mn4+O32?; here all 3d electrons available
	contribute with their spins to the saturation magnetization. The
	ferromagnetic properties can be understood as the result of a strong
	positive Mn3+?Mn4+ exchange interaction combined with a weak Mn3+?Mn3+
	interaction and a negative Mn4+?Mn4+ interaction. The Mn3+?Mn4+ interaction,
	presumably of the indirect exchange type, is thought to be the first
	clear example of positive exchange interaction in oxidic substances.
	},
  doi = {http://dx.doi.org/10.1016/0031-8914(50)90033-4},
  issn = {0031-8914},
  url = {http://www.sciencedirect.com/science/article/pii/0031891450900334}
}

@ARTICLE{kumar_low-temperature_2002,
  author = {Kumar, D. and Sankar, J. and Narayan, J. and Singh, Rajiv K. and
	Majumdar, A. K.},
  title = {Low-temperature resistivity minima in colossal magnetoresistive {La0.7Ca0.3MnO3}
	thin films},
  journal = {Phys. Rev. B},
  year = {2002},
  volume = {65},
  pages = {094407},
  number = {9},
  month = feb,
  __markedentry = {[vijay:1]},
  abstract = {The low-temperature magnetoresistance of {La0.7Ca0.3MnO3} ({LCMO)}
	thin films has been investigated using a four-probe dc technique
	with a 5 T superconducting magnet. Thin film samples of {LCMO} were
	prepared in situ using a pulsed laser deposition technique. The results
	obtained from the high-resolution low-temperature (5–50 K) measurements,
	carried out on various samples differing widely in their resistivities,
	have shown distinct minima at Tm in the resistivity versus temperature
	plots for all fields. The depth of the resistance minima was found
	to increase with an increase in applied magnetic field H, while Tm
	versus H curves showed maxima at around 2 T. We have fitted the resistivity
	versus temperature data for all H to an expression that contains
	three terms, namely, residual resistivity, inelastic scattering,
	and electron-electron (e−e) interaction and Kondo effects. We conclude
	that the e−e interaction effect is the dominant mechanism for the
	negative temperature coefficient of resistivity of these colossal
	magnetoresistance ({CMR)} materials at low temperatures.},
  doi = {10.1103/PhysRevB.65.094407},
  file = {Full Text PDF:files/59/Kumar et al. - 2002 - Low-temperature resistivity minima in colossal mag.pdf:application/pdf;APS Snapshot:files/41/Kumar et al. - 2002 - Low-temperature resistivity minima in colossal mag.html:text/html},
  owner = {vijay},
  timestamp = {2014.05.24},
  url = {http://link.aps.org/doi/10.1103/PhysRevB.65.094407},
  urldate = {2014-05-24}
}

@ARTICLE{chemrev,
  author = {Malrieu, Jean Paul and Caballol, Rosa and Calzado, Carmen J. and
	de Graaf, Coen and Guihéry, Nathalie},
  title = {Magnetic Interactions in Molecules and Highly Correlated Materials:
	Physical Content, Analytical Derivation, and Rigorous Extraction
	of Magnetic Hamiltonians},
  journal = {Chem. Rev.},
  year = {2014},
  volume = {114},
  pages = {429--492},
  number = {1},
  month = jan,
  doi = {10.1021/cr300500z},
  file = {ACS Full Text PDF w/ Links:files/368/Malrieu et al. - 2014 - Magnetic Interactions in Molecules and Highly Corr.pdf:application/pdf;ACS Full Text Snapshot:files/372/Malrieu et al. - 2014 - Magnetic Interactions in Molecules and Highly Corr.html:text/html},
  issn = {0009-2665},
  owner = {vijay},
  shorttitle = {Magnetic Interactions in Molecules and Highly Correlated Materials},
  timestamp = {2014.11.27},
  url = {http://dx.doi.org/10.1021/cr300500z},
  urldate = {2014-07-18}
}

@ARTICLE{malvezzi_origin_2001,
  author = {Malvezzi, André Luiz and Dagotto, Elbio},
  title = {Origin of spin incommensurability in hole-doped S=1 Y2-{xCaxBaNiO}5
	chains},
  journal = {Phys. Rev. B},
  year = {2001},
  volume = {63},
  pages = {140409},
  number = {14},
  month = mar,
  __markedentry = {[vijay:5]},
  abstract = {Spin incommensurability ({IC}) has been recently experimentally discovered
	in the hole-doped Ni-oxide chain compound Y2-{xCaxBaNiO}5 [G. Xu
	et al., Science 289, 419 (2000)]. Here a two orbital model for this
	material is studied using computational techniques. Spin {IC} is
	observed in a wide range of densities and couplings. The phenomenon
	originates in antiferromagnetic correlations “across holes” dynamically
	generated to improve hole movement, as it occurs in the one-dimensional
	Hubbard model and in recent studies of the two-dimensional extended
	t−J model. The close proximity of ferromagnetic and phase-separated
	states in parameter space is also discussed.},
  doi = {10.1103/PhysRevB.63.140409},
  file = {Full Text PDF:files/235/Malvezzi and Dagotto - 2001 - Origin of spin incommensurability in hole-doped S=.pdf:application/pdf;APS Snapshot:files/236/PhysRevB.63.html:text/html},
  owner = {vijay},
  timestamp = {2014.07.18},
  url = {http://link.aps.org/doi/10.1103/PhysRevB.63.140409},
  urldate = {2014-06-09}
}

@ARTICLE{pati_low-lying_1997,
  author = {Pati, Swapan K. and Ramasesha, S. and Sen, Diptiman},
  title = {Low-lying excited states and low-temperature properties of an alternating
	spin-1–spin-1/2 chain: A density-matrix renormalization-group study},
  journal = {Phys. Rev. B},
  year = {1997},
  volume = {55},
  pages = {8894--8904},
  number = {14},
  month = apr,
  abstract = {We report spin wave and density-matrix renormalization-group ({DMRG)}
	studies of the ground and low-lying excited states of uniform and
	dimerized alternating spin chains. The {DMRG} procedure is also employed
	to obtain low-temperature thermodynamic properties of the system.
	The ground state of a {2N} spin system with spin-1 and spin- alternating
	from site to site and interacting via an antiferromagnetic exchange
	is found to be ferrimagnetic with total spin {sG=N/2} from both {DMRG}
	and spin wave analysis. Both the studies also show that there is
	a gapless excitation to a state with spin {sG-1} and a gapped excitation
	to a state with spin {sG+1.} Surprisingly, the correlation length
	in the ground state is found to be very small from both the studies
	for this gapless system. For this very reason, we show that the ground
	state can be described by a variational ansatz of the product type.
	{DMRG} analysis shows that the chain is susceptible to a conditional
	spin-Peierls' instability. The {DMRG} studies of magnetization, magnetic
	susceptibility (χ), and specific heat show strong magnetic-field
	dependence. The product {χT} shows a minimum as a function of temperature
	(T) at low-magnetic fields and the minimum vanishes at high-magnetic
	fields. This low-field behavior is in agreement with earlier experimental
	observations. The specific heat shows a maximum as a function of
	temperature and the height of the maximum increases sharply at high-magnetic
	fields. It is hoped that these studies will motivate experimental
	studies at high-magnetic fields.},
  doi = {10.1103/PhysRevB.55.8894},
  file = {Full Text PDF:files/109/Pati et al. - 1997 - Low-lying excited states and low-temperature prope.pdf:application/pdf;APS Snapshot:files/133/Pati et al. - 1997 - Low-lying excited states and low-temperature prope.html:text/html},
  owner = {vijay},
  shorttitle = {Low-lying excited states and low-temperature properties of an alternating
	spin-1–spin-1/2 chain},
  timestamp = {2014.07.18},
  url = {http://link.aps.org/doi/10.1103/PhysRevB.55.8894},
  urldate = {2014-05-31}
}

@ARTICLE{Ramirez13p21,
  author = {F. E. N. Ramirez and E. Francisquini and J. A. Souza},
  title = {Nature of short-range ordre in paramagnetic state of manganites},
  journal = {J. Alloys Compd.},
  year = {2013},
  volume = {571},
  pages = {21 - 24},
  number = {3},
  __markedentry = {[vijay:5]},
  doi = {10.1016/j.jallcom.2013.03.045},
  file = {:/home/vijay/Documents/references_galore/curie_weiss_2.pdf::},
  keywords = {manganites},
  owner = {vijay},
  timestamp = {2014.07.18},
  url = {www.elsevier.com/locate/jalcom}
}

@ARTICLE{sasaki_hole-induced_2005,
  author = {Sasaki, Tomoyuki and Yokoo, Tetsuya and Katano, Susumu and Akimitsu,
	Jun},
  title = {Hole-induced Novel Spin State within Haldane Gap in Nd2-{xCaxBaNiO}5},
  journal = {J. Phys. Soc. Jpn.},
  year = {2005},
  volume = {74},
  pages = {267--270},
  number = {1},
  month = jan,
  abstract = {Static and dynamical spin properties in a hole-doped one-dimensional
	Haldane chain system have been investigated by magnetization and
	neutron scattering measurements on antiferromagnetic Nd 2- x Ca x
	{BaNiO} 5 . Spin-glass-like weak ferromagnetic behavior and an incommensurate
	dynamical structure were observed, which are similar to those reported
	in the Haldane system Y 2- x Ca x {BaNiO} 5 . A novel excitation,
	moreover, has been revealed at 10 {meV} because of complete the shift
	in the Haldane gap due to three-dimensional antiferromagnetic ordering,
	which possibly provides a total picture of a hole-induced quantum
	Haldane chain.},
  doi = {10.1143/JPSJ.74.267},
  file = {Full Text PDF:files/225/Sasaki et al. - 2005 - Hole-induced Novel Spin State within Haldane Gap i.pdf:application/pdf;Snapshot:files/226/JPSJ.74.html:text/html},
  issn = {0031-9015},
  owner = {vijay},
  timestamp = {2014.07.18},
  url = {http://journals.jps.jp/doi/abs/10.1143/JPSJ.74.267},
  urldate = {2014-06-09}
}

@ARTICLE{Souza08p054411,
  author = {J. A. Souza and H. Terashita and E. Granado and R. F. Jardim and
	N. F. Oliveira Jr and R. Muccillo},
  title = {Polaron liquid-gas crossover at the orthorhombic-rhombohedral transition
	of manganites},
  journal = {Phys. Rev.},
  year = {2008},
  volume = {78},
  pages = {054411},
  number = {75},
  __markedentry = {[vijay:1]},
  doi = {10.1103/PhysRevB.78.054411},
  file = {:/home/vijay/Documents/references_galore/lcmo_exp_7.pdf::},
  owner = {vijay},
  timestamp = {2014.07.18}
}

@ARTICLE{Stripes98p2150,
  author = {J. M. Tranquada},
  title = {CHARGE STRIPES AND ANTIFERROMAGNETISM IN INSULATING NICKELATES AND
	SUPERCONDUCTING CUPRATES},
  journal = {J. Phys. Chem. Solids},
  year = {1998},
  volume = {59},
  pages = {2150 - 2154},
  number = {10 - 12},
  abstract = {Neutron and X-ray scattering studies have provided strong evidence
	for coupled spatial modulations of},
  file = {:/home/vijay/Documents/references_galore/nickelates/review.pdf::},
  keywords = {antiferromagnetism, layered nickelates and cuprates, change order},
  owner = {vijay},
  timestamp = {2014.11.25}
}

@ARTICLE{xu_holes_2000,
  author = {Xu, Guangyong and Aeppli, G. and Bisher, M. E. and Broholm, C. and
	DiTusa, J. F. and Frost, C. D. and Ito, T. and Oka, K. and Paul,
	R. L. and Takagi, H. and Treacy, M. M. J.},
  title = {Holes in a Quantum Spin Liquid},
  journal = {Science},
  year = {2000},
  volume = {289},
  pages = {419--422},
  number = {5478},
  month = jul,
  __markedentry = {[vijay:1]},
  abstract = {Magnetic neutron scattering provides evidence for nucleation of antiferromagnetic
	droplets around impurities in a doped nickel oxide–based quantum
	magnet. The undoped parent compound contains a spin liquid with a
	cooperative singlet ground state and a gap in the magnetic excitation
	spectrum. Calcium doping creates excitations below the gap with an
	incommensurate structure factor. We show that weakly interacting
	antiferromagnetic droplets with a central phase shift of π and a
	size controlled by the correlation length of the quantum liquid can
	account for the data. The experiment provides a quantitative impression
	of the magnetic polarization cloud associated with holes in a doped
	transition metal oxide.},
  doi = {10.1126/science.289.5478.419},
  file = {Full Text PDF:files/228/Xu et al. - 2000 - Holes in a Quantum Spin Liquid.pdf:application/pdf;Snapshot:files/230/419.html:text/html},
  issn = {0036-8075, 1095-9203},
  language = {en},
  owner = {vijay},
  pmid = {10903195},
  timestamp = {2014.07.18},
  url = {http://www.sciencemag.org/content/289/5478/419},
  urldate = {2014-06-09}
}

@ARTICLE{kohn,
  author = {Kohn, Walter},
  title = {Theory of the Insulating State},
  journal = {Phys. Rev.},
  year = {1964},
  volume = {133},
  pages = {A171--A181},
  month = {Jan},
  doi = {10.1103/PhysRev.133.A171},
  issue = {1A},
  numpages = {11},
  publisher = {American Physical Society},
  url = {http://link.aps.org/doi/10.1103/PhysRev.133.A171}
}

@ARTICLE{resta_electron_1999,
  author = {Resta, Raffaele and Sorella, Sandro},
  title = {Electron {Localization} in the {Insulating} {State}},
  journal = {Physical Review Letters},
  year = {1999},
  volume = {82},
  pages = {370--373},
  number = {2},
  month = jan,
  abstract = {The insulating state of matter is characterized by the excitation
	spectrum, but also by qualitative features of the electronic ground
	state. The insulating ground wave function in fact (i) sustains macroscopic
	polarization, and (ii) is localized. We give a sharp definition of
	the latter concept and we show how the two basic features stem from
	essentially the same formalism. Our approach to localization is exemplified
	by means of a two-band Hubbard model in one dimension. In the noninteracting
	limit, the wave function localization is measured by the spread of
	the Wannier orbitals.},
  doi = {10.1103/PhysRevLett.82.370},
  file = {APS Snapshot:files/909/PhysRevLett.82.html:text/html;Full Text PDF:files/908/Resta and Sorella - 1999 - Electron Localization in the Insulating State.pdf:application/pdf},
  url = {http://link.aps.org/doi/10.1103/PhysRevLett.82.370},
  urldate = {2015-07-28}
}

@ARTICLE{nagaoka,
  author = {Yosuke Nagaoka},
  title = {Ground state of correlated electrons in a narrow almost half-filled
	s band },
  journal = {Solid State Communications },
  year = {1965},
  volume = {3},
  pages = {409 - 412},
  number = {12},
  abstract = {We consider a system of conduction electrons in an almost half-filled
	s band with an infinitely strong ?-function type repulsive potential,
	and with non-vanishing transfer matrix elements only between nearest
	neighbors. We find rigorously that the totally polarized ferromagnetic
	state is the ground state for sc and bcc and for fcc and hcp with
	Ne &gt; N, Ne and N being respectively the number of electrons and
	atoms, and that it is not the ground state for fcc and hcp with Ne
	&lt; N. },
  doi = {http://dx.doi.org/10.1016/0038-1098(65)90266-8},
  issn = {0038-1098},
  url = {http://www.sciencedirect.com/science/article/pii/0038109865902668}
}

@ARTICLE{ammon_spin-1_2000,
  author = {Ammon, Beat and Imada, Masatoshi},
  title = {Spin-1 Chain Doped with Mobile S=1/2 Fermions},
  journal = {Phys. Rev. Lett.},
  year = {2000},
  volume = {85},
  pages = {1056--1059},
  number = {5},
  month = jul,
  abstract = {We investigate the doping of a two-orbital chain with mobile S=1/2
	fermions as a valid model for {Y2−xCaxBaNiO5.} The S=1 spins are
	stabilized by strong, ferromagnetic Hund's rule couplings. We calculate
	correlation functions and thermodynamic quantities by density matrix
	renormalization group methods and find a new hierarchy of energy
	scales in the spin sector upon doping. Gapless spin excitations are
	generated at a lower energy scale by interactions among itinerant
	polarons created by each hole and coexist with the larger scale of
	the gapful spin-liquid background of the S=1 chain accompanied by
	a finite string order parameter.},
  doi = {10.1103/PhysRevLett.85.1056},
  owner = {vijay},
  timestamp = {2014.07.18},
  url = {http://link.aps.org/doi/10.1103/PhysRevLett.85.1056},
  urldate = {2014-06-06}
}

@ARTICLE{aandh,
  author = {Anderson, P. W. and Hasegawa, H.},
  title = {Considerations on Double Exchange},
  journal = {Phys. Rev.},
  year = {1955},
  volume = {100},
  pages = {675--681},
  number = {2},
  month = oct,
  abstract = {Zener has suggested a type of interaction between the spins of magnetic
	ions which he named "double exchange." This occurs indirectly by
	means of spin coupling to mobile electrons which travel from one
	ion to the next. We have calculated this interaction for a pair of
	ions with general spin S and with general transfer integral, b, and
	internal exchange integral J.},
  doi = {10.1103/PhysRev.100.675},
  file = {APS Snapshot:files/367/Anderson and Hasegawa - 1955 - Considerations on Double Exchange.html:text/html;Full Text PDF:files/340/Anderson and Hasegawa - 1955 - Considerations on Double Exchange.pdf:application/pdf},
  owner = {vijay},
  timestamp = {2014.11.27},
  url = {http://link.aps.org/doi/10.1103/PhysRev.100.675},
  urldate = {2014-07-18}
}

@article{gp1,
	title = {Double exchange in iron-sulfur clusters and a proposed spin-dependent transfer mechanism},
	volume = {61},
	url = {http://www.degruyter.com/view/j/pac.1989.61.issue-5/pac198961050805/pac198961050805.xml},
	number = {5},
	urldate = {2015-06-20},
	journal = {Pure and Applied Chemistry},
	author = {Girerd, J.-J. and Papaefthymiou, V. and Surerus, K. K. and Munck, E.},
	month = jan,
	year = {1989},
	pages = {805--816},
	file = {Full Text PDF:files/720/Girerd et al. - 1989 - Double exchange in iron-sulfur clusters and a prop.pdf:application/pdf;Snapshot:files/721/pac198961050805.html:text/html}
}

@article{gp2,
	title = {Moessbauer study of {D}. gigas ferredoxin {II} and spin-coupling model for {Fe}3S4 cluster with valence delocalization},
	volume = {109},
	issn = {0002-7863},
	url = {http://dx.doi.org/10.1021/ja00249a037},
	doi = {10.1021/ja00249a037},
	number = {15},
	urldate = {2015-06-20},
	journal = {Journal of the American Chemical Society},
	author = {Papaefthymiou, V. and Girerd, J. J. and Moura, I. and Moura, J. J. G. and Muenck, E.},
	month = jul,
	year = {1987},
	pages = {4703--4710},
	file = {ACS Full Text PDF:files/717/Papaefthymiou et al. - 1987 - Moessbauer study of D. gigas ferredoxin II and spi.pdf:application/pdf;ACS Full Text Snapshot:files/718/ja00249a037.html:text/html}
}

@ARTICLE{PhysRevB.74.014432,
  author = {Bastardis, Roland and Guih{\'e}ry, Nathalie and de Graaf, Coen},
  title = {Ab initio},
  journal = {Phys. Rev. B},
  year = {2006},
  volume = {74},
  pages = {014432},
  month = {Jul},
  doi = {10.1103/PhysRevB.74.014432},
  issue = {1},
  numpages = {10},
  publisher = {American Physical Society},
  url = {http://link.aps.org/doi/10.1103/PhysRevB.74.014432}
}

@ARTICLE{batista_spin_1998-1,
  author = {Batista, C. D. and Aligia, A. A. and Eroles, J.},
  title = {Spin dynamics of hole-doped Y2BaNiO5},
  journal = {{EPL}},
  year = {1998},
  volume = {43},
  pages = {71},
  number = {1},
  month = jul,
  abstract = {Starting from a multiband Hamiltonian containing the relevant Ni and
	O orbitals, we derive an effective Hamiltonian Heff for the low-energy
	physics of doped Y2BaNiO5. For hole doping, Heff describes O fermions
	interacting with S = 1 Ni spins in a chain, and cannot be further
	reduced to a simple one-band model. Using numerical techniques, we
	obtain a dynamical spin structure factor with weight inside the Haldane
	gap. The nature of these low-energy excitations is identified and
	the emerging physical picture is consistent with most of the experimental
	information on Y2 − {xCaxBaNiO}5.},
  doi = {10.1209/epl/i1998-00321-x},
  file = {Full Text PDF:files/204/Batista et al. - 1998 - Spin dynamics of hole-doped Y2BaNiO5.pdf:application/pdf;Snapshot:files/184/Batista et al. - 1998 - Spin dynamics of hole-doped Y2BaNiO5.html:text/html},
  issn = {0295-5075},
  language = {en},
  owner = {vijay},
  timestamp = {2014.07.18},
  url = {http://iopscience.iop.org/0295-5075/43/1/071},
  urldate = {2014-06-07}
}

@ARTICLE{lamas_combined_2011,
  author = {Lamas, C. A. and Capponi, S. and Pujol, P.},
  title = {Combined analytical and numerical approach to study magnetization
	plateaux in doped quasi-one-dimensional antiferromagnets},
  journal = {Physical Review B},
  year = {2011},
  volume = {84},
  pages = {115125},
  number = {11},
  month = sep,
  note = {00003},
  abstract = {We investigate the magnetic properties of quasi-one-dimensional quantum
	spin-S antiferromagnets. We use a combination of analytical and numerical
	techniques to study the presence of plateaux in the magnetization
	curve. The analytical technique consists in a path integral formulation
	in terms of coherent states. This technique can be extended to the
	presence of doping and has the advantage of a much better control
	for large spins than the usual bosonization technique. We discuss
	the appearance of doping-dependent plateaux in the magnetization
	curves for spin-S chains and ladders. The analytical results are
	complemented by a density matrix renormalization group (DMRG) study
	for a trimerized spin-1/2 and anisotropic spin-3/2 doped chains.},
  doi = {10.1103/PhysRevB.84.115125},
  file = {APS Snapshot:files/674/Lamas et al. - 2011 - Combined analytical and numerical approach to stud.html:text/html;Full Text PDF:files/314/Lamas et al. - 2011 - Combined analytical and numerical approach to stud.pdf:application/pdf},
  owner = {vijay},
  timestamp = {2015.06.16},
  url = {http://link.aps.org/doi/10.1103/PhysRevB.84.115125},
  urldate = {2014-07-17}
}

@ARTICLE{garcia_charge_2002,
  author = {Garcia, D. J. and Hallberg, K. and Batista, C. D. and Capponi, S.
	and Poilblanc, D. and Avignon, M. and Alascio, B.},
  title = {Charge and spin inhomogeneous phases in the ferromagnetic {Kondo}
	lattice model},
  journal = {Physical Review B},
  year = {2002},
  volume = {65},
  pages = {134444},
  number = {13},
  month = mar,
  __markedentry = {[vijay:1]},
  abstract = {We study numerically the one-dimensional ferromagnetic Kondo lattice.
	This model is widely used to describe nickel and manganese perovskites.
	Due to the competition between double and superexchange, we find
	a region where the formation of magnetic polarons induces a charge-ordered
	state. This ordering is present even in the absence of any intersite
	Coulomb repulsion. There is an insulating gap associated to the charge
	structure formation. We also study the insulator-metal transition
	induced by a magnetic field, which removes simultaneously both charge
	and spin ordering.},
  doi = {10.1103/PhysRevB.65.134444},
  owner = {vijay},
  timestamp = {2015.06.16},
  url = {http://link.aps.org/doi/10.1103/PhysRevB.65.134444},
  urldate = {2015-06-11}
}

@ARTICLE{batlogg_haldane_1994,
  author = {Batlogg, B. and Cheong, S-W. and Rupp Jr., L. W.},
  title = {Haldane spin state in Y2Ba(Ni, Zn or Mg)O5},
  journal = {Physica B},
  year = {1994},
  volume = {194–196, Part 1},
  pages = {173--174},
  month = feb,
  __markedentry = {[vijay:1]},
  abstract = {We have identified Y2BaNiO5 as a new Haldane state chain compound
	with a magnetic excitation gap Δmin/{kB} of 100±5K and Δmin/{\textbar}J{\textbar}≈0.3.
	Single crystal results of χ(T→0) reveal a splitting of the excited
	magnetic state. The S=1 chains have been severed in a controlled
	way by the substitution of Zn or Mg for Ni, and the resulting modifications
	of χ(T) are presented.},
  doi = {10.1016/0921-4526(94)90416-2},
  file = {ScienceDirect Full Text PDF:files/220/Batlogg et al. - 1994 - Haldane spin state in Y2Ba(Ni, Zn or Mg)O5.pdf:application/pdf;ScienceDirect Snapshot:files/199/Batlogg et al. - 1994 - Haldane spin state in Y2Ba(Ni, Zn or Mg)O5.html:text/html},
  issn = {0921-4526},
  owner = {vijay},
  timestamp = {2014.07.18},
  url = {http://www.sciencedirect.com/science/article/pii/0921452694904162},
  urldate = {2014-06-07}
}

@ARTICLE{Causa98p2,
  author = {M. T. Causa and M. Tovar and A. Caneiro and F. Prado and G. Ibanez
	and C. A. Ramos and A. Butera and B. Alascio},
  title = {High-temperature spin dynamics in CMR manganites: ESR and magnetization},
  journal = {Phys. Rev.},
  year = {1998},
  volume = {58},
  pages = {2 - 4},
  number = {6},
  __markedentry = {[vijay:1]},
  file = {:/home/vijay/Documents/references_galore/cmr_1.pdf::},
  owner = {vijay},
  timestamp = {2014.07.18}
}

@article{cmrTc,
  title = {Colossal magnetoresistance behavior and ESR studies of ${\mathrm{La}}_{1-x}{\mathrm{Te}}_{x}{\mathrm{MnO}}_{3}$ $(0.04&lt;~x&lt;~0.2)$},
  author = {Tan, G. T. and Dai, S. and Duan, P. and Zhou, Y. L. and Lu, H. B. and Chen, Z. H.},
  journal = {Phys. Rev. B},
  volume = {68},
  issue = {1},
  pages = {014426},
  numpages = {5},
  year = {2003},
  month = {Jul},
  publisher = {American Physical Society},
  doi = {10.1103/PhysRevB.68.014426},
  url = {http://link.aps.org/doi/10.1103/PhysRevB.68.014426}
}

@ARTICLE{dagotto_colossal_2001,
  author = {Dagotto, Elbio and Hotta, Takashi and Moreo, Adriana},
  title = {Colossal magnetoresistant materials: the key role of phase separation},
  journal = {Phys. Rep.},
  year = {2001},
  volume = {344},
  pages = {1--153},
  number = {1–3},
  month = apr,
  abstract = {The study of the manganese oxides, widely known as manganites, that
	exhibit the “colossal” magnetoresistance effect is among the
	main areas of research within the area of strongly correlated electrons.
	After considerable theoretical effort in recent years, mainly guided
	by computational and mean-field studies of realistic models, considerable
	progress has been achieved in understanding the curious properties
	of these compounds. These recent studies suggest that the ground
	states of manganite models tend to be intrinsically inhomogeneous
	due to the presence of strong tendencies toward phase separation,
	typically involving ferromagnetic metallic and antiferromagnetic
	charge and orbital ordered insulating domains. Calculations of the
	resistivity versus temperature using mixed states lead to a good
	agreement with experiments. The mixed-phase tendencies have two origins:
	(i) electronic phase separation between phases with different densities
	that lead to nanometer scale coexisting clusters, and (ii) disorder-induced
	phase separation with percolative characteristics between equal-density
	phases, driven by disorder near first-order metal–insulator transitions.
	The coexisting clusters in the latter can be as large as a micrometer
	in size. It is argued that a large variety of experiments reviewed
	in detail here contain results compatible with the theoretical predictions.
	The main phenomenology of mixed-phase states appears to be independent
	of the fine details of the model employed, since the microscopic
	origin of the competing phases does not influence the results at
	the phenomenological level. However, it is quite important to clarify
	the electronic properties of the various manganite phases based on
	microscopic Hamiltonians, including strong electron–phonon {Jahn–Teller}
	and/or Coulomb interactions. Thus, several issues are discussed here
	from the microscopic viewpoint as well, including the phase diagrams
	of manganite models, the stabilization of the charge/orbital/spin
	ordered half-doped correlated electronics ({CE)-states}, the importance
	of the naively small Heisenberg coupling among localized spins, the
	setup of accurate mean-field approximations, the existence of a new
	temperature scale T∗ where clusters start forming above the Curie
	temperature, the presence of stripes in the system, and many others.
	However, much work remains to be carried out, and a list of open
	questions is included here. It is also argued that the mixed-phase
	phenomenology of manganites may appear in a large variety of compounds
	as well, including ruthenates, diluted magnetic semiconductors, and
	others. It is concluded that manganites reveal such a wide variety
	of interesting physical phenomena that their detailed study is quite
	important for progress in the field of correlated electrons.},
  doi = {10.1016/S0370-1573(00)00121-6},
  file = {ScienceDirect Full Text PDF:files/33/Dagotto et al. - 2001 - Colossal magnetoresistant materials the key role .pdf:application/pdf;ScienceDirect Snapshot:files/36/Dagotto et al. - 2001 - Colossal magnetoresistant materials the key role .html:text/html},
  issn = {0370-1573},
  keywords = {Colossal magnetoresistance, Computational physics, Inhomogeneities,
	Manganites, Phase separation},
  owner = {vijay},
  shorttitle = {Colossal magnetoresistant materials},
  timestamp = {2014.07.18},
  url = {http://www.sciencedirect.com/science/article/pii/S0370157300001216},
  urldate = {2014-05-21}
}

@ARTICLE{rincon_exotic_2014,
  author = {Rincón, Julián and Moreo, Adriana and Alvarez, Gonzalo and Dagotto,
	Elbio},
  title = {Exotic {Magnetic} {Order} in the {Orbital}-{Selective} {Mott} {Regime}
	of {Multiorbital} {Systems}},
  journal = {Physical Review Letters},
  year = {2014},
  volume = {112},
  pages = {106405},
  number = {10},
  month = mar,
  __markedentry = {[vijay:1]},
  abstract = {The orbital-selective Mott phase of multiorbital Hubbard models has
	been extensively analyzed before using static and dynamical mean-field
	approximations. In parallel, the properties of block states (antiferromagnetically
	coupled ferromagnetic spin clusters) in Fe-based superconductors
	have also been much discussed. The present effort uses numerically
	exact techniques in one-dimensional systems to report the observation
	of block states within the orbital-selective Mott phase regime, connecting
	two seemingly independent areas of research, and providing analogies
	with the physics of double-exchange models.},
  doi = {10.1103/PhysRevLett.112.106405},
  file = {APS Snapshot:files/310/Rincón et al. - 2014 - Exotic Magnetic Order in the Orbital-Selective Mot.html:text/html;Full Text PDF:files/694/Rincón et al. - 2014 - Exotic Magnetic Order in the Orbital-Selective Mot.pdf:application/pdf},
  owner = {vijay},
  timestamp = {2015.06.16},
  url = {http://link.aps.org/doi/10.1103/PhysRevLett.112.106405},
  urldate = {2015-06-15}
}

@ARTICLE{darriet_compound_1993,
  author = {Darriet, J. and Regnault, L. P.},
  title = {The compound Y2BaNiO5: A new example of a haldane gap in A S = 1
	magnetic chain},
  journal = {Solid State Commun.},
  year = {1993},
  volume = {86},
  pages = {409--412},
  number = {7},
  month = may,
  __markedentry = {[vijay:5]},
  abstract = {The crystal structure of Y2BaNiO5 is characterized by the existence
	of isolated linear magnetic chains. Nickel is located within compressed
	oxygen octahedra sharing two opposite corners to form ({NiO}5)n chains
	along the a-axis separated by Y and Ba atoms. Within a large temperature
	range, the magnetic susceptibility can be fitted using a one dimensional
	S = 1 Heisenberg model. The best fit is obtained for J/k ≈ −285
	K (Tmax ≈ 410 K). The three dimensional long range magnetic ordering
	has not been observed by neutron diffraction down to 1.8 K, implying
	an inter-intrachain coupling ratio J'/J ⪡ 10−2. Inelastic neutron
	scattering experiments have given evidences for a singlet ground
	state and two gaps in the excitation spectrum at energies Δxy ≈
	8.5 {meV} and Δz ≈ 16 {meV}. Our experimental data are interpreted
	quantitatively within the framework of the Haldane conjecture for
	the S = 1 antiferromagnetic chain.},
  doi = {10.1016/0038-1098(93)90455-V},
  file = {ScienceDirect Full Text PDF:files/186/Darriet and Regnault - 1993 - The compound Y2BaNiO5 A new example of a haldane .pdf:application/pdf;ScienceDirect Snapshot:files/219/Darriet and Regnault - 1993 - The compound Y2BaNiO5 A new example of a haldane .html:text/html},
  issn = {0038-1098},
  owner = {vijay},
  shorttitle = {The compound Y2BaNiO5},
  timestamp = {2014.07.18},
  url = {http://www.sciencedirect.com/science/article/pii/003810989390455V},
  urldate = {2014-06-07}
}

@ARTICLE{nathalie1,
  author = {Guihéry, Nathalie and Malrieu, Jean Paul},
  title = {The double exchange mechanism revisited: An ab initio study of the
	[Ni2(napy)4Br2]+ complex},
  journal = {J. Chem. Phys},
  year = {2003},
  volume = {119},
  pages = {8956-8965},
  number = {17},
  doi = {http://dx.doi.org/10.1063/1.1614249},
  url = {http://scitation.aip.org/content/aip/journal/jcp/119/17/10.1063/1.1614249}
}

@ARTICLE{ito_charge_2001,
  author = {Ito, T. and Yamaguchi, H. and Oka, K. and Kojima, K. M. and Eisaki,
	H. and Uchida, S.},
  title = {Charge dynamics of doped holes in one-dimensional S=1 Haldane-gap
	system Y2-{xCaxBaNiO5}},
  journal = {Phys. Rev. B},
  year = {2001},
  volume = {64},
  pages = {060401},
  number = {6},
  month = jun,
  abstract = {The evolution of the anisotropic optical spectra of Y2-{xCaxBaNiO5}
	with x demonstrates that holes are doped into the parent insulator
	and that the dynamics of the holes is one dimensional. In the conductivity
	for E‖ chain a single peak develops with doping, centered at E∼1
	{eV} and within the charge-transfer gap of the parent insulator.
	The peak energy is almost unchanged and its spectral weight increases
	in proportion to x. This feature indicates that the doped holes form
	local bound states rather than a charge ordered state as is realized
	in the two-dimensional La2-{xSrxNiO4+δ.}},
  doi = {10.1103/PhysRevB.64.060401},
  file = {Full Text PDF:files/46/Ito et al. - 2001 - Charge dynamics of doped holes in one-dimensional .pdf:application/pdf;APS Snapshot:files/42/Ito et al. - 2001 - Charge dynamics of doped holes in one-dimensional .html:text/html},
  owner = {vijay},
  timestamp = {2014.07.18},
  url = {http://link.aps.org/doi/10.1103/PhysRevB.64.060401},
  urldate = {2014-05-16}
}

@ARTICLE{Ito00p6,
  author = {T. Ito and H. Yamaguchi and K. Oka and K. M. Kojima and H. Eisaki
	and S. Uchida},
  title = {Charge dynamics of doped holes in one-dimensional S=1 Haldane-gap
	system Y2-xCaxBaNiO5},
  journal = {Phys. Rev.},
  year = {2000},
  volume = {64},
  pages = {6 - 9},
  number = {75},
  __markedentry = {[vijay:5]},
  doi = {10.1103/PhysRevB.64.060401},
  file = {/home/vijay/Documents/references_galore/nickelates/1D_2D_comparison.pdf:/home/vijay/Documents/references_galore/nickelates/1D_2D_comparison.pdf:PDF},
  owner = {vijay},
  timestamp = {2014.07.18}
}

@ARTICLE{izyumov_double_2001,
  author = {Izyumov, Yurii A. and Skryabin, Yu N.},
  title = {Double exchange model and the unique properties of the manganites},
  journal = {Phys.-Usp.},
  year = {2001},
  volume = {44},
  pages = {109},
  number = {2},
  month = feb,
  abstract = {In this review the double exchange ({DE)} model forming a basis for
	the description of the physics of colossal magnetoresistance manganites
	is discussed. For a limiting case of exchange interaction which is
	large compared with the band width, the effective Hamiltonian of
	the {DE} model is derived from that of the sd-exchange model. Since
	this Hamiltonian is very complicated, the dynamical mean field approximation,
	successful for other strongly correlated systems, is found to be
	more suitable for describing the model of interest. Two simplified
	versions of the {DE} model, both capable of accounting for a wide
	range of physical properties, are proposed — one using classical
	localized spins and the other involving quantum spins but no transverse
	spin fluctuations. A temperature–electron concentration phase diagram
	for a system with consideration for the domain of phase separation
	is constructed, whose basic features are shown to be in qualitative
	agreement with experimental data for the manganites, as also are
	the temperature and electron concentration dependences of their electrical
	resistivity, magnetization, and spectral characteristics. At the
	quantitative level, introducing additional electron–lattice interaction
	yields a good agreement. A number of yet unresolved problems in the
	physics of manganites, including the mechanism of temperature- or
	doping-induced metal–insulator phase transition and the nature
	of charge ordering, are also discussed. By comparing predictions
	made by computing approach with the experimental data, the adequacy
	of the {DE} model is assessed and its drawbacks are analyzed. Numerous
	recent theoretical studies of the unique properties of this broad
	class of strongly correlated systems are summarized in this review.},
  doi = {10.1070/PU2001v044n02ABEH000840},
  file = {Full Text PDF:files/56/Izyumov and Skryabin - 2001 - Double exchange model and the unique properties of.pdf:application/pdf;Snapshot:files/43/Izyumov and Skryabin - 2001 - Double exchange model and the unique properties of.html:text/html},
  issn = {1063-7869},
  language = {en},
  owner = {vijay},
  timestamp = {2014.11.25},
  url = {http://iopscience.iop.org/1063-7869/44/2/R01},
  urldate = {2014-05-20}
}

@ARTICLE{Jonker1953120,
  author = {G.H. Jonker and J.H. Van Santen},
  title = {Magnetic compounds wtth perovskite structure III. ferromagnetic compounds
	of cobalt },
  journal = {Physica },
  year = {1953},
  volume = {19},
  pages = {120 - 130},
  number = {1?12},
  abstract = {Synopsis Polycrystalline mixed crystals (La, Sr) CoO3 have been prepared.
	Perovskite structure is found for all compositions. Ferromagnetism
	is observed for intermediate Sr concentrations. Curves are given
	for the saturation magnetizations, the paramagnetic Curie temperatures,
	and the effective paramagnetic moments as a function of composition.
	It is suggested that the ferromagnetism observed is caused essentially
	by a positive Co3+ ? Co4+ interaction. The sign of the exchange interaction
	is discussed in connection with the theories of Anderson and Polder,
	and of Zener. },
  doi = {http://dx.doi.org/10.1016/S0031-8914(53)80011-X},
  issn = {0031-8914},
  url = {http://www.sciencedirect.com/science/article/pii/S003189145380011X}
}

@ARTICLE{Jonker1950337,
  author = {G.H. Jonker and J.H. Van Santen},
  title = {Ferromagnetic compounds of manganese with perovskite structure },
  journal = {Physica },
  year = {1950},
  volume = {16},
  pages = {337 - 349},
  number = {3},
  abstract = {Various manganites of the general formula La3+Mn3+O32?-Me2+Mn4+O32?
	have been prepared in the form of polycrystalline products. Perovskite
	structures were found, i.a. for all mixed crystals LaMnO3?CaMnO3,
	for LaMnO3?SrMnO3 containing up to 70% SrMnO3, and for LaMnO3?BaMnO3
	containing less than 50% BaMnO3. The mixed crystals with perovskite
	structure are ferromagnetic. Curves for the Curie temperature versus
	composition and saturation versus composition are given for LaMnO3?CaMnO3,
	LaMnO3?SrMnO3, and LaMnO3?BaMnO3. Both types of curves show maxima
	between 25 and 40% Me2+Mn4+O32?; here all 3d electrons available
	contribute with their spins to the saturation magnetization. The
	ferromagnetic properties can be understood as the result of a strong
	positive Mn3+?Mn4+ exchange interaction combined with a weak Mn3+?Mn3+
	interaction and a negative Mn4+?Mn4+ interaction. The Mn3+?Mn4+ interaction,
	presumably of the indirect exchange type, is thought to be the first
	clear example of positive exchange interaction in oxidic substances.
	},
  doi = {http://dx.doi.org/10.1016/0031-8914(50)90033-4},
  issn = {0031-8914},
  url = {http://www.sciencedirect.com/science/article/pii/0031891450900334}
}

@ARTICLE{kumar_low-temperature_2002,
  author = {Kumar, D. and Sankar, J. and Narayan, J. and Singh, Rajiv K. and
	Majumdar, A. K.},
  title = {Low-temperature resistivity minima in colossal magnetoresistive {La0.7Ca0.3MnO3}
	thin films},
  journal = {Phys. Rev. B},
  year = {2002},
  volume = {65},
  pages = {094407},
  number = {9},
  month = feb,
  __markedentry = {[vijay:1]},
  abstract = {The low-temperature magnetoresistance of {La0.7Ca0.3MnO3} ({LCMO)}
	thin films has been investigated using a four-probe dc technique
	with a 5 T superconducting magnet. Thin film samples of {LCMO} were
	prepared in situ using a pulsed laser deposition technique. The results
	obtained from the high-resolution low-temperature (5–50 K) measurements,
	carried out on various samples differing widely in their resistivities,
	have shown distinct minima at Tm in the resistivity versus temperature
	plots for all fields. The depth of the resistance minima was found
	to increase with an increase in applied magnetic field H, while Tm
	versus H curves showed maxima at around 2 T. We have fitted the resistivity
	versus temperature data for all H to an expression that contains
	three terms, namely, residual resistivity, inelastic scattering,
	and electron-electron (e−e) interaction and Kondo effects. We conclude
	that the e−e interaction effect is the dominant mechanism for the
	negative temperature coefficient of resistivity of these colossal
	magnetoresistance ({CMR)} materials at low temperatures.},
  doi = {10.1103/PhysRevB.65.094407},
  file = {Full Text PDF:files/59/Kumar et al. - 2002 - Low-temperature resistivity minima in colossal mag.pdf:application/pdf;APS Snapshot:files/41/Kumar et al. - 2002 - Low-temperature resistivity minima in colossal mag.html:text/html},
  owner = {vijay},
  timestamp = {2014.05.24},
  url = {http://link.aps.org/doi/10.1103/PhysRevB.65.094407},
  urldate = {2014-05-24}
}

@ARTICLE{chemrev,
  author = {Malrieu, Jean Paul and Caballol, Rosa and Calzado, Carmen J. and
	de Graaf, Coen and Guihéry, Nathalie},
  title = {Magnetic Interactions in Molecules and Highly Correlated Materials:
	Physical Content, Analytical Derivation, and Rigorous Extraction
	of Magnetic Hamiltonians},
  journal = {Chem. Rev.},
  year = {2014},
  volume = {114},
  pages = {429--492},
  number = {1},
  month = jan,
  doi = {10.1021/cr300500z},
  file = {ACS Full Text PDF w/ Links:files/368/Malrieu et al. - 2014 - Magnetic Interactions in Molecules and Highly Corr.pdf:application/pdf;ACS Full Text Snapshot:files/372/Malrieu et al. - 2014 - Magnetic Interactions in Molecules and Highly Corr.html:text/html},
  issn = {0009-2665},
  owner = {vijay},
  shorttitle = {Magnetic Interactions in Molecules and Highly Correlated Materials},
  timestamp = {2014.11.27},
  url = {http://dx.doi.org/10.1021/cr300500z},
  urldate = {2014-07-18}
}

@ARTICLE{malvezzi_origin_2001,
  author = {Malvezzi, André Luiz and Dagotto, Elbio},
  title = {Origin of spin incommensurability in hole-doped S=1 Y2-{xCaxBaNiO}5
	chains},
  journal = {Phys. Rev. B},
  year = {2001},
  volume = {63},
  pages = {140409},
  number = {14},
  month = mar,
  __markedentry = {[vijay:5]},
  abstract = {Spin incommensurability ({IC}) has been recently experimentally discovered
	in the hole-doped Ni-oxide chain compound Y2-{xCaxBaNiO}5 [G. Xu
	et al., Science 289, 419 (2000)]. Here a two orbital model for this
	material is studied using computational techniques. Spin {IC} is
	observed in a wide range of densities and couplings. The phenomenon
	originates in antiferromagnetic correlations “across holes” dynamically
	generated to improve hole movement, as it occurs in the one-dimensional
	Hubbard model and in recent studies of the two-dimensional extended
	t−J model. The close proximity of ferromagnetic and phase-separated
	states in parameter space is also discussed.},
  doi = {10.1103/PhysRevB.63.140409},
  file = {Full Text PDF:files/235/Malvezzi and Dagotto - 2001 - Origin of spin incommensurability in hole-doped S=.pdf:application/pdf;APS Snapshot:files/236/PhysRevB.63.html:text/html},
  owner = {vijay},
  timestamp = {2014.07.18},
  url = {http://link.aps.org/doi/10.1103/PhysRevB.63.140409},
  urldate = {2014-06-09}
}

@ARTICLE{pati_low-lying_1997,
  author = {Pati, Swapan K. and Ramasesha, S. and Sen, Diptiman},
  title = {Low-lying excited states and low-temperature properties of an alternating
	spin-1–spin-1/2 chain: A density-matrix renormalization-group study},
  journal = {Phys. Rev. B},
  year = {1997},
  volume = {55},
  pages = {8894--8904},
  number = {14},
  month = apr,
  abstract = {We report spin wave and density-matrix renormalization-group ({DMRG)}
	studies of the ground and low-lying excited states of uniform and
	dimerized alternating spin chains. The {DMRG} procedure is also employed
	to obtain low-temperature thermodynamic properties of the system.
	The ground state of a {2N} spin system with spin-1 and spin- alternating
	from site to site and interacting via an antiferromagnetic exchange
	is found to be ferrimagnetic with total spin {sG=N/2} from both {DMRG}
	and spin wave analysis. Both the studies also show that there is
	a gapless excitation to a state with spin {sG-1} and a gapped excitation
	to a state with spin {sG+1.} Surprisingly, the correlation length
	in the ground state is found to be very small from both the studies
	for this gapless system. For this very reason, we show that the ground
	state can be described by a variational ansatz of the product type.
	{DMRG} analysis shows that the chain is susceptible to a conditional
	spin-Peierls' instability. The {DMRG} studies of magnetization, magnetic
	susceptibility (χ), and specific heat show strong magnetic-field
	dependence. The product {χT} shows a minimum as a function of temperature
	(T) at low-magnetic fields and the minimum vanishes at high-magnetic
	fields. This low-field behavior is in agreement with earlier experimental
	observations. The specific heat shows a maximum as a function of
	temperature and the height of the maximum increases sharply at high-magnetic
	fields. It is hoped that these studies will motivate experimental
	studies at high-magnetic fields.},
  doi = {10.1103/PhysRevB.55.8894},
  file = {Full Text PDF:files/109/Pati et al. - 1997 - Low-lying excited states and low-temperature prope.pdf:application/pdf;APS Snapshot:files/133/Pati et al. - 1997 - Low-lying excited states and low-temperature prope.html:text/html},
  owner = {vijay},
  shorttitle = {Low-lying excited states and low-temperature properties of an alternating
	spin-1–spin-1/2 chain},
  timestamp = {2014.07.18},
  url = {http://link.aps.org/doi/10.1103/PhysRevB.55.8894},
  urldate = {2014-05-31}
}

@ARTICLE{Ramirez,
  author = {F. E. N. Ramirez and E. Francisquini and J. A. Souza},
  title = {Nature of short-range ordre in paramagnetic state of manganites},
  journal = {J. Alloys Compd.},
  year = {2013},
  volume = {571},
  pages = {21 - 24},
  number = {3},
  __markedentry = {[vijay:5]},
  doi = {10.1016/j.jallcom.2013.03.045},
  file = {:/home/vijay/Documents/references_galore/curie_weiss_2.pdf::},
  keywords = {manganites},
  owner = {vijay},
  timestamp = {2014.07.18},
  url = {www.elsevier.com/locate/jalcom}
}

@ARTICLE{sippel,
  author = {Sippel, P. and Krohns, S. and Thoms, E. and Ruff, E. and Riegg, S.
	and Kirchhain, H. and Schrettle, F. and Reller, A. and Lunkenheimer,
	P. and Loidl, A.},
  title = {Dielectric signature of charge order in lanthanum nickelates},
  journal = {The European Physical Journal B},
  year = {2012},
  volume = {85},
  pages = {1--8},
  number = {7},
  month = jul,
  __markedentry = {[vijay:1]},
  abstract = {Three charge-ordering lanthanum nickelates La2¿x A x NiO4, substituted
	with specific amounts of A = Sr, Ca, and Ba to achieve commensurate
	charge order, are investigated using broadband dielectric spectroscopy
	up to GHz frequencies. The transition temperatures of the samples
	are characterized by additional specific heat and magnetic susceptibility
	measurements. We find colossal magnitudes of the dielectric constant
	for all three compounds and strong relaxation features, which partly
	are of Maxwell-Wagner type arising from electrode polarization. Quite
	unexpectedly, the temperature-dependent colossal dielectric constants
	of these materials exhibit distinct anomalies at the charge-order
	transitions. This phenomenon is ascribed to a variation of intrinsic
	material properties affecting the formation of depletion layers at
	the electrode-sample interfaces.},
  doi = {10.1140/epjb/e2012-30183-2},
  issn = {1434-6028, 1434-6036},
  keywords = {Condensed Matter Physics, Fluid- and Aerodynamics, Physics, general,
	Solid State Physics, Solid State and Materials, Statistical Physics,
	Dynamical Systems and Complexity},
  language = {en},
  owner = {vijay},
  timestamp = {2015.05.26},
  url = {http://link.springer.com/article/10.1140/epjb/e2012-30183-2},
  urldate = {2015-05-17TZ}
}

@ARTICLE{sasaki_hole-induced_2005,
  author = {Sasaki, Tomoyuki and Yokoo, Tetsuya and Katano, Susumu and Akimitsu,
	Jun},
  title = {Hole-induced Novel Spin State within Haldane Gap in Nd2-{xCaxBaNiO}5},
  journal = {J. Phys. Soc. Jpn.},
  year = {2005},
  volume = {74},
  pages = {267--270},
  number = {1},
  month = jan,
  abstract = {Static and dynamical spin properties in a hole-doped one-dimensional
	Haldane chain system have been investigated by magnetization and
	neutron scattering measurements on antiferromagnetic Nd 2- x Ca x
	{BaNiO} 5 . Spin-glass-like weak ferromagnetic behavior and an incommensurate
	dynamical structure were observed, which are similar to those reported
	in the Haldane system Y 2- x Ca x {BaNiO} 5 . A novel excitation,
	moreover, has been revealed at 10 {meV} because of complete the shift
	in the Haldane gap due to three-dimensional antiferromagnetic ordering,
	which possibly provides a total picture of a hole-induced quantum
	Haldane chain.},
  doi = {10.1143/JPSJ.74.267},
  file = {Full Text PDF:files/225/Sasaki et al. - 2005 - Hole-induced Novel Spin State within Haldane Gap i.pdf:application/pdf;Snapshot:files/226/JPSJ.74.html:text/html},
  issn = {0031-9015},
  owner = {vijay},
  timestamp = {2014.07.18},
  url = {http://journals.jps.jp/doi/abs/10.1143/JPSJ.74.267},
  urldate = {2014-06-09}
}

@ARTICLE{Souza08p054411,
  author = {J. A. Souza and H. Terashita and E. Granado and R. F. Jardim and
	N. F. Oliveira Jr and R. Muccillo},
  title = {Polaron liquid-gas crossover at the orthorhombic-rhombohedral transition
	of manganites},
  journal = {Phys. Rev.},
  year = {2008},
  volume = {78},
  pages = {054411},
  number = {75},
  __markedentry = {[vijay:1]},
  doi = {10.1103/PhysRevB.78.054411},
  file = {:/home/vijay/Documents/references_galore/lcmo_exp_7.pdf::},
  owner = {vijay},
  timestamp = {2014.07.18}
}

@ARTICLE{Stripes98p2150,
  author = {J. M. Tranquada},
  title = {CHARGE STRIPES AND ANTIFERROMAGNETISM IN INSULATING NICKELATES AND
	SUPERCONDUCTING CUPRATES},
  journal = {J. Phys. Chem. Solids},
  year = {1998},
  volume = {59},
  pages = {2150 - 2154},
  number = {10 - 12},
  abstract = {Neutron and X-ray scattering studies have provided strong evidence
	for coupled spatial modulations of},
  file = {:/home/vijay/Documents/references_galore/nickelates/review.pdf::},
  keywords = {antiferromagnetism, layered nickelates and cuprates, change order},
  owner = {vijay},
  timestamp = {2014.11.25}
}

@ARTICLE{xu_holes_2000,
  author = {Xu, Guangyong and Aeppli, G. and Bisher, M. E. and Broholm, C. and
	DiTusa, J. F. and Frost, C. D. and Ito, T. and Oka, K. and Paul,
	R. L. and Takagi, H. and Treacy, M. M. J.},
  title = {Holes in a Quantum Spin Liquid},
  journal = {Science},
  year = {2000},
  volume = {289},
  pages = {419--422},
  number = {5478},
  month = jul,
  __markedentry = {[vijay:1]},
  abstract = {Magnetic neutron scattering provides evidence for nucleation of antiferromagnetic
	droplets around impurities in a doped nickel oxide–based quantum
	magnet. The undoped parent compound contains a spin liquid with a
	cooperative singlet ground state and a gap in the magnetic excitation
	spectrum. Calcium doping creates excitations below the gap with an
	incommensurate structure factor. We show that weakly interacting
	antiferromagnetic droplets with a central phase shift of π and a
	size controlled by the correlation length of the quantum liquid can
	account for the data. The experiment provides a quantitative impression
	of the magnetic polarization cloud associated with holes in a doped
	transition metal oxide.},
  doi = {10.1126/science.289.5478.419},
  file = {Full Text PDF:files/228/Xu et al. - 2000 - Holes in a Quantum Spin Liquid.pdf:application/pdf;Snapshot:files/230/419.html:text/html},
  issn = {0036-8075, 1095-9203},
  language = {en},
  owner = {vijay},
  pmid = {10903195},
  timestamp = {2014.07.18},
  url = {http://www.sciencemag.org/content/289/5478/419},
  urldate = {2014-06-09}
}

@ARTICLE{zener1,
  author = {Zener, Clarence},
  title = {Interaction between the $d$-Shells in the Transition Metals. II.
	Ferromagnetic Compounds of Manganese with Perovskite Structure},
  journal = {Phys. Rev.},
  year = {1951},
  volume = {82},
  pages = {403--405},
  month = {May},
  doi = {10.1103/PhysRev.82.403},
  issue = {3},
  numpages = {0},
  publisher = {American Physical Society},
  url = {http://link.aps.org/doi/10.1103/PhysRev.82.403}
}

@ARTICLE{kohn,
  author = {Kohn, Walter},
  title = {Theory of the Insulating State},
  journal = {Phys. Rev.},
  year = {1964},
  volume = {133},
  pages = {A171--A181},
  month = {Jan},
  doi = {10.1103/PhysRev.133.A171},
  issue = {1A},
  numpages = {11},
  publisher = {American Physical Society},
  url = {http://link.aps.org/doi/10.1103/PhysRev.133.A171}
}

@ARTICLE{jin_colossal_1999,
  author = {Jin, Sungho},
  title = {Colossal magnetoresistance in La-Ca-Mn-O},
  journal = {Metals and Materials},
  year = {1999},
  volume = {5},
  pages = {533--537},
  number = {6},
  month = dec,
  __markedentry = {[vijay:4]},
  abstract = {When an external magnetic field is applied to the colossal magnetoresistance
	({CMR)} materials such as La-Ca-Mn-O, a very large change in electrical
	conductivity by several orders of magnitude is obtained. The magnetoresistance
	is strongly temperature-dependent, and exhibits a sharp peak below
	room temperature, which can be shifted by adjusting the composition
	or processing parameters. The control of lattice geometry or strain,
	e.g., by chemical substitution, epitaxial growth or post-deposition
	anneal of thin films appears to be crucial in obtaining the {CMR}
	properties. The orders of magnitude change in electrical resistivity
	could be useful for various magnetic and electric device applications.},
  doi = {10.1007/BF03026300},
  file = {Full Text:files/49/Jin - 1999 - Colossal magnetoresistance in La-Ca-Mn-O.pdf:application/pdf;Snapshot:files/49/Jin - 1999 - Colossal magnetoresistance in La-Ca-Mn-O.html:text/html},
  issn = {1598-9623},
  keywords = {Characterization and Evaluation of Materials, Continuum Mechanics
	and Mechanics of Materials, electrical conductivity, Engineering
	Thermodynamics, Heat and Mass Transfer, Magnetism, Magnetic Materials,
	magnetoresistance, Metallic Materials, Operating Procedures, Materials
	Treatment, thin films},
  language = {en},
  owner = {vijay},
  timestamp = {2014.05.24},
  url = {http://link.springer.com/article/10.1007/BF03026300},
  urldate = {2014-05-24}
}

@ARTICLE{schiffer_low_1995,
  author = {Schiffer, P. and Ramirez, A. P. and Bao, W. and Cheong, S-W.},
  title = {Low {Temperature} {Magnetoresistance} and the {Magnetic} {Phase}
	{Diagram} of \$\{{\textbackslash}mathrm\{{La}\}\}\_\{1- {\textbackslash}mathit\{x\}\}\{{\textbackslash}mathrm\{{Ca}\}\}\_\{{\textbackslash}mathit\{x\}\}\{{\textbackslash}mathrm\{{MnO}\}\}\_\{3\}\$},
  journal = {Physical Review Letters},
  year = {1995},
  volume = {75},
  pages = {3336--3339},
  number = {18},
  month = oct,
  __markedentry = {[vijay:1]},
  abstract = {The complete phase diagram of a ?colossal? magnetoresistance material
	( La1?xCaxMnO3) was obtained for the first time through magnetization
	and resistivity measurements over a broad range of temperatures and
	concentrations. Near x=0.50, the ground state changes from a ferromagnetic
	(FM) conductor to an antiferromagnetic (AFM) insulator, leading to
	a strongly first order AFM transition with supercooling of ?30\%
	TN at x=0.50. An unexpectedly large magnetoresistance is seen at
	low temperatures in the FM phase, and is largely attributed to unusual
	domain wall scattering.},
  doi = {10.1103/PhysRevLett.75.3336},
  file = {APS Snapshot:files/699/PhysRevLett.75.html:text/html;Full Text PDF:files/698/Schiffer et al. - 1995 - Low Temperature Magnetoresistance and the Magnetic.pdf:application/pdf},
  url = {http://link.aps.org/doi/10.1103/PhysRevLett.75.3336},
  urldate = {2015-06-16}
}

@ARTICLE{lannuzel_magnetoelastic_2004,
  author = {Lannuzel, F.-X. and Janod, E. and Payen, C. and Corraze, B. and Braithwaite,
	D. and Chauvet, O.},
  title = {Magnetoelastic polarons in the hole-doped quasi-one-dimensional model
	system \$\{{\textbackslash}mathrm\{{Y}\}\}\_\{2\$-\$\{\}x\}\{{\textbackslash}mathrm\{{Ca}\}\}\_\{x\}{\textbackslash}mathrm\{{Ba}\}{\textbackslash}mathrm\{{Ni}\}\{{\textbackslash}mathrm\{{O}\}\}\_\{5\}\$},
  journal = {Physical Review B},
  year = {2004},
  volume = {70},
  pages = {155111},
  number = {15},
  month = oct,
  abstract = {Charge transport in the hole-doped quasi-one-dimensional model system
	Y2¿xCaxBaNiO5 (x¿0.15) is investigated in the 50¿300K temperature
	range. The resistivity temperature dependence is characterized by
	a constant activation energy Ea¿kB¿1830K at room temperature
	while Ea decreases upon cooling. We suggest that Ea measures the
	binding energy of the doped holes which form magneto-acoustic polarons
	when polarizing the neighboring Ni spins. A semiclassical model is
	proposed which allows one to relate the electrical measurements and
	the bulk magnetic susceptibility. This model gives a picture of the
	spin-charge-lattice relation in this inhomogeneously doped quasi-1D
	system and explains its unusual one-particle charge excitation spectrum
	close to the Fermi level.},
  doi = {10.1103/PhysRevB.70.155111},
  file = {APS Snapshot:files/409/Lannuzel et al. - 2004 - Magnetoelastic polarons in the hole-doped quasi-on.html:text/html},
  owner = {vijay},
  timestamp = {2015.06.19},
  url = {http://link.aps.org/doi/10.1103/PhysRevB.70.155111},
  urldate = {2015-04-11}
}

@ARTICLE{white_hole_1997,
  author = {White, Steven R. and Scalapino, D. J.},
  title = {Hole and pair structures in the t-{J} model},
  journal = {Physical Review B},
  year = {1997},
  volume = {55},
  pages = {6504--6517},
  number = {10},
  month = mar,
  abstract = {Using numerical results from density matrix renormalization group
	(DMRG) calculations for the t-J model, on systems as large as 10×7,
	we examine the structure of the one and two hole ground states in
	ladder systems and in two dimensional clusters. A simple theoretical
	framework is used to explain why holes bind in pairs in two-dimensional
	antiferromagnets. For the case J/t=0.5, which we have studied, the
	hole pairs reside predominantly on a 2×2 core plaquette with the
	probability that the holes are on diagonal sites greater than nearest-neighbor
	sites. There is a strong singlet bond connecting the spins on the
	two remaining sites of the plaquette. We find that a general characteristic
	of dynamic holes in an antiferromagnet is the presence of frustrating
	antiferromagnetic bonds connecting next-nearest-neighbor sites across
	the holes. Pairs of holes bind in order to share the frustrating
	bonds. At low doping, in addition to hole pairs, there are two additional
	low-energy structures which spontaneously form on certain finite
	systems. The first is an undoped L×2 spin-liquid region, or ladder.
	The second is a hole moving along a one dimensional chain of sites.
	At higher doping we expect that hole pairing is always favored.},
  doi = {10.1103/PhysRevB.55.6504},
  file = {APS Snapshot:files/375/White and Scalapino - 1997 - Hole and pair structures in the t-J model.html:text/html;Full Text PDF:files/626/White and Scalapino - 1997 - Hole and pair structures in the t-J model.pdf:application/pdf},
  owner = {vijay},
  timestamp = {2015.06.20},
  url = {http://link.aps.org/doi/10.1103/PhysRevB.55.6504},
  urldate = {2015-01-30}
}

@ARTICLE{ammon_doped_2001,
  author = {Ammon, Beat and Imada, Masatoshi},
  title = {Doped {Two} {Orbital} {Chains} with {Strong} {Hund}'s {Rule} {Couplings}
	- {Ferromagnetism}, {Spin} {Gap}, {Singlet} and {Triplet} {Pairings}},
  journal = {Journal of the Physical Society of Japan},
  year = {2001},
  volume = {70},
  pages = {547--557},
  number = {2},
  month = feb,
  note = {00008},
  abstract = {Different models for doping of two-orbital chains with mobile S =1/2
	fermions and strong, ferromagnetic (FM) Hund's rule couplings stabilizing
	the S =1 spins are investigated by density matrix renormalization
	group (DMRG) methods. The competition between antiferromagnetic (AF)
	and FM order leads to a rich phase diagram with a narrow FM region
	for weak AF couplings and strongly enhanced triplet pairing correlations.
	Without a level difference between the orbitals, the spin gap persists
	upon doping, whereas gapless spin excitations are generated by interactions
	among itinerant polarons in the presence of a level difference. In
	the charge sector we find dominant singlet pairing correlations without
	a level difference, whereas upon the inclusion of a Coulomb repulsion
	between the orbitals or with a level difference, charge density wave
	(CDW) correlations decay slowest. The string correlation functions
	remain finite upon doping for all models.},
  doi = {10.1143/JPSJ.70.547},
  file = {Full Text PDF:files/581/Ammon and Imada - 2001 - Doped Two Orbital Chains with Strong Hund's Rule C.pdf:application/pdf;Snapshot:files/638/Ammon and Imada - 2001 - Doped Two Orbital Chains with Strong Hund's Rule C.html:text/html},
  issn = {0031-9015},
  owner = {vijay},
  timestamp = {2015.06.20},
  url = {http://journals.jps.jp/doi/abs/10.1143/JPSJ.70.547},
  urldate = {2014-06-06}
}

@ARTICLE{ammon_effect_2000,
  author = {Ammon, Beat and Imada, Masatoshi},
  title = {Effect of the {Orbital} {Level} {Difference} in {Doped} {Spin}-1
	{Chains}},
  journal = {Journal of the Physical Society of Japan},
  year = {2000},
  volume = {69},
  pages = {1946--1949},
  number = {7},
  month = jul,
  note = {00003},
  __markedentry = {[vijay:1]},
  abstract = {The doping of a two-orbital chain with mobile S =1/2 fermions and
	strong Hund's rule couplings stabilizing the S =1 spins strongly
	depends on the presence of a level difference among these orbitals.
	Using density matrix renormalization group (DMRG) methods, we find
	a finite spin gap upon doping and dominant pairing correlations without
	level difference, whereas the presence of a level difference leads
	to dominant charge density wave (CDW) correlations with gapless spin-excitations.
	The string correlation function also shows qualitative differences
	between the two models.},
  doi = {10.1143/JPSJ.69.1946},
  file = {Full Text PDF:files/503/Ammon and Imada - 2000 - Effect of the Orbital Level Difference in Doped Sp.pdf:application/pdf;Snapshot:files/594/Ammon and Imada - 2000 - Effect of the Orbital Level Difference in Doped Sp.html:text/html},
  issn = {0031-9015},
  owner = {vijay},
  timestamp = {2015.06.20},
  url = {http://journals.jps.jp/doi/abs/10.1143/JPSJ.69.1946},
  urldate = {2014-06-20}
}

@ARTICLE{alvarez_conductivity_2002,
  author = {Alvarez, J. V. and Gros, Claudius},
  title = {Conductivity of quantum spin chains: {A} quantum {Monte} {Carlo}
	approach},
  journal = {Physical Review B},
  year = {2002},
  volume = {66},
  pages = {094403},
  number = {9},
  month = sep,
  note = {00029},
  __markedentry = {[vijay:4]},
  abstract = {We discuss zero-frequency transport properties of various spin-1/2
	chains. We show that a careful analysis of quantum Monte Carlo data
	on the imaginary axis allows to distinguish between intrinsic ballistic
	and diffusive transport. We determine the Drude weight, current-relaxation
	lifetime, and the mean free path for integrable and nonintegrable
	quantum spin chains. We discuss, in addition, some phenomenological
	relations between various transport-coefficients and thermal response
	functions.},
  doi = {10.1103/PhysRevB.66.094403},
  file = {APS Snapshot:files/399/Alvarez and Gros - 2002 - Conductivity of quantum spin chains A quantum Mon.html:text/html;Full Text PDF:files/674/Alvarez and Gros - 2002 - Conductivity of quantum spin chains A quantum Mon.pdf:application/pdf},
  owner = {vijay},
  shorttitle = {Conductivity of quantum spin chains},
  timestamp = {2015.06.20},
  url = {http://link.aps.org/doi/10.1103/PhysRevB.66.094403},
  urldate = {2014-05-30}
}

@ARTICLE{jin_effect_2005,
  author = {Jin, Fengping and Xu, Zhaoxin and Ying, Heping and Zheng, Bo},
  title = {On the effect of a regular {S} = 1 dilution of {S} = 1/2 antiferromagnetic
	{Heisenberg} chains obtained from quantum {Monte} {Carlo} simulations},
  journal = {Journal of Physics: Condensed Matter},
  year = {2005},
  volume = {17},
  pages = {5541},
  number = {36},
  month = sep,
  note = {00000},
  __markedentry = {[vijay:4]},
  abstract = {The effect of an S1 = 1 regular dilution in an S2 = 1/2 isotropic
	antiferromagnetic chain is investigated with quantum Monte Carlo
	simulations. Our numerical results show that there exist two kinds
	of ground state phases with different variations of the S1 = 1 concentration.
	When the effective spin in a unit cell is half-integer, the ground
	state is ferromagnetic with a gapless energy spectrum, and the magnetism
	is continuously weakened as the spin S1 concentration ? decreases.
	When the effective spin in a unit cell is integer, however, a non-magnetic
	ground state with a gapped energy spectrum emerges, and the gap decays
	gradually, with .},
  doi = {10.1088/0953-8984/17/36/010},
  file = {Full Text PDF:files/668/Jin et al. - 2005 - On the effect of a regular S = 1 dilution of S = 1.pdf:application/pdf;Snapshot:files/669/010.html:text/html},
  issn = {0953-8984},
  language = {en},
  owner = {vijay},
  timestamp = {2015.06.20},
  url = {http://iopscience.iop.org/0953-8984/17/36/010},
  urldate = {2014-06-09}
}

@ARTICLE{dagotto_static_1992,
  author = {Dagotto, E. and Moreo, A. and Ortolani, F. and Poilblanc, D. and
	Riera, J.},
  title = {Static and dynamical properties of doped {Hubbard} clusters},
  journal = {Physical Review B},
  year = {1992},
  volume = {45},
  pages = {10741--10760},
  number = {18},
  month = may,
  note = {00249},
  abstract = {We study the t-J and the Hubbard models at zero temperature using
	exact-diagonalization techniques on ?10 × ?10 and 4×4 sites clusters.
	Quantum Monte Carlo simulation results on larger lattices are also
	presented. All electronic fillings have been analyzed for the three
	models. We have measured equal-time correlation functions corresponding
	to various types of order (ranging from ??standard?? staggered spin
	order to more ??exotic?? possibilities like chiral order), as well
	as various dynamical properties of these models. Upper bounds for
	the critical hole doping (xc), where long-range antiferromagnetic
	order disappears, are presented. It was found that xc is very small
	in agreement with experiments for the high-Tc superconductors. For
	example, in the t-J model, xc{\textless}0.08 at J/t=0.4. However,
	short-distance spin correlations are important up to much higher
	dopings producing a sharp well-defined spin-wave-like peak in S(q=(?,?),?).
	Regarding the possibility of phase separation in the Hubbard model,
	we have studied the behavior of the density of particles, ?n?, as
	a function of the chemical potential, using the Lanczos method on
	a 4×4 Hubbard cluster, finding no indications of phase separation
	for any value of U/t. Then, we conclude that the t-J model at small
	J/t should not phase separate.},
  doi = {10.1103/PhysRevB.45.10741},
  file = {APS Snapshot:files/484/Dagotto et al. - 1992 - Static and dynamical properties of doped Hubbard c.html:text/html},
  owner = {vijay},
  timestamp = {2015.06.20},
  url = {http://link.aps.org/doi/10.1103/PhysRevB.45.10741},
  urldate = {2014-05-30}
}

@ARTICLE{das_comparison_2004,
  author = {Das, J. and Mahajan, A. V. and Bobroff, J. and Alloul, H. and Alet,
	F. and Sørensen, E. S.},
  title = {Comparison of \${S}=0\$ and \${S}={\textbackslash}frac\{1\}\{2\}\$
	impurities in the {Haldane} chain compound \$\{{\textbackslash}mathrm\{{Y}\}\}\_\{2\}\{{\textbackslash}mathrm\{{BaNiO}\}\}\_\{5\}\$},
  journal = {Physical Review B},
  year = {2004},
  volume = {69},
  pages = {144404},
  number = {14},
  month = apr,
  abstract = {We present the effect of Zn (S=0) and Cu (S=1/2) substitution at the
	Ni site of S=1 Haldane chain compound Y2BaNiO5. 89Y nuclear-magnetic
	resonance (NMR) allows us to measure the local magnetic susceptibility
	at different distances from the defects. The 89Y NMR spectrum consists
	of one central peak and several less intense satellite peaks. The
	central peak represents the chain sites far from the defect. Its
	shift measures the uniform susceptibility, which displays a Haldane
	gap ??100K and it corresponds to an antiferromagnetic (AF) coupling
	J?260K between the nearest neighbor Ni spins. Zn or Cu substitution
	does not affect the Haldane gap. The satellites, which are evenly
	distributed on the two sides of the central peak, probe the antiferromagnetic
	staggered magnetization near the substituted site. The spatial variation
	of the induced magnetization is found to decay exponentially from
	the impurity for both Zn and Cu for T{\textgreater}50K. Its extension
	is found identical for both impurities and corresponds accurately
	to the correlation length ?(T) determined by Monte Carlo simulations
	for the pure compound. In the case of nonmagnetic Zn, the temperature
	dependence of the induced magnetization is consistent with a Curie
	law with an ?effective? spin S=0.4 on each side of Zn. This staggered
	effect is quantitatively well accounted for in all the explored range
	by quantum Monte Carlo (QMC) computations of the spinless-defect-induced
	magnetism. In the case of magnetic Cu, the similarity of the induced
	magnetism to the Zn case implies a weak coupling of the Cu spin to
	the nearest-neighbor Ni spins. The slight reduction of about 20?30\%
	of the induced polarization with respect to Zn is reproduced by QMC
	computations by considering an antiferromagnetic coupling of strength
	J?=0.1J?0.2J between the S=1/2 Cu spin and nearest-neighbor Ni spin.
	Macroscopic susceptibility measurements confirm these results as
	they display a clear Curie contribution due to the impurities nearly
	proportional to their concentration. This contribution is indeed
	close to that of two spin half for Zn substitution. The Curie contribution
	is smaller in the Cu case, which confirms that the coupling between
	Cu and near-neighbor Ni is antiferromagnetic.},
  doi = {10.1103/PhysRevB.69.144404},
  file = {APS Snapshot:files/564/Das et al. - 2004 - Comparison of \$S=0\$ and \$S=frac 1 2 \$ impurities.html:text/html},
  owner = {vijay},
  timestamp = {2015.06.20},
  url = {http://link.aps.org/doi/10.1103/PhysRevB.69.144404},
  urldate = {2015-05-16}
}

@ARTICLE{matsumoto_effects_2005,
  author = {Matsumoto, Munehisa and Takayama, Hajime},
  title = {Effects of {Impurities} in {Quasi}-{One}-{Dimensional} {S} = 1 {Antiferromagnets}},
  journal = {Progress of Theoretical Physics Supplement},
  year = {2005},
  volume = {159},
  pages = {412--416},
  month = may,
  note = {00000},
  abstract = {For the weakly coupled S = 1 antiferromagnetic Heisenberg chains on
	a simple cubic lattice, the effects of magnetic impurities are investigated
	by the quantum Monte Carlo method with the continuous-time loop algorithm.
	The transition temperatures of the impurity-induced phase transitions
	for magnetic impurities with S = 1/2, 3/2 and 2 are determined and
	compared with the transition temperature induced by the non-magnetic
	impurities. Implications on the experimental results are discussed.},
  doi = {10.1143/PTPS.159.412},
  file = {Full Text PDF:files/387/Matsumoto and Takayama - 2005 - Effects of Impurities in Quasi-One-Dimensional S =.pdf:application/pdf;Snapshot:files/569/412.html:text/html},
  issn = {0375-9687,},
  language = {en},
  owner = {vijay},
  timestamp = {2015.06.20},
  url = {http://ptps.oxfordjournals.org/content/159/412},
  urldate = {2014-06-09}
}

@ARTICLE{mezzacapo_variational_2011,
  author = {Mezzacapo, Fabio},
  title = {Variational study of a mobile hole in a two-dimensional quantum antiferromagnet
	using entangled-plaquette states},
  journal = {Physical Review B},
  year = {2011},
  volume = {83},
  pages = {115111},
  number = {11},
  month = mar,
  abstract = {We study the properties of a mobile hole in the t?J model on the square
	lattice by means of variational Monte Carlo simulations based on
	the entangled-plaquette ansatz. Our energy estimates for small lattices
	reproduce available exact results. We obtain values for the hole
	energy dispersion curve on large lattices in quantitative agreement
	with earlier findings based on the most reliable numerical techniques.
	Accurate estimates of the hole spectral weight are provided.},
  doi = {10.1103/PhysRevB.83.115111},
  file = {APS Snapshot:files/305/Mezzacapo - 2011 - Variational study of a mobile hole in a two-dimens.html:text/html;Full Text PDF:files/479/Mezzacapo - 2011 - Variational study of a mobile hole in a two-dimens.pdf:application/pdf},
  owner = {vijay},
  timestamp = {2015.06.20},
  url = {http://link.aps.org/doi/10.1103/PhysRevB.83.115111},
  urldate = {2014-11-18}
}

@ARTICLE{sorella_superconductivity_2002,
  author = {Sorella, S. and Martins, G. B. and Becca, F. and Gazza, C. and Capriotti,
	L. and Parola, A. and Dagotto, E.},
  title = {Superconductivity in the {Two}-{Dimensional} \${\textbackslash}mathit\{t\}-{\textbackslash}mathit\{{J}\}\$
	{Model}},
  journal = {Physical Review Letters},
  year = {2002},
  volume = {88},
  pages = {117002},
  number = {11},
  month = feb,
  abstract = {Using computational techniques, it is shown that pairing is a robust
	property of hole-doped antiferromagnetic insulators. In one dimension
	and for two-leg ladder systems, a BCS-like variational wave function
	with long-bond spin singlets and a Jastrow factor provides an accurate
	representation of the ground state of the t?J model, even though
	strong quantum fluctuations destroy the off-diagonal superconducting
	long-range order in this case. However, in two dimensions it is argued?and
	numerically confirmed using several techniques, especially quantum
	Monte Carlo?that quantum fluctuations are not strong enough to suppress
	superconductivity.},
  doi = {10.1103/PhysRevLett.88.117002},
  file = {APS Snapshot:files/304/Sorella et al. - 2002 - Superconductivity in the Two-Dimensional \$mathit .html:text/html;Full Text PDF:files/443/Sorella et al. - 2002 - Superconductivity in the Two-Dimensional \$mathit .pdf:application/pdf},
  owner = {vijay},
  timestamp = {2015.06.20},
  url = {http://link.aps.org/doi/10.1103/PhysRevLett.88.117002},
  urldate = {2015-04-21}
}

@Misc{petsc-web-page,
            author = {Satish Balay and Shrirang Abhyankar and Mark~F. Adams and Jed Brown and Peter Brune
                      and Kris Buschelman and Lisandro Dalcin and Victor Eijkhout and William~D. Gropp
                      and Dinesh Kaushik and Matthew~G. Knepley
                      and Lois Curfman McInnes and Karl Rupp and Barry~F. Smith
                      and Stefano Zampini and Hong Zhang},
            title =  {{PETS}c {W}eb page},
            url =    {http://www.mcs.anl.gov/petsc},
            howpublished = {\url{http://www.mcs.anl.gov/petsc}},
            year = {2015}
}

@TechReport{petsc-user-ref,
  author = {Satish Balay and Shrirang Abhyankar and Mark~F. Adams and Jed Brown and Peter Brune
            and Kris Buschelman and Lisandro Dalcin and Victor Eijkhout and William~D. Gropp
            and Dinesh Kaushik and Matthew~G. Knepley
            and Lois Curfman McInnes and Karl Rupp and Barry~F. Smith
            and Stefano Zampini and Hong Zhang},
  title  = {{PETS}c Users Manual},
  institution = {Argonne National Laboratory},
  year   = 2015,
  number = {ANL-95/11 - Revision 3.6},
  url    = {http://www.mcs.anl.gov/petsc}
}

@InProceedings{petsc-efficient,
  Author = "Satish Balay and William D. Gropp and Lois Curfman McInnes and Barry F.  Smith",
  Title = "Efficient Management of Parallelism in Object Oriented Numerical Software Libraries",
  Booktitle = "Modern Software Tools in Scientific Computing",
  Editor = "E. Arge and A. M. Bruaset and H. P. Langtangen",
  Pages = "163--202",
  Publisher = "Birkh{\"{a}}user Press",
  Year = "1997"
}

@Article{Hernandez:2005:SSF,
   author  = "Vicente Hernandez and Jose E. Roman and Vicente Vidal",
   title   = "{SLEPc}: A Scalable and Flexible Toolkit for the Solution of Eigenvalue
              Problems",
   journal = "{ACM} Trans. Math. Software",
   volume  = "31",
   number  = "3",
   pages   = "351--362",
   year    = "2005"
}

@Article{Hernandez:2003:SSL,
   author  = "V. Hernandez and J. E. Roman and V. Vidal",
   title   = "{SLEPc}: {S}calable {L}ibrary for {E}igenvalue {P}roblem {C}omputations",
   journal = "Lecture Notes in Computer Science",
   volume  = "2565",
   pages   = "377--391",
   year    = "2003"
}

@TechReport{slepc-users-manual,
   author  = "J. E. Roman and C. Campos and E. Romero and A. Tomas",
   title   = "{SLEPc} Users Manual",
   number  = "DSIC-II/24/02 - Revision 3.6",
   institution = "D. Sistemes Inform\`atics i Computaci\'o,
                  Universitat Polit\`ecnica de Val\`encia",
   year    = "2015"
}

@Article{Campos:2012:SSS,
  author   = "C. Campos and J. E. Roman",
  title    = "Strategies for Spectrum Slicing based on Restarted {Lanczos} Methods",
  journal  = "Numer. Algorithms",
  volume   = "60",
  number   = "2",
  pages    = "279--295",
  year     = "2012"
}

@article{stewart_addendum_2002,
    title = {Addendum to "{A} {Krylov}--{Schur} {Algorithm} for {Large} {Eigenproblems}"},
    volume = {24},
    issn = {0895-4798},
    url = {http://epubs.siam.org/doi/abs/10.1137/S0895479802403150},
    doi = {10.1137/S0895479802403150},
    abstract = {In this addendum to an earlier paper by the author, it is shown how to compute a Krylov   decomposition corresponding to an arbitrary Rayleigh quotient. This decomposition can be used to restart  an Arnoldi process, with a selection of the Ritz vectors corresponding to that Rayleigh quotient.,  In    this addendum to an earlier paper by the author, it is shown how to compute a Krylov decomposition        corresponding to an arbitrary Rayleigh quotient. This decomposition can be used to restart an Arnoldi     process, with a selection of the Ritz vectors corresponding to that Rayleigh quotient.},
    number = {2},
    urldate = {2015-06-25},
    journal = {SIAM Journal on Matrix Analysis and Applications},
    author = {Stewart, G.},
    month = jan,
    year = {2002},
    pages = {599--601},
    file = {Full Text PDF:files/738/Stewart - 2002 - Addendum to A Krylov--Schur Algorithm for Large E.   pdf:application/pdf;Snapshot:files/740/S0895479802403150.html:text/html}
}

@article{stewart_krylov--schur_2002,
    title = {A {Krylov}--{Schur} {Algorithm} for {Large} {Eigenproblems}},
    volume = {23},
    issn = {0895-4798},
    url = {http://epubs.siam.org/doi/abs/10.1137/S0895479800371529},
    doi = {10.1137/S0895479800371529},
    abstract = {Sorensen's implicitly restarted Arnoldi algorithm is one of the most successful and       flexible methods for finding a few eigenpairs of a large matrix. However, the need to preserve the        structure of the Arnoldi decomposition on which the algorithm is based restricts the range of             transformations that can be performed on the decomposition. In consequence, it is difficult to deflate    converged Ritz vectors from the decomposition. Moreover, the potential forward instability of the         implicit QR algorithm can cause unwanted Ritz vectors to persist in the computation. In this paper we     introduce a general Krylov decomposition that solves both problems in a natural and efficient manner.,    Sorensen's implicitly restarted Arnoldi algorithm is one of the most successful and flexible methods for  finding a few eigenpairs of a large matrix. However, the need to preserve the structure of the Arnoldi    decomposition on which the algorithm is based restricts the range of transformations that can be          performed on the decomposition. In consequence, it is difficult to deflate converged Ritz vectors from    the decomposition. Moreover, the potential forward instability of the implicit QR algorithm can cause     unwanted Ritz vectors to persist in the computation. In this paper we introduce a general Krylov          decomposition that solves both problems in a natural and efficient manner.},
    number = {3},
    urldate = {2015-06-25},
    journal = {SIAM Journal on Matrix Analysis and Applications},
    author = {Stewart, G.},
    month = jan,
    year = {2002},
    pages = {601--614},
    file = {Full Text PDF:files/736/Stewart - 2002 - A Krylov--Schur Algorithm for Large Eigenproblems.   pdf:application/pdf;Snapshot:files/739/S0895479800371529.html:text/html}
}

@article{lanczos1950iteration,
  title={An Iteration Method for the Solution of the Eigenvalue Problem of Linear Differential and Integral Operators1},
  author={Lanczos, Cornelius},
  journal={Journal of Research of the National Bureau of Standards},
  volume={45},
  number={4},
  year={1950}
}

@ARTICLE{verstraete_matrix_2008,
  author = {Verstraete, F. and Murg, V. and Cirac, J. I.},
  title = {Matrix product states, projected entangled pair states, and variational
    renormalization group methods for quantum spin systems},
  journal = {Advances in Physics},
  year = {2008},
  volume = {57},
  pages = {143--224},
  number = {2},
  month = mar,
  abstract = {This article reviews recent developments in the theoretical understanding
    and the numerical implementation of variational renormalization group
    methods using matrix product states and projected entangled pair
    states.},
  doi = {10.1080/14789940801912366},
  file = {Snapshot:files/727/14789940801912366.html:text/html},
  issn = {0001-8732},
  owner = {vijay},
  timestamp = {2015.06.25},
  url = {http://dx.doi.org/10.1080/14789940801912366},
  urldate = {2015-06-25}
}

@article{bauer_alps_2011,
    title = {The {ALPS} project release 2.0: open source software for strongly correlated systems},
    volume = {2011},
    issn = {1742-5468},
    shorttitle = {The {ALPS} project release 2.0},
    url = {http://iopscience.iop.org/1742-5468/2011/05/P05001},
    doi = {10.1088/1742-5468/2011/05/P05001},
    abstract = {We present release 2.0 of the ALPS (Algorithms and Libraries for Physics Simulations)     project, an open source software project to develop libraries and application programs for the simulation of strongly correlated quantum lattice models such as quantum magnets, lattice bosons, and strongly       correlated fermion systems. The code development is centered on common XML and HDF5 data formats,         libraries to simplify and speed up code development, common evaluation and plotting tools, and simulation programs. The programs enable non-experts to start carrying out serial or parallel numerical simulations  by providing basic implementations of the important algorithms for quantum lattice models: classical and  quantum Monte Carlo (QMC) using non-local updates, extended ensemble simulations, exact and full          diagonalization (ED), the density matrix renormalization group (DMRG) both in a static version and a      dynamic time-evolving block decimation (TEBD) code, and quantum Monte Carlo solvers for dynamical mean    field theory (DMFT). The ALPS libraries provide a powerful framework for programmers to develop their own applications, which, for instance, greatly simplify the steps of porting a serial code onto a parallel,   distributed memory machine. Major changes in release 2.0 include the use of HDF5 for binary data,         evaluation tools in Python, support for the Windows operating system, the use of CMake as build system    and binary installation packages for Mac OS X and Windows, and integration with the VisTrails workflow    provenance tool. The software is available from our web server at http://alps.comp-phys.org/.},
    language = {en},
    number = {05}, 
    urldate = {2015-06-25},
    journal = {Journal of Statistical Mechanics: Theory and Experiment},
    author = {Bauer, B. and Carr, L. D. and Evertz, H. G. and Feiguin, A. and Freire, J. and Fuchs, S.    and Gamper, L. and Gukelberger, J. and Gull, E. and Guertler, S. and Hehn, A. and Igarashi, R. and        Isakov, S. V. and Koop, D. and Ma, P. N. and Mates, P. and Matsuo, H. and Parcollet, O. and Pawłowski, G. and Picon, J. D. and Pollet, L. and Santos, E. and Scarola, V. W. and Schollwöck, U. and Silva, C. and    Surer, B. and Todo, S. and Trebst, S. and Troyer, M. and Wall, M. L. and Werner, P. and Wessel, S.},
    month = may,
    year = {2011},
    keywords = {classical monte carlo simulations, density matrix renormalization group calculations,     quantum Monte Carlo simulations, quantum phase transitions (theory)},
    pages = {P05001},
    file = {Full Text PDF:files/733/Bauer et al. - 2011 - The ALPS project release 2.0 open source        software.pdf:application/pdf;Snapshot:files/734/P05001.html:text/html}
}

@ARTICLE{tsutsui_differences_1999,
  author = {Tsutsui, K. and Koshibae, W. and Maekawa, S.},
  title = {Differences in optical conductivity between one- and two-dimensional
	doped nickelates},
  journal = {Physical Review B},
  year = {1999},
  volume = {59},
  pages = {9729--9732},
  number = {15},
  month = apr,
  __markedentry = {[vijay:5]},
  abstract = {We study the optical conductivity in doped nickelates, and find the
	dramatic difference of the spectrum in the gap (??4 eV) between one-
	(1D) and two-dimensional (2D) nickelates. The difference is shown
	to be caused by the dependence of hopping integral on dimensionality.
	The theoretical results explain consistently the experimental data
	in 1D and 2D nickelates, Y2?xCaxBaNiO5 and La2?xSrxNiO4, respectively.
	The relation between the spectrum in the x-ray aborption experiments
	and the optical conductivity in La2?xSrxNiO4 is discussed.},
  doi = {10.1103/PhysRevB.59.9729},
  file = {APS Snapshot:files/624/Tsutsui et al. - 1999 - Differences in optical conductivity between one- a.html:text/html;Full Text PDF:files/271/Tsutsui et al. - 1999 - Differences in optical conductivity between one- a.pdf:application/pdf},
  owner = {vijay},
  timestamp = {2015.06.20},
  url = {http://link.aps.org/doi/10.1103/PhysRevB.59.9729},
  urldate = {2015-03-09}
}

@ARTICLE{fagot-revurat_photoemission_2003,
  author = {Fagot-Revurat, Y. and Malterre, D. and Lannuzel, F. -X. and Janod,
	E. and Payen, C. and Gavioli, L. and Bertran, F.},
  title = {Photoemission spectroscopy study of the hole-doped {Haldane} chain
	{Y}2?{xSrxBaNiO}5},
  journal = {Nuclear Instruments and Methods in Physics Research Section B: Beam
	Interactions with Materials and Atoms},
  year = {2003},
  volume = {200},
  pages = {242--247},
  month = jan,
  abstract = {In this paper, we present photoemission experiments on the hole-doped
	Haldane chain compound Y2?xSrxBaNiO5. By using the photon energy
	dependence of the photoemission cross section, we identified the
	symmetry of the first ionisation states (d type). Hole doping in
	this system leads to a significant increase in the spectral weight
	at the top of the valence band without any change in the vicinity
	of the Fermi energy. This behavior, never observed in other charge
	transfer oxides, could result from the Ni3d?O2p hybridization enhancement
	due to the shortening of the relevant Ni?O distance with doping.},
  doi = {10.1016/S0168-583X(02)01684-1},
  file = {ScienceDirect Full Text PDF:files/841/Fagot-Revurat et al. - 2003 - Photoemission             spectroscopy study of the hole-doped.pdf:application/pdf;ScienceDirect Snapshot:files/842/                S0168583X02016841.html:text/html},
  issn = {0168-583X},
  keywords = {Charge transfer insulator, Hole doping, Photoemission spectroscopy,
	Strongly correlated systems},
  owner = {vijay},
  series = {Proceedings of the {E}-{MRS} 2002 {Symposium} {I} on {Synchrotron}
	{Radiation} and {Materials} {Science}},
  timestamp = {2015.07.08},
  url = {http://www.sciencedirect.com/science/article/pii/S0168583X02016841},
  urldate = {2015-07-08}
}

@ARTICLE{hu_electronic_2002,
  author = {Hu, Z. and Knupfer, M. and Kielwein, M. and Rößler, U. K. and Golden,
	M. S. and Fink, J. and Groot, F. M. F. de and Ito, T. and Oka, K.
	and Kaindl, G.},
  title = {The electronic structure of the doped one-dimensional transition
	metal oxide {Y} 2 - x {Ca} x {BaNiO} 5 studied using {X}-ray absorption},
  journal = {The European Physical Journal B - Condensed Matter and Complex Systems},
  year = {2002},
  volume = {26},
  pages = {449--453},
  number = {4},
  month = apr,
  abstract = {A strong anisotropic distribution of the holes in Ni 3 d and O 2 p
	orbitals is observed in the polarization dependent O 1 s and Ni 2
	p 3/2 X-ray absorption spectroscopy of the linear-chain nickelate
	Y2-xCaxBaNiO5 (x = 0, 0.05, 0.1, 0.2), which demonstrates the one-dimensional
	nature of the electronic state in these compounds. The holes introduced
	by Ca-doping occupy both O 2 p and Ni 3 d orbitals along the NiO5
	chains. By comparing the experimental Ni 2 p 3/2 absorption spectra
	of Y2- xCaxBaNiO5 to those from charge transfer multiplet calculations
	we can derive the orbital character of the additional holes to be
	of ?60\% O2 p and ?40\% Ni 3 d.},
  doi = {10.1140/epjb/e20020113},
  file = {Full Text PDF:files/844/Hu et al. - 2002 - The electronic structure of the doped one-        dimensio.pdf:application/pdf;Snapshot:files/845/e20020113.html:text/html},
  issn = {1434-6028, 1434-6036},
  keywords = {PACS. 78.70.Dm X-ray absorption spectra ? 71.28.+d Narrow-band systems;
	intermediate- valence solids ? 79.60.-i Photoemission and photoelectron
	spectra},
  language = {en},
  owner = {vijay},
  timestamp = {2015.07.08},
  url = {http://link.springer.com/article/10.1140/epjb/e20020113},
  urldate = {2015-07-08}
}

@ARTICLE{tedoldi_magnetic_1998,
  author = {Tedoldi, F. and Rigamonti, A. and Brugna, C. and Corti, M. and Lascialfari,
	A. and Capsoni, D. and Massarotti, V.},
  title = {Magnetic properties and spin dynamics in hole-doped {S}=1 {AF} chain:
	89Y {NMR} and susceptibility in {Y}2?{xCaxBaNiO}5},
  journal = {Journal of Applied Physics},
  year = {1998},
  volume = {83},
  pages = {6605--6607},
  number = {11},
  month = jun,
  note = {00000},
  __markedentry = {[vijay:5]},
  abstract = {Hole-doping effects in Haldane chain have been studied by means of
	magnetic susceptibility and 89 Y nuclear magnetic resonance in Y
	2?x Ca x BaNiO 5 for x=0, x=0.06 and x=0.18. In nominally pure YBNO
	detailed information on this prototype of Haldane systems is extracted.
	In the charge-doped compounds the 89 Y relaxation rates indicate
	that holes induce low-energy excitations with an effective spectral
	density having a structure characterized by a narrow central peak.},
  doi = {10.1063/1.367608},
  file = {Full Text PDF:files/448/Tedoldi et al. - 1998 - Magnetic properties and spin dynamics in hole-dope.pdf:application/pdf;Snapshot:files/401/Tedoldi et al. - 1998 - Magnetic properties and spin dynamics in hole-dope.html:text/html},
  issn = {0021-8979, 1089-7550},
  keywords = {Dynamic magnetic susceptibility, Magnetic susceptibilities, Nuclear
	magnetic resonance, Properties of nuclei},
  owner = {vijay},
  shorttitle = {Magnetic properties and spin dynamics in hole-doped {S}=1 {AF} chain},
  timestamp = {2015.06.20},
  url = {http://scitation.aip.org/content/aip/journal/jap/83/11/10.1063/1.367608},
  urldate = {2014-06-06}
}

@ARTICLE{nasani_structural_2015,
  author = {Nasani, Narendar and Ramasamy, Devaraj and Antunes, Isabel and Singh,
	Budhendra and Fagg, Duncan P.},
  title = {Structural and electrical properties of strontium substituted {Y}2BaNiO5},
  journal = {Journal of Alloys and Compounds},
  year = {2015},
  volume = {620},
  pages = {91--96},
  month = jan,
  abstract = {The Y2−xSrxBaNiO5 (x = 0, 0.1, 0.2 and 0.3) acceptor substituted
	system has been synthesized by solid state reaction. Structural and
	microstructural properties have been characterized by X-ray diffraction
	(XRD) and scanning electron microscopy (SEM), respectively. Lattice
	volume is shown to decrease linearly with increasing Sr content until
	composition x = 0.2, highlighting the limit of the solid solution.
	The electrical response in the temperature range (700–100 °C)
	was assessed by A.C. impedance spectroscopy in wet and dry O2 and
	N2 atmospheres. Conductivity measurements as a function of oxygen
	partial pressure (pO2) were also performed. The data reveal that
	the conductivity Y2BaNiO5 can be increased by one and half orders
	magnitude by Sr-doping and is independent of both water vapour and
	oxygen partial pressures (pH2O and pO2). The low activation energy
	for electrical conduction (0.216–0.240 eV) suggests a thermally
	activated electron hopping mechanism, while the observed pO2 and
	pH2O independence of conductivity suggests that charge compensation
	for Sr is predominantly by formation of Ni3+ rather than formation
	of oxygen vacancies.},
  doi = {10.1016/j.jallcom.2014.09.127},
  issn = {0925-8388},
  keywords = {Barium yttrium nickelate, Electrical conductivity, Haldane energy
	gap, Protonic ceramics, Solid solution},
  owner = {vijay},
  timestamp = {2015.05.26},
  url = {http://www.sciencedirect.com/science/article/pii/S0925838814023007},
  urldate = {2015-05-06TZ}
}

@ARTICLE{imanaka_esr_1998,
  author = {Imanaka, Y. and Miura, N. and Nojiri, H. and Luther, S. and Ortenberg,
	M. V. and Yokoo, T. and Akimitsu, J.},
  title = {{ESR} study of the {Haldane} gap system {Y}2BaNiO5 in high magnetic
	fields},
  journal = {Physica B: Condensed Matter},
  year = {1998},
  volume = {246?247},
  pages = {561--564},
  month = may,
  note = {00000},
  abstract = {We performed the ESR experiments in Y2BaNiO5, which has been proposed
	as a good candidate for the Haldane gap system, over a wide magnetic
	field range up to 100 T. We observed a relatively broad EPR signal
	at temperatures above the gap energy and another resonance peak which
	predominates the spectra below 100 K. We also found some more resonant
	peaks, which exhibit an infinite energy when extrapolated to zero
	magnetic field. The gap energy is estimated to be 5.22 meV. This
	resonance seems to correspond to the transition between the singlet
	state and the triplet state, because the intensity of this peak becomes
	stronger as the temperature is lowered. However, the gap energy obtained
	in the present experiment is smaller than the Haldane gap obtained
	from the neutron inelastic scattering.},
  doi = {10.1016/S0921-4526(97)00987-3},
  file = {ScienceDirect Full Text PDF:files/303/Imanaka et al. - 1998 - ESR study of the Haldane gap system Y2BaNiO5 in hi.pdf:application/pdf;ScienceDirect Snapshot:files/378/S0921452697009873.html:text/html},
  issn = {0921-4526},
  keywords = {ESR, Haldane gap, High magnetic field, Y2BaNiO5},
  owner = {vijay},
  timestamp = {2015.06.20},
  url = {http://www.sciencedirect.com/science/article/pii/S0921452697009873},
  urldate = {2014-06-09}
}

@ARTICLE{ni_interesting_2004,
  author = {Ni, Chunlin and Dang, Dongbin and Song, You and Gao, Song and Li,
	Yizhi and Ni, Zhaoping and Tian, Zhengfang and Wen, Lili and Meng,
	Qingjin},
  title = {An interesting magnetic behavior in molecular solid containing one-dimensional
	{Ni}({III}) chain},
  journal = {Chemical Physics Letters},
  year = {2004},
  volume = {396},
  pages = {353--358},
  number = {4?6},
  month = oct,
  note = {00093},
  abstract = {The preparation, crystal structure and magnetic properties of a new
	ion-pair complex, [BrFBzNH2Py][Ni(mnt)2] (1) [BrFBzNH2Py+ = 1-(4?-bromo-2?-flurobenzyl)-4-aminopyridinium,
	mnt2? = maleonitriledithiolate] are reported. The {\textless}img
	height="17" border="0" style="vertical-align:bottom" width="67" alt="View
	the MathML source" title="View the MathML source" src="http://origin-ars.els-cdn.com/content/image/1-s2.0-S0009261404012655-si1.gif"{\textgreater}Ni(mnt)2-
	anions and [BrFBzNH2Py]+ cations of 1 form completely segregated
	uniform stacking columns. The intrachain Ni?Ni separation is 4.045
	Å in the {\textless}img height="17" border="0" style="vertical-align:bottom"
	width="67" alt="View the MathML source" title="View the MathML source"
	src="http://origin-ars.els-cdn.com/content/image/1-s2.0-S0009261404012655-si2.gif"{\textgreater}Ni(mnt)2-
	stacking column. Magnetic susceptibility measurements for 1 in the
	temperature range 2.0?300 K show the occurrence of significant ferromagnetic
	interaction in the high-temperature phase (HT), spin gap in the low-temperature
	phase (LT) and weak ferromagnetism due to spin canting below 5 K.},
  doi = {10.1016/j.cplett.2004.08.060},
  file = {ScienceDirect Full Text PDF:files/498/Ni et al. - 2004 - An interesting magnetic behavior in molecular soli.pdf:application/pdf;ScienceDirect Snapshot:files/444/Ni et al. - 2004 - An interesting magnetic behavior in molecular soli.html:text/html},
  issn = {0009-2614},
  owner = {vijay},
  timestamp = {2015.06.20},
  url = {http://www.sciencedirect.com/science/article/pii/S0009261404012655},
  urldate = {2014-05-31}
}

@ARTICLE{woodward_2001,
  author = {Woodward, Jonathan D. and Backov, Rénal and Abboud, Khalil A. and
	Talham, Daniel R.},
  title = {[{Ni}(terpy)({H} $_{\textrm{2}}$ {O})]- \textit{trans} -[{Ni}-(?-{CN})
	$_{\textrm{2}}$ -({CN}) $_{\textrm{2}}$ ] $_{\textrm{ \textit{n}
	}}$ , a one-dimensional linear tetracyanonickelate chain},
  journal = {Acta Crystallographica Section C Crystal Structure Communications},
  year = {2001},
  volume = {57},
  pages = {1027--1029},
  number = {9},
  month = sep,
  doi = {10.1107/S0108270101009234},
  issn = {0108-2701},
  owner = {vijay},
  timestamp = {2015.06.20},
  url = {http://scripts.iucr.org/cgi-bin/paper?S0108270101009234},
  urldate = {2015-04-11}
}

@ARTICLE{tranquada_simultaneous_1994,
  author = {Tranquada, J. M. and Buttrey, D. J. and Sachan, V. and Lorenzo, J.
	E.},
  title = {Simultaneous {Ordering} of {Holes} and {Spins} in \$\{{\textbackslash}mathrm\{{La}\}\}\_\{2\}\${Ni}\$\{{\textbackslash}mathrm\{{O}\}\}\_\{4.125\}\$},
  journal = {Physical Review Letters},
  year = {1994},
  volume = {73},
  pages = {1003--1006},
  number = {7},
  month = aug,
  abstract = {We report a single-crystal neutron diffraction study of the incommensurate
	magnetic ordering that occurs in La2NiO4.125 below 110 K. Besides
	the magnetic first and third harmonic Bragg peaks, we have also observed
	second harmonic peaks associated with charge ordering. The magnitude
	of the incommensurate splitting, ?, is strongly temperaure dependent.
	Lock-in behavior indicates that ? tends to rational fractions, while
	regions of continuous variation suggest a devil's staircase. Analysis
	of these features indicates that the holes, induced by the excess
	oxygen, order in domain walls that form antiphase boundaries between
	antiferromagnetic domains.},
  doi = {10.1103/PhysRevLett.73.1003},
  file = {APS Snapshot:files/856/PhysRevLett.73.html:text/html},
  owner = {vijay},
  timestamp = {2015.07.08},
  url = {http://link.aps.org/doi/10.1103/PhysRevLett.73.1003},
  urldate = {2015-07-08}
}

@ARTICLE{yoshizawa_stripe_2000,
  author = {Yoshizawa, H. and Kakeshita, T. and Kajimoto, R. and Tanabe, T. and
	Katsufuji, T. and Tokura, Y.},
  title = {Stripe order at low temperatures in \$\{{\textbackslash}mathrm\{{La}\}\}\_\{2-x\}\{{\textbackslash}mathrm\{{Sr}\}\}\_\{x\}\{{\textbackslash}mathrm\{{NiO}\}\}\_\{4\}\$
	with \$0.289{\textbackslash}ensuremath\{{\textbackslash}lesssim\}x{\textbackslash}ensuremath\{{\textbackslash}lesssim\}0.5\$},
  journal = {Physical Review B},
  year = {2000},
  volume = {61},
  pages = {R854--R857},
  number = {2},
  month = jan,
  abstract = {The stripe order in La2?xSrxNiO4+? with 0.289?x?0.5 was studied with
	neutron-scattering technique. At low temperatures, all samples exhibit
	hole stripe order. Incommensurability ? of the stripe order is approximately
	linear in the hole concentration nh=x+2? up to x=1/2, where ? denotes
	the off stoichiometry of oxygen atoms. The charge and spin ordering
	temperatures exhibit maxima at nh=13, and both decrease beyond nh{\textgreater}13.
	For 13{\textless}{\textasciitilde}nh?12, the stripe ordering consists
	of the mixture of the ?=13 stripe order and the nh=12 charge/spin
	order.},
  doi = {10.1103/PhysRevB.61.R854},
  file = {APS Snapshot:files/853/PhysRevB.61.html:text/html;Full Text PDF:files/852/Yoshizawa et al. - 2000 - Stripe order at low temperatures in \$ mathrm La .pdf:application/pdf},
  owner = {vijay},
  timestamp = {2015.07.08},
  url = {http://link.aps.org/doi/10.1103/PhysRevB.61.R854},
  urldate = {2015-07-08}
}

@ARTICLE{mizokawa_description_1998,
  author = {Mizokawa, T. and Fujimori, A.},
  title = {Description of {Spin} and {Charge} {Domain} {Walls} in {Doped} {Perovskite}-{Type}
	\$3{\textbackslash}mathit\{d\}\$ {Transition}-{Metal} {Oxides} {Based}
	on {Superexchange} {Interaction}},
  journal = {Physical Review Letters},
  year = {1998},
  volume = {80},
  pages = {1320--1323},
  number = {6},
  month = feb,
  abstract = {Spin and charge ordered states with domain walls (DW) in two- and
	three-dimensional perovskite-type 3d transition-metal oxides have
	been investigated. We show that the relative stability of the DW
	perpendicular to the (1,0,0), (1,1,0), and (1,1,1) directions in
	three dimensions and those along the (1,0) and (1,1) directions in
	two dimensions is systematically and qualitatively understood in
	terms of superexchange interaction.},
  doi = {10.1103/PhysRevLett.80.1320},
  file = {APS Snapshot:files/859/PhysRevLett.80.html:text/html;Full Text PDF:files/858/Mizokawa and      Fujimori - 1998 - Description of Spin and Charge Domain Walls in Dop.pdf:application/pdf},
  owner = {vijay},
  timestamp = {2015.07.08},
  url = {http://link.aps.org/doi/10.1103/PhysRevLett.80.1320},
  urldate = {2015-07-08}
}


@BOOK{helgaker2014molecular,
  title = {Molecular electronic-structure theory},
  publisher = {John Wiley \& Sons},
  year = {2014},
  author = {Helgaker, Trygve and Jorgensen, Poul and Olsen, Jeppe},
  owner = {vijay},
  timestamp = {2015.07.09}
}

@ARTICLE{qlalgo,
  author = {Bowdler, Hilary and Martin, R. S. and Reinsch, Dr C. and Wilkinson,
	Dr J. H.},
  title = {{TheQR} {andQL} algorithms for symmetric matrices},
  journal = {Numerische Mathematik},
  year = {1968},
  volume = {11},
  pages = {293--306},
  number = {4},
  month = may,
  doi = {10.1007/BF02166681},
  file = {Full Text PDF:files/861/Bowdler et al. - 1968 - TheQR andQL algorithms for symmetric matrices.pdf:application/pdf;Snapshot:files/862/BF02166681.html:text/html},
  issn = {0029-599X, 0945-3245},
  keywords = {Appl.Mathematics/Computational Methods of Engineering, Mathematical
	and Computational Physics, Mathematical Methods in Physics, Mathematics,
	general, Numerical Analysis, Numerical and Computational Methods},
  language = {en},
  owner = {vijay},
  timestamp = {2015.07.13},
  url = {http://link.springer.com/article/10.1007/BF02166681},
  urldate = {2015-07-13}
}

@ARTICLE{householder,
  author = {Martin, Dr R. S. and Reinsch, Dr C. and Wilkinson, J. H.},
  title = {Householder's tridiagonalization of a symmetric matrix},
  journal = {Numerische Mathematik},
  year = {1968},
  volume = {11},
  pages = {181--195},
  number = {3},
  month = mar,
  doi = {10.1007/BF02161841},
  file = {Full Text PDF:files/864/Martin et al. - 1968 - Householder's tridiagonalization of a symmetric ma.pdf:application/pdf;Snapshot:files/865/BF02161841.html:text/html},
  issn = {0029-599X, 0945-3245},
  keywords = {Appl.Mathematics/Computational Methods of Engineering, Mathematical
	and Computational Physics, Mathematical Methods in Physics, Mathematics,
	general, Numerical Analysis, Numerical and Computational Methods},
  language = {en},
  owner = {vijay},
  timestamp = {2015.07.13},
  url = {http://link.springer.com/article/10.1007/BF02161841},
  urldate = {2015-07-13}
}

@INCOLLECTION{intelmkl,
  author = {Wang, Endong and Zhang, Qing and Shen, Bo and Zhang, Guangyong and
	Lu, Xiaowei and Wu, Qing and Wang, Yajuan},
  title = {Intel Math Kernel Library},
  booktitle = {High-Performance Computing on the Intel{\textregistered} Xeon Phi},
  publisher = {Springer},
  year = {2014},
  pages = {167--188},
  owner = {vijay},
  timestamp = {2015.07.13}
}

@ARTICLE{nesbet1965algorithm,
  author = {Nesbet, R-K\_},
  title = {Algorithm for diagonalization of large matrices},
  journal = {The Journal of Chemical Physics},
  year = {1965},
  volume = {43},
  pages = {311--312},
  number = {1},
  owner = {vijay},
  publisher = {AIP Publishing},
  timestamp = {2015.07.13}
}

@ARTICLE{shavitt1973,
  author = {Shavitt, I and Bender, CF and Pipano, A and Hosteny, RP},
  title = {The iterative calculation of several of the lowest or highest eigenvalues
	and corresponding eigenvectors of very large symmetric matrices},
  journal = {Journal of Computational Physics},
  year = {1973},
  volume = {11},
  pages = {90--108},
  number = {1},
  owner = {vijay},
  publisher = {Elsevier},
  timestamp = {2015.07.13}
}

@ARTICLE{davidson,
  author = {Davidson, Ernest R},
  title = {The iterative calculation of a few of the lowest eigenvalues and
	corresponding eigenvectors of large real-symmetric matrices},
  journal = {Journal of Computational Physics},
  year = {1975},
  volume = {17},
  pages = {87--94},
  number = {1},
  owner = {vijay},
  publisher = {Elsevier},
  timestamp = {2015.07.13}
}

@ARTICLE{arnoldi,
  author = {Arnoldi, Walter Edwin},
  title = {The principle of minimized iterations in the solution of the matrix
	eigenvalue problem},
  journal = {Quarterly of Applied Mathematics},
  year = {1951},
  volume = {9},
  pages = {17--29},
  number = {1},
  owner = {vijay},
  publisher = {AMER MATHEMATICAL SOC 201 CHARLES ST, PROVIDENCE, RI 02940-2213},
  timestamp = {2015.07.13}
}

@INPROCEEDINGS{yamada_16.447_2005,
  author = {Yamada, S. and Imamura, T. and Machida, M.},
  title = {16.447 {TFlops} and 159-{Billion}-dimensional {Exact}-diagonalization
	for {Trapped} {Fermion}- {Hubbard} {Model} on the {Earth} {Simulator}},
  booktitle = {Supercomputing, 2005. {Proceedings} of the {ACM}/{IEEE} {SC} 2005
	{Conference}},
  year = {2005},
  pages = {44--44},
  month = nov,
  abstract = {In order to study a possibility of superfluidity in trapped atomic
	Fermi gases loaded on optical lattices, we implement an exact diagonalization
	code for the trapped Hubbard model on the Earth Simulator. Comparing
	two diagonalization algorithms, we find that the performance of the
	preconditioned conjugate gradient (PCG) method is 1.5 times superior
	to the conventional Lanczos one since the PCG method can conceal
	the communication overhead much more efficiently. Consequently, the
	PCG method shows 16.447 TFlops (50.2\% of the peak) on 512 nodes.
	On the other hand, we succeed in solving a 159-billion- dimensional
	matrix by using the conventional Lanczos method. To our knowledge,
	this dimension is a world- record. Numerical results reveal that
	an unconventional type of superfluidity specific to the confined
	system develops under repulsive interaction.},
  doi = {10.1109/SC.2005.1},
  file = {IEEE Xplore Abstract Record:files/879/cookiedetectresponse.html:text/html;IEEE Xplore Full Text  PDF:files/878/Yamada et al. - 2005 - 16.447 TFlops and 159-Billion-dimensional Exact-di.pdf:application/  pdf},
  keywords = {Atom optics, Charge carrier processes, Earth, Electrons, Gases, Lattices,
	Neutrons, Permission, Physics, Protons},
  owner = {vijay},
  timestamp = {2015.07.16}
}

@misc{homecode1,
  author       = {vijay gopal Chilkuri and
                  Nathalie Guihery and
                  Nicolas Suaud and
                  Anthony Scemama},
  title        = {DEHam: First Public Beta},
  month        = jul,
  year         = {2015},
  url          = {http://dx.doi.org/10.5281/zenodo.20450},
  howpublished = {\url{http://dx.doi.org/10.5281/zenodo.20450}}
}

@INPROCEEDINGS{cgc1,
  author = {Edmonds, AR},
  title = {Unitary Symmetry in Theories of Elementary Particles: The Reduction
	of Products of Representations of the Groups U (3) and SU (3)},
  booktitle = {Proceedings of the Royal Society of London A: Mathematical, Physical
	and Engineering Sciences},
  year = {1962},
  volume = {268},
  number = {1335},
  pages = {567--579},
  organization = {The Royal Society},
  owner = {vijay},
  timestamp = {2015.07.23}
}

@ARTICLE{cgc2,
  author = {Rashid, MA},
  title = {The reduction of the product of three 8-dimensional representations
	of U (3) and SU (3) into irreducible representations},
  journal = {Il Nuovo Cimento Series 10},
  year = {1962},
  volume = {26},
  pages = {118--133},
  number = {1},
  owner = {vijay},
  publisher = {Springer},
  timestamp = {2015.07.23}
}

@ARTICLE{van_den_berg_orbital_2012,
  author = {van den Berg, T. L. and Lombardo, P. and Kuzian, R. O. and Hayn,
	R.},
  title = {Orbital polaron in double-exchange ferromagnets},
  journal = {Physical Review B},
  year = {2012},
  volume = {86},
  pages = {235114},
  number = {23},
  month = dec,
  __markedentry = {[vijay:4]},
  abstract = {We investigate the spectral properties of the two-orbital Hubbard
	model, including the pair hopping term, by means of the dynamical
	mean field method. This Hamiltonian describes materials in which
	ferromagnetism is realized by the double-exchange mechanism, as for
	instance manganites, nickelates, or diluted magnetic semiconductors.
	The spectral function of the unoccupied states is characterized by
	a specific equidistant three peak structure. We emphasize the importance
	of the double hopping term on the spectral properties. We show the
	existence of a ferromagnetic phase due to electron doping near n=1
	by the double-exchange mechanism. A quasiparticle excitation at the
	Fermi energy is found that we attribute to what we will call an orbital
	polaron. We derive an effective spin-pseudospin Hamiltonian for the
	two-orbital double-exchange model at n=1 filling to explain the existence
	and dynamics of this quasiparticle.},
  doi = {10.1103/PhysRevB.86.235114},
  file = {APS Snapshot:files/566/van den Berg et al. - 2012 - Orbital polaron in double-exchange ferromagnets.html:text/html;Full Text PDF:files/302/van den Berg et al. - 2012 - Orbital polaron in double-exchange ferromagnets.pdf:application/pdf},
  owner = {vijay},
  timestamp = {2015.06.20},
  url = {http://link.aps.org/doi/10.1103/PhysRevB.86.235114},
  urldate = {2014-11-18}
}

@ARTICLE{batista_electron-doped_1998,
  author = {Batista, C. D. and Eroles, J. and Avignon, M. and Alascio, B.},
  title = {Electron-doped manganese perovskites:?{The} magnetic polaron state},
  journal = {Physical Review B},
  year = {1998},
  volume = {58},
  pages = {R14689--R14692},
  number = {22},
  month = dec,
  note = {00033},
  __markedentry = {[vijay:1]},
  abstract = {Using the Lanczos method in linear chains we study the ground state
	of the double exchange model including an antiferromagnetic superexchange
	in the low concentration limit. We find that this ground state is
	always inhomogeneous, containing ferromagnetic polarons. The extension
	of the polaron spin distortion, the dispersion relation and its trapping
	by impurities, are studied for different values of the superexchange
	interaction and magnetic field. We also find repulsive polaron-polaron
	interaction.},
  doi = {10.1103/PhysRevB.58.R14689},
  file = {APS Snapshot:files/474/Batista et al. - 1998 - Electron-doped manganese perovskites The magnetic.html:text/html;Full Text PDF:files/327/Batista et al. - 1998 - Electron-doped manganese perovskites The magnetic.pdf:application/pdf},
  owner = {vijay},
  shorttitle = {Electron-doped manganese perovskites},
  timestamp = {2015.06.20},
  url = {http://link.aps.org/doi/10.1103/PhysRevB.58.R14689},
  urldate = {2014-06-06}
}

@ARTICLE{navarro_spin-polarons_2012,
  author = {Navarro, O. and Vallejo, E. and Avignon, M.},
  title = {Spin-polarons in an exchange model},
  journal = {International Journal of Modern Physics B},
  year = {2012},
  volume = {26},
  pages = {1250048},
  number = {09},
  month = apr,
  __markedentry = {[vijay:1]},
  abstract = {Spin-polarons are obtained using an Ising-like exchange model consisting
	of double and super-exchange interactions in low-dimensional systems.
	At zero temperature, a new phase separation between small magnetic
	polarons, one conduction electron self-trapped in a magnetic domain
	of two or three sites, and the antiferromagnetic phase was previously
	reported. On the other hand the important effect of temperature was
	missed. Temperature diminishes Boltzmann probability allowing excited
	states in the system. Static magnetic susceptibility and short-range
	spin?spin correlations at zero magnetic field were calculated to
	explore the spin-polaron formation. At high temperature Curie?Weiss
	behavior is obtained and compared with the Curie-like behavior observed
	in the nickelate one-dimensional compound Y2-nCanBaNiO5.},
  doi = {10.1142/S0217979212500488},
  file = {Full Text PDF:files/426/Navarro et al. - 2012 - Spin-polarons in an exchange model.pdf:application/pdf;Snapshot:files/635/Navarro et al. - 2012 - Spin-polarons in an exchange model.html:text/html},
  issn = {0217-9792},
  owner = {vijay},
  timestamp = {2015.06.20},
  url = {http://www.worldscientific.com/doi/abs/10.1142/S0217979212500488},
  urldate = {2015-06-15}
}

@ARTICLE{riera_phase_1997,
  author = {Riera, Jose and Hallberg, Karen and Dagotto, Elbio},
  title = {Phase {Diagram} of {Electronic} {Models} for {Transition} {Metal}
	{Oxides} in {One} {Dimension}},
  journal = {Physical Review Letters},
  year = {1997},
  volume = {79},
  pages = {713--716},
  number = {4},
  month = jul,
  abstract = {The zero temperature phase diagram of the ferromagnetic Kondo model
	in one dimension is studied using numerical techniques, especially
	at large Hund coupling. A robust region of fully saturated ferromagnetism
	(FM) is identified at all densities. Phase separation between hole-rich
	and hole-poor regions and a paramagnetic regime with quasilocalized
	holes were also observed. It is argued that these phases will also
	appear in two and three dimensions. Our results apply both to manganites
	and one-dimensional compounds such as Y2?xCaxBaNiO5. As the transition
	metal ion spin grows, the hole mobility rapidly decreases, explaining
	the differences between Cu oxides and Mn oxides.},
  doi = {10.1103/PhysRevLett.79.713},
  file = {APS Snapshot:files/538/Riera et al. - 1997 - Phase Diagram of Electronic Models for Transition .html:text/html;Full Text PDF:files/518/Riera et al. - 1997 - Phase Diagram of Electronic Models for Transition .pdf:application/pdf},
  owner = {vijay},
  timestamp = {2015.06.20},
  url = {http://link.aps.org/doi/10.1103/PhysRevLett.79.713},
  urldate = {2015-06-09}
}

@ARTICLE{tasaki_extension_1989,
  author = {Tasaki, Hal},
  title = {Extension of {Nagaoka}'s theorem on the large- {\textbackslash}textit\{{U}\}
	{Hubbard} model},
  journal = {Physical Review B},
  year = {1989},
  volume = {40},
  pages = {9192--9193},
  number = {13},
  month = nov,
  abstract = {An extension is given of Nagaoka?s theorem on the existence of ferromagnetism
	in the large-U Hubbard model with precisely one hole. The present
	extension covers a large class of models with arbitrary non-negative
	hopping matrix elements and arbitrary spin-independent interactions.},
  doi = {10.1103/PhysRevB.40.9192},
  file = {APS Snapshot:files/113/PhysRevB.40.html:text/html;Full Text PDF:files/108/Tasaki - 1989 -    Extension of Nagaoka's theorem on the large-    t.pdf:application/pdf},
  url = {http://link.aps.org/doi/10.1103/PhysRevB.40.9192},
  urldate = {2015-07-23}
}

@ARTICLE{nagaoka,
  author = {Nagaoka, Yosuke},
  title = {Ground state of correlated electrons in a narrow almost half-filled
	s band},
  journal = {Solid State Communications},
  year = {1965},
  volume = {3},
  pages = {409--412},
  number = {12},
  month = dec,
  abstract = {We consider a system of conduction electrons in an almost half-filled
	s band with an infinitely strong ?-function type repulsive potential,
	and with non-vanishing transfer matrix elements only between nearest
	neighbors. We find rigorously that the totally polarized ferromagnetic
	state is the ground state for sc and bcc and for fcc and hcp with
	Ne \&gt; N, Ne and N being respectively the number of electrons and
	atoms, and that it is not the ground state for fcc and hcp with Ne
	\&lt; N.},
  doi = {10.1016/0038-1098(65)90266-8},
  file = {ScienceDirect Full Text PDF:files/142/Nagaoka - 1965 - Ground state of correlated electrons   in a narrow a.pdf:application/pdf;ScienceDirect Snapshot:files/152/0038109865902668.html:text/html},
  issn = {0038-1098},
  url = {http://www.sciencedirect.com/science/article/pii/0038109865902668},
  urldate = {2015-07-23}
}

@ARTICLE{costamagna_magnetic_2008,
  author = {Costamagna, S. and Riera, J. A.},
  title = {Magnetic and transport properties of the one-dimensional ferromagnetic
	{Kondo} lattice model with an impurity},
  journal = {Physical Review B},
  year = {2008},
  volume = {77},
  pages = {045302},
  number = {4},
  month = jan,
  note = {00002},
  abstract = {We have studied the ferromagnetic Kondo lattice model (FKLM) with
	an Anderson impurity on finite chains with numerical techniques.
	We are particularly interested in the metallic ferromagnetic phase
	of the FKLM. This model could describe either a quantum dot coupled
	to one-dimensional ferromagnetic leads made with manganites or a
	substitutional transition metal impurity in a MnO chain. We determined
	the region in parameter space where the impurity is empty, half filled,
	or doubly occupied and, hence, where it is magnetic or nonmagnetic.
	The most important result is that we found, for a wide range of impurity
	parameters and electron densities where the impurity is magnetic,
	a singlet phase located between two saturated ferromagnetic phases
	which correspond approximately to the empty and doubly occupied impurity
	states. Transport properties behave, in general, as expected as a
	function of the impurity occupancy, and they provide a test for a
	recently developed numerical approach to compute the conductance.
	The results obtained could be, in principle, reproduced experimentally
	in already existent related nanoscopic devices or in impurity doped
	MnO nanotubes.},
  doi = {10.1103/PhysRevB.77.045302},
  file = {Full Text PDF:files/562/Costamagna and Riera - 2008 - Magnetic and transport properties of the one-dimen.pdf:application/pdf},
  owner = {vijay},
  timestamp = {2015.06.20},
  url = {http://link.aps.org/doi/10.1103/PhysRevB.77.045302},
  urldate = {2014-07-16}
}

@ARTICLE{costamagna_numerical_2008-2,
  author = {Costamagna, S. and Riera, J. A.},
  title = {Numerical study of finite size effects in the one-dimensional two-impurity
	{Anderson} model},
  journal = {Physical Review B},
  year = {2008},
  volume = {77},
  pages = {235103},
  number = {23},
  month = jun,
  note = {00007},
  abstract = {We study the two-impurity Anderson model on finite chains using numerical
	techniques. We discuss the departure of magnetic correlations as
	a function of the interimpurity distance from a pure 2kF oscillation
	due to open boundary conditions. We observe qualitatively different
	behaviors in the interimpurity spin correlations and in transport
	properties at different values of the impurity couplings. We relate
	these different behaviors to a change in the relative dominance between
	the Kondo effect and the Ruderman-Kittel-Kasuya-Yoshida (RKKY) interaction.
	We also observe that when RKKY dominates, there is a definite relation
	between interimpurity magnetic correlations and transport properties.
	In this case, there is a recovery of 2kF periodicity when the on-site
	Coulomb repulsion on the chain is increased at quarter filling. The
	present results could be relevant for electronic nanodevices implementing
	a nonlocal control between two quantum dots that could be located
	at variable distance along a wire.},
  doi = {10.1103/PhysRevB.77.235103},
  file = {APS Snapshot:files/544/Costamagna and Riera - 2008 - Numerical study of finite size effects in the one-.html:text/html;Full Text PDF:files/416/Costamagna and Riera - 2008 - Numerical study of finite size effects in the one-.pdf:application/pdf},
  owner = {vijay},
  timestamp = {2015.06.20},
  url = {http://link.aps.org/doi/10.1103/PhysRevB.77.235103},
  urldate = {2014-07-17}
}

@ARTICLE{koga_hole-doping_2002,
  author = {Koga, Akihisa and Kawakami, Norio and Sigrist, Manfred},
  title = {Hole-doping effects on an {S}=1 ladder system},
  journal = {Physica B: Condensed Matter},
  year = {2002},
  volume = {312--313},
  pages = {606--608},
  month = mar,
  note = {00000},
  abstract = {Some zero-temperature properties of doped S=1 spin ladder systems
	are reported here. We study the low-energy states for small hole
	concentrations by means of the series expansion method . One-hole
	doping generates two kinds of the low-energy states distinguished
	by their spin, S=12 or 32, and the characteristic dispersion relation.
	In particular, we show that the one-hole state with S=32 is a composite
	particle, i.e. a bound state of an S=12 hole and a spin triplet excitation.},
  doi = {10.1016/S0921-4526(01)01153-X},
  file = {ScienceDirect Full Text PDF:files/539/Koga et al. - 2002 - Hole-doping effects on an S=1 ladder system.pdf:application/pdf;ScienceDirect Snapshot:files/613/Koga et al. - 2002 - Hole-doping effects on an S=1 ladder system.html:text/html},
  issn = {0921-4526},
  keywords = {Bound state, S=1 ladder, Series expansion},
  owner = {vijay},
  series = {The {International} {Conference} on {Strongly} {Correlated} {Electron}
	{Systems}},
  timestamp = {2015.06.20},
  url = {http://www.sciencedirect.com/science/article/pii/S092145260101153X},
  urldate = {2014-06-07}
}


@ARTICLE{malvezzi_influence_1999,
  author = {Malvezzi, A. L. and Yunoki, S. and Dagotto, E.},
  title = {Influence of nearest-neighbor {Coulomb} interactions on the phase
	diagram of the ferromagnetic {Kondo} model},
  journal = {Physical Review B},
  year = {1999},
  volume = {59},
  pages = {7033--7042},
  number = {10},
  month = mar,
  abstract = {The influence of a nearest-neighbor Coulomb repulsion of strength
	V on the properties of the ferromagnetic Kondo model is analyzed
	using computational techniques. The Hamiltonian studied here is defined
	on a chain using localized S=1/2 spins, and one orbital per site.
	Special emphasis is given to the influence of the Coulomb repulsion
	on the regions of phase separation recently discovered in this family
	of models, as well as on the double-exchange-induced ferromagnetic
	ground state. When phase separation dominates at V=0, the Coulomb
	interaction breaks the large domains of the two competing phases
	into small ?islands? of one phase embedded into the other. This is
	in agreement with several experimental results, as discussed in the
	text. Vestiges of the original phase separation regime are found
	in the spin structure factor as incommensurate peaks, even at large
	values of V. In the ferromagnetic regime close to density n=0.5,
	the Coulomb interaction induces tendencies to charge ordering without
	altering the fully polarized character of the state. This regime
	of ?charge-ordered ferromagnetism? may be related with experimental
	observations of a similar phase by Chen and Cheong [Phys. Rev. Lett.
	76, 4042 (1996)]. Our results reinforce the recently introduced notion
	[see, e.g., S. Yunoki et al., Phys. Rev. Lett. 80, 845 (1998)] that
	in realistic models for manganites analyzed with unbiased many-body
	techniques, the ground state properties arise from a competition
	between ferromagnetism and phase-separation?charge-ordering tendencies.},
  doi = {10.1103/PhysRevB.59.7033},
  file = {APS Snapshot:files/421/Malvezzi et al. - 1999 - Influence of nearest-neighbor Coulomb interactions.html:text/html;Full Text PDF:files/653/Malvezzi et al. - 1999 - Influence of nearest-neighbor Coulomb interactions.pdf:application/pdf},
  owner = {vijay},
  timestamp = {2015.06.20},
  url = {http://link.aps.org/doi/10.1103/PhysRevB.59.7033},
  urldate = {2014-11-06}
}

@ARTICLE{bastardis_ab_2006,
  author = {Bastardis, Roland and Guih{\'e}ry, Nathalie and de Graaf, Coen},
  title = {Ab initio study of the {Zener} polaron spectrum of half-doped manganites:
	{Comparison} of several model {Hamiltonians}},
  journal = {Physical Review B},
  year = {2006},
  volume = {74},
  pages = {014432},
  number = {1},
  month = jul,
  note = {00010},
  abstract = {The low-energy spectrum of the Zener polaron in half-doped manganite
	is studied by means of correlated ab initio calculations. It is shown
	that the electronic structure of the low-energy states results from
	a subtle interplay between double-exchange configurations and O 2p?
	to Mn 3d charge-transfer configurations that obey a Heisenberg logic.
	The comparison of the calculated spectrum to those predicted by the
	Zener Hamiltonian reveals that this simple description does not correctly
	reproduces the Zener polaron physics. A better reproduction of the
	calculated spectrum is obtained with either a Heisenberg model that
	considers a purely magnetic oxygen or the Girerd-Papaefthymiou double-exchange
	model. An additional significant improvement is obtained when different
	antiferromagnetic contributions are combined with the double-exchange
	model, showing that the Zener polaron spectrum is actually ruled
	by a refined double-exchange mechanism where non-Hund atomic states
	play a non-negligible role. Finally, eight states of a different
	nature have been found to be intercalated in the double-exchange
	spectrum. These states exhibit an O to Mn charge transfer, implying
	a second O 2p orbital of approximate ? character instead of the usual
	? symmetry. A small mixing of the two families of states occurs,
	accounting for the final ordering of the states.},
  doi = {10.1103/PhysRevB.74.014432},
  file = {APS Snapshot:files/501/Bastardis et al. - 2006 - Ab initio study of the Zener polaron spectrum of h.html:text/html;Full Text PDF:files/541/Bastardis et al. - 2006 - Ab initio study of the Zener polaron spectrum of h.pdf:application/pdf},
  owner = {vijay},
  shorttitle = {Ab initio study of the {Zener} polaron spectrum of half-doped manganites},
  timestamp = {2015.06.20},
  url = {http://link.aps.org/doi/10.1103/PhysRevB.74.014432},
  urldate = {2014-07-18}
}

@Misc{interp2d,
  author       = {David Zaslavsky and Vijay Gopal Chilkuri},
  title        = {{I}nterp2d},
  year         = {2015},
  url          = {https://github.com/diazona/interp2d},
  howpublished = {\url{https://github.com/diazona/interp2d}}
}

@ARTICLE{ditusa_magnetic_1994,
  author = {DiTusa, J. F. and Cheong, S-W. and Park, J. -H. and Aeppli, G. and
	Broholm, C. and Chen, C. T.},
  title = {Magnetic and {Charge} {Dynamics} in a {Doped} {One}-{Dimensional}
	{Transition} {Metal} {Oxide}},
  journal = {Physical Review Letters},
  year = {1994},
  volume = {73},
  pages = {1857--1860},
  number = {13},
  month = sep,
  note = {00148},
  __markedentry = {[vijay:1]},
  abstract = {We have measured the electrical resistivity, polarized x-ray absorption,
	and magnetic neutron scattering for Y2?xCaxBaNi1?yZnyO5 to determine
	how doping affects the charge and spin dynamics of a Haldane chain
	compound. While Zn doping, which severs the NiO chains, increases
	the resistivity beyond that of the pure material, Ca doping introduces
	holes, residing mainly in the 2pz, orbital of the oxygens in the
	NiO chains. Both dopants lead to simple finite size effects above
	the Haldane gap. In addition, we have discovered that Ca doping yields
	substantial magnetic states below the Haldane gap.},
  doi = {10.1103/PhysRevLett.73.1857},
  file = {APS Snapshot:files/586/DiTusa et al. - 1994 - Magnetic and Charge Dynamics in a Doped One-Dimens.html:text/html;Full Text PDF:files/324/DiTusa et al. - 1994 - Magnetic and Charge Dynamics in a Doped One-Dimens.pdf:application/pdf},
  owner = {vijay},
  timestamp = {2015.06.20},
  url = {http://link.aps.org/doi/10.1103/PhysRevLett.73.1857},
  urldate = {2014-06-07}
}

@ARTICLE{bendazzoli_total_2014,
  author = {Bendazzoli, Gian Luigi and El Khatib, Muammar and Evangelisti, Stefano
	and Leininger, Thierry},
  title = {The total {Position} {Spread} in mixed-valence compounds: {A} study
	on the {H}4+ model system},
  journal = {Journal of Computational Chemistry},
  year = {2014},
  volume = {35},
  pages = {802--808},
  number = {10},
  month = apr,
  abstract = {The behavior of the Total Position Spread (TPS) tensor, which is the
	second moment cumulant of the total position operator, is investigated
	in the case of a mixed-valence model system. The system consists
	of two H2 molecules placed at a distance D. If D is larger than about
	4 bohr, the singly ionized system shows a mixed-valence character.
	It is shown that the magnitude of the TPS has a strong peak in the
	region of the avoided crossing. We believe that the TPS can be a
	powerful tool to characterize the behavior of the electrons in realistic
	mixed-valence compounds. © 2014 Wiley Periodicals, Inc.},
  copyright = {Copyright © 2014 Wiley Periodicals, Inc.},
  doi = {10.1002/jcc.23557},
  file = {Full Text PDF:files/811/Bendazzoli et al. - 2014 - The total Position Spread in mixed-valence    compoun.pdf:application/pdf;Snapshot:files/814/full.html:text/html},
  issn = {1096-987X},
  keywords = {full CI, H4+, Localization Tensor, mixed-valence systems, Total Position
	Spread tensor},
  language = {en},
  shorttitle = {The total {Position} {Spread} in mixed-valence compounds},
  url = {http://onlinelibrary.wiley.com/doi/10.1002/jcc.23557/abstract},
  urldate = {2015-07-27}
}

@ARTICLE{bendazzoli_asymptotic_2012,
  author = {Bendazzoli, Gian Luigi and Evangelisti, Stefano and Monari, Antonio},
  title = {Asymptotic analysis of the localization spread and polarizability
	of 1-{D} noninteracting electrons},
  journal = {International Journal of Quantum Chemistry},
  year = {2012},
  volume = {112},
  pages = {653--664},
  number = {3},
  month = feb,
  abstract = {According to the modern Theory of the Insulating State [Resta, J Chem
	Phys 2006, 124, 104104], the metallic behavior of a N-electron system
	with open boundary conditions is characterized by a localization
	spread ??? diverging in the thermodynamic limit. This quantity, which
	is the second-moment cumulant of the position operator (per electron),
	cannot in general be evaluated in closed form but for simple model
	systems. In this article, we perform an asymptotic analysis of ???
	for a gas of N non- interacting electrons in a 1-Dimensional box
	and a Hückel chain of N equivalent sites. The asymptotic behavior
	of the closely related polarizability tensor is also investigated
	for these exactly solvable models. © 2011 Wiley Periodicals, Inc.
	Int J Quantum Chem, 2011},
  copyright = {Copyright © 2011 Wiley Periodicals, Inc.},
  doi = {10.1002/qua.23036},
  file = {Full Text PDF:files/253/Bendazzoli et al. - 2012 - Asymptotic analysis of the localization       spread and.pdf:application/pdf;Snapshot:files/356/abstract.html:text/html},
  issn = {1097-461X},
  keywords = {1D electron gas, Huckel chain, insulating state, particle in a box},
  language = {en},
  url = {http://onlinelibrary.wiley.com/doi/10.1002/qua.23036/abstract},
  urldate = {2015-07-27}
}

@ARTICLE{bendazzoli_kohns_2010,
  author = {Bendazzoli, Gian Luigi and Evangelisti, Stefano and Monari, Antonio
	and Resta, Raffaele},
  title = {Kohn?s localization in the insulating state: {One}-dimensional lattices,
	crystalline versus disordered},
  journal = {The Journal of Chemical Physics},
  year = {2010},
  volume = {133},
  pages = {064703},
  number = {6},
  month = aug,
  abstract = {The qualitative difference between insulators and metals stems from
	the nature of the low- lying excitations, but also?according to Kohn?s
	theory [W. Kohn, Phys. Rev.133, A171 (1964)]?from a different organization
	of the electrons in their ground state: electrons are localized in
	insulators and delocalized in metals. We adopt a quantitative measure
	of such localization, by means of a ?localization length? ? , finite
	in insulators and divergent in metals. We perform simulations over
	a one-dimensional binary alloy model, in a tight-binding scheme.
	In the ordered case the model is either a band insulator or a band
	metal, whereas in the disordered case it is an Anderson insulator.
	The results show indeed a localized/delocalized ground state in the
	insulating/metallic cases, as expected. More interestingly, we find
	a significant difference between the two insulating cases: band versus
	Anderson. The insulating behavior is due to two very different scattering
	mechanisms; we show that the corresponding values of ? differ by
	a large factor for the same alloy composition. We also investigate
	the organization of the electrons in the many body ground state from
	the viewpoint of the density matrices and of Boys? theory of localization.},
  doi = {10.1063/1.3467877},
  file = {Full Text PDF:files/609/Bendazzoli et al. - 2010 - Kohn?s localization in the insulating state   One-d.pdf:application/pdf;Snapshot:files/620/1.html:text/html},
  issn = {0021-9606, 1089-7690},
  keywords = {Band structure, Boundary value problems, Fermi levels, Ground states,
	Insulators},
  shorttitle = {Kohn?s localization in the insulating state},
  url = {http://scitation.aip.org/content/aip/journal/jcp/133/6/10.1063/1.3467877},
  urldate = {2015-07-27}
}

@ARTICLE{kohn_theory_1964,
  author = {Kohn, Walter},
  title = {Theory of the {Insulating} {State}},
  journal = {Physical Review},
  year = {1964},
  volume = {133},
  pages = {A171--A181},
  number = {1A},
  month = jan,
  abstract = {In this paper a new and more comprehensive characterization of the
	insulating state of matter is developed. This characterization includes
	the conventional insulators with energy gap as well as systems discussed
	by Mott which, in band theory, would be metals. The essential property
	is this: Every low-lying wave function ? of an insulating ring breaks
	up into a sum of functions, ?=?????M, which are localized in disconnected
	regions of the many-particle configuration space and have essentially
	vanishing overlap. This property is the analog of localization for
	a single particle and leads directly to the electrical properties
	characteristic of insulators. An Appendix deals with a soluble model
	exhibiting a transition between an insulating and a conducting state.},
  doi = {10.1103/PhysRev.133.A171},
  file = {APS Snapshot:files/855/PhysRev.133.html:text/html},
  url = {http://link.aps.org/doi/10.1103/PhysRev.133.A171},
  urldate = {2015-07-28}
}

@ARTICLE{resta_insulating_2011,
  author = {Resta, R.},
  title = {The insulating state of matter: a geometrical theory},
  journal = {The European Physical Journal B},
  year = {2011},
  volume = {79},
  pages = {121--137},
  number = {2},
  month = jan,
  abstract = {In 1964 Kohn published the milestone paper ?Theory of the insulating
	state?, according to which insulators and metals differ in their
	ground state. Even before the system is excited by any probe, a different
	organization of the electrons is present in the ground state and
	this is the key feature discriminating between insulators and metals.
	However, the theory of the insulating state remained somewhat incomplete
	until the late 1990s; this review addresses the recent developments.
	The many-body ground wavefunction of any insulator is characterized
	by means of geometrical concepts (Berry phase, connection, curvature,
	Chern number, quantum metric). Among them, it is the quantum metric
	which sharply characterizes the insulating state of matter. The theory
	deals on a common ground with several kinds of insulators: band insulators,
	Mott insulators, Anderson insulators, quantum Hall insulators, Chern
	and topological insulators.},
  doi = {10.1140/epjb/e2010-10874-4},
  file = {Full Text PDF:files/451/Resta - 2011 - The insulating state of matter a geometrical theo.pdf:application/pdf;Snapshot:files/298/Resta - 2011 - The insulating state of matter a geometrical theo.html:text/html},
  issn = {1434-6028, 1434-6036},
  keywords = {Condensed Matter Physics, Fluid- and Aerodynamics, Physics, general,
	Solid State Physics, Statistical Physics, Dynamical Systems and Complexity},
  language = {en},
  owner = {vijay},
  shorttitle = {The insulating state of matter},
  timestamp = {2015.06.20},
  url = {http://link.springer.com/article/10.1140/epjb/e2010-10874-4},
  urldate = {2014-12-02}
}

@ARTICLE{yang_concept_1962,
  author = {Yang, C. N.},
  title = {Concept of {Off}-{Diagonal} {Long}-{Range} {Order} and the {Quantum}
	{Phases} of {Liquid} {He} and of {Superconductors}},
  journal = {Reviews of Modern Physics},
  year = {1962},
  volume = {34},
  pages = {694--704},
  number = {4},
  month = oct,
  __markedentry = {[vijay:1]},
  abstract = {DOI:},
  doi = {10.1103/RevModPhys.34.694},
  file = {APS Snapshot:files/906/RevModPhys.34.html:text/html;Full Text PDF:files/905/Yang - 1962 -       Concept of Off-Diagonal Long-Range Order and the Q.pdf:application/pdf},
  url = {http://link.aps.org/doi/10.1103/RevModPhys.34.694},
  urldate = {2015-07-28}
}

@ARTICLE{malrieu_recipe_2012,
  author = {Malrieu, Jean-Paul and Trinquier, Georges},
  title = {A {Recipe} for {Geometry} {Optimization} of {Diradicalar} {Singlet}
	{States} from {Broken}- {Symmetry} {Calculations}},
  journal = {The Journal of Physical Chemistry A},
  year = {2012},
  volume = {116},
  pages = {8226--8237},
  number = {31},
  month = aug,
  abstract = {The equilibrium geometries of the singlet and triplet states of diradicals
	may be somewhat different, which may have an influence on their magnetic
	properties. The single-determinantal methods, such as Hartree?Fock
	or Kohn?Sham density functional theory, in general rely on broken-symmetry
	solutions to approach the singlet-state energy and geometry. An approximate
	spin decontamination is rather easy for the energy of this state
	but is rarely performed for its geometry optimization. We suggest
	simple procedures to estimate the optimized geometry and energy of
	a spin-decontaminated singlet, the accuracies of which are tested
	on a few organic diradicals. This technique can be generalized to
	interactions between higher-spin units or to multispin systems.},
  doi = {10.1021/jp303825x},
  file = {ACS Full Text PDF w/ Links:files/820/Malrieu and Trinquier - 2012 - A Recipe for Geometry       Optimization of Diradicalar .pdf:application/pdf;ACS Full Text Snapshot:files/910/jp303825x.html:text/    html},
  issn = {1089-5639},
  url = {http://dx.doi.org/10.1021/jp303825x},
  urldate = {2015-08-07}
}

@ARTICLE{boilleau_possible_2009,
  author = {Boilleau, Corentin and Suaud, Nicolas and Bastardis, Roland and Guih{\'e}ry,
	Nathalie and Malrieu, Jean Paul},
  title = {Possible use of {DFT} approaches for the determination of double
	exchange interactions},
  journal = {Theoretical Chemistry Accounts},
  year = {2009},
  volume = {126},
  pages = {231--241},
  number = {3-4},
  month = nov,
  abstract = {DFT calculations are performed on a model mixed-valence system presenting
	a double exchange phenomenon. Due to the intrinsic multireference
	character of the lowest Ms components of the spin states, it is shown
	that the interactions involved in the double-exchange model cannot
	be simply extracted from the DFT energies as it is sometimes done.
	It is, however, possible to extract from different DFT single determinant
	energies the interactions of a generalized Hubbard Hamiltonian, from
	which, in a second step, the double-exchange spectrum may be evaluated.
	The problems generated by the charge and spin polarization are discussed
	in both symmetric and non symmetric geometries, and the sensitivity
	of the results to the choice of the density functional is illustrated.},
  doi = {10.1007/s00214-009-0671-4},
  file = {Full Text PDF:files/941/Boilleau et al. - 2009 - Possible use of DFT approaches for the         determinati.pdf:application/pdf;Snapshot:files/942/10.html:text/html},
  issn = {1432-881X, 1432-2234},
  keywords = {DFT calculations, double exchange, Inorganic Chemistry, Magnetic systems,
	Organic Chemistry, Parameter extraction, Physical Chemistry, Theoretical
	and Computational Chemistry, Theory of magnetism},
  language = {en},
  url = {http://link.springer.com/article/10.1007/s00214-009-0671-4},
  urldate = {2015-08-07}
}

@ARTICLE{ovchinnikov_multiplicity_1978,
  author = {Ovchinnikov, Alexandr A.},
  title = {Multiplicity of the ground state of large alternant organic molecules
	with conjugated bonds},
  journal = {Theoretica chimica acta},
  year = {1978},
  volume = {47},
  pages = {297--304},
  number = {4},
  month = dec,
  abstract = {The multiplicity and the full spin of the ground state of large alternate
	molecules with conjugated bonds are considered. It is strictly shown
	that if the numbers of starred and unstarred atoms (say, carbon)
	differ from each other the full spin of the molecule is more than
	zero. Some possible planar and linear molecules having the full spin
	to be proportional to their sizes are presented. Particularly, they
	would be ferromagnets at infinite sizes.},
  doi = {10.1007/BF00549259},
  file = {Full Text PDF:files/944/Ovchinnikov - 1978 - Multiplicity of the ground state of large alternan.pdf:application/pdf;Snapshot:files/945/BF00549259.html:text/html},
  issn = {0040-5744, 1432-2234},
  keywords = {Inorganic Chemistry, Organic Chemistry, Organic molecules, large alternant
	?, Physical Chemistry, Theoretical and Computational Chemistry},
  language = {en},
  url = {http://link.springer.com/article/10.1007/BF00549259},
  urldate = {2015-08-07}
}

@ARTICLE{soda_ab_2000,
  author = {Soda, T. and Kitagawa, Y. and Onishi, T. and Takano, Y. and Shigeta,
	Y. and Nagao, H. and Yoshioka, Y. and Yamaguchi, K.},
  title = {Ab initio computations of effective exchange integrals for {H}?{H},
	{H}?{He}?{H} and {Mn}2O2 complex: comparison of broken-symmetry approaches},
  journal = {Chemical Physics Letters},
  year = {2000},
  volume = {319},
  pages = {223--230},
  number = {3?4},
  month = mar,
  abstract = {Ab initio calculations of effective exchange interactions between
	spins are performed for H?H, H?He?H and a simplified model of binuclear
	manganese oxide, Mn2O2, by using the spin-unrestricted Hartree?Fock
	(UHF), spin-polarized density functional (DFT) and UHF + DFT hybrid
	methods. The scopes and limitations of these broken-symmetry approaches
	are discussed in relation to several computational schemes of effective
	exchange integrals (Jab). The natural orbitals (UNO or DNO) of the
	UHF, DFT and hybrid DFT solutions for magnetic clusters are used
	for interpretation of the superexchange interactions in Mn2O2 complexes.},
  doi = {10.1016/S0009-2614(00)00166-4},
  file = {ScienceDirect Full Text PDF:files/951/Soda et al. - 2000 - Ab initio computations of effective  exchange integ.pdf:application/pdf;ScienceDirect Snapshot:files/952/S0009261400001664.html:text/html},
  issn = {0009-2614},
  shorttitle = {Ab initio computations of effective exchange integrals for {H}?{H},
	{H}?{He}?{H} and {Mn}2O2 complex},
  url = {http://www.sciencedirect.com/science/article/pii/S0009261400001664},
  urldate = {2015-08-11}
}

@INCOLLECTION{yamaguchi_ab-initio_1986,
  author = {Yamaguchi, K. and Takahara, Y. and Fueno, T.},
  title = {Ab-{Initio} {Molecular} {Orbital} {Studies} of {Structure} and {Reactivity}
	of {Transition} {Metal}-{OXO} {Compounds}},
  booktitle = {Applied {Quantum} {Chemistry}},
  publisher = {Springer Netherlands},
  year = {1986},
  editor = {Jr, Vedene H. Smith and III, Henry F. Schaefer and Morokuma, Keiji},
  pages = {155--184},
  abstract = {Ab-initio molecular orbital (m.o.) calculations were carried out to
	elucidate electronic structures and reactivities of transition metal-oxo
	compounds. It was found (1) that the oxygens in these complexes exhibit
	dual properties, electrophilic and nucleophilic, which are determined
	by the formal oxidation number of the transition metals and ligands
	involved and (2) that the diradical characters are not negligible
	for the weak ?-bonds between transition metals and oxygens. The energy
	differences among the metal 1,4-diradical (MDR), zwitterion (MZW)
	and perepoxide (MPE) Intermediates were calculated to be not so large
	as in the case of singlet oxygen reactions. Thus the reaction mechanisms
	for epoxidations of olefins seem variable depending on the types
	of olefins and reaction conditions employed.},
  copyright = {©1986 Springer Science+Business Media B.V.},
  file = {Full Text PDF:files/959/Yamaguchi et al. - 1986 - Ab-Initio Molecular Orbital Studies of         Structure a.pdf:application/pdf;Snapshot:files/960/978-94-009-4746-7_11.html:text/html},
  isbn = {978-94-010-8609-7 978-94-009-4746-7},
  keywords = {Physical Chemistry},
  language = {en},
  url = {http://link.springer.com/chapter/10.1007/978-94-009-4746-7_11},
  urldate = {2015-08-12}
}

@ARTICLE{trinquier_designing_2011,
  author = {Trinquier, Georges and Suaud, Nicolas and Guih\'ery, Nathalie and Malrieu,
	Jean-Paul},
  title = {Designing magnetic organic lattices from high-spin polycyclic units},
  journal = {Chemphyschem: A European Journal of Chemical Physics and Physical
	Chemistry},
  year = {2011},
  volume = {12},
  pages = {3020--3036},
  number = {16},
  month = nov,
  abstract = {This work addresses the conception of purely organic magnetic materials
	by properly bridging high-spin polycyclic hydrocarbons A and B, through
	covalent ligands L. The strategy varies two degrees of freedom that
	govern the magnetic character of the A-L--B sequence, namely, the
	bridge response to spin polarization and the relative signs of spin
	density on carbon atoms to which the bridge is attached. Topological
	prescriptions based on Ovchinnikov's rule are proposed to predict
	ground-state spin multiplicities of various A-L-B sets. The relevance
	of these guiding principles is essentially confirmed through DFT
	calculations on dimers connected by conjugated bridges. The transferability
	of interunit magnetic couplings to larger assemblies is further checked,
	the building blocks tending to maintain their high-spin character
	whatever the environment. Such local designs open the way to periodic
	lattices of ferromagnetic, antiferromagnetic, ferrimagnetic, or paramagnetic
	materials.},
  doi = {10.1002/cphc.201100311},
  issn = {1439-7641},
  language = {eng},
  pmid = {22021220}
}

@ARTICLE{trinquier_theoretical_2010,
  author = {Trinquier, Georges and Suaud, Nicolas and Malrieu, Jean-Paul},
  title = {Theoretical design of high-spin polycyclic hydrocarbons},
  journal = {Chemistry (Weinheim an Der Bergstrasse, Germany)},
  year = {2010},
  volume = {16},
  pages = {8762--8772},
  number = {29},
  month = aug,
  abstract = {High-spin organic structures can be obtained from fused polycyclic
	hydrocarbons, by converting selected peripheral HC(sp(2)) sites into
	H(2)C(sp(3)) ones, guided by Ovchinnikov's rule. Theoretical investigation
	is performed on a few examples of such systems, involving three to
	twelve fused rings, and maintaining threefold symmetry. Unrestricted
	DFT (UDFT) calculations, including geometry optimizations, confirm
	the high-spin multiplicity of the ground state. Spin-density distributions
	and low-energy spectra are further studied through geometry-dependent
	Heisenberg-Hamiltonian diagonalizations and explicit correlated ab
	initio treatments, which all agree on the high-spin character of
	the suggested structures, and locate the low-lying states at significantly
	higher energies. In particular, the lowest- lying state of lower
	multiplicity is always found to be higher than kT at room temperature
	(at least ten times higher). Simplification of the ferromagnetic
	organization based on sets of semilocalized nonbonding orbitals is
	proposed. Molecular architectures are thus conceived in which the
	ferromagnetically-coupled unpaired electrons tally up to one third
	of the involved conjugated carbons. Connecting such building blocks
	should provide bidimensional materials endowed with robust magnetic
	properties.},
  doi = {10.1002/chem.201000044},
  issn = {1521-3765},
  language = {eng},
  pmid = {20572170}
}

@ARTICLE{rajca1994organic,
  author = {Rajca, Andrzej},
  title = {Organic Diradicals and Polyradicals: From Spin Coupling to Magnetism?},
  journal = {Chemical reviews},
  year = {1994},
  volume = {94},
  pages = {871--893},
  number = {4},
  publisher = {ACS Publications}
}

@ARTICLE{rajca1999very,
  author = {Rajca, Andrzej and Rajca, Suchada and Wongsriratanakul, Jirawat},
  title = {Very high-spin organic polymer: $\pi$-Conjugated hydrocarbon network
	with average spin of S? 40},
  journal = {Journal of the American Chemical Society},
  year = {1999},
  volume = {121},
  pages = {6308--6309},
  number = {26},
  publisher = {ACS Publications}
}

@ARTICLE{rajca_organic_2004-2,
  author = {Rajca, Andrzej and Wongsriratanakul, Jirawat and Rajca, Suchada},
  title = {Organic spin clusters: macrocyclic-macrocyclic polyarylmethyl polyradicals
	with very high spin {S} = 5-13},
  journal = {Journal of the American Chemical Society},
  year = {2004},
  volume = {126},
  pages = {6608--6626},
  number = {21},
  month = jun,
  abstract = {Synthesis and magnetic studies of a new class of organic spin clusters,
	possessing alternating connectivity of unequal spins, are described.
	Polyarylmethyl polyether precursors to the spin clusters, with linear
	and branched connectivity between calix[4]arene-based macrocycles,
	are prepared via modular, multistep syntheses. Their molecular connectivity
	and stereoisomerism are analyzed using NMR spectroscopy. The absolute
	masses (4-10 kDa) are determined by FABMS and GPC/MALS. Small angle
	neutron scattering (SANS) provides the radii of gyration of 1.2-1.8
	nm. The corresponding polyradicals with 15, 22, and 36 triarylmethyls,
	which are prepared and studied as solutions in tetrahydrofuran-d(8),
	may be described as S' = 7/2, 1/2, 7/2 spin trimer (average S = 5-6),
	S' = 7/2, 1/2, 6/2, 1/2, 7/2 spin pentamer (average S = 7-9), and
	spin nonamer (average S = 11-13), respectively, as determined by
	SQUID magnetometry and numerical fits to linear combinations of the
	Brillouin functions. For spin trimer and pentamer, the quantitative
	magnetization data are fit to new percolation models, based upon
	random distributions of chemical defects and ferromagnetic vs antiferromagnetic
	couplings. The value of S = 13 is the highest for an organic molecule.},
  doi = {10.1021/ja031549b},
  issn = {0002-7863},
  language = {eng},
  pmid = {15161289},
  shorttitle = {Organic spin clusters}
}

@ARTICLE{rajca_magnetic_2001,
  author = {Rajca, Andrzej and Wongsriratanakul, Jirawat and Rajca, Suchada},
  title = {Magnetic {Ordering} in an {Organic} {Polymer}},
  journal = {Science},
  year = {2001},
  volume = {294},
  pages = {1503--1505},
  number = {5546},
  month = nov,
  abstract = {We describe preparation and magnetic properties of an organic ?-conjugated
	polymer with very large magnetic moment and magnetic order at low
	temperatures. The polymer is designed with a large density of cross-links
	and alternating connectivity of radical modules with unequal spin
	quantum numbers (S), macrocyclicS = 2 and, cross-linking S = ½ modules,
	which permits large net S values for either ferromagnetic or antiferromagnetic
	exchange couplings between the modules. In the highly cross-linked
	polymer, an effective magnetic moment corresponding to an average
	S of about 5000 and slow reorientation of the magnetization by a
	small magnetic field (less than or equal to 1 oersted) below a temperature
	of about 10 kelvin are found. Qualitatively, this magnetic behavior
	is comparable to that of insulating spin glasses and blocked superparamagnets.},
  doi = {10.1126/science.1065477},
  file = {Full Text PDF:files/934/Rajca et al. - 2001 - Magnetic Ordering in an Organic Polymer.pdf:      application/pdf;Snapshot:files/936/1503.html:text/html},
  issn = {0036-8075, 1095-9203},
  language = {en},
  pmid = {11711668},
  url = {http://www.sciencemag.org/content/294/5546/1503},
  urldate = {2015-08-07}
}

@ARTICLE{rajca_organic_2004,
  author = {Rajca, Andrzej and Wongsriratanakul, Jirawat and Rajca, Suchada and
	Cerny, Ronald L.},
  title = {Organic spin clusters: annelated macrocyclic polyarylmethyl polyradicals
	and a polymer with very high spin {S}=6-18},
  journal = {Chemistry (Weinheim an Der Bergstrasse, Germany)},
  year = {2004},
  volume = {10},
  pages = {3144--3157},
  number = {13},
  month = jul,
  abstract = {Synthesis and magnetic studies of annelated macrocyclic polyradicals
	and a related high- spin polymer with macrocyclic repeat units are
	described. Polyarylmethyl polyether precursors to the polyradicals
	and the related polymer are prepared by using Negishi cross-coupling
	of difunctionalized calix[4]arene-based macrocycles. The three lowest
	homologues, with high degree of monodispersity, are tetradecaether
	(14-ether) 3-(OCH(3))(14), octacosaether (28-ether) 4-(OCH(3))(28),
	and dotetracontaether (42-ether) 5-(OCH(3))(42), in which 2, 4, and
	6 calix[4]arene-based macrocycles are annelated to the center macrocycle,
	respectively. The evidence for their annelated structures (ladder
	connectivities) is based upon FAB-MS and the (1)H NMR based end-group
	analysis. The absolute masses (4-12 kDa) were determined by FAB-MS
	and GPC/MALS. Small angle neutron scattering (SANS) provides the
	radii of gyration of 1.7, 2.0, and 3.2 nm for 4-(OCH(3))(28), 5-(OCH(3))(42),
	and polymer 6-(OCH(3))(n), respectively. The corresponding polyarylmethyl
	polyradicals 3 and 4, and polymer 6 possess average values of S approximately
	6-7, S approximately 10, and S approximately 18, respectively, as
	determined by SQUID magnetometry and numerical fits to linear combinations
	of Brillouin functions. The quantitative values of magnetization
	at saturation and of magnetic susceptibilities indicate that about
	40-60 \% of unpaired electrons are present at low temperatures (T=1.8-5
	K). For polyradical 3, the variable temperature magnetic data are
	fit to the Heisenberg Hamiltonian based model. The variable magnetic
	field data at low temperatures are also fit to a percolation-based
	model for organic spin cluster, with random distribution of chemical
	defects, and ferromagnetic versus antiferromagnetic couplings, providing
	quantitative agreement between the experiment and the theory. For
	polyradical 3 (with S approximately 6-7), annealing at room temperature
	for 0.5 h leads to a polyradical with S approximately 5.},
  doi = {10.1002/chem.200306036},
  issn = {0947-6539},
  keywords = {Free Radicals, Macrocyclic Compounds, Magnetic Resonance Spectroscopy,
	Magnetics, Molecular Structure, Polymers, Spectrometry, Mass, Fast
	Atom Bombardment},
  language = {eng},
  pmid = {15224323},
  shorttitle = {Organic spin clusters}
}

@MISC{g09,
  author = {M. J. Frisch and G. W. Trucks and H. B. Schlegel and G. E. Scuseria
	and M. A. Robb and J. R. Cheeseman and G. Scalmani and V. Barone
	and B. Mennucci and G. A. Petersson and H. Nakatsuji and M. Caricato
	and X. Li and H. P. Hratchian and A. F. Izmaylov and J. Bloino and
	G. Zheng and J. L. Sonnenberg and M. Hada and M. Ehara and K. Toyota
	and R. Fukuda and J. Hasegawa and M. Ishida and T. Nakajima and Y.
	Honda and O. Kitao and H. Nakai and T. Vreven and Montgomery, {Jr.},
	J. A. and J. E. Peralta and F. Ogliaro and M. Bearpark and J. J.
	Heyd and E. Brothers and K. N. Kudin and V. N. Staroverov and R.
	Kobayashi and J. Normand and K. Raghavachari and A. Rendell and J.
	C. Burant and S. S. Iyengar and J. Tomasi and M. Cossi and N. Rega
	and J. M. Millam and M. Klene and J. E. Knox and J. B. Cross and
	V. Bakken and C. Adamo and J. Jaramillo and R. Gomperts and R. E.
	Stratmann and O. Yazyev and A. J. Austin and R. Cammi and C. Pomelli
	and J. W. Ochterski and R. L. Martin and K. Morokuma and V. G. Zakrzewski
	and G. A. Voth and P. Salvador and J. J. Dannenberg and S. Dapprich
	and A. D. Daniels and Ö. Farkas and J. B. Foresman and J. V. Ortiz
	and J. Cioslowski and D. J. Fox},
  title = {Gaussian?09 {R}evision {D}.01},
  note = {Gaussian Inc. Wallingford CT 2009}
}

@ARTICLE{ferre2015spin,
  author = {Ferr{\'e}, Nicolas and Guih{\'e}ry, Nathalie and Malrieu, Jean-Paul},
  title = {Spin decontamination of broken-symmetry density functional theory
	calculations: deeper insight and new formulations},
  journal = {Physical Chemistry Chemical Physics},
  year = {2015},
  volume = {17},
  pages = {14375--14382},
  number = {22},
  publisher = {Royal Society of Chemistry}
}

@ARTICLE{clar_aromatic_1953,
  author = {Clar, E. and Stewart, D. G.},
  title = {Aromatic {Hydrocarbons}. {LXV}. {Triangulene} {Derivatives}1},
  journal = {Journal of the American Chemical Society},
  year = {1953},
  volume = {75},
  pages = {2667--2672},
  number = {11},
  month = jun,
  doi = {10.1021/ja01107a035},
  file = {ACS Full Text PDF w/ Links:files/969/Clar and Stewart - 1953 - Aromatic Hydrocarbons. LXV.       Triangulene Derivative.pdf:application/pdf;ACS Full Text Snapshot:files/970/ja01107a035.html:text/html},
  issn = {0002-7863},
  url = {http://dx.doi.org/10.1021/ja01107a035},
  urldate = {2015-08-17}
}

@ARTICLE{clar_aromatic_1954,
  author = {Clar, E. and Stewart, D. G.},
  title = {Aromatic {Hydrocarbons}. {LXVIII}. {Triangulene} {Derivatives}. {Part}
	{II}1},
  journal = {Journal of the American Chemical Society},
  year = {1954},
  volume = {76},
  pages = {3504--3507},
  number = {13},
  month = jul,
  doi = {10.1021/ja01642a044},
  file = {ACS Full Text PDF:files/972/Clar and Stewart - 1954 - Aromatic Hydrocarbons. LXVIII. Triangulene Derivat.pdf:application/pdf;ACS Full Text Snapshot:files/973/ja01642a044.html:text/html},
  issn = {0002-7863},
  url = {http://dx.doi.org/10.1021/ja01642a044},
  urldate = {2015-08-17}
}

@ARTICLE{inoue_first_2001,
  author = {Inoue, Jun and Fukui, Kozo and Kubo, Takashi and Nakazawa, Shigeaki
	and Sato, Kazunobu and Shiomi, Daisuke and Morita, Yasushi and Yamamoto,
	Kagetoshi and Takui, Takeji and Nakasuji, Kazuhiro},
  title = {The {First} {Detection} of a {Clar}'s {Hydrocarbon}, 2,6,10-{Tri}-tert-{Butyltriangulene}:?
	{A} {Ground}-{State} {Triplet} of {Non}-{Kekulé} {Polynuclear} {Benzenoid}
	{Hydrocarbon}},
  journal = {Journal of the American Chemical Society},
  year = {2001},
  volume = {123},
  pages = {12702--12703},
  number = {50},
  month = dec,
  doi = {10.1021/ja016751y},
  file = {ACS Full Text PDF w/ Links:files/981/Inoue et al. - 2001 - The First Detection of a Clar's     Hydrocarbon, 2,6,1.pdf:application/pdf;ACS Full Text Snapshot:files/982/ja016751y.html:text/html},
  issn = {0002-7863},
  shorttitle = {The {First} {Detection} of a {Clar}'s {Hydrocarbon}, 2,6,10-{Tri}-tert-
	{Butyltriangulene}},
  url = {http://dx.doi.org/10.1021/ja016751y},
  urldate = {2015-08-17}
}

@ARTICLE{shultz_one-electron_2001,
  author = {Shultz, David A. and Kumar, R. Krishna},
  title = {One-{Electron} {Reduction} of an {Antiferromagnetically} {Coupled}
	{Triradical} {Yields} a {Mixed}-{Valent} {Biradical} with {Enhanced}
	{Ferromagnetic} {Coupling}},
  journal = {Journal of the American Chemical Society},
  year = {2001},
  volume = {123},
  pages = {6431--6432},
  number = {26},
  month = jul,
  doi = {10.1021/ja010630g},
  file = {ACS Full Text PDF w/ Links:files/966/Shultz and Kumar - 2001 - One-Electron Reduction of an      Antiferromagnetically.pdf:application/pdf;ACS Full Text Snapshot:files/967/ja010630g.html:text/html},
  issn = {0002-7863},
  url = {http://dx.doi.org/10.1021/ja010630g},
  urldate = {2015-08-17}
}

@ARTICLE{laursen_1998,
  author = {Laursen, Bo W. and Krebs, Frederik C. and Nielsen, Merete F. and
	Bechgaard, Klaus and Christensen, Jørn B. and Harrit, Niels},
  title = {2,6,10-{Tris}(dialkylamino)trioxatriangulenium {Ions}. {Synthesis},
	{Structure}, and {Properties} of {Exceptionally} {Stable} {Carbenium}
	{Ions}},
  journal = {Journal of the American Chemical Society},
  year = {1998},
  volume = {120},
  pages = {12255--12263},
  number = {47},
  month = dec,
  abstract = {A general synthetic route to a novel type of triamino-substituted
	planar carbenium ions (5) is reported. The synthetic method is based
	on a facile and selective nucleophilic aromatic substitution on the
	tris(2,4,6-trimethoxyphenyl)carbenium ion (1) with amines and gives
	access to a wide variety of more complex amino-substituted carbenium
	ions. X-ray crystallography shows that the 2,6,10-tris(N- pyrrolidinyl)-4,8,12-trioxatriangulenium
	ion (5b) is planar and forms segregated stacks of cations and PF6
	anions in the solid phase. The stability of the 2,6,10-tris(diethylamino)-4,8,12-trioxatriangulenium
	ion 5a is expressed as the pKR+ value, which is determined in strongly
	basic nonaqueous solution on the basis of a new acidity function
	C\_. The pKR+ value of 5a is measured to be 19.7, which is 10 orders
	of magnitude higher than the values found for the most stable carbenium
	ions previously reported. Electrochemical reduction of compound 5a
	leads to rapid dimerization. Two consecutive one-electron oxidations
	are identified by cyclic voltammetry.},
  doi = {10.1021/ja982550r},
  issn = {0002-7863},
  url = {http://dx.doi.org/10.1021/ja982550r},
  urldate = {2015-08-17}
}

@ARTICLE{lofthagen_synthesis_1992,
  author = {Lofthagen, Michael and VernonClark, Russell and Baldridge, Kim K.
	and Siegel, Jay S.},
  title = {Synthesis of trioxatricornan and derivatives. {Useful} keystones
	for the construction of rigid molecular cavities},
  journal = {The Journal of Organic Chemistry},
  year = {1992},
  volume = {57},
  pages = {61--69},
  number = {1},
  month = jan,
  doi = {10.1021/jo00027a015},
  issn = {0022-3263},
  url = {http://dx.doi.org/10.1021/jo00027a015},
  urldate = {2015-08-17}
}

@ARTICLE{bencini_electron_1978,
  author = {Bencini, A. and Gatteschi, D. and Sacconi, L.},
  title = {Electron spin resonance investigation of the mixed-valence dinuclear
	tetra(.mu.-1,8- naphthyridine-{N},{N}')-bis(bromonickel) tetraphenylborate
	complex},
  journal = {Inorganic Chemistry},
  year = {1978},
  volume = {17},
  pages = {2670--2672},
  number = {9},
  month = sep,
  doi = {10.1021/ic50187a058},
  file = {ACS Full Text PDF:files/993/Bencini et al. - 1978 - Electron spin resonance investigation of  the mixed.pdf:application/pdf;ACS Full Text Snapshot:files/994/ic50187a058.html:text/html},
  issn = {0020-1669},
  url = {http://dx.doi.org/10.1021/ic50187a058},
  urldate = {2015-08-17}
}

@ARTICLE{gatteschi_binuclear_1973,
  author = {Gatteschi, D. and Mealli, C. and Sacconi, L.},
  title = {Binuclear complex of 1.5 valent nickel},
  journal = {Journal of the American Chemical Society},
  year = {1973},
  volume = {95},
  pages = {2736--2738},
  number = {8},
  month = apr,
  doi = {10.1021/ja00789a083},
  file = {ACS Full Text PDF:files/987/Gatteschi et al. - 1973 - Binuclear complex of 1.5 valent nickel. pdf:application/pdf;ACS Full Text Snapshot:files/988/ja00789a083.html:text/html},
  issn = {0002-7863},
  url = {http://dx.doi.org/10.1021/ja00789a083},
  urldate = {2015-08-17}
}

@ARTICLE{sacconi_synthesis_1974,
  author = {Sacconi, Luigi. and Mealli, Carlo. and Gatteschi, Dante.},
  title = {Synthesis and characterization of 1,8-naphthyridine complexes of
	1.5-valent nickel},
  journal = {Inorganic Chemistry},
  year = {1974},
  volume = {13},
  pages = {1985--1991},
  number = {8},
  month = aug,
  doi = {10.1021/ic50138a039},
  file = {ACS Full Text PDF:files/990/Sacconi et al. - 1974 - Synthesis and characterization of 1,8-    naphthyridin.pdf:application/pdf;ACS Full Text Snapshot:files/991/ic50138a039.html:text/html},
  issn = {0020-1669},
  url = {http://dx.doi.org/10.1021/ic50138a039},
  urldate = {2015-08-17}
}

@ARTICLE{ding_mossbauer_1990,
  author = {Ding, Xiao-Qi and Bominaar, Emile L. and Bill, Eckhard and Winkler,
	Heiner and Trautwein, Alfred X. and Drüeke, Stefan and Chaudhuri,
	Phalguni and Wieghardt, Karl},
  title = {Mössbauer and electron paramagnetic resonance study of the double?exchange
	and {Heisenberg}?exchange interactions in a novel binuclear {Fe}({II}/{III})
	delocalized?valence compound},
  journal = {The Journal of Chemical Physics},
  year = {1990},
  volume = {92},
  pages = {178--186},
  number = {1},
  month = jan,
  abstract = {In this paper we present the characterization by UV?VIS, Mössbauer,
	and EPRspectroscopy of [L2Fe2(??OH)3](ClO4)2?2CH3OH?2H2O, with L=N,N?,N??trimethyl?1,4,7?triazacyclononane,
	a novel dimeric iron compound, which is shown to possess a central
	exchange?coupled delocalized?valence Fe(II/III) pair. Complete delocalization
	of the excess electron in the dimeric iron center is concluded from
	the indistinguishability of the two iron sites in Mössbauer spectroscopy.
	Mössbauer, EPR, and magnetic susceptibility data imply a system spin
	S t =9/2 for the ground state. This finding is explained as being
	a consequence of the double?exchange interaction which is generated
	by the delocalized electron. Experimental values obtained from UV?VIS,
	Mössbauer, and EPRspectroscopy are for the double?exchange parameter
	B=1300 cm? 1, the g factorsg x,y =2.04 and g z =2.3, the parameters
	for zero?field splitting D=4 cm? 1 and E?0 cm? 1, and for the hyperfine
	parameters ?E Q =?2.14 mm s? 1, A x,y =?21.2 T, A z =?27 T, and ?=0.74
	mm s? 1. From our temperature?dependent studies we assign to the
	first excited state a spin?octet with an excitation energy ?\&gt;175
	cm? 1. From this value a lower bound of ?235 cm? 1 has been deduced
	for the exchange?coupling constant J. In the framework of a simplified
	description of the iron atoms by unperturbed 3d orbitals, the values
	of the Atensor components as well as the quadrupole splitting ?E
	Q can be interpreted in a consistent manner by assuming the excess
	electron being delocalized over two d ? orbitals centered at the
	two iron sites of the dimer and directed along the iron?iron axis
	as the z direction.},
  doi = {10.1063/1.458460},
  file = {Full Text PDF:files/999/Ding et al. - 1990 - Mössbauer and electron paramagnetic resonance stud.pdf:application/pdf;Snapshot:files/1000/1.html:text/html},
  issn = {0021-9606, 1089-7690},
  keywords = {Electron paramagnetic resonance spectroscopy, Electron spectroscopy,
	Exchange interactions, Excitation energies, Iron},
  url = {http://scitation.aip.org/content/aip/journal/jcp/92/1/10.1063/1.458460},
  urldate = {2015-08-17}
}

@ARTICLE{drueke_novel_1989,
  author = {Dr\"ueke, S. and Chaudhuri, P. and Pohl, K. and Wieghardt, K. and Ding,
	X.-Q. and Bill, E. and Sawaryn, A. and Trautwein, A. X. and Winkler,
	H. and Gurman, S. J.},
  title = {The novel mixed-valence, exchange-coupled, class {III} dimer [{L}2Fe2(µ-{OH})3]2+({L}
	={N}, {N}?,{N}?-trimethyl-1,4,7-triazacyclononane)},
  journal = {Journal of the Chemical Society, Chemical Communications},
  year = {1989},
  pages = {59--62},
  number = {1},
  month = jan,
  abstract = {The reaction of Fe(CIO4)2·6H2O with N,N?,N,?-trimethyl-1,4,7-triazacyclononane
	(L) in methanol affords, in the presence of a small amount of oxygen,
	the deep blue binuclear complex [L2Fe2(µ- OH)3](CIO4)2·2MeOH·2H2O
	which was characterized by EXAFS, e.s.r., u.v.-visible and Mössbauer
	spectroscopy to be a mixed valence iron (II/III) species of class
	III with an Stot= 9/2 ground state.},
  doi = {10.1039/C39890000059},
  file = {Full Text PDF:files/996/Drüeke et al. - 1989 - The novel mixed-valence, exchange-coupled,     class I.pdf:application/pdf;Snapshot:files/997/c39890000059.html:text/html},
  issn = {0022-4936},
  language = {en},
  url = {http://pubs.rsc.org/en/content/articlelanding/1989/c3/c39890000059},
  urldate = {2015-08-17}
}

@ARTICLE{taratiel_refined_2004,
  author = {Taratiel, David and Guih{\'e}ry, Nathalie},
  title = {A refined model of the double exchange phenomenon: {Test} on the
	stretched {N}2+ molecule},
  journal = {The Journal of Chemical Physics},
  year = {2004},
  volume = {121},
  pages = {7127--7135},
  number = {15},
  month = oct,
  note = {00009},
  abstract = {The N 2 + molecule is studied at different interatomic distances as
	a model molecule for the double exchange mechanism. The energy spectrum
	as well as the wave functions of the lowest states are analyzed and
	confronted both with the usual model of double exchange and with
	a recently proposed refined model. It is shown that the usual model
	fails to reproduce the energy spacings while the refined model is
	valid on a large domain of interatomic distances (in the magnetic
	regime). The study of a model molecule on a large domain of interatomic
	distances makes it possible to systematically investigate several
	regimes associated with different energetic state orderings. The
	perfect agreement between the refined model and the computed energies
	in the whole domain of stretched distances shows its applicability
	to a large number of real compounds. Finally, the respective contributions
	of dynamical and nondynamical correlations are analyzed.},
  doi = {10.1063/1.1786913},
  file = {Full Text PDF:files/573/Taratiel and Guihéry - 2004 - A refined model of the double exchange phenomenon.pdf:application/pdf;Snapshot:files/482/Taratiel and Guihéry - 2004 - A refined model of the double exchange phenomenon.html:text/html},
  issn = {0021-9606, 1089-7690},
  keywords = {Interatomic distances, Numerical modeling, Spectrum analysis, Wave
	functions},
  owner = {vijay},
  shorttitle = {A refined model of the double exchange phenomenon},
  timestamp = {2015.06.20},
  url = {http://scitation.aip.org/content/aip/journal/jcp/121/15/10.1063/1.1786913},
  urldate = {2014-06-20}
}

@ARTICLE{guihery_double_2006,
  author = {Guih{\'e}ry, Nathalie},
  title = {The {Double} {Exchange} {Phenomenon} {Revisited}: {The} [{Re}2OCl10]3?
	{Compound}},
  journal = {Theoretical Chemistry Accounts},
  year = {2006},
  volume = {116},
  pages = {576--586},
  number = {4-5},
  month = mar,
  abstract = {Correlated ab initio calculations have been performed on the [Re2OCl10]3?
	anion. The calculated spectrum does not respect the intervals given
	by the usually accepted double exchange Hamiltonian. Surprisingly
	enough the ground-state happens to be of intermediate spin multiplicity
	(i.e. a quartet) at any level of correlation treatment. A model that
	combines the Anderson and Hasegawa method and the usually used double
	exchange one rationalizes the spectrum calculated both by a configuration
	interaction restricted to the open shell molecular orbitals and at
	a more correlated level of calculation. An alternative analysis of
	the double exchange phenomenon, based on a molecular orbital language,
	is presented. The specific effects of the electronic correlation
	brought by extended active space and by a difference dedicated configuration
	interaction are also analyzed.},
  doi = {10.1007/s00214-006-0103-7},
  file = {Full Text PDF:files/1003/Guihéry - 2006 - The Double Exchange Phenomenon Revisited The [Re2.pdf:application/pdf;Snapshot:files/1004/s00214-006-0103-7.html:text/html},
  issn = {1432-881X, 1432-2234},
  keywords = {Inorganic Chemistry, Organic Chemistry, Physical Chemistry, Theoretical
	and Computational Chemistry},
  language = {en},
  shorttitle = {The {Double} {Exchange} {Phenomenon} {Revisited}},
  url = {http://link.springer.com/article/10.1007/s00214-006-0103-7},
  urldate = {2015-08-20}
}

@ARTICLE{improta_interplay_2004,
  author = {Improta, Roberto and Barone, Vincenzo},
  title = {Interplay of {Electronic}, {Environmental}, and {Vibrational} {Effects}
	in {Determining} the {Hyperfine} {Coupling} {Constants} of {Organic}
	{Free} {Radicals}},
  journal = {Chemical Reviews},
  year = {2004},
  volume = {104},
  pages = {1231--1254},
  number = {3},
  month = mar,
  doi = {10.1021/cr960085f},
  file = {ACS Full Text PDF w/ Links:files/1026/Improta and Barone - 2004 - Interplay of Electronic,    Environmental, and Vibrat.pdf:application/pdf;ACS Full Text Snapshot:files/1028/cr960085f.html:text/html},
  issn = {0009-2665},
  url = {http://dx.doi.org/10.1021/cr960085f},
  urldate = {2015-09-01}
}

@ARTICLE{osiecki_studies_1968,
  author = {Osiecki, Jeanne H. and Ullman, Edwin F.},
  title = {Studies of free radicals. {I}. .alpha.-{Nitronyl} nitroxides, a new
	class of stable radicals},
  journal = {Journal of the American Chemical Society},
  year = {1968},
  volume = {90},
  pages = {1078--1079},
  number = {4},
  month = feb,
  doi = {10.1021/ja01006a053},
  file = {ACS Full Text PDF w/ Links:files/1024/Osiecki and Ullman - 1968 - Studies of free radicals.   I. .alpha.-Nitronyl nitr.pdf:application/pdf;ACS Full Text Snapshot:files/1027/ja01006a053.html:text/html},
  issn = {0002-7863},
  url = {http://dx.doi.org/10.1021/ja01006a053},
  urldate = {2015-09-01}
}

@INCOLLECTION{oakley_cyclic_1988,
  author = {Oakley, Richard T.},
  title = {Cyclic and {Heterocyclic} {Thiazenes}},
  booktitle = {Progress in {Inorganic} {Chemistry}},
  publisher = {John Wiley \& Sons, Inc.},
  year = {1988},
  editor = {Lippard, Stephen J.},
  pages = {299--391},
  abstract = {This chapter contains sections titled: ¬¬¬¬ * Introduction * Binary
	Sulfur-Nitrogen Compounds * Heterocyclic Thiazenes * Reactivity Patterns
	* Concluding Remarks},
  copyright = {Copyright © 1988 by John Wiley \& Sons, Inc.},
  file = {Snapshot:files/1022/summary.html:text/html},
  isbn = {978-0-470-16637-6},
  keywords = {cyclic thiazenes, dithiadiazolyl radicals, heterocyclic thiazenes,
	phosphorus-containing rings, redox reactions},
  language = {en},
  url = {http://onlinelibrary.wiley.com/doi/10.1002/9780470166376.ch4/summary},
  urldate = {2015-09-01}
}

@ARTICLE{kuhn_surprisingly_1963,
  author = {Kuhn, Richard and Trischmann, H.},
  title = {Surprisingly {Stable} {Nitrogenous} {Free} {Radicals}},
  journal = {Angewandte Chemie International Edition in English},
  year = {1963},
  volume = {2},
  pages = {155--155},
  number = {3},
  month = mar,
  copyright = {Copyright © 1963 by Verlag Chemie, GmbH, Germany},
  doi = {10.1002/anie.196301552},
  file = {Full Text PDF:files/1018/Kuhn and Trischmann - 1963 - Surprisingly Stable Nitrogenous Free    Radicals.pdf:application/pdf;Snapshot:files/1019/abstract.html:text/html},
  issn = {1521-3773},
  language = {en},
  url = {http://onlinelibrary.wiley.com/doi/10.1002/anie.196301552/abstract},
  urldate = {2015-09-01}
}

@ARTICLE{lemaire_progress_2010,
  author = {Lemaire, Martin T.},
  title = {Progress and design challenges for high-spin molecules},
  journal = {Pure and Applied Chemistry},
  year = {2010},
  volume = {83},
  pages = {141--149},
  number = {1},
  month = nov,
  __markedentry = {[vijay:5]},
  abstract = {In this short critical review, selected examples of current (within
	the past two years) synthetic efforts toward the construction of
	high-spin molecules are explored, including the use of metal complexes
	containing stable free radical ligands, lanthanide or actinide complexes,
	and other coordination clusters, or a completely different approach,
	taking advantage of non-Heisenberg exchange in fully delocalized
	mixed-valence complexes (spin-dependent delocalization, SDD, or double
	exchange). A description of reported work in this regard is followed
	by a brief general discussion that highlights what the future may
	hold for high-spin molecule design.},
  file = {Full Text PDF:files/1015/Lemaire - 2010 - Progress and design challenges for high-spin molec. pdf:application/pdf;Snapshot:files/1016/pac-con-10-10-20.html:text/html},
  url = {http://www.degruyter.com/view/j/pac.2011.83.issue-1/pac-con-10-10-20/pac-con-10-10-20.xml},
  urldate = {2015-09-01}
}

@ARTICLE{bedurftig_friedel_1998,
  author = {Bed{\"u}rftig, G. and Brendel, B. and Frahm, H. and Noack, R. M.},
  title = {Friedel oscillations in the open {Hubbard} chain},
  journal = {Physical Review B},
  year = {1998},
  volume = {58},
  pages = {10225--10235},
  number = {16},
  month = oct,
  abstract = {Using the density-matrix renormalization-group (DMRG) technique, we
	calculate critical exponents for the one-dimensional Hubbard model
	with open boundary conditions with and without additional boundary
	potentials at both ends. A direct comparison with open boundary condition
	Bethe ansatz calculations provides a good check for the DMRG calculations
	on large system sizes. On the other hand, the DMRG calculations provide
	an independent check of the predictions of conformal field theory,
	which are needed to obtain the critical exponents from the Bethe
	ansatz. From the Bethe ansatz we predict the behavior of the 1/L-corrected
	mean value of the Friedel oscillations (for the density and the magnetization)
	and the characteristic wave vectors, and show numerically that these
	conjectures are fulfilled with and without boundary potentials. The
	quality of the numerical results allows us to determine the behavior
	of the coefficients of the Friedel oscillations as a function of
	the Hubbard interaction.},
  doi = {10.1103/PhysRevB.58.10225},
  file = {APS Snapshot:files/1031/PhysRevB.58.html:text/html},
  url = {http://link.aps.org/doi/10.1103/PhysRevB.58.10225},
  urldate = {2015-09-02}
}

@ARTICLE{schuster_local_2004,
  author = {Schuster, Cosima and Brune, Philipp},
  title = {Local distortions and {Friedel} oscillations in interacting {Fermi}
	chains},
  journal = {physica status solidi (b)},
  year = {2004},
  volume = {241},
  pages = {2043--2054},
  number = {9},
  month = jul,
  abstract = {The interplay between disorder and interaction, especially near metal
	insulator transitions, is a long-standing question. We investigate
	in detail single impurities, in particular, the Friedel oscillations
	induced by them. We study the decay of the Friedel oscillations in
	the one- dimensional Heisenberg and Hubbard model analytically using
	the bosonization technique and numerically using the density matrix
	renormalization group treatment (DMRG). For the Heisenberg chain,
	we confirm the predictions of conformal field theory and bosonization
	for small interaction, but near phase transitions deviations are
	found in form of vanishing or additional oscillations. For the Hubbard
	chain, we study the oscillations ? in the density as well as in the
	magnetization ? in the spin-gap, charge-gap, and Luttinger liquid
	phase. We find an exponential decay or a very slow algebraic decay
	of the oscillations in the gapped phases. In the Luttinger liquid
	phase, we concentrate on the question of logarithmic corrections
	(which occur also in the isotropic Heisenberg antiferromagnet). Differences
	in the behavior near a boundary compared to an impurity are pointed
	out. (© 2004 WILEY-VCH Verlag GmbH \& Co. KGaA, Weinheim)},
  copyright = {Copyright © 2004 WILEY-VCH Verlag GmbH \& Co. KGaA, Weinheim},
  doi = {10.1002/pssb.200404795},
  file = {Full Text PDF:files/1039/Schuster and Brune - 2004 - Local distortions and Friedel            oscillations in inte.pdf:application/pdf;Snapshot:files/1040/abstract.html:text/html},
  issn = {1521-3951},
  keywords = {71.10.Fd, 71.10.Pm},
  language = {en},
  url = {http://onlinelibrary.wiley.com/doi/10.1002/pssb.200404795/abstract},
  urldate = {2015-09-02}
}

@ARTICLE{costamagna_anderson_2006,
  author = {Costamagna, S. and Gazza, C. J. and Torio, M. E. and Riera, J. A.},
  title = {Anderson impurity in the one-dimensional {Hubbard} model for finite-size
	systems},
  journal = {Physical Review B},
  year = {2006},
  volume = {74},
  pages = {195103},
  number = {19},
  month = nov,
  abstract = {An Anderson impurity in a Hubbard model on chains with finite length
	is studied using the density-matrix renormalization-group (DMRG)
	technique. In the first place, we analyzed how the reduction of electron
	density from half filling to quarter filling affects the Kondo resonance
	in the limit of Hubbard repulsion U=0. In general, a weak dependence
	with the electron density was found for the local density of states
	(LDOS), at the impurity except when the impurity, at half filling,
	is close to a mixed- valence regime. Next, in the central part of
	this paper, we studied the effects of finite Hubbard interaction
	on the chain at quarter filling. Our main result is that this interaction
	drives the impurity into a more defined Kondo regime although accompanied
	in most cases by a reduction of the spectral weight of the impurity
	LDOS. Again, for the impurity in the mixed-valence regime, we observed
	an interesting nonmonotonic behavior. We also concluded that the
	conductance, computed for a small finite bias applied to the leads,
	follows the behavior of the impurity LDOS, as in the case of noninteracting
	chains. Finally, we analyzed how the Hubbard interaction and the
	finite chain length affect the spin compensation cloud both at zero
	and at finite temperature, in this case using quantum Monte Carlo
	techniques.},
  doi = {10.1103/PhysRevB.74.195103},
  file = {APS Snapshot:files/1037/PhysRevB.74.html:text/html;Full Text PDF:files/1036/Costamagna et     al. - 2006 - Anderson impurity in the one-dimensional Hubbard m.pdf:application/pdf},
  url = {http://link.aps.org/doi/10.1103/PhysRevB.74.195103},
  urldate = {2015-09-02}
}

@ARTICLE{white_friedel_2002,
  author = {White, Steven R. and Affleck, Ian and Scalapino, Douglas J.},
  title = {Friedel oscillations and charge density waves in chains and ladders},
  journal = {Physical Review B},
  year = {2002},
  volume = {65},
  pages = {165122},
  number = {16},
  month = apr,
  abstract = {The density matrix renormalization (DMRG) group method for ladders
	works much more efficiently with open boundary conditions. One consequence
	of these boundary conditions is ground-state charge density oscillations
	that often appear to be nearly constant in magnitude or to decay
	only slightly away from the boundaries. We analyze these using bosonization
	techniques, relating their detailed form to the correlation exponent
	and distinguishing boundary induced generalized Friedel oscillations
	from true charge density waves. We also discuss a different approach
	to extracting the correlation exponent from the finite size spectrum
	which uses exclusively open boundary conditions and can therefore
	take advantage of data for much larger system sizes. A general discussion
	of the Friedel oscillation wave vectors is given, and a convenient
	Fourier transform technique is used to determine it. DMRG results
	are analyzed on Hubbard and t?J chains and 2 leg t?J ladders. We
	present evidence for the existence of a long-ranged charge density
	wave state in the t?J ladder at a filling of n=0.75 and near J/ t?0.25.},
  doi = {10.1103/PhysRevB.65.165122},
  file = {APS Snapshot:files/1034/PhysRevB.65.html:text/html},
  url = {http://link.aps.org/doi/10.1103/PhysRevB.65.165122},
  urldate = {2015-09-02}
}

@ARTICLE{el_khatib_computing_2014,
  author = {El Khatib, Muammar and Leininger, Thierry and Bendazzoli, Gian Luigi
	and Evangelisti, Stefano},
  title = {Computing the {Position}-{Spread} tensor in the {CAS}-{SCF} formalism},
  journal = {Chemical Physics Letters},
  year = {2014},
  volume = {591},
  pages = {58--63},
  month = jan,
  abstract = {The Total Position Spread (TPS) tensor is a key quantity that describes
	the mobility of the electrons in a molecular system. The computation
	of the TPS tensor has been implemented for CAS-SCF wavefunctions
	in the MOLPRO code. This permits the calculation of this quantity
	for fairly large systems and wavefunctions having a strong multi-reference
	character. In order to illustrate the possibilities of the method,
	we applied the formalism to a mixed-valence Spiro-type system.},
  doi = {10.1016/j.cplett.2013.10.080},
  file = {ScienceDirect Full Text PDF:files/1042/El Khatib et al. - 2014 - Computing the Position-      Spread tensor in the CAS-SC.pdf:application/pdf;ScienceDirect Snapshot:files/1043/S0009261413013602.html: text/html},
  issn = {0009-2614},
  url = {http://www.sciencedirect.com/science/article/pii/S0009261413013602},
  urldate = {2015-09-02}
}

@ARTICLE{khatib_total_2015,
  author = {Khatib, Muammar El and Brea, Oriana and Fertitta, Edoardo and Bendazzoli,
	Gian Luigi and Evangelisti, Stefano and Leininger, Thierry},
  title = {The total position-spread tensor: {Spin} partition},
  journal = {The Journal of Chemical Physics},
  year = {2015},
  volume = {142},
  pages = {094113},
  number = {9},
  month = mar,
  abstract = {The Total Position Spread (TPS) tensor, defined as the second moment
	cumulant of the position operator, is a key quantity to describe
	the mobility of electrons in a molecule or an extended system. In
	the present investigation, the partition of the TPS tensor according
	to spin variables is derived and discussed. It is shown that, while
	the spin-summed TPS gives information on charge mobility, the spin-partitioned
	TPS tensor becomes a powerful tool that provides information about
	spin fluctuations. The case of the hydrogen molecule is treated,
	both analytically, by using a 1s Slater-type orbital, and numerically,
	at Full Configuration Interaction (FCI) level with a V6Z basis set.
	It is found that, for very large inter-nuclear distances, the partitioned
	tensor growths quadratically with the distance in some of the low-lying
	electronic states. This fact is related to the presence of entanglement
	in the wave function. Non-dimerized open chains described by a model
	Hubbard Hamiltonian and linear hydrogen chains H n (n ? 2), composed
	of equally spaced atoms, are also studied at FCI level. The hydrogen
	systems show the presence of marked maxima for the spin-summed TPS
	(corresponding to a high charge mobility) when the inter-nuclear
	distance is about 2 bohrs. This fact can be associated to the presence
	of a Mott transition occurring in this region. The spin-partitioned
	TPS tensor, on the other hand, has a quadratical growth at long distances,
	a fact that corresponds to the high spin mobility in a magnetic system.},
  doi = {10.1063/1.4913734},
  file = {Full Text PDF:files/1045/Khatib et al. - 2015 - The total position-spread tensor Spin         partition.pdf:application/pdf;Snapshot:files/1046/1.html:text/html},
  issn = {0021-9606, 1089-7690},
  shorttitle = {The total position-spread tensor},
  url = {http://scitation.aip.org/content/aip/journal/jcp/142/9/10.1063/1.4913734},
  urldate = {2015-09-02}
}

@ARTICLE{shores_tetracyanide-bridged_2002,
  author = {Shores, Matthew P. and Long, Jeffrey R.},
  title = {Tetracyanide-{Bridged} {Divanadium} {Complexes}:? {Redox} {Switching}
	between {Strong} {Antiferromagnetic} and {Strong} {Ferromagnetic}
	{Coupling}},
  journal = {Journal of the American Chemical Society},
  year = {2002},
  volume = {124},
  pages = {3512--3513},
  number = {14},
  month = apr,
  abstract = {Reaction of [(Me3tacn)V(CF3SO3)3] (Me3tacn = N,N',N''-trimethyl-1,4,7-triazacyclononane)
	with LiCN in DMF results in oligomerization of cyanide to form [(Me3tacn)2V2(CN)4(-C4N4)].
	The structure of this binuclear complex features a planar tetracyanide
	unit bridging two VIV centers via imido type linkages. The conjugated
	pathway provided by the bridging ligand leads to strong antiferromagnetic
	coupling (J = 112 cm-1) and an S = 0 ground state. Reduction of the
	complex with cobaltocene generates the Class III mixed-valence anion
	[(Me3tacn)2V2(CN)4(-C4N4)]1-, wherein resonance exchange induces
	strong ferromagnetic coupling to give a well-isolated S = 3/2 ground
	state.},
  doi = {10.1021/ja025512n},
  file = {ACS Full Text PDF w/ Links:files/1051/Shores and Long - 2002 - Tetracyanide-Bridged           Divanadium Complexes  Redox .pdf:application/pdf;ACS Full Text Snapshot:files/1052/ja025512n.html:text/   html},
  issn = {0002-7863},
  shorttitle = {Tetracyanide-{Bridged} {Divanadium} {Complexes}},
  url = {http://dx.doi.org/10.1021/ja025512n},
  urldate = {2015-09-03}
}

@ARTICLE{vostrikova_high-spin_2008,
  author = {Vostrikova, Kira E.},
  title = {High-spin molecules based on metal complexes of organic free radicals},
  journal = {Coordination Chemistry Reviews},
  year = {2008},
  volume = {252},
  pages = {1409--1419},
  number = {12?14},
  month = jul,
  abstract = {This review presents a survey of the literature dedicated to the design
	of metal complexes of stable free radical ligands that have a ground
	spin state of high multiplicity but excluding extended species. Most
	stable free radicals have a sophisticated chemistry allowing the
	design of multi- site coordination ligands whose metal complexes
	are oligonuclear with a fairly high ground spin state. The versatile
	magnetic behavior of these species associated with the direct bonding
	of metal and organic spin carriers is described. The advantages of
	using organic free radical ligands for building up single- molecule
	magnets (SMM) is discussed.},
  doi = {10.1016/j.ccr.2007.08.024},
  file = {ScienceDirect Full Text PDF:files/1048/Vostrikova - 2008 - High-spin molecules based on metal complexes of or.pdf:application/pdf;ScienceDirect Snapshot:files/1049/S0010854507001853.html:text/html},
  issn = {0010-8545},
  keywords = {Free Radicals, Metal complexes, Paramagnetic ligands, Single-molecule
	magnets},
  url = {http://www.sciencedirect.com/science/article/pii/S0010854507001853},
  urldate = {2015-09-03}
}

@ARTICLE{kubo_note_1982,
  author = {Kubo, Kenn},
  title = {Note on the {Ground} {States} of {Systems} with the {Strong} {Hund}-{Coupling}},
  journal = {Journal of the Physical Society of Japan},
  year = {1982},
  volume = {51},
  pages = {782--786},
  number = {3},
  month = mar,
  abstract = {The effect of the strong Hund coupling on the magnetic property of
	the ground stateis studied based on the double exchange model. For
	one dimensional chains the ground state is proved to have the maximum
	total spin. The effect of the Pauli principle is shown to destroy
	the ferromagnetic ground state by numerical studies of finite loops.
	The result is extended to a model with degenerate bands.},
  doi = {10.1143/JPSJ.51.782},
  file = {Full Text PDF:files/1054/Kubo - 1982 - Note on the Ground States of Systems with the Stro.pdf:application/pdf;Snapshot:files/1055/JPSJ.51.html:text/html},
  issn = {0031-9015},
  url = {http://journals.jps.jp/doi/abs/10.1143/JPSJ.51.782},
  urldate = {2015-09-03}
}

@ARTICLE{bechlars_high-spin_2010,
  author = {Bechlars, Bettina and D'Alessandro, Deanna M. and Jenkins, David
	M. and Iavarone, Anthony T. and Glover, Starla D. and Kubiak, Clifford
	P. and Long, Jeffrey R.},
  title = {High-spin ground states via electron delocalization in mixed-valence
	imidazolate-bridged divanadium complexes},
  journal = {Nature Chemistry},
  year = {2010},
  volume = {2},
  pages = {362--368},
  number = {5},
  month = may,
  __markedentry = {[vijay:3]},
  abstract = {The field of molecular magnetism has grown tremendously since the
	discovery of single- molecule magnets, but it remains centred around
	the superexchange mechanism. The possibility of instead using a double-exchange
	mechanism (based on electron delocalization rather than Heisenberg
	exchange through a non-magnetic bridge) presents a tantalizing prospect
	for synthesizing molecules with high-spin ground states that are
	well isolated in energy. We now demonstrate that magnetic double
	exchange can be sustained by simple imidazolate bridging ligands,
	known to be well suited for the construction of coordination clusters
	and solids. A series of mixed-valence molecules of the type [(PY5Me2)VII(µ-Lbr)
	VIII(PY5Me2)]4+ were synthesized and their electron delocalization
	probed through cyclic voltammetry and spectroelectrochemistry. Magnetic
	susceptibility data reveal a well-isolated S = 5/2 ground state arising
	from double exchange for [(PY5Me2)2V2(µ-5,6-dimethylbenzimidazolate)]4+.
	Combined modelling of the magnetic data and spectral analysis leads
	to an estimate of the double-exchange parameter of B = 220 cm?1 when
	vibronic coupling is taken into account.},
  copyright = {© 2010 Nature Publishing Group},
  doi = {10.1038/nchem.585},
  file = {Full Text PDF:files/1063/Bechlars et al. - 2010 - High-spin ground states via electron        delocalizatio.pdf:application/pdf;Snapshot:files/1064/nchem.585.html:text/html},
  issn = {1755-4330},
  language = {en},
  url = {http://www.nature.com/nchem/journal/v2/n5/full/nchem.585.html},
  urldate = {2015-09-04}
}

@ARTICLE{beltran_directed_2005,
  author = {Beltran, Lianne M. C. and Long, Jeffrey R.},
  title = {Directed {Assembly} of {Metal}?{Cyanide} {Cluster} {Magnets}},
  journal = {Accounts of Chemical Research},
  year = {2005},
  volume = {38},
  pages = {325--334},
  number = {4},
  month = apr,
  __markedentry = {[vijay:3]},
  abstract = {The simple, well-understood coordination chemistry of the cyanide
	ligand is of significant utility in the design of new single-molecule
	magnets. Its preference for bridging two transition metals in a linear
	M??CN?M geometry permits the use of multidentate blocking ligands
	in directing the assembly of specific molecular architectures. This
	approach has been employed in the synthesis of numerous high-nuclearity
	constructs, including simple cubic M4M?4(CN)12 and face-centered
	cubic M8M?6(CN)24 coordination clusters, as well as some unexpected
	cluster geometries featuring as many as 27 metal centers. The ability
	to substitute a range of different transition metal ions into these
	structures enables adjustment of their magnetic properties, facilitating
	creation of high-spin ground states with axial magnetic anisotropy.
	To date, at least four different cyano-bridged single-molecule magnets
	have been characterized, exhibiting spin-reversal barriers as high
	as 25 cm-1. Ultimately, it is envisioned that this strategy might
	lead to molecules possessing much larger barriers with the potential
	for storing information at more practical temperatures.},
  doi = {10.1021/ar040158e},
  file = {ACS Full Text PDF w/ Links:files/1076/Beltran and Long - 2005 - Directed Assembly of          Metal?Cyanide Cluster Magnets.pdf:application/pdf;ACS Full Text Snapshot:files/1077/ar040158e.html:text/  html},
  issn = {0001-4842},
  url = {http://dx.doi.org/10.1021/ar040158e},
  urldate = {2015-09-04}
}

@ARTICLE{dul_redox_2009,
  author = {Dul, Marie-Claire and Pardo, Emilio and Lescouëzec, Rodrigue and
	Chamoreau, Lise-Marie and Villain, Françoise and Journaux, Yves and
	Ruiz-García, Rafael and Cano, Joan and Julve, Miguel and Lloret,
	Francesc and Pasán, Jorge and Ruiz-Pérez, Catalina},
  title = {Redox {Switch}-{Off} of the {Ferromagnetic} {Coupling} in a {Mixed}-{Spin}
	{Tricobalt}({II}) {Triple} {Mesocate}},
  journal = {Journal of the American Chemical Society},
  year = {2009},
  volume = {131},
  pages = {14614--14615},
  number = {41},
  month = oct,
  __markedentry = {[vijay:3]},
  abstract = {A prelude to redox-based, ferromagnetic ?metal?organic switches? is
	exemplified by a new trinuclear oxalamide cobalt triple mesocate
	that presents two redox states (ON and OFF) with dramatically different
	magnetic properties; the two terminal high-spin d7 CoII ions (S =
	3/2) that are ferromagnetically coupled in the homovalent tricobalt(II)
	reduced state (2) become uncoupled in the heterovalent tricobalt(II,III,II)
	oxidized state (2ox) upon one-electron oxidation of the central low-
	spin d7 CoII ion (S = 1/2) to a low-spin d6 CoIII ion (S = 0).},
  doi = {10.1021/ja9052202},
  file = {ACS Full Text PDF w/ Links:files/1066/Dul et al. - 2009 - Redox Switch-Off of the             Ferromagnetic Coupling in .pdf:application/pdf;ACS Full Text Snapshot:files/1067/ja9052202.html:text/html},
  issn = {0002-7863},
  url = {http://dx.doi.org/10.1021/ja9052202},
  urldate = {2015-09-04}
}

@ARTICLE{soncini_molecular_2010,
  author = {Soncini, Alessandro and Mallah, Talal and Chibotaru, Liviu F.},
  title = {Molecular {Spintronics} in {Mixed}-{Valence} {Magnetic} {Dimers}:
	{The} {Double}-{Exchange} {Blockade} {Mechanism}},
  journal = {Journal of the American Chemical Society},
  year = {2010},
  volume = {132},
  pages = {8106--8114},
  number = {23},
  month = jun,
  __markedentry = {[vijay:3]},
  abstract = {We theoretically investigate the charge and spin transport through
	a binuclear FeIIIFeIII iron complex connected to two metallic electrodes.
	During the transport process, the FeIIIFeIII dimer undergoes a one-electron
	reduction (Coulomb blockade transport regime), leading to the reduced
	mixed- valence FeII FeIII species. For such a system, the additional
	electron may be fully delocalized leading to the stabilization of
	the highest spin ground state S = 9/2 by the double exchange mechanism,
	while the original FeIIIFeIII has usually an S = 0 spin ground state
	due to the antiferromagnetic exchange coupling between the two FeIII
	ions. Intuitively, the spin delocalization within the mixed-valence
	complex may be thought to favor charge and spin transport through
	the molecule between the two metallic electrodes. Contrary to such
	an intuitive concept, we find that the increased delocalization leads
	in fact to a blocking of the transport, if the exchange coupling
	interaction within the FeIIIFeIII dimer is antiferromagnetic. This
	is due to the violation of the spin angular momentum conservation,
	where a change of half a unit of spin (?S = 1/2) is allowed between
	two different redox states of the molecule. The result is explained
	in terms of a double-exchange blockade mechanism, triggered by the
	interplay between spin delocalization and antiferromagnetic coupling
	between the magnetic cores. Consequently, ferromagnetically coupled
	dimers are shown not to be affected by the double-exchange blockade
	mechanism. The situation is evocative of the onset and removal of
	giant magnetoresistance in the conductance of diamagnetic layers,
	as a function of the relative alignment of the magnetization of two
	weakly antiferromagnetically coupled ferromagnetic contacts. Numerical
	simulations accounting for the effect of vibronic coupling show that
	the spin current increases as a function of spin delocalization in
	Class I and Class II mixed-valence dimers. The signature of vibronic
	coupling on sequential spin-tunneling processes through Class I and
	Class II mixed-valence systems is identified and discussed.},
  doi = {10.1021/ja101887f},
  file = {ACS Full Text PDF w/ Links:files/1069/Soncini et al. - 2010 - Molecular Spintronics in Mixed- Valence Magnetic Di.pdf:application/pdf;ACS Full Text Snapshot:files/1070/ja101887f.html:text/html},
  issn = {0002-7863},
  shorttitle = {Molecular {Spintronics} in {Mixed}-{Valence} {Magnetic} {Dimers}},
  url = {http://dx.doi.org/10.1021/ja101887f},
  urldate = {2015-09-04}
}

@ARTICLE{zueva_double_2014,
  author = {Zueva, Ekaterina M. and Herchel, Radovan and Borshch, Serguei A.
	and Govor, Evgen V. and Sameera, W. M. C. and McDonald, Ross and
	Singleton, John and Krzystek, Jurek and Trávní?ek, Zden?k and Sanakis,
	Yiannis and McGrady, John E. and Raptis, Raphael G.},
  title = {Double exchange in a mixed-valent octanuclear iron cluster, [{Fe}
	$_{\textrm{8}}$ (? $_{\textrm{4}}$ -{O}) $_{\textrm{4}}$ (?-4-{Cl}-pz)
	$_{\textrm{12}}$ {Cl} $_{\textrm{4}}$ ] $^{\textrm{?}}$},
  journal = {Dalton Trans.},
  year = {2014},
  volume = {43},
  pages = {11269--11276},
  number = {29},
  __markedentry = {[vijay:3]},
  doi = {10.1039/C4DT00020J},
  issn = {1477-9226, 1477-9234},
  language = {en},
  url = {http://xlink.rsc.org/?DOI=C4DT00020J},
  urldate = {2015-09-04}
}

@ARTICLE{girerd_electron_1983,
  author = {Girerd, J.-J.},
  title = {Electron transfer between magnetic ions in mixed valence binuclear
	systems},
  journal = {The Journal of Chemical Physics},
  year = {1983},
  volume = {79},
  pages = {1766--1775},
  number = {4},
  month = aug,
  abstract = {This paper describes how the Hubbard model in the atomic limit implemented
	by taking into account molecular vibrations can give a description
	of mixed valence binuclear systems with both metallic ions simultaneously
	magnetic. A FeIII (high spin) FeII (high spin) binuclear complex
	would constitute an example. In such systems electron transfer and
	electron exchange are expected. If the compound belongs to class
	II (Robin and Day classification) we find that the activation energy
	of the thermal electron transfer and the intensity of the intervalence
	band are spin dependent but that the energies of spin states are
	given by the Heisenberg Hamiltonian with a new expression for the
	exchange parameter. For class III binuclear complexes a quite different
	behavior is found. The energy of the intervalence band is spin dependent
	but energies of spin states are no longer given by the Heisenberg
	Hamiltonian.},
  doi = {10.1063/1.446021},
  file = {Full Text PDF:files/441/Girerd - 1983 - Electron transfer between magnetic ions in mixed v.pdf:application/pdf;Snapshot:files/393/Girerd - 1983 - Electron transfer between magnetic ions in mixed v.html:text/html},
  issn = {0021-9606, 1089-7690},
  keywords = {Activation energies, Electron transfer, Energy transfer, Hubbard model},
  owner = {vijay},
  timestamp = {2015.06.20},
  url = {http://scitation.aip.org/content/aip/journal/jcp/79/4/10.1063/1.446021},
  urldate = {2015-03-18}
}

@BOOK{gatteschi2006molecular,
  title = {Molecular nanomagnets},
  publisher = {Oxford University Press},
  year = {2006},
  author = {Gatteschi, Dante and Sessoli, Roberta and Villain, Jacques}
}

@INCOLLECTION{nelson_x-ray_2001,
  author = {Nelson, C. S. and Zimmermann, M. v and Hill, J. P. and Gibbs, Doon
	and Kiryukhin, V. and Koo, T. Y. and Cheong, S.-W.},
  title = {X-ray {Scattering} {Studies} of {Correlated} {Polarons} in {La}0.7{Ca}0.3{Mn}O3},
  booktitle = {Vibronic {Interactions}: {Jahn}-{Teller} {Effect} in {Crystals} and
	{Molecules}},
  publisher = {Springer Netherlands},
  year = {2001},
  editor = {Kaplan, Michael D. and Zimmerman, George O.},
  number = {39},
  series = {{NATO} {Science} {Series}},
  pages = {209--213},
  copyright = {©2001 Kluwer Academic Publishers},
  file = {Full Text PDF:files/1088/Nelson et al. - 2001 - X-ray Scattering Studies of Correlated        Polarons in.pdf:application/pdf;Snapshot:files/1089/10.html:text/html},
  isbn = {978-1-4020-0045-4 978-94-010-0985-0},
  keywords = {Atomic, Molecular, Optical and Plasma Physics, Biophysics and Biological
	Physics, Condensed Matter Physics},
  language = {en},
  url = {http://link.springer.com/chapter/10.1007/978-94-010-0985-0_22},
  urldate = {2015-09-08}
}

@ARTICLE{tokura_colossal_1999,
  author = {Tokura, Y and Tomioka, Y},
  title = {Colossal magnetoresistive manganites},
  journal = {Journal of Magnetism and Magnetic Materials},
  year = {1999},
  volume = {200},
  pages = {1--23},
  number = {1?3},
  month = oct,
  note = {00813},
  abstract = {Magnetoelectronic features of the perovskite-type manganites are overviewed
	in the light of the mechanism of the colossal magnetoresistance (CMR).
	The essential ingredient of the CMR physics is not only the double-exchange
	interaction but also other competing interactions, such as ferromagnetic/antiferromagnetic
	superexchange interactions and charge/orbital ordering instabilities
	as well as their strong coupling with the lattice deformation. In
	particular, the orbital degree of freedom of the conduction electrons
	in the near-degenerate 3d eg state plays an essential role in producing
	the unconventional metal?insulator phenomena in the manganites via
	strong coupling with spin, charge, and lattice degrees of freedom.
	Insulating or poorly conducting states arise from the long or short-range
	correlations of charge and orbital, but can be mostly melted or turned
	into the orbital-disordered conducting state by application of a
	magnetic field, producing the CMR or the insulator?metal transition.},
  doi = {10.1016/S0304-8853(99)00352-2},
  file = {ScienceDirect Full Text PDF:files/527/Tokura and Tomioka - 1999 - Colossal magnetoresistive manganites.pdf:application/pdf;ScienceDirect Snapshot:files/683/Tokura and Tomioka - 1999 - Colossal magnetoresistive manganites.html:text/html},
  issn = {0304-8853},
  keywords = {Charge ordering, Colossal magnetoresistance, Double-exchange interaction,
	Manganese oxides, Orbital ordering},
  owner = {vijay},
  timestamp = {2015.06.20},
  url = {http://www.sciencedirect.com/science/article/pii/S0304885399003522},
  urldate = {2014-05-24}
}

@ARTICLE{dai2012magnetism,
  author = {Dai, Pengcheng and Hu, Jiangping and Dagotto, Elbio},
  title = {Magnetism and its microscopic origin in iron-based high-temperature
	superconductors},
  journal = {Nature Physics},
  year = {2012},
  volume = {8},
  pages = {709--718},
  number = {10},
  publisher = {Nature Publishing Group}
}

@ARTICLE{shen_preparation_2010,
  author = {Shen, Yongna and Zhao, Hailei and Liu, Xiaotong and Xu, Nansheng},
  title = {Preparation and electrical properties of {Ca}-doped {La}(2){NiO}(4+?)
	cathode materials for {IT}-{SOFC}},
  journal = {Physical chemistry chemical physics: PCCP},
  year = {2010},
  volume = {12},
  pages = {15124--15131},
  number = {45},
  month = dec,
  abstract = {Ca-doped La(2)NiO(4+?) is synthesized via the nitrate-citrate route.
	The effects of Ca substitution for La on the sinterability, lattice
	structure and electrical properties of La(2)NiO(4+?) are investigated.
	Ca-doping is unfavorable for the densification process of La(2-x)Ca(x)NiO(4+?)
	materials. The introduction of Ca leads to the elongation of the
	La-O(2) bond length, which provides more space for the migration
	of oxygen ion in La(2)O(2) rock salt layers. The substitution of
	Ca increases remarkably the electronic conductivity of La(2-x)Ca(x)NiO(4+?).
	With increasing Ca-doping level, both the excess oxygen concentration
	and the activation energy of oxygen ion migration decrease, resulting
	in an optimization where a highest ionic conductivity is presented.
	Ca-doping is charge compensated by the oxidation of Ni(2+) to Ni(3+)
	and the desorption of excess oxygen. The substitution of Ca enhances
	the structural stability of La(2)NiO(4+?) material at high temperatures
	and renders the material a good thermal cycleability. La(1.7)Ca(0.3)NiO(4+?)
	exhibits an excellent chemical compatibility with CGO electrolyte.
	La(2-x)Ca(x)NiO(4+?) is a promising cathode alternative for solid
	oxide fuel cells.},
  doi = {10.1039/c0cp00261e},
  issn = {1463-9084},
  language = {eng},
  owner = {vijay},
  pmid = {20967398},
  timestamp = {2015.06.20}
}

@ARTICLE{dagotto_correlated_1994,
  author = {Dagotto, Elbio},
  title = {Correlated electrons in high-temperature superconductors},
  journal = {Reviews of Modern Physics},
  year = {1994},
  volume = {66},
  pages = {763--840},
  number = {3},
  month = jul,
  abstract = {Theoretical ideas and experimental results concerning high-temperature
	superconductors are reviewed. Special emphasis is given to calculations
	performed with the help of computers applied to models of strongly
	correlated electrons proposed to describe the two-dimensional CuO2
	planes. The review also includes results using several analytical
	techniques. The one- and three-band Hubbard models and the t?J model
	are discussed, and their behavior compared against experiments when
	available. The author found, among the conclusions of the review,
	that some experimentally observed unusual properties of the cuprates
	have a natural explanation through Hubbard-like models. In particular,
	abnormal features like the mid-infrared band of the optical conductivity
	?(?), the new states observed in the gap in photoemission experiments,
	the behavior of the spin correlations with doping, and the presence
	of phase separation in the copper oxide superconductors may be explained,
	at least in part, by these models. Finally, the existence of superconductivity
	in Hubbard-like models is analyzed. Some aspects of the recently
	proposed ideas to describe the cuprates as having a dx2?y2 superconducting
	condensate at low temperatures are discussed. Numerical results favor
	this scenario over others. It is concluded that computational techniques
	provide a useful, unbiased tool for studying the difficult regime
	where electrons are strongly interacting, and that considerable progress
	can be achieved by comparing numerical results against analytical
	predictions for the properties of these models. Future directions
	of the active field of computational studies of correlated electrons
	are briefly discussed.},
  doi = {10.1103/RevModPhys.66.763},
  file = {APS Snapshot:/home/vijay/.zotero/zotero/8zggah0l.default/zotero/storage/JFPW98K4/Dagotto - 1994 - Correlated electrons in high-temperature supercond.html:text/html},
  url = {http://link.aps.org/doi/10.1103/RevModPhys.66.763},
  urldate = {2015-06-20}
}

@ARTICLE{rajca1999very,
  author = {Rajca, Andrzej and Rajca, Suchada and Wongsriratanakul, Jirawat},
  title = {Very high-spin organic polymer: $\pi$-Conjugated hydrocarbon network
	with average spin of S? 40},
  journal = {Journal of the American Chemical Society},
  year = {1999},
  volume = {121},
  pages = {6308--6309},
  number = {26},
  __markedentry = {[vijay:2]},
  publisher = {ACS Publications}
}

@ARTICLE{rajca1993toward,
  author = {Rajca, Andrzej and Utamapanya, Suchada},
  title = {Toward organic synthesis of a magnetic particle: dendritic polyradicals
	with 15 and 31 centers for unpaired electrons},
  journal = {Journal of the American Chemical Society},
  year = {1993},
  volume = {115},
  pages = {10688--10694},
  number = {23},
  __markedentry = {[vijay:2]},
  publisher = {ACS Publications}
}

@ARTICLE{rajca2004organic,
  author = {Rajca, Suchada and Rajca, Andrzej and Wongsriratanakul, Jirawat and
	Butler, Paul and Choi, Sung-min},
  title = {Organic spin clusters. A dendritic-macrocyclic poly (arylmethyl)
	polyradical with very high spin of S= 10 and its derivatives: Synthesis,
	magnetic studies, and small-angle neutron scattering},
  journal = {Journal of the American Chemical Society},
  year = {2004},
  volume = {126},
  pages = {6972--6986},
  number = {22},
  __markedentry = {[vijay:2]},
  publisher = {ACS Publications}
}

@ARTICLE{novoselov_electric_2004,
  author = {Novoselov, K. S. and Geim, A. K. and Morozov, S. V. and Jiang, D.
	and Zhang, Y. and Dubonos, S. V. and Grigorieva, I. V. and Firsov,
	A. A.},
  title = {Electric {Field} {Effect} in {Atomically} {Thin} {Carbon} {Films}},
  journal = {Science},
  year = {2004},
  volume = {306},
  pages = {666--669},
  number = {5696},
  month = oct,
  __markedentry = {[vijay:2]},
  abstract = {We describe monocrystalline graphitic films, which are a few atoms
	thick but are nonetheless stable under ambient conditions, metallic,
	and of remarkably high quality. The films are found to be a two-dimensional
	semimetal with a tiny overlap between valence and conductance bands,
	and they exhibit a strong ambipolar electric field effect such that
	electrons and holes in concentrations up to 1013 per square centimeter
	and with room-temperature mobilities of ?10,000 square centimeters
	per volt-second can be induced by applying gate voltage.},
  doi = {10.1126/science.1102896},
  file = {Full Text PDF:files/1110/Novoselov et al. - 2004 - Electric Field Effect in Atomically Thin      Carbon Fi.pdf:application/pdf;Snapshot:files/1112/666.html:text/html},
  issn = {0036-8075, 1095-9203},
  language = {en},
  pmid = {15499015},
  url = {http://www.sciencemag.org/content/306/5696/666},
  urldate = {2015-09-09}
}

@ARTICLE{novoselov_two-dimensional_2005,
  author = {Novoselov, K. S. and Jiang, D. and Schedin, F. and Booth, T. J. and
	Khotkevich, V. V. and Morozov, S. V. and Geim, A. K.},
  title = {Two-dimensional atomic crystals},
  journal = {Proceedings of the National Academy of Sciences of the United States
	of America},
  year = {2005},
  volume = {102},
  pages = {10451--10453},
  number = {30},
  month = jul,
  __markedentry = {[vijay:2]},
  abstract = {We report free-standing atomic crystals that are strictly 2D and can
	be viewed as individual atomic planes pulled out of bulk crystals
	or as unrolled single-wall nanotubes. By using micromechanical cleavage,
	we have prepared and studied a variety of 2D crystals including single
	layers of boron nitride, graphite, several dichalcogenides, and complex
	oxides. These atomically thin sheets (essentially gigantic 2D molecules
	unprotected from the immediate environment) are stable under ambient
	conditions, exhibit high crystal quality, and are continuous on a
	macroscopic scale.},
  doi = {10.1073/pnas.0502848102},
  file = {Full Text PDF:files/1114/Novoselov et al. - 2005 - Two-dimensional atomic crystals.pdf:          application/pdf;Snapshot:files/1116/10451.html:text/html},
  issn = {0027-8424, 1091-6490},
  keywords = {graphene, layered material},
  language = {en},
  pmid = {16027370},
  url = {http://www.pnas.org/content/102/30/10451},
  urldate = {2015-09-09}
}

@ARTICLE{trinquier_kekule_2015,
  author = {Trinquier, Georges and Malrieu, Jean-Paul},
  title = {Kekulé versus {Lewis}: {When} {Aromaticity} {Prevents} {Electron}
	{Pairing} and {Imposes} {Polyradical} {Character}},
  journal = {Chemistry: A European Journal},
  year = {2015},
  volume = {21},
  pages = {814--828},
  number = {2},
  month = jan,
  abstract = {Some conjugated alternant hydrocarbons, of singlet ground state according
	to Ovchinnikov?s rule, may exhibit strong polyradical character,
	despite admitting complete pairing of electrons in bond orbitals
	between adjacent atoms. Typical organizations of this kind are encountered
	in polycyclic frames supporting two or more extracyclic methylene
	groups. Lewis bond pairing would require quinonization of six-membered
	rings, whereas safeguarding aromaticity proves sufficient to impose
	ground- state open-shell character, that is, the existence of unpaired
	electrons, providing the number of benzene rings to be quinonized
	is larger than two. Several examples built as variations around para-polyphenylene
	frames are examined through unrestricted DFT (UDFT) calculations,
	using various methods for spin decontamination of wavefunctions,
	geometries, and singlet?triplet energy gaps. They all illustrate
	how it is possible to conceive architectures that can be written
	with a closed-shell bond pairing, although they exhibit a large number
	of unpaired electrons. The same analyses also apply to systems in
	which quinonization would not kill but only reduce the number of
	unpaired electrons.},
  copyright = {© 2015 WILEY-VCH Verlag GmbH \& Co. KGaA, Weinheim},
  doi = {10.1002/chem.201403952},
  file = {Full Text PDF:files/1124/Trinquier and Malrieu - 2015 - Kekulé versus Lewis When Aromaticity  Prevents Ele.pdf:application/pdf;Snapshot:files/1125/abstract.html:text/html},
  issn = {1521-3765},
  keywords = {aromaticity, conjugation, density functional calculations, Lewis pairing,
	polyradicals, spin decontamination},
  language = {en},
  shorttitle = {Kekulé versus {Lewis}},
  url = {http://onlinelibrary.wiley.com/doi/10.1002/chem.201403952/abstract},
  urldate = {2015-09-09}
}

@ARTICLE{trinquier_theoretical_2010,
  author = {Trinquier, Georges and Suaud, Nicolas and Malrieu, Jean-Paul},
  title = {Theoretical {Design} of {High}-{Spin} {Polycyclic} {Hydrocarbons}},
  journal = {Chemistry ? A European Journal},
  year = {2010},
  volume = {16},
  pages = {8762--8772},
  number = {29},
  month = aug,
  abstract = {High-spin organic structures can be obtained from fused polycyclic
	hydrocarbons, by converting selected peripheral HC(sp2) sites into
	H2C(sp3) ones, guided by Ovchinnikov?s rule. Theoretical investigation
	is performed on a few examples of such systems, involving three to
	twelve fused rings, and maintaining threefold symmetry. Unrestricted
	DFT (UDFT) calculations, including geometry optimizations, confirm
	the high-spin multiplicity of the ground state. Spin-density distributions
	and low-energy spectra are further studied through geometry-dependent
	Heisenberg?Hamiltonian diagonalizations and explicit correlated ab?initio
	treatments, which all agree on the high-spin character of the suggested
	structures, and locate the low-lying states at significantly higher
	energies. In particular, the lowest- lying state of lower multiplicity
	is always found to be higher than kT at room temperature (at least
	ten times higher). Simplification of the ferromagnetic organization
	based on sets of semilocalized nonbonding orbitals is proposed. Molecular
	architectures are thus conceived in which the ferromagnetically-coupled
	unpaired electrons tally up to one third of the involved conjugated
	carbons. Connecting such building blocks should provide bidimensional
	materials endowed with robust magnetic properties.},
  copyright = {Copyright © 2010 WILEY-VCH Verlag GmbH \& Co. KGaA, Weinheim},
  doi = {10.1002/chem.201000044},
  file = {Full Text PDF:files/1121/Trinquier et al. - 2010 - Theoretical Design of High-Spin Polycyclic Hydroca.pdf:application/pdf;Snapshot:files/1122/abstract.html:text/html},
  issn = {1521-3765},
  keywords = {density functional calculations, high-spin polycyclic hydrocarbons,
	magnetic properties, Ovchinnikov?s rule},
  language = {en},
  url = {http://onlinelibrary.wiley.com/doi/10.1002/chem.201000044/abstract},
  urldate = {2015-09-09}
}

@ARTICLE{perdew_self-interaction_1981,
  author = {Perdew, J. P. and Zunger, Alex},
  title = {Self-interaction correction to density-functional approximations
	for many-electron systems},
  journal = {Physical Review B},
  year = {1981},
  volume = {23},
  pages = {5048--5079},
  number = {10},
  month = may,
  abstract = {The exact density functional for the ground-state energy is strictly
	self-interaction- free (i.e., orbitals demonstrably do not self-interact),
	but many approximations to it, including the local-spin-density (LSD)
	approximation for exchange and correlation, are not. We present two
	related methods for the self-interaction correction (SIC) of any
	density functional for the energy; correction of the self-consistent
	one-electron potenial follows naturally from the variational principle.
	Both methods are sanctioned by the Hohenberg-Kohn theorem. Although
	the first method introduces an orbital-dependent single-particle
	potential, the second involves a local potential as in the Kohn-Sham
	scheme. We apply the first method to LSD and show that it properly
	conserves the number content of the exchange-correlation hole, while
	substantially improving the description of its shape. We apply this
	method to a number of physical problems, where the uncorrected LSD
	approach produces systematic errors. We find systematic improvements,
	qualitative as well as quantitative, from this simple correction.
	Benefits of SIC in atomic calculations include (i) improved values
	for the total energy and for the separate exchange and correlation
	pieces of it, (ii) accurate binding energies of negative ions, which
	are wrongly unstable in LSD, (iii) more accurate electron densities,
	(iv) orbital eigenvalues that closely approximate physical removal
	energies, including relaxation, and (v) correct longrange behavior
	of the potential and density. It appears that SIC can also remedy
	the LSD underestimate of the band gaps in insulators (as shown by
	numerical calculations for the rare-gas solids and CuCl), and the
	LSD overestimate of the cohesive energies of transition metals. The
	LSD spin splitting in atomic Ni and s?d interconfigurational energies
	of transition elements are almost unchanged by SIC. We also discuss
	the admissibility of fractional occupation numbers, and present a
	parametrization of the electron-gas correlation energy at any density,
	based on the recent results of Ceperley and Alder.},
  doi = {10.1103/PhysRevB.23.5048},
  file = {APS Snapshot:files/1131/PhysRevB.23.html:text/html},
  url = {http://link.aps.org/doi/10.1103/PhysRevB.23.5048},
  urldate = {2015-09-09}
}

@comment{jabref-meta: selector_publisher:}

@comment{jabref-meta: psDirectory:/home/vijay/Documents/ups/stage_m1/c
hilkuri/dontorb/dontorb/valencia/thesis/Biblio;}

@comment{jabref-meta: fileDirectory:/home/vijay/Documents/ups/stage_m1
/chilkuri/dontorb/dontorb/valencia/thesis/Biblio;}

@comment{jabref-meta: selector_author:}

@comment{jabref-meta: selector_journal:}

@comment{jabref-meta: selector_keywords:}

@comment{jabref-meta: pdfDirectory:/home/vijay/Documents/ups/stage_m1/
chilkuri/dontorb/dontorb/valencia/thesis/Biblio;}


% This file was created with JabRef 2.7b.
% Encoding: ISO8859_1

@ARTICLE{alvarez_conductivity_2002,
  author = {Alvarez, J. V. and Gros, Claudius},
  title = {Conductivity of quantum spin chains: {A} quantum {Monte} {Carlo}
	approach},
  journal = {Physical Review B},
  year = {2002},
  volume = {66},
  pages = {094403},
  number = {9},
  month = sep,
  note = {00029},
  __markedentry = {[vijay:4]},
  abstract = {We discuss zero-frequency transport properties of various spin-1/2
	chains. We show that a careful analysis of quantum Monte Carlo data
	on the imaginary axis allows to distinguish between intrinsic ballistic
	and diffusive transport. We determine the Drude weight, current-relaxation
	lifetime, and the mean free path for integrable and nonintegrable
	quantum spin chains. We discuss, in addition, some phenomenological
	relations between various transport-coefficients and thermal response
	functions.},
  doi = {10.1103/PhysRevB.66.094403},
  file = {APS Snapshot:files/399/Alvarez and Gros - 2002 - Conductivity of quantum spin chains A quantum Mon.html:text/html;Full Text PDF:files/674/Alvarez and Gros - 2002 - Conductivity of quantum spin chains A quantum Mon.pdf:application/pdf},
  owner = {vijay},
  shorttitle = {Conductivity of quantum spin chains},
  timestamp = {2015.06.20},
  url = {http://link.aps.org/doi/10.1103/PhysRevB.66.094403},
  urldate = {2014-05-30}
}

@ARTICLE{ammon_doped_2001,
  author = {Ammon, Beat and Imada, Masatoshi},
  title = {Doped {Two} {Orbital} {Chains} with {Strong} {Hund}'s {Rule} {Couplings}
	- {Ferromagnetism}, {Spin} {Gap}, {Singlet} and {Triplet} {Pairings}},
  journal = {Journal of the Physical Society of Japan},
  year = {2001},
  volume = {70},
  pages = {547--557},
  number = {2},
  month = feb,
  note = {00008},
  abstract = {Different models for doping of two-orbital chains with mobile S =1/2
	fermions and strong, ferromagnetic (FM) Hund's rule couplings stabilizing
	the S =1 spins are investigated by density matrix renormalization
	group (DMRG) methods. The competition between antiferromagnetic (AF)
	and FM order leads to a rich phase diagram with a narrow FM region
	for weak AF couplings and strongly enhanced triplet pairing correlations.
	Without a level difference between the orbitals, the spin gap persists
	upon doping, whereas gapless spin excitations are generated by interactions
	among itinerant polarons in the presence of a level difference. In
	the charge sector we find dominant singlet pairing correlations without
	a level difference, whereas upon the inclusion of a Coulomb repulsion
	between the orbitals or with a level difference, charge density wave
	(CDW) correlations decay slowest. The string correlation functions
	remain finite upon doping for all models.},
  doi = {10.1143/JPSJ.70.547},
  file = {Full Text PDF:files/581/Ammon and Imada - 2001 - Doped Two Orbital Chains with Strong Hund's Rule C.pdf:application/pdf;Snapshot:files/638/Ammon and Imada - 2001 - Doped Two Orbital Chains with Strong Hund's Rule C.html:text/html},
  issn = {0031-9015},
  owner = {vijay},
  timestamp = {2015.06.20},
  url = {http://journals.jps.jp/doi/abs/10.1143/JPSJ.70.547},
  urldate = {2014-06-06}
}

@ARTICLE{ammon_effect_2000,
  author = {Ammon, Beat and Imada, Masatoshi},
  title = {Effect of the {Orbital} {Level} {Difference} in {Doped} {Spin}-1
	{Chains}},
  journal = {Journal of the Physical Society of Japan},
  year = {2000},
  volume = {69},
  pages = {1946--1949},
  number = {7},
  month = jul,
  note = {00003},
  __markedentry = {[vijay:1]},
  abstract = {The doping of a two-orbital chain with mobile S =1/2 fermions and
	strong Hund's rule couplings stabilizing the S =1 spins strongly
	depends on the presence of a level difference among these orbitals.
	Using density matrix renormalization group (DMRG) methods, we find
	a finite spin gap upon doping and dominant pairing correlations without
	level difference, whereas the presence of a level difference leads
	to dominant charge density wave (CDW) correlations with gapless spin-excitations.
	The string correlation function also shows qualitative differences
	between the two models.},
  doi = {10.1143/JPSJ.69.1946},
  file = {Full Text PDF:files/503/Ammon and Imada - 2000 - Effect of the Orbital Level Difference in Doped Sp.pdf:application/pdf;Snapshot:files/594/Ammon and Imada - 2000 - Effect of the Orbital Level Difference in Doped Sp.html:text/html},
  issn = {0031-9015},
  owner = {vijay},
  timestamp = {2015.06.20},
  url = {http://journals.jps.jp/doi/abs/10.1143/JPSJ.69.1946},
  urldate = {2014-06-20}
}

@ARTICLE{ammon_spin-1_2000,
  author = {Ammon, Beat and Imada, Masatoshi},
  title = {Spin-1 {Chain} {Doped} with {Mobile} {S}=1/2 {Fermions}},
  journal = {Physical Review Letters},
  year = {2000},
  volume = {85},
  pages = {1056--1059},
  number = {5},
  month = jul,
  note = {00010},
  __markedentry = {[vijay:5]},
  abstract = {We investigate the doping of a two-orbital chain with mobile S=1/2
	fermions as a valid model for Y2?xCaxBaNiO5. The S=1 spins are stabilized
	by strong, ferromagnetic Hund's rule couplings. We calculate correlation
	functions and thermodynamic quantities by density matrix renormalization
	group methods and find a new hierarchy of energy scales in the spin
	sector upon doping. Gapless spin excitations are generated at a lower
	energy scale by interactions among itinerant polarons created by
	each hole and coexist with the larger scale of the gapful spin-liquid
	background of the S=1 chain accompanied by a finite string order
	parameter.},
  doi = {10.1103/PhysRevLett.85.1056},
  owner = {vijay},
  timestamp = {2015.06.20},
  url = {http://link.aps.org/doi/10.1103/PhysRevLett.85.1056},
  urldate = {2014-06-06}
}

@ARTICLE{anderson_considerations_1955,
  author = {Anderson, P. W. and Hasegawa, H.},
  title = {Considerations on {Double} {Exchange}},
  journal = {Physical Review},
  year = {1955},
  volume = {100},
  pages = {675--681},
  number = {2},
  month = oct,
  note = {02984},
  abstract = {Zener has suggested a type of interaction between the spins of magnetic
	ions which he named "double exchange." This occurs indirectly by
	means of spin coupling to mobile electrons which travel from one
	ion to the next. We have calculated this interaction for a pair of
	ions with general spin S and with general transfer integral, b, and
	internal exchange integral J.},
  doi = {10.1103/PhysRev.100.675},
  file = {APS Snapshot:files/614/Anderson and Hasegawa - 1955 - Considerations on Double Exchange.html:text/html;Full Text PDF:files/289/Anderson and Hasegawa - 1955 - Considerations on Double Exchange.pdf:application/pdf},
  owner = {vijay},
  timestamp = {2015.06.20},
  url = {http://link.aps.org/doi/10.1103/PhysRev.100.675},
  urldate = {2014-07-18}
}

@ARTICLE{apel_single-hole_2002,
  author = {Apel, W. and Everts, H.-U. and Körner, U.},
  title = {Single-hole dynamics in the t-{J} model},
  journal = {Physical Review B},
  year = {2002},
  volume = {66},
  pages = {174423},
  number = {17},
  month = nov,
  abstract = {The quasiparticle weight of a single hole in an antiferromagnetic
	background is studied in the semiclassical approximation. We start
	from the t?J model, generalize it to arbitrary spin S by employing
	an appropriate coherent state representation for the hole, and derive
	an effective action for the dynamics in the long-wavelength low-energy
	limit. In the same limit, we find an expression for the single-hole
	Green?s function which we evaluate in an 1/S expansion. Our approach
	has the advantage of being applicable in one and in two dimensions.
	We find two qualitatively different results in these two cases: while
	in one dimension our results are compatible with a vanishing quasiparticle
	weight, this weight is found to be finite in two dimensions, indicating
	normal quasiparticle behavior of the hole in this last case.},
  doi = {10.1103/PhysRevB.66.174423},
  file = {APS Snapshot:files/267/Apel et al. - 2002 - Single-hole dynamics in the t-J model.html:text/html;Full Text PDF:files/620/Apel et al. - 2002 - Single-hole dynamics in the t-J model.pdf:application/pdf},
  owner = {vijay},
  timestamp = {2015.06.20},
  url = {http://link.aps.org/doi/10.1103/PhysRevB.66.174423},
  urldate = {2014-11-18}
}

@ARTICLE{arnoldi1951principle,
  author = {Arnoldi, Walter Edwin},
  title = {The principle of minimized iterations in the solution of the matrix
	eigenvalue problem},
  journal = {Quarterly of Applied Mathematics},
  year = {1951},
  volume = {9},
  pages = {17--29},
  number = {1},
  publisher = {AMER MATHEMATICAL SOC 201 CHARLES ST, PROVIDENCE, RI 02940-2213}
}

@ARTICLE{bastardis_ab_2006,
  author = {Bastardis, Roland and Guihéry, Nathalie and de Graaf, Coen},
  title = {Ab initio study of the {Zener} polaron spectrum of half-doped manganites:
	{Comparison} of several model {Hamiltonians}},
  journal = {Physical Review B},
  year = {2006},
  volume = {74},
  pages = {014432},
  number = {1},
  month = jul,
  note = {00010},
  abstract = {The low-energy spectrum of the Zener polaron in half-doped manganite
	is studied by means of correlated ab initio calculations. It is shown
	that the electronic structure of the low-energy states results from
	a subtle interplay between double-exchange configurations and O 2p?
	to Mn 3d charge-transfer configurations that obey a Heisenberg logic.
	The comparison of the calculated spectrum to those predicted by the
	Zener Hamiltonian reveals that this simple description does not correctly
	reproduces the Zener polaron physics. A better reproduction of the
	calculated spectrum is obtained with either a Heisenberg model that
	considers a purely magnetic oxygen or the Girerd-Papaefthymiou double-exchange
	model. An additional significant improvement is obtained when different
	antiferromagnetic contributions are combined with the double-exchange
	model, showing that the Zener polaron spectrum is actually ruled
	by a refined double-exchange mechanism where non-Hund atomic states
	play a non-negligible role. Finally, eight states of a different
	nature have been found to be intercalated in the double-exchange
	spectrum. These states exhibit an O to Mn charge transfer, implying
	a second O 2p orbital of approximate ? character instead of the usual
	? symmetry. A small mixing of the two families of states occurs,
	accounting for the final ordering of the states.},
  doi = {10.1103/PhysRevB.74.014432},
  file = {APS Snapshot:files/501/Bastardis et al. - 2006 - Ab initio study of the Zener polaron spectrum of h.html:text/html;Full Text PDF:files/541/Bastardis et al. - 2006 - Ab initio study of the Zener polaron spectrum of h.pdf:application/pdf},
  owner = {vijay},
  shorttitle = {Ab initio study of the {Zener} polaron spectrum of half-doped manganites},
  timestamp = {2015.06.20},
  url = {http://link.aps.org/doi/10.1103/PhysRevB.74.014432},
  urldate = {2014-07-18}
}

@ARTICLE{bastardis_textitab_2006,
  author = {Bastardis, Roland and Guihéry, Nathalie and de Graaf, Coen},
  title = {{\textbackslash}textit\{{Ab} initio\} study of the {Zener} polaron
	spectrum of half-doped manganites: {Comparison} of several model
	{Hamiltonians}},
  journal = {Physical Review B},
  year = {2006},
  volume = {74},
  pages = {014432},
  number = {1},
  month = jul,
  abstract = {The low-energy spectrum of the Zener polaron in half-doped manganite
	is studied by means of correlated ab initio calculations. It is shown
	that the electronic structure of the low-energy states results from
	a subtle interplay between double-exchange configurations and O 2p?
	to Mn 3d charge-transfer configurations that obey a Heisenberg logic.
	The comparison of the calculated spectrum to those predicted by the
	Zener Hamiltonian reveals that this simple description does not correctly
	reproduces the Zener polaron physics. A better reproduction of the
	calculated spectrum is obtained with either a Heisenberg model that
	considers a purely magnetic oxygen or the Girerd-Papaefthymiou double-exchange
	model. An additional significant improvement is obtained when different
	antiferromagnetic contributions are combined with the double-exchange
	model, showing that the Zener polaron spectrum is actually ruled
	by a refined double-exchange mechanism where non-Hund atomic states
	play a non-negligible role. Finally, eight states of a different
	nature have been found to be intercalated in the double-exchange
	spectrum. These states exhibit an O to Mn charge transfer, implying
	a second O 2p orbital of approximate ? character instead of the usual
	? symmetry. A small mixing of the two families of states occurs,
	accounting for the final ordering of the states.},
  doi = {10.1103/PhysRevB.74.014432},
  owner = {vijay},
  shorttitle = {{\textbackslash}textit\{{Ab} initio\} study of the {Zener} polaron
	spectrum of half-doped manganites},
  timestamp = {2015.06.20},
  url = {http://link.aps.org/doi/10.1103/PhysRevB.74.014432},
  urldate = {2015-03-18}
}

@ARTICLE{bastardis_competition_2006,
  author = {Bastardis, Roland and Guihéry, Nathalie and Suaud, Nicolas and Graaf,
	Coen de},
  title = {Competition between double exchange and purely magnetic {Heisenberg}
	models in mixed valence systems: {Application} to half-doped manganites},
  journal = {The Journal of Chemical Physics},
  year = {2006},
  volume = {125},
  pages = {194708},
  number = {19},
  month = nov,
  abstract = {A truncated Hubbard model is developed for the description of the
	electronic structure of odd-electron TM ? L ? TM units ( TM = transition
	metal and L = ligand ). The model variationally treats both the double
	exchange and purely magnetic Heisenberg configurations. This Hubbard
	model can either be mapped on a purely magnetic Heisenber model in
	which the bridging oxygen is also magnetic or on a double exchange
	model owing to the hybridization of the magnetic and ligand or bitals.
	The purely magnetic Heisenberg model is analytically solved in the
	general case of two metals (having n magnetic orbitals) bridged by
	a magnetic oxygen. The comparison of the analytical expressions of
	the Heisenberg energies to those of the double exchange model reveals
	that the two model spectra are identical except for one state which
	does not belong to the model space of the double exchange Hamiltonian.
	Consequently, the fitting of the model spectra to accurate ab initio
	spectra does not discriminate between the physically different models.
	These concepts are illustrated for the Mn?O?Mn unit (or Zener polaron)
	found in the half-doped manganite Pr 0.6 Ca 0.4 Mn O 3 . It is shown
	that in the present case the projections of the ab initioground statewave
	function onto both model spaces are almost identical provided that
	one uses properly localized orbitals, proving that the magnetic description
	of the Zener polaron and the double exchange viewpoint of the electronic
	structure are equally valid.},
  doi = {10.1063/1.2375119},
  file = {Full Text PDF:files/410/Bastardis et al. - 2006 - Competition between double exchange and purely mag.pdf:application/pdf;Snapshot:files/649/Bastardis et al. - 2006 - Competition between double exchange and purely mag.html:text/html},
  issn = {0021-9606, 1089-7690},
  keywords = {Ab initio calculations, Ground states, Heisenberg model, Hubbard model,
	Wave functions},
  owner = {vijay},
  shorttitle = {Competition between double exchange and purely magnetic {Heisenberg}
	models in mixed valence systems},
  timestamp = {2015.06.20},
  url = {http://scitation.aip.org/content/aip/journal/jcp/125/19/10.1063/1.2375119},
  urldate = {2015-05-06}
}

@ARTICLE{batista_spin_1998,
  author = {Batista, C. D. and Aligia, A. A. and Eroles, J.},
  title = {Spin dynamics of hole-doped {Y}2BaNiO5},
  journal = {EPL (Europhysics Letters)},
  year = {1998},
  volume = {43},
  pages = {71},
  number = {1},
  month = jul,
  note = {00026},
  abstract = {Starting from a multiband Hamiltonian containing the relevant Ni and
	O orbitals, we derive an effective Hamiltonian Heff for the low-energy
	physics of doped Y2BaNiO5. For hole doping, Heff describes O fermions
	interacting with S = 1 Ni spins in a chain, and cannot be further
	reduced to a simple one-band model. Using numerical techniques, we
	obtain a dynamical spin structure factor with weight inside the Haldane
	gap. The nature of these low-energy excitations is identified and
	the emerging physical picture is consistent with most of the experimental
	information on Y2 ? xCaxBaNiO5.},
  doi = {10.1209/epl/i1998-00321-x},
  file = {Full Text PDF:files/390/Batista et al. - 1998 - Spin dynamics of hole-doped Y2BaNiO5.pdf:application/pdf;Snapshot:files/417/Batista et al. - 1998 - Spin dynamics of hole-doped Y2BaNiO5.html:text/html},
  issn = {0295-5075},
  language = {en},
  owner = {vijay},
  timestamp = {2015.06.20},
  url = {http://iopscience.iop.org/0295-5075/43/1/071},
  urldate = {2014-06-06}
}

@ARTICLE{batista_spin_1998-1,
  author = {Batista, C. D. and Aligia, A. A. and Eroles, J.},
  title = {Spin dynamics of hole-doped {Y}2BaNiO5},
  journal = {EPL (Europhysics Letters)},
  year = {1998},
  volume = {43},
  pages = {71},
  number = {1},
  month = jul,
  note = {00026},
  __markedentry = {[vijay:5]},
  abstract = {Starting from a multiband Hamiltonian containing the relevant Ni and
	O orbitals, we derive an effective Hamiltonian Heff for the low-energy
	physics of doped Y2BaNiO5. For hole doping, Heff describes O fermions
	interacting with S = 1 Ni spins in a chain, and cannot be further
	reduced to a simple one-band model. Using numerical techniques, we
	obtain a dynamical spin structure factor with weight inside the Haldane
	gap. The nature of these low-energy excitations is identified and
	the emerging physical picture is consistent with most of the experimental
	information on Y2 ? xCaxBaNiO5.},
  doi = {10.1209/epl/i1998-00321-x},
  file = {Full Text PDF:files/497/Batista et al. - 1998 - Spin dynamics of hole-doped Y2BaNiO5.pdf:application/pdf;Snapshot:files/279/Batista et al. - 1998 - Spin dynamics of hole-doped Y2BaNiO5.html:text/html},
  issn = {0295-5075},
  language = {en},
  owner = {vijay},
  timestamp = {2015.06.20},
  url = {http://iopscience.iop.org/0295-5075/43/1/071},
  urldate = {2014-06-07}
}

@ARTICLE{batista_electron-doped_1998,
  author = {Batista, C. D. and Eroles, J. and Avignon, M. and Alascio, B.},
  title = {Electron-doped manganese perovskites:?{The} magnetic polaron state},
  journal = {Physical Review B},
  year = {1998},
  volume = {58},
  pages = {R14689--R14692},
  number = {22},
  month = dec,
  note = {00033},
  __markedentry = {[vijay:1]},
  abstract = {Using the Lanczos method in linear chains we study the ground state
	of the double exchange model including an antiferromagnetic superexchange
	in the low concentration limit. We find that this ground state is
	always inhomogeneous, containing ferromagnetic polarons. The extension
	of the polaron spin distortion, the dispersion relation and its trapping
	by impurities, are studied for different values of the superexchange
	interaction and magnetic field. We also find repulsive polaron-polaron
	interaction.},
  doi = {10.1103/PhysRevB.58.R14689},
  file = {APS Snapshot:files/474/Batista et al. - 1998 - Electron-doped manganese perovskites The magnetic.html:text/html;Full Text PDF:files/327/Batista et al. - 1998 - Electron-doped manganese perovskites The magnetic.pdf:application/pdf},
  owner = {vijay},
  shorttitle = {Electron-doped manganese perovskites},
  timestamp = {2015.06.20},
  url = {http://link.aps.org/doi/10.1103/PhysRevB.58.R14689},
  urldate = {2014-06-06}
}

@ARTICLE{bauer_alps_2011,
  author = {Bauer, B. and Carr, L. D. and Evertz, H. G. and Feiguin, A. and Freire,
	J. and Fuchs, S. and Gamper, L. and Gukelberger, J. and Gull, E.
	and Guertler, S. and Hehn, A. and Igarashi, R. and Isakov, S. V.
	and Koop, D. and Ma, P. N. and Mates, P. and Matsuo, H. and Parcollet,
	O. and Paw?owski, G. and Picon, J. D. and Pollet, L. and Santos,
	E. and Scarola, V. W. and Schollwöck, U. and Silva, C. and Surer,
	B. and Todo, S. and Trebst, S. and Troyer, M. and Wall, M. L. and
	Werner, P. and Wessel, S.},
  title = {The {ALPS} project release 2.0: open source software for strongly
	correlated systems},
  journal = {Journal of Statistical Mechanics: Theory and Experiment},
  year = {2011},
  volume = {2011},
  pages = {P05001},
  number = {05},
  month = may,
  abstract = {We present release 2.0 of the ALPS (Algorithms and Libraries for Physics
	Simulations) project, an open source software project to develop
	libraries and application programs for the simulation of strongly
	correlated quantum lattice models such as quantum magnets, lattice
	bosons, and strongly correlated fermion systems. The code development
	is centered on common XML and HDF5 data formats, libraries to simplify
	and speed up code development, common evaluation and plotting tools,
	and simulation programs. The programs enable non-experts to start
	carrying out serial or parallel numerical simulations by providing
	basic implementations of the important algorithms for quantum lattice
	models: classical and quantum Monte Carlo (QMC) using non-local updates,
	extended ensemble simulations, exact and full diagonalization (ED),
	the density matrix renormalization group (DMRG) both in a static
	version and a dynamic time-evolving block decimation (TEBD) code,
	and quantum Monte Carlo solvers for dynamical mean field theory (DMFT).
	The ALPS libraries provide a powerful framework for programmers to
	develop their own applications, which, for instance, greatly simplify
	the steps of porting a serial code onto a parallel, distributed memory
	machine. Major changes in release 2.0 include the use of HDF5 for
	binary data, evaluation tools in Python, support for the Windows
	operating system, the use of CMake as build system and binary installation
	packages for Mac OS X and Windows, and integration with the VisTrails
	workflow provenance tool. The software is available from our web
	server at http://alps.comp-phys.org/.},
  doi = {10.1088/1742-5468/2011/05/P05001},
  file = {Full Text PDF:files/733/Bauer et al. - 2011 - The ALPS project release 2.0 open source        software.pdf:application/pdf;Snapshot:files/734/P05001.html:text/html},
  issn = {1742-5468},
  keywords = {classical monte carlo simulations, density matrix renormalization
	group calculations, quantum Monte Carlo simulations, quantum phase
	transitions (theory)},
  language = {en},
  shorttitle = {The {ALPS} project release 2.0},
  url = {http://iopscience.iop.org/1742-5468/2011/05/P05001},
  urldate = {2015-06-25}
}

@ARTICLE{bedurftig_friedel_1998,
  author = {Bedürftig, G. and Brendel, B. and Frahm, H. and Noack, R. M.},
  title = {Friedel oscillations in the open {Hubbard} chain},
  journal = {Physical Review B},
  year = {1998},
  volume = {58},
  pages = {10225--10235},
  number = {16},
  month = oct,
  abstract = {Using the density-matrix renormalization-group (DMRG) technique, we
	calculate critical exponents for the one-dimensional Hubbard model
	with open boundary conditions with and without additional boundary
	potentials at both ends. A direct comparison with open boundary condition
	Bethe ansatz calculations provides a good check for the DMRG calculations
	on large system sizes. On the other hand, the DMRG calculations provide
	an independent check of the predictions of conformal field theory,
	which are needed to obtain the critical exponents from the Bethe
	ansatz. From the Bethe ansatz we predict the behavior of the 1/L-corrected
	mean value of the Friedel oscillations (for the density and the magnetization)
	and the characteristic wave vectors, and show numerically that these
	conjectures are fulfilled with and without boundary potentials. The
	quality of the numerical results allows us to determine the behavior
	of the coefficients of the Friedel oscillations as a function of
	the Hubbard interaction.},
  doi = {10.1103/PhysRevB.58.10225},
  file = {APS Snapshot:files/1031/PhysRevB.58.html:text/html},
  url = {http://link.aps.org/doi/10.1103/PhysRevB.58.10225},
  urldate = {2015-09-02}
}

@ARTICLE{belinicher_two-hole_1997,
  author = {Belinicher, V. I. and Chernyshev, A. L. and Shubin, V. A.},
  title = {Two-hole problem in the \$t\$-\${J}\$ model: {A} canonical transformation
	approach},
  journal = {Physical Review B},
  year = {1997},
  volume = {56},
  pages = {3381--3393},
  number = {6},
  month = aug,
  abstract = {The t-J model in the spinless-fermion representation is studied. An
	effective Hamiltonian for the quasiparticles is derived using a canonical
	transformation approach. It is shown that the rather simple form
	of the transformation generator allows one to take into account the
	effect of hole interactions with the short-range spin waves and to
	describe the single-hole ground state. Obtained results are very
	close to ones of the self-consistent Born approximation. Further
	accounting of the long- range spin-wave interaction is possible on
	a perturbative basis. Spin-wave exchange and an effective interaction
	due to minimization of the number of broken antiferromagnetic bonds
	are included in the effective quasiparticle Hamiltonian. The two-hole
	bound state problem is solved using a Bethe-Salpeter equation. The
	only bound state found to exist in the region of 1{\textless}(t/J){\textless}5
	is the d wave. Both types of the hole-hole interaction are important
	for its formation. A discussion of the possible relation of the obtained
	results to the problem of superconductivity in real systems is presented.},
  doi = {10.1103/PhysRevB.56.3381},
  file = {APS Snapshot:files/839/PhysRevB.56.html:text/html;Full Text PDF:files/838/Belinicher et al. - 1997 - Two-hole problem in the \$t\$-\$J\$ model A canonical.pdf:application/pdf},
  shorttitle = {Two-hole problem in the \$t\$-\${J}\$ model},
  url = {http://link.aps.org/doi/10.1103/PhysRevB.56.3381},
  urldate = {2015-07-08}
}

@ARTICLE{bendazzoli_total_2014,
  author = {Bendazzoli, Gian Luigi and El Khatib, Muammar and Evangelisti, Stefano
	and Leininger, Thierry},
  title = {The total {Position} {Spread} in mixed-valence compounds: {A} study
	on the {H}4+ model system},
  journal = {Journal of Computational Chemistry},
  year = {2014},
  volume = {35},
  pages = {802--808},
  number = {10},
  month = apr,
  abstract = {The behavior of the Total Position Spread (TPS) tensor, which is the
	second moment cumulant of the total position operator, is investigated
	in the case of a mixed-valence model system. The system consists
	of two H2 molecules placed at a distance D. If D is larger than about
	4 bohr, the singly ionized system shows a mixed-valence character.
	It is shown that the magnitude of the TPS has a strong peak in the
	region of the avoided crossing. We believe that the TPS can be a
	powerful tool to characterize the behavior of the electrons in realistic
	mixed-valence compounds. © 2014 Wiley Periodicals, Inc.},
  copyright = {Copyright © 2014 Wiley Periodicals, Inc.},
  doi = {10.1002/jcc.23557},
  file = {Full Text PDF:files/811/Bendazzoli et al. - 2014 - The total Position Spread in mixed-valence    compoun.pdf:application/pdf;Snapshot:files/814/full.html:text/html},
  issn = {1096-987X},
  keywords = {full CI, H4+, Localization Tensor, mixed-valence systems, Total Position
	Spread tensor},
  language = {en},
  shorttitle = {The total {Position} {Spread} in mixed-valence compounds},
  url = {http://onlinelibrary.wiley.com/doi/10.1002/jcc.23557/abstract},
  urldate = {2015-07-27}
}

@INCOLLECTION{bendazzoli_huckel_2004,
  author = {Bendazzoli, Gian Luigi and Evangelisti, Stefano},
  title = {The {Hückel} {Model} of {Polyacetylene} {Revisited}: {Asymptotic}
	{Analysis} of {Peierls} {Instability}},
  booktitle = {Advances in {Quantum} {Chemistry}},
  publisher = {Academic Press},
  year = {2004},
  editor = {E.J. Brandas, {and} E. Brandas},
  volume = {Volume 47},
  series = {A {Tribute} {Volume} in {Honor} of {Professor} {Osvaldo} {Goscinski}},
  pages = {347--368},
  abstract = {We carry out an analysis of the spectrum the Hückel model of dimerized
	polyacetylene, both for cyclic and open chain boundary conditions,
	with special emphasis on the linear polyene with odd number of ?
	bonds. We also perform explicit perturbation expansion up to second
	order in the nuclear displacements. The asymptotic analysis of first-
	and second-order perturbation theory reveals that some behaviours
	assumed in the literature are not correct.},
  file = {ScienceDirect Snapshot:files/563/Bendazzoli et al. - 2004 - The Hückel Model of Polyacetylene    Revisited Asymp.html:text/html},
  isbn = {0065-3276},
  shorttitle = {The {Hückel} {Model} of {Polyacetylene} {Revisited}},
  url = {http://www.sciencedirect.com/science/article/pii/S0065327604470208},
  urldate = {2014-11-05}
}

@ARTICLE{bendazzoli_asymptotic_2012,
  author = {Bendazzoli, Gian Luigi and Evangelisti, Stefano and Monari, Antonio},
  title = {Asymptotic analysis of the localization spread and polarizability
	of 1-{D} noninteracting electrons},
  journal = {International Journal of Quantum Chemistry},
  year = {2012},
  volume = {112},
  pages = {653--664},
  number = {3},
  month = feb,
  abstract = {According to the modern Theory of the Insulating State [Resta, J Chem
	Phys 2006, 124, 104104], the metallic behavior of a N-electron system
	with open boundary conditions is characterized by a localization
	spread ??? diverging in the thermodynamic limit. This quantity, which
	is the second-moment cumulant of the position operator (per electron),
	cannot in general be evaluated in closed form but for simple model
	systems. In this article, we perform an asymptotic analysis of ???
	for a gas of N non- interacting electrons in a 1-Dimensional box
	and a Hückel chain of N equivalent sites. The asymptotic behavior
	of the closely related polarizability tensor is also investigated
	for these exactly solvable models. © 2011 Wiley Periodicals, Inc.
	Int J Quantum Chem, 2011},
  copyright = {Copyright © 2011 Wiley Periodicals, Inc.},
  doi = {10.1002/qua.23036},
  file = {Full Text PDF:files/253/Bendazzoli et al. - 2012 - Asymptotic analysis of the localization       spread and.pdf:application/pdf;Snapshot:files/356/abstract.html:text/html},
  issn = {1097-461X},
  keywords = {1D electron gas, Huckel chain, insulating state, particle in a box},
  language = {en},
  url = {http://onlinelibrary.wiley.com/doi/10.1002/qua.23036/abstract},
  urldate = {2015-07-27}
}

@ARTICLE{bendazzoli_kohns_2010,
  author = {Bendazzoli, Gian Luigi and Evangelisti, Stefano and Monari, Antonio
	and Resta, Raffaele},
  title = {Kohn?s localization in the insulating state: {One}-dimensional lattices,
	crystalline versus disordered},
  journal = {The Journal of Chemical Physics},
  year = {2010},
  volume = {133},
  pages = {064703},
  number = {6},
  month = aug,
  abstract = {The qualitative difference between insulators and metals stems from
	the nature of the low- lying excitations, but also?according to Kohn?s
	theory [W. Kohn, Phys. Rev.133, A171 (1964)]?from a different organization
	of the electrons in their ground state: electrons are localized in
	insulators and delocalized in metals. We adopt a quantitative measure
	of such localization, by means of a ?localization length? ? , finite
	in insulators and divergent in metals. We perform simulations over
	a one-dimensional binary alloy model, in a tight-binding scheme.
	In the ordered case the model is either a band insulator or a band
	metal, whereas in the disordered case it is an Anderson insulator.
	The results show indeed a localized/delocalized ground state in the
	insulating/metallic cases, as expected. More interestingly, we find
	a significant difference between the two insulating cases: band versus
	Anderson. The insulating behavior is due to two very different scattering
	mechanisms; we show that the corresponding values of ? differ by
	a large factor for the same alloy composition. We also investigate
	the organization of the electrons in the many body ground state from
	the viewpoint of the density matrices and of Boys? theory of localization.},
  doi = {10.1063/1.3467877},
  file = {Full Text PDF:files/609/Bendazzoli et al. - 2010 - Kohn?s localization in the insulating state   One-d.pdf:application/pdf;Snapshot:files/620/1.html:text/html},
  issn = {0021-9606, 1089-7690},
  keywords = {Band structure, Boundary value problems, Fermi levels, Ground states,
	Insulators},
  shorttitle = {Kohn?s localization in the insulating state},
  url = {http://scitation.aip.org/content/aip/journal/jcp/133/6/10.1063/1.3467877},
  urldate = {2015-07-27}
}

@ARTICLE{van_den_berg_orbital_2012,
  author = {van den Berg, T. L. and Lombardo, P. and Kuzian, R. O. and Hayn,
	R.},
  title = {Orbital polaron in double-exchange ferromagnets},
  journal = {Physical Review B},
  year = {2012},
  volume = {86},
  pages = {235114},
  number = {23},
  month = dec,
  __markedentry = {[vijay:4]},
  abstract = {We investigate the spectral properties of the two-orbital Hubbard
	model, including the pair hopping term, by means of the dynamical
	mean field method. This Hamiltonian describes materials in which
	ferromagnetism is realized by the double-exchange mechanism, as for
	instance manganites, nickelates, or diluted magnetic semiconductors.
	The spectral function of the unoccupied states is characterized by
	a specific equidistant three peak structure. We emphasize the importance
	of the double hopping term on the spectral properties. We show the
	existence of a ferromagnetic phase due to electron doping near n=1
	by the double-exchange mechanism. A quasiparticle excitation at the
	Fermi energy is found that we attribute to what we will call an orbital
	polaron. We derive an effective spin-pseudospin Hamiltonian for the
	two-orbital double-exchange model at n=1 filling to explain the existence
	and dynamics of this quasiparticle.},
  doi = {10.1103/PhysRevB.86.235114},
  file = {APS Snapshot:files/566/van den Berg et al. - 2012 - Orbital polaron in double-exchange ferromagnets.html:text/html;Full Text PDF:files/302/van den Berg et al. - 2012 - Orbital polaron in double-exchange ferromagnets.pdf:application/pdf},
  owner = {vijay},
  timestamp = {2015.06.20},
  url = {http://link.aps.org/doi/10.1103/PhysRevB.86.235114},
  urldate = {2014-11-18}
}

@ARTICLE{bhattacharyya_generalized_1995,
  author = {Bhattacharyya, B. and Sil, S.},
  title = {The generalized {Hubbard} model in one dimension: a renormalization
	group study},
  journal = {Journal of Physics: Condensed Matter},
  year = {1995},
  volume = {7},
  pages = {6663},
  number = {33},
  month = aug,
  abstract = {We have modified an existing real space renormalization group scheme
	to remove some physically inconsistent features and used the rectified
	method to study the ground state of the one- dimensional Hubbard
	model with nearest-neighbour repulsion and the bond-charge interaction
	for half filling. For the nearest-neighbour interaction we find a
	phase diagram which agrees nicely with the Monte Carlo results. For
	the case of bond-charge interaction we have reproduced some available
	exact results and found some interesting information in the parameter
	regime where no exact solution is available.},
  doi = {10.1088/0953-8984/7/33/006},
  file = {Full Text PDF:files/963/Bhattacharyya and Sil - 1995 - The generalized Hubbard model in one   dimension a .pdf:application/pdf;Snapshot:files/964/006.html:text/html},
  issn = {0953-8984},
  language = {en},
  shorttitle = {The generalized {Hubbard} model in one dimension},
  url = {http://iopscience.iop.org/0953-8984/7/33/006},
  urldate = {2015-08-14}
}

@ARTICLE{bowdler_theqr_1968,
  author = {Bowdler, Hilary and Martin, R. S. and Reinsch, Dr C. and Wilkinson,
	Dr J. H.},
  title = {{TheQR} {andQL} algorithms for symmetric matrices},
  journal = {Numerische Mathematik},
  year = {1968},
  volume = {11},
  pages = {293--306},
  number = {4},
  month = may,
  doi = {10.1007/BF02166681},
  file = {Full Text PDF:files/861/Bowdler et al. - 1968 - TheQR andQL algorithms for symmetric matrices.pdf:application/pdf;Snapshot:files/862/BF02166681.html:text/html},
  issn = {0029-599X, 0945-3245},
  keywords = {Appl.Mathematics/Computational Methods of Engineering, Mathematical
	and Computational Physics, Mathematical Methods in Physics, Mathematics,
	general, Numerical Analysis, Numerical and Computational Methods},
  language = {en},
  url = {http://link.springer.com/article/10.1007/BF02166681},
  urldate = {2015-07-13}
}

@INCOLLECTION{brom_magnetic_2003,
  author = {Brom, Hans B. and Zaanen, Jan},
  title = {Magnetic ordering phenomena and dynamic fluctuations in cuprate superconductors
	and insulating nickelates},
  booktitle = {Handbook of {Magnetic} {Materials}},
  publisher = {Elsevier},
  year = {2003},
  editor = {{K.H.J. Buschow}},
  volume = {Volume 15},
  series = {Handbook of {Magnetic} {Materials}},
  pages = {379--496},
  note = {00010},
  abstract = {This chapter discusses the stripes and resonance peaks of single-
	and multi-layered cuprates and related nickelates. Superconductors
	with surprisingly high superconducting transition temperatures Tc,
	as function of doping, strongly stimulated the interest in the interplay
	between antiferromagnetism and superconductivity. Stripes are associated
	with the highly organized form of electron matter. When a hole moves
	through a dynamical spin system, it causes severe spin frustrations
	that in turn frustrate the free motion of the hole. A single hole
	can only propagate as a quasiparticle because of the presence of
	quantum spin fluctuations repairing the damage in the spin background.
	However, when the density of holes is finite, they can help each
	other repairing the spin damage by concerting their motions. This
	gives rise to the complex ordering phenomena called ?stripes.? The
	formation of stripes may be described with the help of the t?J model.
	When a hole is injected in a 2D S = 1/2 antiferromagnet, it leaves
	behind a string of flipped spins that delocalizes the ?magnetic string.?
	These violate the antiparallel registry of the spins, thereby frustrating
	the delocalization of the hole. At finite hole densities, the holes
	can coordinate their motions in such a way that they help each other
	to avoid the frustrating magnetic strings. An individual hole may
	now move sideways without causing a misoriented spin and by repeating
	these hops kinks propagate along the stripe that cause the stripe
	as a whole to move freely over the plane, unimpeded by the spin system.},
  file = {ScienceDirect Snapshot:files/471/Brom et al. - 2003 - Magnetic ordering phenomena and dynamic fluctuatio.html:text/html},
  isbn = {1567-2719},
  owner = {vijay},
  timestamp = {2015.06.20},
  url = {http://www.sciencedirect.com/science/article/pii/S1567271903150044},
  urldate = {2014-05-29}
}

@ARTICLE{calzado_multi-reference_2003,
  author = {Calzado, Carmen J. and Evangelisti, Stefano and Maynau, Daniel},
  title = {Multi-reference configuration interaction using localized orbitals:
	a test study on {HN}},
  journal = {Journal of Molecular Structure: THEOCHEM},
  year = {2003},
  volume = {621},
  pages = {51--58},
  number = {1?2},
  month = feb,
  abstract = {A method to obtain localized orbitals of CAS-SCF quality has been
	applied to model systems composed of linear Hydrogen chains. Both
	large distance (?magnetic?) and short distance (?metallic?) geometries
	have been investigated. Several possibilities of truncation of the
	complete active space have been tested. The orbitals obtained in
	this way have been used to perform multi-reference configuration
	interaction with single and double excitations (MR-CISD). It is shown
	that the proposed formalism permits a great reduction of the dimensions
	of the problem, without affecting significantly the quality of the
	results.},
  doi = {10.1016/S0166-1280(02)00533-X},
  file = {ScienceDirect Full Text PDF:files/393/Calzado et al. - 2003 - Multi-reference configuration      interaction using lo.pdf:application/pdf;ScienceDirect Snapshot:files/639/Calzado et al. - 2003 - Multi-  reference configuration interaction using lo.html:text/html},
  issn = {0166-1280},
  keywords = {Chains d?Hydrogene, Configuration interaction, Hydrogen chains, Interaction
	de Configurations, Localisation d?Orbitales, Orbital localization},
  series = {2001 {Quitel} {S}.{I}.},
  shorttitle = {Multi-reference configuration interaction using localized orbitals},
  url = {http://www.sciencedirect.com/science/article/pii/S016612800200533X},
  urldate = {2014-11-04}
}

@ARTICLE{costamagna_anderson_2006,
  author = {Costamagna, S. and Gazza, C. J. and Torio, M. E. and Riera, J. A.},
  title = {Anderson impurity in the one-dimensional {Hubbard} model for finite-size
	systems},
  journal = {Physical Review B},
  year = {2006},
  volume = {74},
  pages = {195103},
  number = {19},
  month = nov,
  abstract = {An Anderson impurity in a Hubbard model on chains with finite length
	is studied using the density-matrix renormalization-group (DMRG)
	technique. In the first place, we analyzed how the reduction of electron
	density from half filling to quarter filling affects the Kondo resonance
	in the limit of Hubbard repulsion U=0. In general, a weak dependence
	with the electron density was found for the local density of states
	(LDOS), at the impurity except when the impurity, at half filling,
	is close to a mixed- valence regime. Next, in the central part of
	this paper, we studied the effects of finite Hubbard interaction
	on the chain at quarter filling. Our main result is that this interaction
	drives the impurity into a more defined Kondo regime although accompanied
	in most cases by a reduction of the spectral weight of the impurity
	LDOS. Again, for the impurity in the mixed-valence regime, we observed
	an interesting nonmonotonic behavior. We also concluded that the
	conductance, computed for a small finite bias applied to the leads,
	follows the behavior of the impurity LDOS, as in the case of noninteracting
	chains. Finally, we analyzed how the Hubbard interaction and the
	finite chain length affect the spin compensation cloud both at zero
	and at finite temperature, in this case using quantum Monte Carlo
	techniques.},
  doi = {10.1103/PhysRevB.74.195103},
  file = {APS Snapshot:files/1037/PhysRevB.74.html:text/html;Full Text PDF:files/1036/Costamagna et     al. - 2006 - Anderson impurity in the one-dimensional Hubbard m.pdf:application/pdf},
  url = {http://link.aps.org/doi/10.1103/PhysRevB.74.195103},
  urldate = {2015-09-02}
}

@ARTICLE{costamagna_magnetic_2008,
  author = {Costamagna, S. and Riera, J. A.},
  title = {Magnetic and transport properties of the one-dimensional ferromagnetic
	{Kondo} lattice model with an impurity},
  journal = {Physical Review B},
  year = {2008},
  volume = {77},
  pages = {045302},
  number = {4},
  month = jan,
  note = {00002},
  abstract = {We have studied the ferromagnetic Kondo lattice model (FKLM) with
	an Anderson impurity on finite chains with numerical techniques.
	We are particularly interested in the metallic ferromagnetic phase
	of the FKLM. This model could describe either a quantum dot coupled
	to one-dimensional ferromagnetic leads made with manganites or a
	substitutional transition metal impurity in a MnO chain. We determined
	the region in parameter space where the impurity is empty, half filled,
	or doubly occupied and, hence, where it is magnetic or nonmagnetic.
	The most important result is that we found, for a wide range of impurity
	parameters and electron densities where the impurity is magnetic,
	a singlet phase located between two saturated ferromagnetic phases
	which correspond approximately to the empty and doubly occupied impurity
	states. Transport properties behave, in general, as expected as a
	function of the impurity occupancy, and they provide a test for a
	recently developed numerical approach to compute the conductance.
	The results obtained could be, in principle, reproduced experimentally
	in already existent related nanoscopic devices or in impurity doped
	MnO nanotubes.},
  doi = {10.1103/PhysRevB.77.045302},
  file = {Full Text PDF:files/562/Costamagna and Riera - 2008 - Magnetic and transport properties of the one-dimen.pdf:application/pdf},
  owner = {vijay},
  timestamp = {2015.06.20},
  url = {http://link.aps.org/doi/10.1103/PhysRevB.77.045302},
  urldate = {2014-07-16}
}

@ARTICLE{costamagna_numerical_2008-2,
  author = {Costamagna, S. and Riera, J. A.},
  title = {Numerical study of finite size effects in the one-dimensional two-impurity
	{Anderson} model},
  journal = {Physical Review B},
  year = {2008},
  volume = {77},
  pages = {235103},
  number = {23},
  month = jun,
  note = {00007},
  abstract = {We study the two-impurity Anderson model on finite chains using numerical
	techniques. We discuss the departure of magnetic correlations as
	a function of the interimpurity distance from a pure 2kF oscillation
	due to open boundary conditions. We observe qualitatively different
	behaviors in the interimpurity spin correlations and in transport
	properties at different values of the impurity couplings. We relate
	these different behaviors to a change in the relative dominance between
	the Kondo effect and the Ruderman-Kittel-Kasuya-Yoshida (RKKY) interaction.
	We also observe that when RKKY dominates, there is a definite relation
	between interimpurity magnetic correlations and transport properties.
	In this case, there is a recovery of 2kF periodicity when the on-site
	Coulomb repulsion on the chain is increased at quarter filling. The
	present results could be relevant for electronic nanodevices implementing
	a nonlocal control between two quantum dots that could be located
	at variable distance along a wire.},
  doi = {10.1103/PhysRevB.77.235103},
  file = {APS Snapshot:files/544/Costamagna and Riera - 2008 - Numerical study of finite size effects in the one-.html:text/html;Full Text PDF:files/416/Costamagna and Riera - 2008 - Numerical study of finite size effects in the one-.pdf:application/pdf},
  owner = {vijay},
  timestamp = {2015.06.20},
  url = {http://link.aps.org/doi/10.1103/PhysRevB.77.235103},
  urldate = {2014-07-17}
}

@ARTICLE{dagotto_complexity_2005,
  author = {Dagotto, Elbio},
  title = {Complexity in {Strongly} {Correlated} {Electronic} {Systems}},
  journal = {Science},
  year = {2005},
  volume = {309},
  pages = {257--262},
  number = {5732},
  month = jul,
  abstract = {A wide variety of experimental results and theoretical investigations
	in recent years have convincingly demonstrated that several transition
	metal oxides and other materials have dominant states that are not
	spatially homogeneous. This occurs in cases in which several physical
	interactions?spin, charge, lattice, and/or orbital?are simultaneously
	active. This phenomenon causes interesting effects, such as colossal
	magnetoresistance, and it also appears crucial to understand the
	high-temperature superconductors. The spontaneous emergence of electronic
	nanometer-scale structures in transition metal oxides, and the existence
	of many competing states, are properties often associated with complex
	matter where nonlinearities dominate, such as soft materials and
	biological systems. This electronic complexity could have potential
	consequences for applications of correlated electronic materials,
	because not only charge (semiconducting electronic), or charge and
	spin (spintronics) are of relevance, but in addition the lattice
	and orbital degrees of freedom are active, leading to giant responses
	to small perturbations. Moreover, several metallic and insulating
	phases compete, increasing the potential for novel behavior.},
  doi = {10.1126/science.1107559},
  file = {Full Text PDF:files/440/Dagotto - 2005 - Complexity in Strongly Correlated Electronic Syste.pdf:application/pdf;Snapshot:files/285/Dagotto - 2005 - Complexity in Strongly Correlated Electronic Syste.html:text/html},
  issn = {0036-8075, 1095-9203},
  language = {en},
  owner = {vijay},
  pmid = {16002608},
  timestamp = {2015.06.20},
  url = {http://www.sciencemag.org/content/309/5732/257},
  urldate = {2015-01-30}
}

@ARTICLE{dagotto_open_2005,
  author = {Dagotto, Elbio},
  title = {Open questions in {CMR} manganites, relevance of clustered states
	and analogies with other compounds including the cuprates},
  journal = {New Journal of Physics},
  year = {2005},
  volume = {7},
  pages = {67},
  number = {1},
  month = feb,
  note = {00281},
  abstract = {This is an informal paper that contains a list of 'things we know'
	and 'things we do not know' in manganites and other compounds. It
	is adapted from the Conclusions chapter of a recent book by the author,
	Nanoscale Phase Separation and Colossal Magnetoresistance. The Physics
	of Manganites and Related Compounds (Berlin: Springer-Verlag; 2002),
	but it also contains a summary of some of the most important recent
	results in the field. It is argued that the current main theoretical
	and experimental frameworks to rationalize the results of recent
	manganite investigations are based on the discovery of tendencies
	towards nanoscale inhomogeneous states, both in experiments and in
	simulations of models. The colossal magnetoresistance effect appears
	to be closely linked to these mixed-phase tendencies, although considerably
	more work is needed to fully confirm these ideas. The paper also
	includes information on cuprates, diluted magnetic semiconductors,
	relaxor ferroelectrics, cobaltites and organic and heavy fermion
	superconductors. These materials potentially share some common phenomenology
	with the manganites, such as a temperature scale T* above the ordering
	temperature where anomalous behaviour starts. Many of these materials
	also present low-temperature phase competition. The possibility of
	colossal-like effects in compounds that do not involve ferromagnets
	is briefly discussed. In particular, colossal effects in cuprates
	are explained. Overall, it is concluded that inhomogeneous 'clustered'
	states should be considered as a new paradigm in condensed matter
	physics, since their presence appears to be far more common than
	previously anticipated.},
  doi = {10.1088/1367-2630/7/1/067},
  issn = {1367-2630},
  language = {en},
  owner = {vijay},
  timestamp = {2015.06.20},
  url = {http://iopscience.iop.org/1367-2630/7/1/067},
  urldate = {2014-07-16}
}

@ARTICLE{dagotto_correlated_1994,
  author = {Dagotto, Elbio},
  title = {Correlated electrons in high-temperature superconductors},
  journal = {Reviews of Modern Physics},
  year = {1994},
  volume = {66},
  pages = {763--840},
  number = {3},
  month = jul,
  abstract = {Theoretical ideas and experimental results concerning high-temperature
	superconductors are reviewed. Special emphasis is given to calculations
	performed with the help of computers applied to models of strongly
	correlated electrons proposed to describe the two-dimensional CuO2
	planes. The review also includes results using several analytical
	techniques. The one- and three-band Hubbard models and the t?J model
	are discussed, and their behavior compared against experiments when
	available. The author found, among the conclusions of the review,
	that some experimentally observed unusual properties of the cuprates
	have a natural explanation through Hubbard-like models. In particular,
	abnormal features like the mid-infrared band of the optical conductivity
	?(?), the new states observed in the gap in photoemission experiments,
	the behavior of the spin correlations with doping, and the presence
	of phase separation in the copper oxide superconductors may be explained,
	at least in part, by these models. Finally, the existence of superconductivity
	in Hubbard-like models is analyzed. Some aspects of the recently
	proposed ideas to describe the cuprates as having a dx2?y2 superconducting
	condensate at low temperatures are discussed. Numerical results favor
	this scenario over others. It is concluded that computational techniques
	provide a useful, unbiased tool for studying the difficult regime
	where electrons are strongly interacting, and that considerable progress
	can be achieved by comparing numerical results against analytical
	predictions for the properties of these models. Future directions
	of the active field of computational studies of correlated electrons
	are briefly discussed.},
  doi = {10.1103/RevModPhys.66.763},
  file = {APS Snapshot:/home/vijay/.zotero/zotero/8zggah0l.default/zotero/storage/JFPW98K4/Dagotto - 1994 - Correlated electrons in high-temperature supercond.html:text/html},
  url = {http://link.aps.org/doi/10.1103/RevModPhys.66.763},
  urldate = {2015-06-20}
}

@ARTICLE{dagotto_colossal_2001,
  author = {Dagotto, Elbio and Hotta, Takashi and Moreo, Adriana},
  title = {Colossal magnetoresistant materials: the key role of phase separation},
  journal = {Physics Reports},
  year = {2001},
  volume = {344},
  pages = {1--153},
  number = {1?3},
  month = apr,
  note = {00000},
  abstract = {The study of the manganese oxides, widely known as manganites, that
	exhibit the ?colossal? magnetoresistance effect is among the main
	areas of research within the area of strongly correlated electrons.
	After considerable theoretical effort in recent years, mainly guided
	by computational and mean-field studies of realistic models, considerable
	progress has been achieved in understanding the curious properties
	of these compounds. These recent studies suggest that the ground
	states of manganite models tend to be intrinsically inhomogeneous
	due to the presence of strong tendencies toward phase separation,
	typically involving ferromagnetic metallic and antiferromagnetic
	charge and orbital ordered insulating domains. Calculations of the
	resistivity versus temperature using mixed states lead to a good
	agreement with experiments. The mixed-phase tendencies have two origins:
	(i) electronic phase separation between phases with different densities
	that lead to nanometer scale coexisting clusters, and (ii) disorder-induced
	phase separation with percolative characteristics between equal-density
	phases, driven by disorder near first-order metal?insulator transitions.
	The coexisting clusters in the latter can be as large as a micrometer
	in size. It is argued that a large variety of experiments reviewed
	in detail here contain results compatible with the theoretical predictions.
	The main phenomenology of mixed-phase states appears to be independent
	of the fine details of the model employed, since the microscopic
	origin of the competing phases does not influence the results at
	the phenomenological level. However, it is quite important to clarify
	the electronic properties of the various manganite phases based on
	microscopic Hamiltonians, including strong electron?phonon Jahn?Teller
	and/or Coulomb interactions. Thus, several issues are discussed here
	from the microscopic viewpoint as well, including the phase diagrams
	of manganite models, the stabilization of the charge/orbital/spin
	ordered half-doped correlated electronics (CE)-states, the importance
	of the naively small Heisenberg coupling among localized spins, the
	setup of accurate mean-field approximations, the existence of a new
	temperature scale T? where clusters start forming above the Curie
	temperature, the presence of stripes in the system, and many others.
	However, much work remains to be carried out, and a list of open
	questions is included here. It is also argued that the mixed-phase
	phenomenology of manganites may appear in a large variety of compounds
	as well, including ruthenates, diluted magnetic semiconductors, and
	others. It is concluded that manganites reveal such a wide variety
	of interesting physical phenomena that their detailed study is quite
	important for progress in the field of correlated electrons.},
  doi = {10.1016/S0370-1573(00)00121-6},
  file = {ScienceDirect Full Text PDF:files/269/Dagotto et al. - 2001 - Colossal magnetoresistant materials the key role .pdf:application/pdf;ScienceDirect Snapshot:files/306/Dagotto et al. - 2001 - Colossal magnetoresistant materials the key role .html:text/html},
  issn = {0370-1573},
  keywords = {Colossal magnetoresistance, Computational physics, Inhomogeneities,
	manganites, Phase separation},
  owner = {vijay},
  shorttitle = {Colossal magnetoresistant materials},
  timestamp = {2015.06.20},
  url = {http://www.sciencedirect.com/science/article/pii/S0370157300001216},
  urldate = {2014-05-21}
}

@ARTICLE{dagotto_strongly_1990,
  author = {Dagotto, Elbio and Joynt, Robert and Moreo, Adriana and Bacci, Silvia
	and Gagliano, Eduardo},
  title = {Strongly correlated electronic systems with one hole: {Dynamical}
	properties},
  journal = {Physical Review B},
  year = {1990},
  volume = {41},
  pages = {9049--9073},
  number = {13},
  month = may,
  abstract = {The spectral functions of one hole in the t-J and one-band Hubbard
	models are calculated using exact diagonalization techniques on small
	lattices. Results for the t-Jz model are also presented. For the
	t-J model we found that there is a quasiparticle at the bottom of
	the hole spectrum with an energy well approximated by Eh=-3.17+2.83J0.73
	(for 0.1?J?1.0, t=1) on a 4×4 lattice. The rest of the spectrum is
	not incoherent: We identified at least two other peaks following
	a similar power-law behavior with J. We speculate that the J dependence
	of the results can be explained by a model where the hole is trapped
	in a confining potential as in the Ising limit. The bandwidth of
	the hole is linear in J in the region 0.1?J?0.4 although a power-law
	behavior is not excluded. The spectral weight of the quasiparticle
	grows like J0.5 in the same region. We present new analytical results
	in the large J/t limit to understand the motion of the hole: In perturbation
	theory it can be shown that the momentum of the hole at large J/t
	is k=(?,?) changing to k=(?/2,?/2) at intermediate J/t in agreement
	with numerical and spin- waves results. We show analytically and
	numerically that the bandwidth of the quasiparticle is of order t
	in the large J/t limit. This result corresponds to a spin-liquid
	state. The one-hole spectral function of the Hubbard model is obtained
	for lattices with 8 and 10 sites. A quasiparticle is also observed
	in this case. The bandwidth and the relation with the t-J model are
	discussed and a comparison with recent Monte Carlo results is made.
	We also review and extend previous results for the ground-state properties
	of the t-J model.},
  doi = {10.1103/PhysRevB.41.9049},
  file = {APS Snapshot:files/836/PhysRevB.41.html:text/html;Full Text PDF:files/835/Dagotto et al. -     1990 - Strongly correlated electronic systems with one ho.pdf:application/pdf},
  shorttitle = {Strongly correlated electronic systems with one hole},
  url = {http://link.aps.org/doi/10.1103/PhysRevB.41.9049},
  urldate = {2015-07-08}
}

@ARTICLE{dagotto_static_1992,
  author = {Dagotto, E. and Moreo, A. and Ortolani, F. and Poilblanc, D. and
	Riera, J.},
  title = {Static and dynamical properties of doped {Hubbard} clusters},
  journal = {Physical Review B},
  year = {1992},
  volume = {45},
  pages = {10741--10760},
  number = {18},
  month = may,
  note = {00249},
  abstract = {We study the t-J and the Hubbard models at zero temperature using
	exact-diagonalization techniques on ?10 × ?10 and 4×4 sites clusters.
	Quantum Monte Carlo simulation results on larger lattices are also
	presented. All electronic fillings have been analyzed for the three
	models. We have measured equal-time correlation functions corresponding
	to various types of order (ranging from ??standard?? staggered spin
	order to more ??exotic?? possibilities like chiral order), as well
	as various dynamical properties of these models. Upper bounds for
	the critical hole doping (xc), where long-range antiferromagnetic
	order disappears, are presented. It was found that xc is very small
	in agreement with experiments for the high-Tc superconductors. For
	example, in the t-J model, xc{\textless}0.08 at J/t=0.4. However,
	short-distance spin correlations are important up to much higher
	dopings producing a sharp well-defined spin-wave-like peak in S(q=(?,?),?).
	Regarding the possibility of phase separation in the Hubbard model,
	we have studied the behavior of the density of particles, ?n?, as
	a function of the chemical potential, using the Lanczos method on
	a 4×4 Hubbard cluster, finding no indications of phase separation
	for any value of U/t. Then, we conclude that the t-J model at small
	J/t should not phase separate.},
  doi = {10.1103/PhysRevB.45.10741},
  file = {APS Snapshot:files/484/Dagotto et al. - 1992 - Static and dynamical properties of doped Hubbard c.html:text/html},
  owner = {vijay},
  timestamp = {2015.06.20},
  url = {http://link.aps.org/doi/10.1103/PhysRevB.45.10741},
  urldate = {2014-05-30}
}

@ARTICLE{dagotto_surprises_1996-1,
  author = {Dagotto, Elbio and Rice, T. M.},
  title = {Surprises on the {Way} from {One}- to {Two}-{Dimensional} {Quantum}
	{Magnets}: {The} {Ladder} {Materials}},
  journal = {Science},
  year = {1996},
  volume = {271},
  pages = {618--623},
  number = {5249},
  month = feb,
  abstract = {To make the transition from the quasi-long-range order in a chain
	of antiferromagnetically coupled S = 1/2 spins to the true long-range
	order that occurs in a plane, one can assemble chains to make ladders
	of increasing width. Surprisingly, this crossover between one and
	two dimensions is not at all smooth. Ladders with an even number
	of legs have purely short-range magnetic order and a finite energy
	gap to all magnetic excitations. Predictions of this ground state
	have now been verified experimentally. Holes doped into these ladders
	are predicted to pair and possibly superconduct.},
  doi = {10.1126/science.271.5249.618},
  file = {Snapshot:files/530/Dagotto and Rice - 1996 - Surprises on the Way from One- to Two-Dimensional .html:text/html},
  issn = {0036-8075, 1095-9203},
  language = {en},
  owner = {vijay},
  shorttitle = {Surprises on the {Way} from {One}- to {Two}-{Dimensional} {Quantum}
	{Magnets}},
  timestamp = {2015.06.20},
  url = {http://www.sciencemag.org/content/271/5249/618},
  urldate = {2015-06-09}
}

@ARTICLE{dagotto_spin_1996,
  author = {Dagotto, E. and Riera, J. and Sandvik, A. and Moreo, A.},
  title = {Spin {Dynamics} of {Hole} {Doped} {Y}2-{xCaxBaNiO}5},
  journal = {Physical Review Letters},
  year = {1996},
  volume = {76},
  pages = {1731--1734},
  number = {10},
  month = mar,
  note = {00000},
  abstract = {We propose an electronic model for the recently discovered hole doped
	compound Y2?xCaxBaNiO5. From a multiband Hamiltonian with oxygen
	and nickel orbitals, a one band model is discussed. Holes are described
	using Zhang-Rice-like S=12 states at the nickels propagating on a
	S=1 spin chain. Using numerical techniques to calculate the dynamical
	spin structure factor S(q,?) in a realistic regime of couplings,
	spectral weight in the Haldane gap is observed in agreement with
	neutron scattering data. The case of static defects relevant for
	Zn-doped chains is also discussed. Ferromagnetic states at high hole
	mobility are favored in our model, contrary to what occurs in the
	1D t-J model.},
  doi = {10.1103/PhysRevLett.76.1731},
  owner = {vijay},
  timestamp = {2015.06.20},
  url = {http://link.aps.org/doi/10.1103/PhysRevLett.76.1731},
  urldate = {2014-06-06}
}

@ARTICLE{dagotto_spin_1996-1,
  author = {Dagotto, E. and Riera, J. and Sandvik, A. and Moreo, A.},
  title = {Spin {Dynamics} of {Hole} {Doped} {Y}2-{xCaxBaNiO}5},
  journal = {Physical Review Letters},
  year = {1996},
  volume = {76},
  pages = {1731--1734},
  number = {10},
  month = mar,
  note = {00000},
  __markedentry = {[vijay:1]},
  abstract = {We propose an electronic model for the recently discovered hole doped
	compound Y2?xCaxBaNiO5. From a multiband Hamiltonian with oxygen
	and nickel orbitals, a one band model is discussed. Holes are described
	using Zhang-Rice-like S=12 states at the nickels propagating on a
	S=1 spin chain. Using numerical techniques to calculate the dynamical
	spin structure factor S(q,?) in a realistic regime of couplings,
	spectral weight in the Haldane gap is observed in agreement with
	neutron scattering data. The case of static defects relevant for
	Zn-doped chains is also discussed. Ferromagnetic states at high hole
	mobility are favored in our model, contrary to what occurs in the
	1D t-J model.},
  doi = {10.1103/PhysRevLett.76.1731},
  file = {APS Snapshot:files/580/Dagotto et al. - 1996 - Spin Dynamics of Hole Doped Y2-xCaxBaNiO5.html:text/html;Full Text PDF:files/316/Dagotto et al. - 1996 - Spin Dynamics of Hole Doped Y2-xCaxBaNiO5.pdf:application/pdf},
  owner = {vijay},
  timestamp = {2015.06.20},
  url = {http://link.aps.org/doi/10.1103/PhysRevLett.76.1731},
  urldate = {2014-07-18}
}

@ARTICLE{das_comparison_2004,
  author = {Das, J. and Mahajan, A. V. and Bobroff, J. and Alloul, H. and Alet,
	F. and Sørensen, E. S.},
  title = {Comparison of \${S}=0\$ and \${S}={\textbackslash}frac\{1\}\{2\}\$
	impurities in the {Haldane} chain compound \$\{{\textbackslash}mathrm\{{Y}\}\}\_\{2\}\{{\textbackslash}mathrm\{{BaNiO}\}\}\_\{5\}\$},
  journal = {Physical Review B},
  year = {2004},
  volume = {69},
  pages = {144404},
  number = {14},
  month = apr,
  abstract = {We present the effect of Zn (S=0) and Cu (S=1/2) substitution at the
	Ni site of S=1 Haldane chain compound Y2BaNiO5. 89Y nuclear-magnetic
	resonance (NMR) allows us to measure the local magnetic susceptibility
	at different distances from the defects. The 89Y NMR spectrum consists
	of one central peak and several less intense satellite peaks. The
	central peak represents the chain sites far from the defect. Its
	shift measures the uniform susceptibility, which displays a Haldane
	gap ??100K and it corresponds to an antiferromagnetic (AF) coupling
	J?260K between the nearest neighbor Ni spins. Zn or Cu substitution
	does not affect the Haldane gap. The satellites, which are evenly
	distributed on the two sides of the central peak, probe the antiferromagnetic
	staggered magnetization near the substituted site. The spatial variation
	of the induced magnetization is found to decay exponentially from
	the impurity for both Zn and Cu for T{\textgreater}50K. Its extension
	is found identical for both impurities and corresponds accurately
	to the correlation length ?(T) determined by Monte Carlo simulations
	for the pure compound. In the case of nonmagnetic Zn, the temperature
	dependence of the induced magnetization is consistent with a Curie
	law with an ?effective? spin S=0.4 on each side of Zn. This staggered
	effect is quantitatively well accounted for in all the explored range
	by quantum Monte Carlo (QMC) computations of the spinless-defect-induced
	magnetism. In the case of magnetic Cu, the similarity of the induced
	magnetism to the Zn case implies a weak coupling of the Cu spin to
	the nearest-neighbor Ni spins. The slight reduction of about 20?30\%
	of the induced polarization with respect to Zn is reproduced by QMC
	computations by considering an antiferromagnetic coupling of strength
	J?=0.1J?0.2J between the S=1/2 Cu spin and nearest-neighbor Ni spin.
	Macroscopic susceptibility measurements confirm these results as
	they display a clear Curie contribution due to the impurities nearly
	proportional to their concentration. This contribution is indeed
	close to that of two spin half for Zn substitution. The Curie contribution
	is smaller in the Cu case, which confirms that the coupling between
	Cu and near-neighbor Ni is antiferromagnetic.},
  doi = {10.1103/PhysRevB.69.144404},
  file = {APS Snapshot:files/564/Das et al. - 2004 - Comparison of \$S=0\$ and \$S=frac 1 2 \$ impurities.html:text/html},
  owner = {vijay},
  timestamp = {2015.06.20},
  url = {http://link.aps.org/doi/10.1103/PhysRevB.69.144404},
  urldate = {2015-05-16}
}

@ARTICLE{davidson1975iterative,
  author = {Davidson, Ernest R},
  title = {The iterative calculation of a few of the lowest eigenvalues and
	corresponding eigenvectors of large real-symmetric matrices},
  journal = {Journal of Computational Physics},
  year = {1975},
  volume = {17},
  pages = {87--94},
  number = {1},
  publisher = {Elsevier}
}

@ARTICLE{duan_one-dimensional_2010,
  author = {Duan, Hai-Bao and Ren, Xiao-Ming and Meng, Qing-Jin},
  title = {One-dimensional (1D) [{Ni}(mnt)2]?-based spin-{Peierls}-like complexes:
	{Structural}, magnetic and transition properties},
  journal = {Coordination Chemistry Reviews},
  year = {2010},
  volume = {254},
  pages = {1509--1522},
  number = {13?14},
  month = jul,
  note = {00063},
  __markedentry = {[vijay:4]},
  abstract = {1D spin-Peierls-like complexes assembled from [Ni(mnt)2]? with ?-shaped
	1-(4?-R-benzyl)pyridinium derivatives (R represents a substituent)
	are reviewed, with data on their crystal structures, magnetic properties
	under ambient conditions as well as under pressure, and the nature
	of the paramagnetic-to-nonmagnetic transition. In this series of
	1D spin systems, the correlation between the magnetic exchange and
	the anion stacking pattern is addressed by application of density
	functional theory (DFT) combined with a broken-symmetry approach.
	The qualitative relationship between the transition enthalpy change
	and the variation of the magnetic susceptibility in the low-temperature
	phase is determined. The influence of nonmagnetic doping on the structural
	and magnetic properties and the magnetic transitions are reported.
	Furthermore, the effect of the substituent group in the phenyl ring
	of the cation on the transition temperature and the origin of the
	transition are discussed.},
  doi = {10.1016/j.ccr.2009.12.021},
  file = {ScienceDirect Full Text PDF:files/357/Duan et al. - 2010 - One-dimensional (1D) [Ni(mnt)2]?-based spin-Peierl.pdf:application/pdf;ScienceDirect Snapshot:files/645/Duan et al. - 2010 - One-dimensional (1D) [Ni(mnt)2]?-based spin-Peierl.html:text/html},
  issn = {0010-8545},
  keywords = {Bis(maleonitriledithiolato)nickelate mono-anion, Nonmagnetic doping,
	Pressure effect, Spin-Peierls-like transition, Substituent effect},
  owner = {vijay},
  series = {Dithiolenes and non-innocent redox-active ligands {COST} {Action}
	{D}35 {Conference} 2009},
  shorttitle = {One-dimensional (1D) [{Ni}(mnt)2]?-based spin-{Peierls}-like complexes},
  timestamp = {2015.06.20},
  url = {http://www.sciencedirect.com/science/article/pii/S0010854509003348},
  urldate = {2014-05-31}
}

@INPROCEEDINGS{edmonds1962unitary,
  author = {Edmonds, AR},
  title = {Unitary Symmetry in Theories of Elementary Particles: The Reduction
	of Products of Representations of the Groups U (3) and SU (3)},
  booktitle = {Proceedings of the Royal Society of London A: Mathematical, Physical
	and Engineering Sciences},
  year = {1962},
  volume = {268},
  number = {1335},
  pages = {567--579},
  organization = {The Royal Society}
}

@ARTICLE{el_khatib_computing_2014,
  author = {El Khatib, Muammar and Leininger, Thierry and Bendazzoli, Gian Luigi
	and Evangelisti, Stefano},
  title = {Computing the {Position}-{Spread} tensor in the {CAS}-{SCF} formalism},
  journal = {Chemical Physics Letters},
  year = {2014},
  volume = {591},
  pages = {58--63},
  month = jan,
  abstract = {The Total Position Spread (TPS) tensor is a key quantity that describes
	the mobility of the electrons in a molecular system. The computation
	of the TPS tensor has been implemented for CAS-SCF wavefunctions
	in the MOLPRO code. This permits the calculation of this quantity
	for fairly large systems and wavefunctions having a strong multi-reference
	character. In order to illustrate the possibilities of the method,
	we applied the formalism to a mixed-valence Spiro-type system.},
  doi = {10.1016/j.cplett.2013.10.080},
  file = {ScienceDirect Full Text PDF:files/1042/El Khatib et al. - 2014 - Computing the Position-      Spread tensor in the CAS-SC.pdf:application/pdf;ScienceDirect Snapshot:files/1043/S0009261413013602.html: text/html},
  issn = {0009-2614},
  url = {http://www.sciencedirect.com/science/article/pii/S0009261413013602},
  urldate = {2015-09-02}
}

@ARTICLE{elser_ground_1990,
  author = {Elser, Veit and Huse, David and Shraiman, Boris and Siggia, Eric},
  title = {Ground state of a mobile vacancy in a quantum antiferromagnet: {Small}-cluster
	study},
  journal = {Physical Review B},
  year = {1990},
  volume = {41},
  pages = {6715--6723},
  number = {10},
  month = apr,
  abstract = {The ground state of a mobile vacancy in a square-lattice spin-1/2
	Heisenberg antiferromagnet described by the t-J Hamiltonian is investigated
	numerically for an 18-site cluster. The ground state has spin 1/2,
	and the bottom of the vacancy band is at momentum k=(±?/2,±?/2),
	for t/J{\textless}4, which is consistent with the spin-wave perturbation
	theory. The structure of the spin configuration in the vicinity of
	the vacancy is examined using a number of correlation functions.
	The latter reveal the existence of a dipolar distortion of the staggered
	magnetization, as suggested by the semiclassical theory.},
  doi = {10.1103/PhysRevB.41.6715},
  file = {APS Snapshot:files/628/Elser et al. - 1990 - Ground state of a mobile vacancy in a quantum anti.html:text/html;Full Text PDF:files/633/Elser et al. - 1990 - Ground state of a mobile vacancy in a quantum anti.pdf:application/pdf},
  owner = {vijay},
  shorttitle = {Ground state of a mobile vacancy in a quantum antiferromagnet},
  timestamp = {2015.06.20},
  url = {http://link.aps.org/doi/10.1103/PhysRevB.41.6715},
  urldate = {2014-12-11}
}

@ARTICLE{esquivel_phenomenological_2009,
  author = {Esquivel, Rodolfo O. and Flores-Gallegos, Nelson and Iuga, Cristina
	and Carrera, Edmundo M. and Angulo, Juan Carlos and Antolín, Juan},
  title = {Phenomenological description of the transition state, and the bond
	breaking and bond forming processes of selected elementary chemical
	reactions: an information-theoretic study},
  journal = {Theoretical Chemistry Accounts},
  year = {2009},
  volume = {124},
  pages = {445--460},
  number = {5-6},
  month = nov,
  abstract = {Theoretic-information measures of the Shannon type are employed to
	describe the course of the simplest hydrogen abstraction and the
	identity SN2 exchange chemical reactions. For these elementary chemical
	processes, the transition state is detected and the bond breaking/forming
	regions are revealed. A plausibility argument of the former is provided
	and verified numerically. It is shown that the information entropy
	profiles posses much more chemically meaningful structure than the
	profile of the total energy for these chemical reactions. Our results
	support the concept of a continuum of transient of Zewail and Polanyi
	for the transition state rather than a single state, which is also
	in agreement with reaction force analyses. This is performed by following
	the intrinsic reaction coordinate (IRC) path calculated at the MP2
	level of theory from which Shannon entropies in position and momentum
	spaces at the QCISD(T)/6-311++G(3df,2p) level are determined. Several
	selected descriptors of the density are utilized to support the observations,
	such as the molecular electrostatic potential, the hardness, the
	dipole moment along with geometrical parameters.},
  doi = {10.1007/s00214-009-0641-x},
  file = {Full Text PDF:files/460/Esquivel et al. - 2009 - Phenomenological description of the transition sta.pdf:application/pdf;Snapshot:files/509/Esquivel et al. - 2009 - Phenomenological description of the transition sta.html:text/html},
  issn = {1432-881X, 1432-2234},
  keywords = {Ab initio calculations, Chemical reaction, Information theory, Inorganic
	Chemistry, Organic Chemistry, Physical Chemistry, Reaction mechanisms,
	Theoretical and Computational Chemistry},
  language = {en},
  owner = {vijay},
  shorttitle = {Phenomenological description of the transition state, and the bond
	breaking and bond forming processes of selected elementary chemical
	reactions},
  timestamp = {2015.06.20},
  url = {http://link.springer.com/article/10.1007/s00214-009-0641-x},
  urldate = {2014-10-30}
}

@ARTICLE{frahm_doping_1998,
  author = {Frahm, Holger and Pfannmüller, Markus P. and Tsvelik, A. M.},
  title = {Doping of a {Spin}-1 {Chain}: {An} {Integrable} {Model}},
  journal = {Physical Review Letters},
  year = {1998},
  volume = {81},
  pages = {2116--2119},
  number = {10},
  month = sep,
  note = {00031},
  abstract = {An exactly soluble model describing a spin S=1 antiferromagnetic chain
	doped with mobile S=1/2 carriers is constructed. In its continuum
	limit the undoped state is described by three gapless Majorana fermions
	composing the SU(2) triplet. Doping adds a scalar charge field and
	a singlet Majorana fermion with different velocity. We argue that
	this mode survives when the Haldane gap is added.},
  doi = {10.1103/PhysRevLett.81.2116},
  file = {APS Snapshot:files/549/Frahm et al. - 1998 - Doping of a Spin-1 Chain An Integrable Model.html:text/html;Full Text PDF:files/348/Frahm et al. - 1998 - Doping of a Spin-1 Chain An Integrable Model.pdf:application/pdf},
  owner = {vijay},
  shorttitle = {Doping of a {Spin}-1 {Chain}},
  timestamp = {2015.06.20},
  url = {http://link.aps.org/doi/10.1103/PhysRevLett.81.2116},
  urldate = {2014-06-06}
}

@ARTICLE{fujimoto_weak-coupling_1995,
  author = {Fujimoto, Satoshi and Kawakami, Norio},
  title = {Weak-coupling approach to hole-doped {S}=1 {Haldane} systems},
  journal = {Physical Review B},
  year = {1995},
  volume = {52},
  pages = {6189--6192},
  number = {9},
  month = sep,
  note = {00017},
  abstract = {As a weak-coupling analogue of hole-doped S=1 Haldane systems, we
	study two models for coupled chains via Hund coupling; coupled Hubbard
	chains, and a Hubbard chain coupled with an S=1/2 Heisenberg chain.
	The fixed point properties of these models are investigated by using
	bosonization and renormalization group methods. The effect of randomness
	on these fixed points is also studied. It is found that the presence
	of the disorder parameter inherent in the Haldane state in the former
	model suppresses the Anderson localization for weak randomness, and
	stabilizes the Tomonaga-Luttinger liquid state with the spin gap.},
  doi = {10.1103/PhysRevB.52.6189},
  file = {APS Snapshot:files/637/Fujimoto and Kawakami - 1995 - Weak-coupling approach to hole-doped S=1 Haldane s.html:text/html;Full Text PDF:files/351/Fujimoto and Kawakami - 1995 - Weak-coupling approach to hole-doped S=1 Haldane s.pdf:application/pdf},
  owner = {vijay},
  timestamp = {2015.06.20},
  url = {http://link.aps.org/doi/10.1103/PhysRevB.52.6189},
  urldate = {2014-06-07}
}

@ARTICLE{garcia_charge_2002,
  author = {Garcia, D. J. and Hallberg, K. and Batista, C. D. and Capponi, S.
	and Poilblanc, D. and Avignon, M. and Alascio, B.},
  title = {Charge and spin inhomogeneous phases in the ferromagnetic {Kondo}
	lattice model},
  journal = {Physical Review B},
  year = {2002},
  volume = {65},
  pages = {134444},
  number = {13},
  month = mar,
  __markedentry = {[vijay:1]},
  abstract = {We study numerically the one-dimensional ferromagnetic Kondo lattice.
	This model is widely used to describe nickel and manganese perovskites.
	Due to the competition between double and superexchange, we find
	a region where the formation of magnetic polarons induces a charge-ordered
	state. This ordering is present even in the absence of any intersite
	Coulomb repulsion. There is an insulating gap associated to the charge
	structure formation. We also study the insulator-metal transition
	induced by a magnetic field, which removes simultaneously both charge
	and spin ordering.},
  doi = {10.1103/PhysRevB.65.134444},
  owner = {vijay},
  timestamp = {2015.06.20},
  url = {http://link.aps.org/doi/10.1103/PhysRevB.65.134444},
  urldate = {2015-06-11}
}

@ARTICLE{de_gennes_effects_1960,
  author = {de Gennes, P. -G.},
  title = {Effects of {Double} {Exchange} in {Magnetic} {Crystals}},
  journal = {Physical Review},
  year = {1960},
  volume = {118},
  pages = {141--154},
  number = {1},
  month = apr,
  abstract = {This paper discusses some effects of mobile electrons in some antiferromagnetic
	lattices. It is shown that these electrons (or holes) always give
	rise to a distortion of the ground state spin arrangement, since
	electron transfer lowers the energy by a term of first order in the
	distortion angles. In the most typical cases this results in: (a)
	a nonzero spontaneous moment in low fields; (b) a lack of saturation
	in high fields; (c) simultaneous occurrence of "ferromagnetic" and
	"antiferromagnetic" lines in neutron diffraction patterns; (d) both
	ferromagnetic and antiferromagnetic branches in the spin wave spectra.
	Some of these properties have indeed been observed in compounds of
	mixed valency such as the manganites with low Mn4+ content. Similar
	considerations apply at finite temperatures, at least for the (most
	widespread) case where only the bottom of the carrier band is occupied
	at all temperatures of interest. The free energy is computed by a
	variational procedure, using simple carrier wave functions and an
	extension of the molecular field approximation. It is found that
	the canted arrangements are stable up to a well-defined temperature
	T1. Above T1 the system is either antiferromagnetic or ferromagnetic,
	depending upon the relative amount of mobile electrons. This behavior
	is not qualitatively modified when the carriers which are responsible
	for double exchange fall into bound states around impurity ions of
	opposite charge. Such bound states, however, will give rise to local
	inhomogeneities in the spin distortion, and to diffuse magnetic peaks
	in the neutron diffraction pattern. The possibility of observing
	these peaks and of eliminating the spurious spin-wave scattering
	is discussed in an Appendix.},
  doi = {10.1103/PhysRev.118.141},
  file = {APS Snapshot:files/472/de Gennes - 1960 - Effects of Double Exchange in Magnetic Crystals.html:text/html;Full Text PDF:files/492/de Gennes - 1960 - Effects of Double Exchange in Magnetic Crystals.pdf:application/pdf},
  owner = {vijay},
  timestamp = {2015.06.20},
  url = {http://link.aps.org/doi/10.1103/PhysRev.118.141},
  urldate = {2015-06-12}
}

@ARTICLE{girerd_electron_1983,
  author = {Girerd, J.-J.},
  title = {Electron transfer between magnetic ions in mixed valence binuclear
	systems},
  journal = {The Journal of Chemical Physics},
  year = {1983},
  volume = {79},
  pages = {1766--1775},
  number = {4},
  month = aug,
  abstract = {This paper describes how the Hubbard model in the atomic limit implemented
	by taking into account molecular vibrations can give a description
	of mixed valence binuclear systems with both metallic ions simultaneously
	magnetic. A FeIII (high spin) FeII (high spin) binuclear complex
	would constitute an example. In such systems electron transfer and
	electron exchange are expected. If the compound belongs to class
	II (Robin and Day classification) we find that the activation energy
	of the thermal electron transfer and the intensity of the intervalence
	band are spin dependent but that the energies of spin states are
	given by the Heisenberg Hamiltonian with a new expression for the
	exchange parameter. For class III binuclear complexes a quite different
	behavior is found. The energy of the intervalence band is spin dependent
	but energies of spin states are no longer given by the Heisenberg
	Hamiltonian.},
  doi = {10.1063/1.446021},
  file = {Full Text PDF:files/441/Girerd - 1983 - Electron transfer between magnetic ions in mixed v.pdf:application/pdf;Snapshot:files/393/Girerd - 1983 - Electron transfer between magnetic ions in mixed v.html:text/html},
  issn = {0021-9606, 1089-7690},
  keywords = {Activation energies, Electron transfer, Energy transfer, Hubbard model},
  owner = {vijay},
  timestamp = {2015.06.20},
  url = {http://scitation.aip.org/content/aip/journal/jcp/79/4/10.1063/1.446021},
  urldate = {2015-03-18}
}

@ARTICLE{guihery_double_2006,
  author = {Guihéry, Nathalie},
  title = {The {Double} {Exchange} {Phenomenon} {Revisited}: {The} [{Re}2OCl10]3?
	{Compound}},
  journal = {Theoretical Chemistry Accounts},
  year = {2006},
  volume = {116},
  pages = {576--586},
  number = {4-5},
  month = sep,
  note = {00009},
  abstract = {Correlated ab initio calculations have been performed on the [Re2OCl10]3?
	anion. The calculated spectrum does not respect the intervals given
	by the usually accepted double exchange Hamiltonian. Surprisingly
	enough the ground-state happens to be of intermediate spin multiplicity
	(i.e. a quartet) at any level of correlation treatment. A model that
	combines the Anderson and Hasegawa method and the usually used double
	exchange one rationalizes the spectrum calculated both by a configuration
	interaction restricted to the open shell molecular orbitals and at
	a more correlated level of calculation. An alternative analysis of
	the double exchange phenomenon, based on a molecular orbital language,
	is presented. The specific effects of the electronic correlation
	brought by extended active space and by a difference dedicated configuration
	interaction are also analyzed.},
  doi = {10.1007/s00214-006-0103-7},
  file = {Full Text PDF:files/634/Guihéry - 2006 - The Double Exchange Phenomenon Revisited The [Re2.pdf:application/pdf;Snapshot:files/663/Guihéry - 2006 - The Double Exchange Phenomenon Revisited The [Re2.html:text/html},
  issn = {1432-881X, 1432-2234},
  keywords = {Inorganic Chemistry, Organic Chemistry, Physical Chemistry, Theoretical
	and Computational Chemistry},
  language = {en},
  owner = {vijay},
  shorttitle = {The {Double} {Exchange} {Phenomenon} {Revisited}},
  timestamp = {2015.06.20},
  url = {http://link.springer.com/article/10.1007/s00214-006-0103-7},
  urldate = {2014-06-20}
}

@ARTICLE{guihery_double_2003,
  author = {Guihéry, Nathalie and Malrieu, Jean Paul},
  title = {The double exchange mechanism revisited: {An} ab initio study of
	the [{Ni}2(napy)4Br2]+ complex},
  journal = {The Journal of Chemical Physics},
  year = {2003},
  volume = {119},
  pages = {8956--8965},
  number = {17},
  month = nov,
  note = {00028},
  abstract = {The results of extensive ab initioconfiguration interaction (CI) calculations
	of the spectrum of the [ Ni 2 ( napy ) 4 Br 2 ] + complex are reported
	and analyzed. This complex can be seen as the simplest system exhibiting
	the so-called double-exchange phenomenon. This effect is usually
	rationalized in the minimal valence space defined by the partially
	occupied orbitals. The analysis reveals that the leading antiferromagnetic
	contributions implies the atomic excited (non-Hund) states through
	a mechanism proposed by Anderson et al. and Blondin et al. but the
	energy spacings deviate significantly from those predicted by the
	usually accepted model Hamiltonian. An analytic derivation explains
	these deviations and an alternative modelization of the spectrum
	is proposed. The extensive CI calculations also reveal the existence
	of another mechanism involving low-lying virtual orbitals with a
	large component on the 4s of the Ni atoms and that strongly stabilizes
	the upper excited states.},
  doi = {10.1063/1.1614249},
  file = {Full Text PDF:files/362/Guihéry and Malrieu - 2003 - The double exchange mechanism revisited An ab ini.pdf:application/pdf;Snapshot:files/608/Guihéry and Malrieu - 2003 - The double exchange mechanism revisited An ab ini.html:text/html},
  issn = {0021-9606, 1089-7690},
  keywords = {Ab initio calculations, Antiferromagnetism, Configuration interaction,
	Excited states, Nickel},
  owner = {vijay},
  shorttitle = {The double exchange mechanism revisited},
  timestamp = {2015.06.20},
  url = {http://scitation.aip.org/content/aip/journal/jcp/119/17/10.1063/1.1614249},
  urldate = {2014-07-18}
}

@ARTICLE{gutzwiller_effect_1963,
  author = {Gutzwiller, Martin},
  title = {Effect of {Correlation} on the {Ferromagnetism} of {Transition} {Metals}},
  journal = {Physical Review Letters},
  year = {1963},
  volume = {10},
  pages = {159--162},
  number = {5},
  month = mar,
  abstract = {DOI: http://dx.doi.org/10.1103/PhysRevLett.10.159},
  doi = {10.1103/PhysRevLett.10.159},
  file = {APS Snapshot:files/612/Gutzwiller - 1963 - Effect of Correlation on the Ferromagnetism of Tra.html:text/html;Full Text PDF:files/542/Gutzwiller - 1963 - Effect of Correlation on the Ferromagnetism of Tra.pdf:application/pdf},
  owner = {vijay},
  timestamp = {2015.06.20},
  url = {http://link.aps.org/doi/10.1103/PhysRevLett.10.159},
  urldate = {2014-12-11}
}

@BOOK{helgaker2014molecular,
  title = {Molecular electronic-structure theory},
  publisher = {John Wiley \& Sons},
  year = {2014},
  author = {Helgaker, Trygve and Jorgensen, Poul and Olsen, Jeppe}
}

@ARTICLE{imada_metal-insulator_1995,
  author = {Imada, Masatoshi},
  title = {Metal-{Insulator} {Transition} of {Correlated} {Systems} and {Origin}
	of {Unusual} {Metal}},
  journal = {Journal of the Physical Society of Japan},
  year = {1995},
  volume = {64},
  pages = {2954--2969},
  number = {8},
  month = aug,
  note = {00073},
  abstract = {Transitions between the Mott insulator and metals in clean systems
	are analyzed with the scaling theory in terms of quantum critical
	phenomena. In the case of generic transitions controlled by the filling,
	the scaling theory is established from analyses of the Drude weight
	and the compressibility based on hyperscaling. In the transition
	to the Mott insulator, a new universality class with the correlation
	length exponent ?{\textless}1/2 and the dynamical exponent z {\textgreater}2
	is derived, which is in contrast to the transition to the band insulator
	characterized by ?=1/2 and z =2. The unusual exponents lead to various
	anomalous characters of metals near the Mott insulator such as strong
	suppression of the degeneracy temperature, and nonlinear dependence
	of the Drude weight on the doping concentration. Remarkable properties
	are also found in the specific heat, the compressibility and spin
	correlations. A mechanism of high temperature superconductivity is
	discussed in terms of the release from the suppressed coherence.},
  doi = {10.1143/JPSJ.64.2954},
  file = {Full Text PDF:files/309/Imada - 1995 - Metal-Insulator Transition of Correlated Systems a.pdf:application/pdf;Snapshot:files/488/Imada - 1995 - Metal-Insulator Transition of Correlated Systems a.html:text/html},
  issn = {0031-9015},
  owner = {vijay},
  timestamp = {2015.06.20},
  url = {http://journals.jps.jp/doi/abs/10.1143/JPSJ.64.2954},
  urldate = {2014-05-30}
}

@ARTICLE{izyumov_strongly_1997,
  author = {Izyumov, Yurii A.},
  title = {Strongly correlated electrons: the t-{J} model},
  journal = {Physics-Uspekhi},
  year = {1997},
  volume = {40},
  pages = {445},
  number = {5},
  month = may,
  abstract = {A systematic study is made of the t-J model as a working model for
	copper oxide high-Tc superconductors. The main focus is on the near-half-
	filling region (low hole concentrations) relevant to these materials.
	The theory of the magnetic polaron, which is a charge carrier traveling
	in an antiferromagnetic matrix, and the theory of antiferromagnetic
	ordering are discussed in a unified framework. The spin liquid state
	beyond the antiferromagnetic phase is examined. The Hamiltonian parameters
	? hole concentration phase diagram for the model is described and
	compared with that for the Hubbard model in the strong correlation
	limit. Two extensions of the t-J model, the t-t'-J model and the
	three-center interaction model, are discussed. Mostly analityc strong
	correlation techniques are employed in this review.},
  doi = {10.1070/PU1997v040n05ABEH000234},
  file = {Full Text PDF:files/283/Izyumov - 1997 - Strongly correlated electrons the t-J model.pdf:application/pdf;Snapshot:files/446/Izyumov - 1997 - Strongly correlated electrons the t-J model.html:text/html},
  issn = {1063-7869},
  language = {en},
  owner = {vijay},
  shorttitle = {Strongly correlated electrons},
  timestamp = {2015.06.20},
  url = {http://iopscience.iop.org/1063-7869/40/5/R01},
  urldate = {2014-12-11}
}

@ARTICLE{izyumov_double_2001,
  author = {Izyumov, Yurii A. and Skryabin, Yu N.},
  title = {Double exchange model and the unique properties of the manganites},
  journal = {Physics-Uspekhi},
  year = {2001},
  volume = {44},
  pages = {109},
  number = {2},
  month = feb,
  note = {00105},
  __markedentry = {[vijay:3]},
  abstract = {In this review the double exchange (DE) model forming a basis for
	the description of the physics of colossal magnetoresistance manganites
	is discussed. For a limiting case of exchange interaction which is
	large compared with the band width, the effective Hamiltonian of
	the DE model is derived from that of the sd-exchange model. Since
	this Hamiltonian is very complicated, the dynamical mean field approximation,
	successful for other strongly correlated systems, is found to be
	more suitable for describing the model of interest. Two simplified
	versions of the DE model, both capable of accounting for a wide range
	of physical properties, are proposed ? one using classical localized
	spins and the other involving quantum spins but no transverse spin
	fluctuations. A temperature?electron concentration phase diagram
	for a system with consideration for the domain of phase separation
	is constructed, whose basic features are shown to be in qualitative
	agreement with experimental data for the manganites, as also are
	the temperature and electron concentration dependences of their electrical
	resistivity, magnetization, and spectral characteristics. At the
	quantitative level, introducing additional electron?lattice interaction
	yields a good agreement. A number of yet unresolved problems in the
	physics of manganites, including the mechanism of temperature- or
	doping-induced metal?insulator phase transition and the nature of
	charge ordering, are also discussed. By comparing predictions made
	by computing approach with the experimental data, the adequacy of
	the DE model is assessed and its drawbacks are analyzed. Numerous
	recent theoretical studies of the unique properties of this broad
	class of strongly correlated systems are summarized in this review.},
  doi = {10.1070/PU2001v044n02ABEH000840},
  file = {Full Text PDF:files/677/Izyumov and Skryabin - 2001 - Double exchange model and the unique properties of.pdf:application/pdf;Snapshot:files/430/Izyumov and Skryabin - 2001 - Double exchange model and the unique properties of.html:text/html},
  issn = {1063-7869},
  language = {en},
  owner = {vijay},
  timestamp = {2015.06.20},
  url = {http://iopscience.iop.org/1063-7869/44/2/R01},
  urldate = {2014-05-20}
}

@ARTICLE{jin_effect_2005,
  author = {Jin, Fengping and Xu, Zhaoxin and Ying, Heping and Zheng, Bo},
  title = {On the effect of a regular {S} = 1 dilution of {S} = 1/2 antiferromagnetic
	{Heisenberg} chains obtained from quantum {Monte} {Carlo} simulations},
  journal = {Journal of Physics: Condensed Matter},
  year = {2005},
  volume = {17},
  pages = {5541},
  number = {36},
  month = sep,
  note = {00000},
  __markedentry = {[vijay:4]},
  abstract = {The effect of an S1 = 1 regular dilution in an S2 = 1/2 isotropic
	antiferromagnetic chain is investigated with quantum Monte Carlo
	simulations. Our numerical results show that there exist two kinds
	of ground state phases with different variations of the S1 = 1 concentration.
	When the effective spin in a unit cell is half-integer, the ground
	state is ferromagnetic with a gapless energy spectrum, and the magnetism
	is continuously weakened as the spin S1 concentration ? decreases.
	When the effective spin in a unit cell is integer, however, a non-magnetic
	ground state with a gapped energy spectrum emerges, and the gap decays
	gradually, with .},
  doi = {10.1088/0953-8984/17/36/010},
  file = {Full Text PDF:files/668/Jin et al. - 2005 - On the effect of a regular S = 1 dilution of S = 1.pdf:application/pdf;Snapshot:files/669/010.html:text/html},
  issn = {0953-8984},
  language = {en},
  owner = {vijay},
  timestamp = {2015.06.20},
  url = {http://iopscience.iop.org/0953-8984/17/36/010},
  urldate = {2014-06-09}
}

@ARTICLE{jonker_ferromagnetic_1950,
  author = {Jonker, G. H. and Van Santen, J. H.},
  title = {Ferromagnetic compounds of manganese with perovskite structure},
  journal = {Physica},
  year = {1950},
  volume = {16},
  pages = {337--349},
  number = {3},
  month = mar,
  note = {01768},
  abstract = {Various manganites of the general formula La3+Mn3+O32?-Me2+Mn4+O32?
	have been prepared in the form of polycrystalline products. Perovskite
	structures were found, i.a. for all mixed crystals LaMnO3?CaMnO3,
	for LaMnO3?SrMnO3 containing up to 70\% SrMnO3, and for LaMnO3?BaMnO3
	containing less than 50\% BaMnO3. The mixed crystals with perovskite
	structure are ferromagnetic. Curves for the Curie temperature versus
	composition and saturation versus composition are given for LaMnO3?CaMnO3,
	LaMnO3?SrMnO3, and LaMnO3?BaMnO3. Both types of curves show maxima
	between 25 and 40\% Me2+Mn4+O32?; here all 3d electrons available
	contribute with their spins to the saturation magnetization. The
	ferromagnetic properties can be understood as the result of a strong
	positive Mn3+?Mn4+ exchange interaction combined with a weak Mn3+?Mn3+
	interaction and a negative Mn4+?Mn4+ interaction. The Mn3+?Mn4+ interaction,
	presumably of the indirect exchange type, is thought to be the first
	clear example of positive exchange interaction in oxidic substances.},
  doi = {10.1016/0031-8914(50)90033-4},
  file = {ScienceDirect Full Text PDF:files/606/Jonker and Van Santen - 1950 - Ferromagnetic compounds of manganese with perovski.pdf:application/pdf;ScienceDirect Snapshot:files/284/Jonker and Van Santen - 1950 - Ferromagnetic compounds of manganese with perovski.html:text/html},
  issn = {0031-8914},
  owner = {vijay},
  timestamp = {2015.06.20},
  url = {http://www.sciencedirect.com/science/article/pii/0031891450900334},
  urldate = {2014-07-18}
}

@ARTICLE{kane_motion_1989,
  author = {Kane, C. L. and Lee, P. A. and Read, N.},
  title = {Motion of a single hole in a quantum antiferromagnet},
  journal = {Physical Review B},
  year = {1989},
  volume = {39},
  pages = {6880--6897},
  number = {10},
  month = apr,
  abstract = {We formulate a quasiparticle theory for a single hole in a quantum
	antiferromagnet in the limit that the Heisenberg exchange energy
	is much less than the hopping matrix element, J?t. We consider the
	ground state of the spins to be either a quantum Néel state or a
	d-wave resonating-valence-bond (RVB) state. We show in a self-consistent
	perturbation theory that the hole spectrum is strongly renormalized
	by the interactions with spin excitations. The hole can be described
	by a narrow quasiparticle band located at an energy of order -t with
	a quasiparticle residue of order J/t and a bandwidth of order J.
	Above the quasiparticle band is an incoherent band of width of order
	t. Our results indicate that the energy scale for any coherent phenomenon
	involving the holes is ?J, where ? is the doping concentration. In
	the Néel state we perform a spin-wave expansion on an anisotropic
	Heisenberg model. In the Ising limit we reproduce previously known
	results and then expand perturbatively about that limit. In this
	expansion we find that the holes have a quasiparticle residue of
	Jz/t and a bandwidth of J?. In the Heisenberg limit we employ a ??dominant
	pole?? approximation in which we ignore contributions to the self-energy
	from the incoherent part of the hole spectrum. A similar technique
	is used to study the d-wave RVB state. The relevance of our results
	to recent optical experiments is discussed.},
  doi = {10.1103/PhysRevB.39.6880},
  file = {APS Snapshot:files/350/Kane et al. - 1989 - Motion of a single hole in a quantum antiferromagn.html:text/html;Full Text PDF:files/322/Kane et al. - 1989 - Motion of a single hole in a quantum antiferromagn.pdf:application/pdf},
  owner = {vijay},
  timestamp = {2015.06.20},
  url = {http://link.aps.org/doi/10.1103/PhysRevB.39.6880},
  urldate = {2014-11-18}
}

@ARTICLE{khatib_total_2015,
  author = {Khatib, Muammar El and Brea, Oriana and Fertitta, Edoardo and Bendazzoli,
	Gian Luigi and Evangelisti, Stefano and Leininger, Thierry},
  title = {The total position-spread tensor: {Spin} partition},
  journal = {The Journal of Chemical Physics},
  year = {2015},
  volume = {142},
  pages = {094113},
  number = {9},
  month = mar,
  abstract = {The Total Position Spread (TPS) tensor, defined as the second moment
	cumulant of the position operator, is a key quantity to describe
	the mobility of electrons in a molecule or an extended system. In
	the present investigation, the partition of the TPS tensor according
	to spin variables is derived and discussed. It is shown that, while
	the spin-summed TPS gives information on charge mobility, the spin-partitioned
	TPS tensor becomes a powerful tool that provides information about
	spin fluctuations. The case of the hydrogen molecule is treated,
	both analytically, by using a 1s Slater-type orbital, and numerically,
	at Full Configuration Interaction (FCI) level with a V6Z basis set.
	It is found that, for very large inter-nuclear distances, the partitioned
	tensor growths quadratically with the distance in some of the low-lying
	electronic states. This fact is related to the presence of entanglement
	in the wave function. Non-dimerized open chains described by a model
	Hubbard Hamiltonian and linear hydrogen chains H n (n ? 2), composed
	of equally spaced atoms, are also studied at FCI level. The hydrogen
	systems show the presence of marked maxima for the spin-summed TPS
	(corresponding to a high charge mobility) when the inter-nuclear
	distance is about 2 bohrs. This fact can be associated to the presence
	of a Mott transition occurring in this region. The spin-partitioned
	TPS tensor, on the other hand, has a quadratical growth at long distances,
	a fact that corresponds to the high spin mobility in a magnetic system.},
  doi = {10.1063/1.4913734},
  file = {Full Text PDF:files/1045/Khatib et al. - 2015 - The total position-spread tensor Spin         partition.pdf:application/pdf;Snapshot:files/1046/1.html:text/html},
  issn = {0021-9606, 1089-7690},
  shorttitle = {The total position-spread tensor},
  url = {http://scitation.aip.org/content/aip/journal/jcp/142/9/10.1063/1.4913734},
  urldate = {2015-09-02}
}

@ARTICLE{koga_hole-doping_2002,
  author = {Koga, Akihisa and Kawakami, Norio and Sigrist, Manfred},
  title = {Hole-doping effects on an {S}=1 ladder system},
  journal = {Physica B: Condensed Matter},
  year = {2002},
  volume = {312?313},
  pages = {606--608},
  month = mar,
  note = {00000},
  abstract = {Some zero-temperature properties of doped S=1 spin ladder systems
	are reported here. We study the low-energy states for small hole
	concentrations by means of the series expansion method . One-hole
	doping generates two kinds of the low-energy states distinguished
	by their spin, S=12 or 32, and the characteristic dispersion relation.
	In particular, we show that the one-hole state with S=32 is a composite
	particle, i.e. a bound state of an S=12 hole and a spin triplet excitation.},
  doi = {10.1016/S0921-4526(01)01153-X},
  file = {ScienceDirect Full Text PDF:files/539/Koga et al. - 2002 - Hole-doping effects on an S=1 ladder system.pdf:application/pdf;ScienceDirect Snapshot:files/613/Koga et al. - 2002 - Hole-doping effects on an S=1 ladder system.html:text/html},
  issn = {0921-4526},
  keywords = {Bound state, S=1 ladder, Series expansion},
  owner = {vijay},
  series = {The {International} {Conference} on {Strongly} {Correlated} {Electron}
	{Systems}},
  timestamp = {2015.06.20},
  url = {http://www.sciencedirect.com/science/article/pii/S092145260101153X},
  urldate = {2014-06-07}
}

@ARTICLE{kohn_theory_1964,
  author = {Kohn, Walter},
  title = {Theory of the {Insulating} {State}},
  journal = {Physical Review},
  year = {1964},
  volume = {133},
  pages = {A171--A181},
  number = {1A},
  month = jan,
  abstract = {In this paper a new and more comprehensive characterization of the
	insulating state of matter is developed. This characterization includes
	the conventional insulators with energy gap as well as systems discussed
	by Mott which, in band theory, would be metals. The essential property
	is this: Every low-lying wave function ? of an insulating ring breaks
	up into a sum of functions, ?=?????M, which are localized in disconnected
	regions of the many-particle configuration space and have essentially
	vanishing overlap. This property is the analog of localization for
	a single particle and leads directly to the electrical properties
	characteristic of insulators. An Appendix deals with a soluble model
	exhibiting a transition between an insulating and a conducting state.},
  doi = {10.1103/PhysRev.133.A171},
  file = {APS Snapshot:files/855/PhysRev.133.html:text/html},
  url = {http://link.aps.org/doi/10.1103/PhysRev.133.A171},
  urldate = {2015-07-28}
}

@ARTICLE{kubo_note_1982,
  author = {Kubo, Kenn},
  title = {Note on the {Ground} {States} of {Systems} with the {Strong} {Hund}-{Coupling}},
  journal = {Journal of the Physical Society of Japan},
  year = {1982},
  volume = {51},
  pages = {782--786},
  number = {3},
  month = mar,
  __markedentry = {[vijay:3]},
  abstract = {The effect of the strong Hund coupling on the magnetic property of
	the ground stateis studied based on the double exchange model. For
	one dimensional chains the ground state is proved to have the maximum
	total spin. The effect of the Pauli principle is shown to destroy
	the ferromagnetic ground state by numerical studies of finite loops.
	The result is extended to a model with degenerate bands.},
  doi = {10.1143/JPSJ.51.782},
  file = {Full Text PDF:files/1054/Kubo - 1982 - Note on the Ground States of Systems with the Stro.pdf:application/pdf;Snapshot:files/1055/JPSJ.51.html:text/html},
  issn = {0031-9015},
  url = {http://journals.jps.jp/doi/abs/10.1143/JPSJ.51.782},
  urldate = {2015-09-03}
}

@ARTICLE{labeguerie_is_2008,
  author = {Labèguerie, Pierre and Boilleau, Corentin and Bastardis, Roland and
	Suaud, Nicolas and Guihéry, Nathalie and Malrieu, Jean-Paul},
  title = {Is it possible to determine rigorous magnetic {Hamiltonians} in spin
	s=1 systems from density functional theory calculations?},
  journal = {The Journal of Chemical Physics},
  year = {2008},
  volume = {129},
  pages = {154110},
  number = {15},
  month = oct,
  note = {00012},
  abstract = {The variational energies of broken-symmetry single determinants are
	frequently used (especially in the Kohn?Sham density functional theory)
	to determine the magnetic coupling between open-shell metal ions
	in molecular complexes or periodic lattices. Most applications extract
	the information from the solutions of m s max and m s min eigenvalues
	of S ? z magnetic spin momentum, assuming that a mapping of these
	energies on the energies of an Ising Hamiltonian is grounded. This
	approach is unable to predict the possible importance of deviations
	from the simplest form of the Heisenberg Hamiltonians. For systems
	involving s = 1 magnetic centers, it cannot provide an estimate of
	neither the biquadratic exchange integral nor the three-body operator
	interaction that has recently been proven to be of the same order
	of magnitude [Phys. Rev. B70, 132412 (2007)]. The present work shows
	that one may use other broken-symmetry solutions of intermediate
	values of m s to evaluate the amplitude of these additional terms.
	The here-derived equations rely on the assumption that an extended
	Hubbard-type Hamiltonian rules the interactions between the magnetic
	electrons. Numerical illustrations on a model problem of two O 2
	molecules and a fragment of the La 2 NiO 4 lattice are reported.
	The results obtained using a variable percentage of Fock exchange
	in the BLYP functional are compared to those provided by elaborate
	wave function calculations. The relevant percentage of Fock exchange
	is system dependent but a mean value of 30\% leads to acceptable
	amplitudes of the effective exchange interaction.},
  doi = {10.1063/1.2993263},
  file = {Full Text PDF:files/696/Labèguerie et al. - 2008 - Is it possible to determine rigorous magnetic Hami.pdf:application/pdf;Snapshot:files/609/Labèguerie et al. - 2008 - Is it possible to determine rigorous magnetic Hami.html:text/html},
  issn = {0021-9606, 1089-7690},
  keywords = {Ab initio calculations, Density functional theory, Nickel, Polarization,
	Wave functions},
  owner = {vijay},
  timestamp = {2015.06.20},
  url = {http://scitation.aip.org/content/aip/journal/jcp/129/15/10.1063/1.2993263},
  urldate = {2014-05-19}
}

@ARTICLE{lamas_combined_2011-1,
  author = {Lamas, C. A. and Capponi, S. and Pujol, P.},
  title = {Combined analytical and numerical approach to study magnetization
	plateaux in doped quasi-one-dimensional antiferromagnets},
  journal = {Physical Review B},
  year = {2011},
  volume = {84},
  pages = {115125},
  number = {11},
  month = sep,
  abstract = {We investigate the magnetic properties of quasi-one-dimensional quantum
	spin-S antiferromagnets. We use a combination of analytical and numerical
	techniques to study the presence of plateaux in the magnetization
	curve. The analytical technique consists in a path integral formulation
	in terms of coherent states. This technique can be extended to the
	presence of doping and has the advantage of a much better control
	for large spins than the usual bosonization technique. We discuss
	the appearance of doping-dependent plateaux in the magnetization
	curves for spin-S chains and ladders. The analytical results are
	complemented by a density matrix renormalization group (DMRG) study
	for a trimerized spin-1/2 and anisotropic spin-3/2 doped chains.},
  doi = {10.1103/PhysRevB.84.115125},
  file = {APS Snapshot:files/405/Lamas et al. - 2011 - Combined analytical and numerical approach to stud.html:text/html},
  owner = {vijay},
  timestamp = {2015.06.20},
  url = {http://link.aps.org/doi/10.1103/PhysRevB.84.115125},
  urldate = {2015-06-11}
}

@ARTICLE{lanczos1950iteration,
  author = {Lanczos, Cornelius},
  title = {An Iteration Method for the Solution of the Eigenvalue Problem of
	Linear Differential and Integral Operators1},
  journal = {Journal of Research of the National Bureau of Standards},
  year = {1950},
  volume = {45},
  number = {4}
}

@ARTICLE{lee_charge_2001,
  author = {Lee, S.-H. and Cheong, S.-W. and Yamada, K. and Majkrzak, C. F.},
  title = {Charge and canted spin order in {La}2-{xSrxNiO}4 (x=0.275 and 13)},
  journal = {Physical Review B},
  year = {2001},
  volume = {63},
  pages = {060405},
  number = {6},
  month = jan,
  note = {00000},
  abstract = {Polarized neutron diffraction on La2-xSrxNiO4 (x=0.275 and 1/3) reveals
	that the spins in the ordered phases are canted in the NiO2 plane
	away from the charge and spin stripe direction. The deviation angle
	is larger for x=1/3 than for x=0.275. Furthermore, for the optimal
	x=1/3 stoichiometry, an enhancement of the charge contribution and
	a larger canting of the spins occur below 50 K, which indicates a
	further lock in of the doped holes.},
  doi = {10.1103/PhysRevB.63.060405},
  file = {APS Snapshot:files/388/Lee et al. - 2001 - Charge and canted spin order in La2-xSrxNiO4 (x=0..html:text/html;Full Text PDF:files/685/Lee et al. - 2001 - Charge and canted spin order in La2-xSrxNiO4 (x=0..pdf:application/pdf},
  owner = {vijay},
  timestamp = {2015.06.20},
  url = {http://link.aps.org/doi/10.1103/PhysRevB.63.060405},
  urldate = {2014-05-31}
}

@ARTICLE{maciag_mixed_2006,
  author = {Maci¸ag, A. and Wróbel, P.},
  title = {Mixed singlet-triplet superconducting state in doped antiferromagnets},
  journal = {physica status solidi (b)},
  year = {2006},
  volume = {243},
  pages = {512--529},
  number = {2},
  month = feb,
  abstract = {We analyze symmetry mixing in the superconducting (SC) order parameter
	of planar cuprates. The behavior of thermal conductivity observed
	in some systems doped with magnetic impurities or in some systems
	exposed to external magnetic field seems to indicate that such symmetry
	mixing takes place. We discuss this phenomenon in the framework of
	the spin polaron model (SPM). We assume that antiferromagnetic (AF)
	correlations, which are at least of short range, tend to confine
	motion of holes which have been created in the AF spin background.
	The nature of the propagation of quasiparticles which are hole-like
	and the nature of the interaction between quasiparticles is determined
	by a tendency to restore the local AF order. It is known that two
	holes in the t ?J model (tJ M) form bound states with dx 2?y2 or
	p-wave symmetry. The d-wave bound state has lower energy and is the
	ground state. The mixing of d-wave symmetry with p-wave symmetry
	takes place in the SC order parameter at some range of finite values
	of the doping parameter. That range lies at the applicability verge
	of the SPM, where AF correlation are already very short. On the other
	hand, these correlations may be strengthened by above mentioned external
	factors, which seems to explain why symmetry mixing is observed in
	this case. (© 2006 WILEY-VCH Verlag GmbH \& Co. KGaA, Weinheim)},
  copyright = {Copyright © 2006 WILEY-VCH Verlag GmbH \& Co. KGaA, Weinheim},
  doi = {10.1002/pssb.200541003},
  file = {Full Text PDF:files/456/Maci¸ag and Wróbel - 2006 - Mixed singlet-triplet superconducting state in dop.pdf:application/pdf;Snapshot:files/286/Maci¸ag and Wróbel - 2006 - Mixed singlet-triplet superconducting state in dop.html:text/html},
  issn = {1521-3951},
  keywords = {71.27.+a, 74.20.Mn, 74.20.Rp, 74.72.?h},
  language = {en},
  owner = {vijay},
  timestamp = {2015.06.20},
  url = {http://onlinelibrary.wiley.com/doi/10.1002/pssb.200541003/abstract},
  urldate = {2014-11-18}
}

@ARTICLE{malrieu_magnetic_2014,
  author = {Malrieu, Jean Paul and Caballol, Rosa and Calzado, Carmen J. and
	de Graaf, Coen and Guihéry, Nathalie},
  title = {Magnetic {Interactions} in {Molecules} and {Highly} {Correlated}
	{Materials}: {Physical} {Content}, {Analytical} {Derivation}, and
	{Rigorous} {Extraction} of {Magnetic} {Hamiltonians}},
  journal = {Chemical Reviews},
  year = {2014},
  volume = {114},
  pages = {429--492},
  number = {1},
  month = jan,
  note = {00000},
  doi = {10.1021/cr300500z},
  file = {ACS Full Text PDF w/ Links:files/644/Malrieu et al. - 2014 - Magnetic Interactions in Molecules and Highly Corr.pdf:application/pdf;ACS Full Text Snapshot:files/694/Malrieu et al. - 2014 - Magnetic Interactions in Molecules and Highly Corr.html:text/html},
  issn = {0009-2665},
  owner = {vijay},
  shorttitle = {Magnetic {Interactions} in {Molecules} and {Highly} {Correlated} {Materials}},
  timestamp = {2015.06.20},
  url = {http://dx.doi.org/10.1021/cr300500z},
  urldate = {2014-07-18}
}

@ARTICLE{malrieu_single_2012,
  author = {Malrieu, Jean-Paul},
  title = {Single reference {Coupled} {Cluster} treatment of nearly degenerate
	problems: {Cohesive} energy of antiferromagnetic lattices of spin
	1 centers},
  journal = {Chemical Physics},
  year = {2012},
  volume = {401},
  pages = {130--135},
  month = jun,
  note = {00001},
  abstract = {Lattices of antiferromagnetically coupled spins, ruled by Heisenberg
	Hamiltonians, are intrinsically highly degenerate systems. The present
	work tries to estimate the ground state energy of regular bipartite
	spin lattices of S = 1 sites from a single reference Coupled Cluster
	expansion starting from a Néel function, taken as reference. The
	simultaneous changes of spin momentum on adjacent sites play the
	role of the double excitations in molecular electronic problems.
	Propagation of the spin changes plays the same role as the triple
	excitations. The treatment takes care of the deviation of multiple
	excitation energies from additivity. Specific difficulties appear
	for 1D chains, which are not due to a near degeneracy between the
	reference and the vectors which directly interact with it but to
	the complexity of the processes which lead to the low energy configurations
	where a consistent reversed-Néel domain is created inside the Néel
	starting spin wave. Despite these difficulties a reasonable value
	of the cohesive energy is obtained.},
  doi = {10.1016/j.chemphys.2011.11.005},
  file = {ScienceDirect Full Text PDF:files/632/Malrieu - 2012 - Single reference Coupled Cluster treatment of near.pdf:application/pdf;ScienceDirect Snapshot:files/438/Malrieu - 2012 - Single reference Coupled Cluster treatment of near.html:text/html},
  issn = {0301-0104},
  keywords = {Coupled Cluster, Heisenberg Hamiltonians, Magnetic lattices, Perturbation
	Theory},
  owner = {vijay},
  series = {Recent advances in electron correlation methods and applications},
  shorttitle = {Single reference {Coupled} {Cluster} treatment of nearly degenerate
	problems},
  timestamp = {2015.06.20},
  url = {http://www.sciencedirect.com/science/article/pii/S0301010411004861},
  urldate = {2014-06-20}
}

@ARTICLE{malvezzi_origin_2001,
  author = {Malvezzi, André Luiz and Dagotto, Elbio},
  title = {Origin of spin incommensurability in hole-doped {S}=1 {Y}2-{xCaxBaNiO}5
	chains},
  journal = {Physical Review B},
  year = {2001},
  volume = {63},
  pages = {140409},
  number = {14},
  month = mar,
  note = {00000},
  abstract = {Spin incommensurability (IC) has been recently experimentally discovered
	in the hole-doped Ni-oxide chain compound Y2-xCaxBaNiO5 [G. Xu et
	al., Science 289, 419 (2000)]. Here a two orbital model for this
	material is studied using computational techniques. Spin IC is observed
	in a wide range of densities and couplings. The phenomenon originates
	in antiferromagnetic correlations ?across holes? dynamically generated
	to improve hole movement, as it occurs in the one-dimensional Hubbard
	model and in recent studies of the two-dimensional extended t?J model.
	The close proximity of ferromagnetic and phase-separated states in
	parameter space is also discussed.},
  doi = {10.1103/PhysRevB.63.140409},
  file = {APS Snapshot:files/355/PhysRevB.63.html:text/html;Full Text PDF:files/336/Malvezzi and Dagotto - 2001 - Origin of spin incommensurability in hole-doped S=.pdf:application/pdf},
  owner = {vijay},
  timestamp = {2015.06.20},
  url = {http://link.aps.org/doi/10.1103/PhysRevB.63.140409},
  urldate = {2014-06-09}
}

@ARTICLE{malvezzi_origin_2001-1,
  author = {Malvezzi, André Luiz and Dagotto, Elbio},
  title = {Origin of spin incommensurability in hole-doped \${S}=1\{{\textbackslash}mathrm\{\}{\textbackslash}mathrm\{{Y}\}\}\_\{2-x\}\{{\textbackslash}mathrm\{{Ca}\}\}\_\{x\}\{{\textbackslash}mathrm\{{BaNiO}\}\}\_\{5\}\$
	chains},
  journal = {Physical Review B},
  year = {2001},
  volume = {63},
  pages = {140409},
  number = {14},
  month = mar,
  __markedentry = {[vijay:1]},
  abstract = {Spin incommensurability (IC) has been recently experimentally discovered
	in the hole-doped Ni-oxide chain compound Y2?xCaxBaNiO5 [G. Xu et
	al., Science 289, 419 (2000)]. Here a two orbital model for this
	material is studied using computational techniques. Spin IC is observed
	in a wide range of densities and couplings. The phenomenon originates
	in antiferromagnetic correlations ?across holes? dynamically generated
	to improve hole movement, as it occurs in the one-dimensional Hubbard
	model and in recent studies of the two-dimensional extended t?J model.
	The close proximity of ferromagnetic and phase-separated states in
	parameter space is also discussed.},
  doi = {10.1103/PhysRevB.63.140409},
  file = {APS Snapshot:files/408/Malvezzi and Dagotto - 2001 - Origin of spin incommensurability in hole-doped \$S.html:text/html;Full Text PDF:files/386/Malvezzi and Dagotto - 2001 - Origin of spin incommensurability in hole-doped \$S.pdf:application/pdf},
  owner = {vijay},
  timestamp = {2015.06.20},
  url = {http://link.aps.org/doi/10.1103/PhysRevB.63.140409},
  urldate = {2015-04-10}
}

@ARTICLE{malvezzi_influence_1999,
  author = {Malvezzi, A. L. and Yunoki, S. and Dagotto, E.},
  title = {Influence of nearest-neighbor {Coulomb} interactions on the phase
	diagram of the ferromagnetic {Kondo} model},
  journal = {Physical Review B},
  year = {1999},
  volume = {59},
  pages = {7033--7042},
  number = {10},
  month = mar,
  abstract = {The influence of a nearest-neighbor Coulomb repulsion of strength
	V on the properties of the ferromagnetic Kondo model is analyzed
	using computational techniques. The Hamiltonian studied here is defined
	on a chain using localized S=1/2 spins, and one orbital per site.
	Special emphasis is given to the influence of the Coulomb repulsion
	on the regions of phase separation recently discovered in this family
	of models, as well as on the double-exchange-induced ferromagnetic
	ground state. When phase separation dominates at V=0, the Coulomb
	interaction breaks the large domains of the two competing phases
	into small ?islands? of one phase embedded into the other. This is
	in agreement with several experimental results, as discussed in the
	text. Vestiges of the original phase separation regime are found
	in the spin structure factor as incommensurate peaks, even at large
	values of V. In the ferromagnetic regime close to density n=0.5,
	the Coulomb interaction induces tendencies to charge ordering without
	altering the fully polarized character of the state. This regime
	of ?charge-ordered ferromagnetism? may be related with experimental
	observations of a similar phase by Chen and Cheong [Phys. Rev. Lett.
	76, 4042 (1996)]. Our results reinforce the recently introduced notion
	[see, e.g., S. Yunoki et al., Phys. Rev. Lett. 80, 845 (1998)] that
	in realistic models for manganites analyzed with unbiased many-body
	techniques, the ground state properties arise from a competition
	between ferromagnetism and phase-separation?charge-ordering tendencies.},
  doi = {10.1103/PhysRevB.59.7033},
  file = {APS Snapshot:files/421/Malvezzi et al. - 1999 - Influence of nearest-neighbor Coulomb interactions.html:text/html;Full Text PDF:files/653/Malvezzi et al. - 1999 - Influence of nearest-neighbor Coulomb interactions.pdf:application/pdf},
  owner = {vijay},
  timestamp = {2015.06.20},
  url = {http://link.aps.org/doi/10.1103/PhysRevB.59.7033},
  urldate = {2014-11-06}
}

@ARTICLE{martin_householders_1968,
  author = {Martin, Dr R. S. and Reinsch, Dr C. and Wilkinson, J. H.},
  title = {Householder's tridiagonalization of a symmetric matrix},
  journal = {Numerische Mathematik},
  year = {1968},
  volume = {11},
  pages = {181--195},
  number = {3},
  month = mar,
  doi = {10.1007/BF02161841},
  file = {Full Text PDF:files/864/Martin et al. - 1968 - Householder's tridiagonalization of a symmetric ma.pdf:application/pdf;Snapshot:files/865/BF02161841.html:text/html},
  issn = {0029-599X, 0945-3245},
  keywords = {Appl.Mathematics/Computational Methods of Engineering, Mathematical
	and Computational Physics, Mathematical Methods in Physics, Mathematics,
	general, Numerical Analysis, Numerical and Computational Methods},
  language = {en},
  url = {http://link.springer.com/article/10.1007/BF02161841},
  urldate = {2015-07-13}
}

@ARTICLE{matsumoto_effects_2005,
  author = {Matsumoto, Munehisa and Takayama, Hajime},
  title = {Effects of {Impurities} in {Quasi}-{One}-{Dimensional} {S} = 1 {Antiferromagnets}},
  journal = {Progress of Theoretical Physics Supplement},
  year = {2005},
  volume = {159},
  pages = {412--416},
  month = may,
  note = {00000},
  abstract = {For the weakly coupled S = 1 antiferromagnetic Heisenberg chains on
	a simple cubic lattice, the effects of magnetic impurities are investigated
	by the quantum Monte Carlo method with the continuous-time loop algorithm.
	The transition temperatures of the impurity-induced phase transitions
	for magnetic impurities with S = 1/2, 3/2 and 2 are determined and
	compared with the transition temperature induced by the non-magnetic
	impurities. Implications on the experimental results are discussed.},
  doi = {10.1143/PTPS.159.412},
  file = {Full Text PDF:files/387/Matsumoto and Takayama - 2005 - Effects of Impurities in Quasi-One-Dimensional S =.pdf:application/pdf;Snapshot:files/569/412.html:text/html},
  issn = {0375-9687,},
  language = {en},
  owner = {vijay},
  timestamp = {2015.06.20},
  url = {http://ptps.oxfordjournals.org/content/159/412},
  urldate = {2014-06-09}
}

@ARTICLE{mezzacapo_variational_2011,
  author = {Mezzacapo, Fabio},
  title = {Variational study of a mobile hole in a two-dimensional quantum antiferromagnet
	using entangled-plaquette states},
  journal = {Physical Review B},
  year = {2011},
  volume = {83},
  pages = {115111},
  number = {11},
  month = mar,
  abstract = {We study the properties of a mobile hole in the t?J model on the square
	lattice by means of variational Monte Carlo simulations based on
	the entangled-plaquette ansatz. Our energy estimates for small lattices
	reproduce available exact results. We obtain values for the hole
	energy dispersion curve on large lattices in quantitative agreement
	with earlier findings based on the most reliable numerical techniques.
	Accurate estimates of the hole spectral weight are provided.},
  doi = {10.1103/PhysRevB.83.115111},
  file = {APS Snapshot:files/305/Mezzacapo - 2011 - Variational study of a mobile hole in a two-dimens.html:text/html;Full Text PDF:files/479/Mezzacapo - 2011 - Variational study of a mobile hole in a two-dimens.pdf:application/pdf},
  owner = {vijay},
  timestamp = {2015.06.20},
  url = {http://link.aps.org/doi/10.1103/PhysRevB.83.115111},
  urldate = {2014-11-18}
}

@ARTICLE{moca_su3_2012,
  author = {Moca, C?t?lin Pa?cu and Alex, Arne and von Delft, Jan and Zaránd,
	Gergely},
  title = {{SU}(3) {Anderson} impurity model: {A} numerical renormalization
	group approach exploiting non-{Abelian} symmetries},
  journal = {Physical Review B},
  year = {2012},
  volume = {86},
  pages = {195128},
  number = {19},
  month = nov,
  note = {00004},
  abstract = {We show how the density-matrix numerical renormalization group method
	can be used in combination with non-Abelian symmetries such as SU(N).
	The decomposition of the direct product of two irreducible representations
	requires the use of a so-called outer multiplicity label. We apply
	this scheme to the SU(3) symmetrical Anderson model, for which we
	analyze the finite size spectrum, determine local fermionic, spin,
	superconducting, and trion spectral functions, and also compute the
	temperature dependence of the conductance. Our calculations reveal
	a rich Fermi liquid structure.},
  doi = {10.1103/PhysRevB.86.195128},
  file = {APS Snapshot:files/557/Moca et al. - 2012 - SU(3) Anderson impurity model A numerical renorma.html:text/html;Full Text PDF:files/593/Moca et al. - 2012 - SU(3) Anderson impurity model A numerical renorma.pdf:application/pdf},
  owner = {vijay},
  shorttitle = {{SU}(3) {Anderson} impurity model},
  timestamp = {2015.06.20},
  url = {http://link.aps.org/doi/10.1103/PhysRevB.86.195128},
  urldate = {2014-08-07}
}

@ARTICLE{monari_metal-insulator_2008,
  author = {Monari, Antonio and Bendazzoli, Gian Luigi and Evangelisti, Stefano},
  title = {The metal-insulator transition in dimerized {Hückel} chains},
  journal = {The Journal of Chemical Physics},
  year = {2008},
  volume = {129},
  pages = {134104},
  number = {13},
  month = oct,
  abstract = {The metal-insulator transition is investigated in the case of linear
	chains described by a one-electron Hückel Hamiltonian. In these systems,
	the transition is a consequence of a dimerization of the chain bond
	length, which induces a similar dimerization of the hopping integral.
	Three indicators of the chain character are considered: The highest
	occupied molecular orbital?lowest unoccupied molecular orbital gap,
	the polarizability, and the localization tensor. In the case of even
	open chains, the behavior of the large chains depends in a crucial
	way on the alternating structure of the hopping integrals. If the
	ending atoms of the chain are weakly bonded to their neighbors, the
	energy spectrum of the Hamiltonian shows two quasidegenerated eigenvalues,
	and all the indicators would predict a (spurious) metallic behavior.
	It is shown that if the corresponding eigenvectors are removed from
	the Hamiltonian, the ordinary insulating behavior of alternating
	chains is recovered.},
  doi = {10.1063/1.2987702},
  file = {Full Text PDF:files/311/Monari et al. - 2008 - The metal-insulator transition in dimerized       Hückel.pdf:application/pdf;Snapshot:files/306/Monari et al. - 2008 - The metal-insulator transition in    dimerized Hückel.html:text/html},
  issn = {0021-9606, 1089-7690},
  keywords = {Band gap, Eigenvalues, Ground states, Polarizability, Tensor methods},
  url = {http://scitation.aip.org/content/aip/journal/jcp/129/13/10.1063/1.2987702},
  urldate = {2014-11-05}
}

@ARTICLE{mott_spin-polaron_1990,
  author = {Mott, N.F.},
  title = {The spin-polaron theory of high-{T} c superconductivity},
  journal = {Advances in Physics},
  year = {1990},
  volume = {39},
  pages = {55--81},
  number = {1},
  month = feb,
  abstract = {An outline is given of the model for some high-temperature superconductors
	which assumes that the carriers are holes in the (hybridized) oxygen
	2p band and form ?spin polarons? with the moments on the copper atoms.
	A comparison is made with observations of spin polarons in Gd3-x
	v x S4 and with the properties of La1-x Sr x VO3 in relation to those
	of La2-x Sr x CuO4. It is assumed, following several authors, that
	in the superconductors the polarons form bipolarons, which are bosons,
	and a comparison is made with some other treatments of this hypothesis.
	It is proposed that in many such superconductors the boson, essentially
	a pair of these holes, moves in an impurity band, and that normally
	all the polarons (fermions) form bipolarons; the fermions repel each
	other on the same site (positive Hubbard U) but attract when on adjacent
	sites; the critical temperature T c is then that at which the Bose
	gas becomes non-degenerate. In such materials a non-degenerate gas
	of bosons would carry the current above T c as first suggested by
	Alexandrov et al. (1986). The linear increase in the resistivity
	above T c is explained on this hypothesis. The effective mass of
	the bipolaron is, we believe, large (?20?30m e). The copper 3d9 moments
	in the superconducting range resonate between their two orientations
	as a consequence of the motion of the carriers, as they do in the
	description by Brinkman and Rice (1970) of highly correlated metals.
	Spin polarons, we believe, form only when this is so, but not in
	the antiferromagnetic range of x. A discussion is given of the resistivity
	above T c, thermopower above T c, and of the nature of the superconducting
	gap as shown by tunnelling. We confine our discussion to the materials
	containing copper, excluding for instance cubic Ba1-x K x BiO3, and
	possibly any superconductor containing bismuth, where the bosons
	may be Bi3+.},
  doi = {10.1080/00018739000101471},
  file = {Snapshot:files/489/Mott - 1990 - The spin-polaron theory of high-T c superconductiv.html:text/html},
  issn = {0001-8732},
  owner = {vijay},
  timestamp = {2015.06.20},
  url = {http://dx.doi.org/10.1080/00018739000101471},
  urldate = {2014-12-11}
}

@ARTICLE{nagaoka_ground_1965,
  author = {Nagaoka, Yosuke},
  title = {Ground state of correlated electrons in a narrow almost half-filled
	s band},
  journal = {Solid State Communications},
  year = {1965},
  volume = {3},
  pages = {409--412},
  number = {12},
  month = dec,
  abstract = {We consider a system of conduction electrons in an almost half-filled
	s band with an infinitely strong ?-function type repulsive potential,
	and with non-vanishing transfer matrix elements only between nearest
	neighbors. We find rigorously that the totally polarized ferromagnetic
	state is the ground state for sc and bcc and for fcc and hcp with
	Ne \&gt; N, Ne and N being respectively the number of electrons and
	atoms, and that it is not the ground state for fcc and hcp with Ne
	\&lt; N.},
  doi = {10.1016/0038-1098(65)90266-8},
  file = {ScienceDirect Full Text PDF:files/142/Nagaoka - 1965 - Ground state of correlated electrons   in a narrow a.pdf:application/pdf;ScienceDirect Snapshot:files/152/0038109865902668.html:text/html},
  issn = {0038-1098},
  url = {http://www.sciencedirect.com/science/article/pii/0038109865902668},
  urldate = {2015-07-23}
}

@ARTICLE{narozhny_transport_1998,
  author = {Narozhny, B. N. and Millis, A. J. and Andrei, N.},
  title = {Transport in the {XXZ} model},
  journal = {Physical Review B},
  year = {1998},
  volume = {58},
  pages = {R2921--R2924},
  number = {6},
  month = aug,
  note = {00093},
  abstract = {We present evidence suggesting that spin transport in the gapless
	phase of the S=1/2 XXZ model is ballistic rather than diffusive.
	We map the model onto a spinless fermion model whose charge stiffness
	determines the spin transport of the original model. By means of
	exact numerical diagonalization and finite size scaling we study
	both the stiffness and the level statistics. We show that the stiffness
	is nonzero at any temperature so that the transport is ballistic.
	Our results suggest that the nonzero stiffness arises because even
	in the presence of umklapp scattering a nonzero fraction of states
	remains degenerate in the thermodynamic limit.},
  doi = {10.1103/PhysRevB.58.R2921},
  file = {APS Snapshot:files/521/Narozhny et al. - 1998 - Transport in the XXZ model.html:text/html;Full Text PDF:files/423/Narozhny et al. - 1998 - Transport in the XXZ model.pdf:application/pdf},
  owner = {vijay},
  timestamp = {2015.06.20},
  url = {http://link.aps.org/doi/10.1103/PhysRevB.58.R2921},
  urldate = {2014-05-30}
}

@ARTICLE{nataf_exact_2014,
  author = {Nataf, Pierre and Mila, Frédéric},
  title = {Exact {Diagonalization} of {Heisenberg} {SU}({N}) {Models}},
  journal = {Physical Review Letters},
  year = {2014},
  volume = {113},
  pages = {127204},
  number = {12},
  month = sep,
  abstract = {Building on advanced results on permutations, we show that it is possible
	to construct, for each irreducible representation of SU(N), an orthonormal
	basis labeled by the set of standard Young tableaux in which the
	matrix of the Heisenberg SU(N) model (the quantum permutation of
	N-color objects) takes an explicit and extremely simple form. Since
	the relative dimension of the full Hilbert space to that of the singlet
	space on n sites increases very fast with N, this formulation allows
	us to extend exact diagonalizations of finite clusters to much larger
	values of N than accessible so far. Using this method, we show that,
	on the square lattice, there is long-range color order for SU(5),
	spontaneous dimerization for SU(8), and evidence in favor of a quantum
	liquid for SU(10).},
  doi = {10.1103/PhysRevLett.113.127204},
  file = {APS Snapshot:files/675/Nataf and Mila - 2014 - Exact Diagonalization of Heisenberg SU(N) Models.html:text/html;Full Text PDF:files/433/Nataf and Mila - 2014 - Exact Diagonalization of Heisenberg SU(N) Models.pdf:application/pdf},
  owner = {vijay},
  timestamp = {2015.06.20},
  url = {http://link.aps.org/doi/10.1103/PhysRevLett.113.127204},
  urldate = {2014-10-29}
}

@ARTICLE{navarro_spin-polarons_2012,
  author = {Navarro, O. and Vallejo, E. and Avignon, M.},
  title = {Spin-polarons in an exchange model},
  journal = {International Journal of Modern Physics B},
  year = {2012},
  volume = {26},
  pages = {1250048},
  number = {09},
  month = apr,
  __markedentry = {[vijay:1]},
  abstract = {Spin-polarons are obtained using an Ising-like exchange model consisting
	of double and super-exchange interactions in low-dimensional systems.
	At zero temperature, a new phase separation between small magnetic
	polarons, one conduction electron self-trapped in a magnetic domain
	of two or three sites, and the antiferromagnetic phase was previously
	reported. On the other hand the important effect of temperature was
	missed. Temperature diminishes Boltzmann probability allowing excited
	states in the system. Static magnetic susceptibility and short-range
	spin?spin correlations at zero magnetic field were calculated to
	explore the spin-polaron formation. At high temperature Curie?Weiss
	behavior is obtained and compared with the Curie-like behavior observed
	in the nickelate one-dimensional compound Y2-nCanBaNiO5.},
  doi = {10.1142/S0217979212500488},
  file = {Full Text PDF:files/426/Navarro et al. - 2012 - Spin-polarons in an exchange model.pdf:application/pdf;Snapshot:files/635/Navarro et al. - 2012 - Spin-polarons in an exchange model.html:text/html},
  issn = {0217-9792},
  owner = {vijay},
  timestamp = {2015.06.20},
  url = {http://www.worldscientific.com/doi/abs/10.1142/S0217979212500488},
  urldate = {2015-06-15}
}

@ARTICLE{nesbet1965algorithm,
  author = {Nesbet, R-K\_},
  title = {Algorithm for diagonalization of large matrices},
  journal = {The Journal of Chemical Physics},
  year = {1965},
  volume = {43},
  pages = {311--312},
  number = {1},
  publisher = {AIP Publishing}
}

@ARTICLE{ogata_phase_1991,
  author = {Ogata, Masao and Luchini, M. U. and Sorella, S. and Assaad, F. F.},
  title = {Phase diagram of the one-dimensional {\textbackslash}textit\{t\}
	- {\textbackslash}textit\{{J}\} model},
  journal = {Physical Review Letters},
  year = {1991},
  volume = {66},
  pages = {2388--2391},
  number = {18},
  month = may,
  abstract = {The phase diagram of the one-dimensional t-J model is investigated
	by analyzing the results of exact diagonalization and the exact solutions
	at J/t=0 and 2. Phase separation takes place above a critical value
	of J around Jc/t=2.5?3.5 depending on the electron density. In the
	small-J region, Tomonaga-Luttinger liquid theory holds and its correlation
	exponents are calculated as a function of J/t and the electron density.
	Superconducting correlations become dominant in a region between
	the solvable case (J/t=2) and phase separation. A spin-gap region
	is also found at low density.},
  doi = {10.1103/PhysRevLett.66.2388},
  file = {APS Snapshot:files/377/Ogata et al. - 1991 - Phase diagram of the one-dimensional     textit t.html:text/html;Full Text PDF:files/455/Ogata et al. - 1991 - Phase diagram of the one-dimensional     textit t.pdf:application/pdf},
  owner = {vijay},
  timestamp = {2015.06.20},
  url = {http://link.aps.org/doi/10.1103/PhysRevLett.66.2388},
  urldate = {2015-04-21}
}

@ARTICLE{oles_fingerprints_2012,
  author = {Ole?, Andrzej M.},
  title = {Fingerprints of spin?orbital entanglement in transition metal oxides},
  journal = {Journal of Physics: Condensed Matter},
  year = {2012},
  volume = {24},
  pages = {313201},
  number = {31},
  month = aug,
  abstract = {The concept of spin?orbital entanglement on superexchange bonds in
	transition metal oxides is introduced and explained on several examples.
	It is shown that spin?orbital entanglement in superexchange models
	destabilizes the long-range (spin and orbital) order and may lead
	either to a disordered spin-liquid state or to novel phases at low
	temperature which arise from strongly frustrated interactions. Such
	novel ground states cannot be described within the conventionally
	used mean field theory which separates spin and orbital degrees of
	freedom. Even in cases where the ground states are disentangled,
	spin?orbital entanglement occurs in excited states and may become
	crucial for a correct description of physical properties at finite
	temperature. As an important example of this behaviour we present
	spin?orbital entanglement in the RV O3 perovskites, with R = La,Pr,?,Y
	b,Lu, where the finite temperature properties of these compounds
	can be understood only using entangled states: (i) the thermal evolution
	of the optical spectral weights, (ii) the dependence of the transition
	temperatures for the onset of orbital and magnetic order on the ionic
	radius in the phase diagram of the RV O3 perovskites, and (iii) the
	dimerization observed in the magnon spectra for the C-type antiferromagnetic
	phase of Y V O3. Finally, it is shown that joint spin?orbital excitations
	in an ordered phase with coexisting antiferromagnetic and alternating
	orbital order introduce topological constraints for the hole propagation
	and will thus radically modify the transport properties in doped
	Mott insulators where hole motion implies simultaneous spin and orbital
	excitations.},
  doi = {10.1088/0953-8984/24/31/313201},
  file = {Full Text PDF:files/394/Ole? - 2012 - Fingerprints of spin?orbital entanglement in trans.pdf:application/pdf;Snapshot:files/400/Ole? - 2012 - Fingerprints of spin?orbital entanglement in trans.html:text/html},
  issn = {0953-8984},
  language = {en},
  owner = {vijay},
  timestamp = {2015.06.20},
  url = {http://iopscience.iop.org/0953-8984/24/31/313201},
  urldate = {2014-11-18}
}

@ARTICLE{pati_low-lying_1997,
  author = {Pati, Swapan K. and Ramasesha, S. and Sen, Diptiman},
  title = {Low-lying excited states and low-temperature properties of an alternating
	spin-1?spin-1/2 chain: {A} density-matrix renormalization-group study},
  journal = {Physical Review B},
  year = {1997},
  volume = {55},
  pages = {8894--8904},
  number = {14},
  month = apr,
  note = {00179},
  abstract = {We report spin wave and density-matrix renormalization-group (DMRG)
	studies of the ground and low-lying excited states of uniform and
	dimerized alternating spin chains. The DMRG procedure is also employed
	to obtain low-temperature thermodynamic properties of the system.
	The ground state of a 2N spin system with spin-1 and spin- alternating
	from site to site and interacting via an antiferromagnetic exchange
	is found to be ferrimagnetic with total spin sG=N/2 from both DMRG
	and spin wave analysis. Both the studies also show that there is
	a gapless excitation to a state with spin sG-1 and a gapped excitation
	to a state with spin sG+1. Surprisingly, the correlation length in
	the ground state is found to be very small from both the studies
	for this gapless system. For this very reason, we show that the ground
	state can be described by a variational ansatz of the product type.
	DMRG analysis shows that the chain is susceptible to a conditional
	spin-Peierls' instability. The DMRG studies of magnetization, magnetic
	susceptibility (?), and specific heat show strong magnetic-field
	dependence. The product ?T shows a minimum as a function of temperature
	(T) at low-magnetic fields and the minimum vanishes at high-magnetic
	fields. This low-field behavior is in agreement with earlier experimental
	observations. The specific heat shows a maximum as a function of
	temperature and the height of the maximum increases sharply at high-magnetic
	fields. It is hoped that these studies will motivate experimental
	studies at high-magnetic fields.},
  doi = {10.1103/PhysRevB.55.8894},
  file = {APS Snapshot:files/487/Pati et al. - 1997 - Low-lying excited states and low-temperature prope.html:text/html;Full Text PDF:files/323/Pati et al. - 1997 - Low-lying excited states and low-temperature prope.pdf:application/pdf},
  owner = {vijay},
  shorttitle = {Low-lying excited states and low-temperature properties of an alternating
	spin-1?spin-1/2 chain},
  timestamp = {2015.06.20},
  url = {http://link.aps.org/doi/10.1103/PhysRevB.55.8894},
  urldate = {2014-05-31}
}

@ARTICLE{penc_propagating_1995,
  author = {Penc, Karlo and Shiba, Hiroyuki},
  title = {Propagating {S}=1/2 particles in {S}=1 {Haldane}-gap systems},
  journal = {Physical Review B},
  year = {1995},
  volume = {52},
  pages = {R715--R718},
  number = {2},
  month = jul,
  note = {00030},
  abstract = {Inspired by the recent experiments on Y2-xCaxBaNiO5, we discuss the
	dispersion relation of the S=1/2 particles in the S=1 Heisenberg
	and valence bond solid model in the limit of small hopping amplitudes.
	The effective S=1/2 edge spins mix with the spin of the impurity
	resulting in one fourfold and two twofold degenerate bands. We briefly
	discuss the interaction between the S=1/2 particles arising from
	the background.},
  doi = {10.1103/PhysRevB.52.R715},
  file = {APS Snapshot:files/567/Penc and Shiba - 1995 - Propagating S=12 particles in S=1 Haldane-gap sys.html:text/html;Full Text PDF:files/690/Penc and Shiba - 1995 - Propagating S=12 particles in S=1 Haldane-gap sys.pdf:application/pdf},
  owner = {vijay},
  timestamp = {2015.06.20},
  url = {http://link.aps.org/doi/10.1103/PhysRevB.52.R715},
  urldate = {2014-06-07}
}

@ARTICLE{perdew_self-interaction_1981,
  author = {Perdew, J. P. and Zunger, Alex},
  title = {Self-interaction correction to density-functional approximations
	for many-electron systems},
  journal = {Physical Review B},
  year = {1981},
  volume = {23},
  pages = {5048--5079},
  number = {10},
  month = may,
  abstract = {The exact density functional for the ground-state energy is strictly
	self-interaction- free (i.e., orbitals demonstrably do not self-interact),
	but many approximations to it, including the local-spin-density (LSD)
	approximation for exchange and correlation, are not. We present two
	related methods for the self-interaction correction (SIC) of any
	density functional for the energy; correction of the self-consistent
	one-electron potenial follows naturally from the variational principle.
	Both methods are sanctioned by the Hohenberg-Kohn theorem. Although
	the first method introduces an orbital-dependent single-particle
	potential, the second involves a local potential as in the Kohn-Sham
	scheme. We apply the first method to LSD and show that it properly
	conserves the number content of the exchange-correlation hole, while
	substantially improving the description of its shape. We apply this
	method to a number of physical problems, where the uncorrected LSD
	approach produces systematic errors. We find systematic improvements,
	qualitative as well as quantitative, from this simple correction.
	Benefits of SIC in atomic calculations include (i) improved values
	for the total energy and for the separate exchange and correlation
	pieces of it, (ii) accurate binding energies of negative ions, which
	are wrongly unstable in LSD, (iii) more accurate electron densities,
	(iv) orbital eigenvalues that closely approximate physical removal
	energies, including relaxation, and (v) correct longrange behavior
	of the potential and density. It appears that SIC can also remedy
	the LSD underestimate of the band gaps in insulators (as shown by
	numerical calculations for the rare-gas solids and CuCl), and the
	LSD overestimate of the cohesive energies of transition metals. The
	LSD spin splitting in atomic Ni and s?d interconfigurational energies
	of transition elements are almost unchanged by SIC. We also discuss
	the admissibility of fractional occupation numbers, and present a
	parametrization of the electron-gas correlation energy at any density,
	based on the recent results of Ceperley and Alder.},
  doi = {10.1103/PhysRevB.23.5048},
  file = {APS Snapshot:files/1131/PhysRevB.23.html:text/html},
  url = {http://link.aps.org/doi/10.1103/PhysRevB.23.5048},
  urldate = {2015-09-09}
}

@ARTICLE{poirot_synthesis_2006,
  author = {Poirot, N. and Zaghrioui, M.},
  title = {Synthesis and characterization of calcium-doped lanthanium nickelates
	{La}2?{xCaxNiO}4+? ()},
  journal = {Solid State Sciences},
  year = {2006},
  volume = {8},
  pages = {149--154},
  number = {2},
  month = feb,
  abstract = {We have investigated the influence of Ca doping on chemical and physical
	properties of the La2NiO4+? compounds. Several Ca-doped compounds
	were then prepared by the citrate method. The X-ray diffraction analysis
	indicates that all the obtained compounds are pure with no presence
	of secondary phases. All compounds adopt a tetragonal structure with
	I4/mmm space group. As x increases, the excess oxygen ? decreases.
	Magnetic susceptibility shows anomaly below 200 K, which was attributed
	to charge ordering. Magnetization measurements displayed prominently
	the presence of weak ferromagnetism below 200 K. Most probably, the
	ferromagnetic component modifies the antiferromagnetic fluctuations
	in the NiO2 planes. The origin of the ferromagnetic component must
	be related to a presence of kind of bond-centered stripes.},
  doi = {10.1016/j.solidstatesciences.2005.10.003},
  file = {ScienceDirect Snapshot:files/337/Poirot and Zaghrioui - 2006 - Synthesis and characterization of calcium-doped la.html:text/html},
  issn = {1293-2558},
  keywords = {Ca-doping, Charge and spin ordering, La2NiO4+?, Magnetic susceptibility,
	Oxygen non-stoichiometry, Stripes},
  owner = {vijay},
  timestamp = {2015.06.20},
  url = {http://www.sciencedirect.com/science/article/pii/S1293255805002669},
  urldate = {2015-05-17}
}

@ARTICLE{rashid1962reduction,
  author = {Rashid, MA},
  title = {The reduction of the product of three 8-dimensional representations
	of U (3) and SU (3) into irreducible representations},
  journal = {Il Nuovo Cimento Series 10},
  year = {1962},
  volume = {26},
  pages = {118--133},
  number = {1},
  publisher = {Springer}
}

@ARTICLE{resta_insulating_2011,
  author = {Resta, R.},
  title = {The insulating state of matter: a geometrical theory},
  journal = {The European Physical Journal B},
  year = {2011},
  volume = {79},
  pages = {121--137},
  number = {2},
  month = jan,
  abstract = {In 1964 Kohn published the milestone paper ?Theory of the insulating
	state?, according to which insulators and metals differ in their
	ground state. Even before the system is excited by any probe, a different
	organization of the electrons is present in the ground state and
	this is the key feature discriminating between insulators and metals.
	However, the theory of the insulating state remained somewhat incomplete
	until the late 1990s; this review addresses the recent developments.
	The many-body ground wavefunction of any insulator is characterized
	by means of geometrical concepts (Berry phase, connection, curvature,
	Chern number, quantum metric). Among them, it is the quantum metric
	which sharply characterizes the insulating state of matter. The theory
	deals on a common ground with several kinds of insulators: band insulators,
	Mott insulators, Anderson insulators, quantum Hall insulators, Chern
	and topological insulators.},
  doi = {10.1140/epjb/e2010-10874-4},
  file = {Full Text PDF:files/451/Resta - 2011 - The insulating state of matter a geometrical theo.pdf:application/pdf;Snapshot:files/298/Resta - 2011 - The insulating state of matter a geometrical theo.html:text/html},
  issn = {1434-6028, 1434-6036},
  keywords = {Condensed Matter Physics, Fluid- and Aerodynamics, Physics, general,
	Solid State Physics, Statistical Physics, Dynamical Systems and Complexity},
  language = {en},
  owner = {vijay},
  shorttitle = {The insulating state of matter},
  timestamp = {2015.06.20},
  url = {http://link.springer.com/article/10.1140/epjb/e2010-10874-4},
  urldate = {2014-12-02}
}

@ARTICLE{resta_insulating_2011,
  author = {Resta, R.},
  title = {The insulating state of matter: a geometrical theory},
  journal = {The European Physical Journal B},
  year = {2011},
  volume = {79},
  pages = {121--137},
  number = {2},
  month = jan,
  abstract = {In 1964 Kohn published the milestone paper ?Theory of the insulating
	state?, according to which insulators and metals differ in their
	ground state. Even before the system is excited by any probe, a different
	organization of the electrons is present in the ground state and
	this is the key feature discriminating between insulators and metals.
	However, the theory of the insulating state remained somewhat incomplete
	until the late 1990s; this review addresses the recent developments.
	The many-body ground wavefunction of any insulator is characterized
	by means of geometrical concepts (Berry phase, connection, curvature,
	Chern number, quantum metric). Among them, it is the quantum metric
	which sharply characterizes the insulating state of matter. The theory
	deals on a common ground with several kinds of insulators: band insulators,
	Mott insulators, Anderson insulators, quantum Hall insulators, Chern
	and topological insulators.},
  doi = {10.1140/epjb/e2010-10874-4},
  file = {Full Text PDF:files/742/Resta - 2011 - The insulating state of matter a geometrical theo.pdf:application/pdf;Snapshot:files/769/10.html:text/html},
  issn = {1434-6028, 1434-6036},
  keywords = {Condensed Matter Physics, Fluid- and Aerodynamics, Physics, general,
	Solid State Physics, Statistical Physics, Dynamical Systems and Complexity},
  language = {en},
  shorttitle = {The insulating state of matter},
  url = {http://link.springer.com/article/10.1140/epjb/e2010-10874-4},
  urldate = {2015-07-27}
}

@ARTICLE{resta_why_2002,
  author = {Resta, Raffaele},
  title = {Why are insulators insulating and metals conducting?},
  journal = {Journal of Physics: Condensed Matter},
  year = {2002},
  volume = {14},
  pages = {R625},
  number = {20},
  month = may,
  note = {00097},
  __markedentry = {[vijay:1]},
  abstract = {At variance with what happens in metals, the electronic charge in
	insulators cannot flow freely under an applied dc field, and undergoes
	instead static polarization. These two features arise from the difference
	in nature of the excitation spectra, but also from the difference
	in organization of the electrons in their ground state: electrons
	are localized in insulators and delocalized in metals. Such localization,
	however, is hidden in a rather subtle way in the many-body wavefunction.
	We review the theory of the insulating state, on the basis of electron
	localization, addressing on the same basis all insulators: either
	independent electron or correlated, either crystalline or disordered.
	The starting point is a 1964 milestone paper by Kohn. Significant
	advances occurred from 1999 onwards. These advances are deeply rooted
	in the modern theory of polarization: localization and polarization
	can be regarded as two aspects of the same phenomenon, and stem from
	essentially the same formalism. Starting from the many-body ground
	wavefunction, one defines a dimensionless complex number which vanishes
	in metals and is finite in insulators; in the latter case, its phase
	(the Berry phase) yields the macroscopic polarization, while its
	modulus measures localization. Conductivity features are addressed
	within the same theoretical scheme.},
  doi = {10.1088/0953-8984/14/20/201},
  file = {Full Text PDF:files/619/Resta - 2002 - Why are insulators insulating and metals conductin.pdf:application/pdf;Snapshot:files/623/Resta - 2002 - Why are insulators insulating and metals conductin.html:text/html},
  issn = {0953-8984},
  language = {en},
  owner = {vijay},
  timestamp = {2015.06.20},
  url = {http://iopscience.iop.org/0953-8984/14/20/201},
  urldate = {2014-05-20}
}

@ARTICLE{resta_electron_1999,
  author = {Resta, Raffaele and Sorella, Sandro},
  title = {Electron {Localization} in the {Insulating} {State}},
  journal = {Physical Review Letters},
  year = {1999},
  volume = {82},
  pages = {370--373},
  number = {2},
  month = jan,
  abstract = {The insulating state of matter is characterized by the excitation
	spectrum, but also by qualitative features of the electronic ground
	state. The insulating ground wave function in fact (i) sustains macroscopic
	polarization, and (ii) is localized. We give a sharp definition of
	the latter concept and we show how the two basic features stem from
	essentially the same formalism. Our approach to localization is exemplified
	by means of a two-band Hubbard model in one dimension. In the noninteracting
	limit, the wave function localization is measured by the spread of
	the Wannier orbitals.},
  doi = {10.1103/PhysRevLett.82.370},
  file = {APS Snapshot:files/909/PhysRevLett.82.html:text/html;Full Text PDF:files/908/Resta and Sorella - 1999 - Electron Localization in the Insulating State.pdf:application/pdf},
  url = {http://link.aps.org/doi/10.1103/PhysRevLett.82.370},
  urldate = {2015-07-28}
}

@ARTICLE{riera_phase_1997,
  author = {Riera, Jose and Hallberg, Karen and Dagotto, Elbio},
  title = {Phase {Diagram} of {Electronic} {Models} for {Transition} {Metal}
	{Oxides} in {One} {Dimension}},
  journal = {Physical Review Letters},
  year = {1997},
  volume = {79},
  pages = {713--716},
  number = {4},
  month = jul,
  abstract = {The zero temperature phase diagram of the ferromagnetic Kondo model
	in one dimension is studied using numerical techniques, especially
	at large Hund coupling. A robust region of fully saturated ferromagnetism
	(FM) is identified at all densities. Phase separation between hole-rich
	and hole-poor regions and a paramagnetic regime with quasilocalized
	holes were also observed. It is argued that these phases will also
	appear in two and three dimensions. Our results apply both to manganites
	and one-dimensional compounds such as Y2?xCaxBaNiO5. As the transition
	metal ion spin grows, the hole mobility rapidly decreases, explaining
	the differences between Cu oxides and Mn oxides.},
  doi = {10.1103/PhysRevLett.79.713},
  file = {APS Snapshot:files/538/Riera et al. - 1997 - Phase Diagram of Electronic Models for Transition .html:text/html;Full Text PDF:files/518/Riera et al. - 1997 - Phase Diagram of Electronic Models for Transition .pdf:application/pdf},
  owner = {vijay},
  timestamp = {2015.06.20},
  url = {http://link.aps.org/doi/10.1103/PhysRevLett.79.713},
  urldate = {2015-06-09}
}

@ARTICLE{riera_phase_1997-1,
  author = {Riera, Jose and Hallberg, Karen and Dagotto, Elbio},
  title = {Phase {Diagram} of {Electronic} {Models} for {Transition} {Metal}
	{Oxides} in {One} {Dimension}},
  journal = {Physical Review Letters},
  year = {1997},
  volume = {79},
  pages = {713--716},
  number = {4},
  month = jul,
  note = {00091},
  __markedentry = {[vijay:1]},
  abstract = {The zero temperature phase diagram of the ferromagnetic Kondo model
	in one dimension is studied using numerical techniques, especially
	at large Hund coupling. A robust region of fully saturated ferromagnetism
	(FM) is identified at all densities. Phase separation between hole-rich
	and hole-poor regions and a paramagnetic regime with quasilocalized
	holes were also observed. It is argued that these phases will also
	appear in two and three dimensions. Our results apply both to manganites
	and one-dimensional compounds such as Y2?xCaxBaNiO5. As the transition
	metal ion spin grows, the hole mobility rapidly decreases, explaining
	the differences between Cu oxides and Mn oxides.},
  doi = {10.1103/PhysRevLett.79.713},
  file = {APS Snapshot:files/280/Riera et al. - 1997 - Phase Diagram of Electronic Models for Transition .html:text/html;Full Text PDF:files/376/Riera et al. - 1997 - Phase Diagram of Electronic Models for Transition .pdf:application/pdf},
  owner = {vijay},
  timestamp = {2015.06.20},
  url = {http://link.aps.org/doi/10.1103/PhysRevLett.79.713},
  urldate = {2014-07-16}
}

@ARTICLE{rincon_exotic_2014,
  author = {Rincón, Julián and Moreo, Adriana and Alvarez, Gonzalo and Dagotto,
	Elbio},
  title = {Exotic {Magnetic} {Order} in the {Orbital}-{Selective} {Mott} {Regime}
	of {Multiorbital} {Systems}},
  journal = {Physical Review Letters},
  year = {2014},
  volume = {112},
  pages = {106405},
  number = {10},
  month = mar,
  abstract = {The orbital-selective Mott phase of multiorbital Hubbard models has
	been extensively analyzed before using static and dynamical mean-field
	approximations. In parallel, the properties of block states (antiferromagnetically
	coupled ferromagnetic spin clusters) in Fe-based superconductors
	have also been much discussed. The present effort uses numerically
	exact techniques in one-dimensional systems to report the observation
	of block states within the orbital-selective Mott phase regime, connecting
	two seemingly independent areas of research, and providing analogies
	with the physics of double-exchange models.},
  doi = {10.1103/PhysRevLett.112.106405},
  file = {APS Snapshot:files/311/Rincón et al. - 2014 - Exotic Magnetic Order in the Orbital-Selective Mot.html:text/html;Full Text PDF:files/697/Rincón et al. - 2014 - Exotic Magnetic Order in the Orbital-Selective Mot.pdf:application/pdf},
  owner = {vijay},
  timestamp = {2015.06.20},
  url = {http://link.aps.org/doi/10.1103/PhysRevLett.112.106405},
  urldate = {2015-06-15}
}

@ARTICLE{rubin_comparison_2014,
  author = {Rubin, Nicholas C. and Mazziotti, David A.},
  title = {Comparison of one-dimensional and quasi-one-dimensional {Hubbard}
	models from the variational two-electron reduced-density-matrix method},
  journal = {Theoretical Chemistry Accounts},
  year = {2014},
  volume = {133},
  pages = {1--9},
  number = {7},
  month = jul,
  abstract = {Minimizing the energy of an NN -electron system as a functional of
	a two-electron reduced density matrix (2-RDM), constrained by necessary
	NN -representability conditions (conditions for the 2-RDM to represent
	an ensemble NN -electron quantum system), yields a rigorous lower
	bound to the ground-state energy in contrast to variational wave
	function methods. We characterize the performance of two sets of
	approximate constraints, (2,2)-positivity (DQG) and approximate (2,3)-positivity
	(DQGT) conditions, at capturing correlation in one-dimensional and
	quasi-one-dimensional (ladder) Hubbard models. We find that, while
	both the DQG and DQGT conditions capture both the weak and strong
	correlation limits, the more stringent DQGT conditions improve the
	ground-state energies, the natural occupation numbers, the pair correlation
	function, the effective hopping, and the connected (cumulant) part
	of the 2-RDM. We observe that the DQGT conditions are effective at
	capturing strong electron correlation effects in both one- and quasi-one-dimensional
	lattices for both half filling and less-than-half filling.},
  doi = {10.1007/s00214-014-1492-7},
  file = {Full Text PDF:files/370/Rubin and Mazziotti - 2014 - Comparison of one-dimensional and quasi-one-dimens.pdf:application/pdf;Snapshot:files/648/Rubin and Mazziotti - 2014 - Comparison of one-dimensional and quasi-one-dimens.html:text/html},
  issn = {1432-881X, 1432-2234},
  keywords = {Atomic/Molecular Structure and Spectra, Hubbard models, Inorganic
	Chemistry, N-representability conditions, Organic Chemistry, Physical
	Chemistry, Strong electron correlation, Theoretical and Computational
	Chemistry, Two-electron reduced density matrix},
  language = {en},
  owner = {vijay},
  timestamp = {2015.06.20},
  url = {http://link.springer.com/article/10.1007/s00214-014-1492-7},
  urldate = {2014-10-31}
}

@ARTICLE{sano_bound_2011,
  author = {Sano, Kazuhiro and Takano, Ken?ichi},
  title = {Bound state of a hole and a triplet spin in the t1-t2-{J}1-{J}2 model},
  journal = {Physical Review B},
  year = {2011},
  volume = {83},
  pages = {054421},
  number = {5},
  month = feb,
  abstract = {We show that a hole and a triplet spin form a bound state in a nearly
	half-filled band of the one- and two-dimensional t1?t2?J1?J2 models.
	Numerical calculation indicates that the bound state is a spatially
	small object and moves as a composite particle with spin 1 and charge
	+e in the spin-gapped background. Two bound states repulsively interact
	with each other in a short distance and move independently as long
	as they keep their distance. If a finite density of bound states
	behaves as bosons, the system undergoes the Bose-Einstein condensation,
	which means a superconductivity with charge +e.},
  doi = {10.1103/PhysRevB.83.054421},
  file = {APS Snapshot:files/330/Sano and Takano - 2011 - Bound state of a hole and a triplet spin in the t1.html:text/html;Full Text PDF:files/385/Sano and Takano - 2011 - Bound state of a hole and a triplet spin in the t1.pdf:application/pdf},
  owner = {vijay},
  timestamp = {2015.06.20},
  url = {http://link.aps.org/doi/10.1103/PhysRevB.83.054421},
  urldate = {2014-11-18}
}

@ARTICLE{schuster_local_2004,
  author = {Schuster, Cosima and Brune, Philipp},
  title = {Local distortions and {Friedel} oscillations in interacting {Fermi}
	chains},
  journal = {physica status solidi (b)},
  year = {2004},
  volume = {241},
  pages = {2043--2054},
  number = {9},
  month = jul,
  abstract = {The interplay between disorder and interaction, especially near metal
	insulator transitions, is a long-standing question. We investigate
	in detail single impurities, in particular, the Friedel oscillations
	induced by them. We study the decay of the Friedel oscillations in
	the one- dimensional Heisenberg and Hubbard model analytically using
	the bosonization technique and numerically using the density matrix
	renormalization group treatment (DMRG). For the Heisenberg chain,
	we confirm the predictions of conformal field theory and bosonization
	for small interaction, but near phase transitions deviations are
	found in form of vanishing or additional oscillations. For the Hubbard
	chain, we study the oscillations ? in the density as well as in the
	magnetization ? in the spin-gap, charge-gap, and Luttinger liquid
	phase. We find an exponential decay or a very slow algebraic decay
	of the oscillations in the gapped phases. In the Luttinger liquid
	phase, we concentrate on the question of logarithmic corrections
	(which occur also in the isotropic Heisenberg antiferromagnet). Differences
	in the behavior near a boundary compared to an impurity are pointed
	out. (© 2004 WILEY-VCH Verlag GmbH \& Co. KGaA, Weinheim)},
  copyright = {Copyright © 2004 WILEY-VCH Verlag GmbH \& Co. KGaA, Weinheim},
  doi = {10.1002/pssb.200404795},
  file = {Full Text PDF:files/1039/Schuster and Brune - 2004 - Local distortions and Friedel            oscillations in inte.pdf:application/pdf;Snapshot:files/1040/abstract.html:text/html},
  issn = {1521-3951},
  keywords = {71.10.Fd, 71.10.Pm},
  language = {en},
  url = {http://onlinelibrary.wiley.com/doi/10.1002/pssb.200404795/abstract},
  urldate = {2015-09-02}
}

@ARTICLE{shavitt1973iterative,
  author = {Shavitt, I and Bender, CF and Pipano, A and Hosteny, RP},
  title = {The iterative calculation of several of the lowest or highest eigenvalues
	and corresponding eigenvectors of very large symmetric matrices},
  journal = {Journal of Computational Physics},
  year = {1973},
  volume = {11},
  pages = {90--108},
  number = {1},
  publisher = {Elsevier}
}

@ARTICLE{shelton_antiferromagnetic_1996,
  author = {Shelton, D. G. and Nersesyan, A. A. and Tsvelik, A. M.},
  title = {Antiferromagnetic spin ladders: {Crossover} between spin {\textbackslash}textit\{{S}\}
	=1/2 and {\textbackslash}textit\{{S}\} =1 chains},
  journal = {Physical Review B},
  year = {1996},
  volume = {53},
  pages = {8521--8532},
  number = {13},
  month = apr,
  abstract = {We study a model of two weakly coupled isotropic spin-1/2 Heisenberg
	chains with an antiferromagnetic coupling along the chains (spin
	ladder). It is shown that the system always has a spectral gap and
	the lower lying excitations are triplets. For the case of identical
	chains the model in the continuous limit is shown to be equivalent
	to four decoupled noncritical Ising models with the Z2×SU(2) symmetry.
	For this case we obtain the exact expressions for asymptotics of
	spin-spin correlation functions. It is shown that when the chains
	have different exchange integrals J1?J2 the spectrum at low energies
	is described by the O(3)-nonlinear ? model. We discuss the topological
	order parameter related to the gap formation and give a detailed
	description of the dynamical magnetic susceptibility. © 1996 The
	American Physical Society.},
  doi = {10.1103/PhysRevB.53.8521},
  file = {APS Snapshot:files/481/Shelton et al. - 1996 - Antiferromagnetic spin ladders Crossover between .html:text/html;Full Text PDF:files/618/Shelton et al. - 1996 - Antiferromagnetic spin ladders Crossover between .pdf:application/pdf},
  owner = {vijay},
  shorttitle = {Antiferromagnetic spin ladders},
  timestamp = {2015.06.20},
  url = {http://link.aps.org/doi/10.1103/PhysRevB.53.8521},
  urldate = {2015-04-21}
}

@ARTICLE{shelton_antiferromagnetic_1996-1,
  author = {Shelton, D. G. and Nersesyan, A. A. and Tsvelik, A. M.},
  title = {Antiferromagnetic spin ladders: {Crossover} between spin {\textbackslash}textit\{{S}\}
	=1/2 and {\textbackslash}textit\{{S}\} =1 chains},
  journal = {Physical Review B},
  year = {1996},
  volume = {53},
  pages = {8521--8532},
  number = {13},
  month = apr,
  abstract = {We study a model of two weakly coupled isotropic spin-1/2 Heisenberg
	chains with an antiferromagnetic coupling along the chains (spin
	ladder). It is shown that the system always has a spectral gap and
	the lower lying excitations are triplets. For the case of identical
	chains the model in the continuous limit is shown to be equivalent
	to four decoupled noncritical Ising models with the Z2×SU(2) symmetry.
	For this case we obtain the exact expressions for asymptotics of
	spin-spin correlation functions. It is shown that when the chains
	have different exchange integrals J1?J2 the spectrum at low energies
	is described by the O(3)-nonlinear ? model. We discuss the topological
	order parameter related to the gap formation and give a detailed
	description of the dynamical magnetic susceptibility. © 1996 The
	American Physical Society.},
  doi = {10.1103/PhysRevB.53.8521},
  file = {Full Text PDF:files/429/Shelton et al. - 1996 - Antiferromagnetic spin ladders Crossover between .pdf:application/pdf},
  owner = {vijay},
  shorttitle = {Antiferromagnetic spin ladders},
  timestamp = {2015.06.20},
  url = {http://link.aps.org/doi/10.1103/PhysRevB.53.8521},
  urldate = {2015-04-21}
}

@ARTICLE{sorella_superconductivity_2002,
  author = {Sorella, S. and Martins, G. B. and Becca, F. and Gazza, C. and Capriotti,
	L. and Parola, A. and Dagotto, E.},
  title = {Superconductivity in the {Two}-{Dimensional} \${\textbackslash}mathit\{t\}-{\textbackslash}mathit\{{J}\}\$
	{Model}},
  journal = {Physical Review Letters},
  year = {2002},
  volume = {88},
  pages = {117002},
  number = {11},
  month = feb,
  abstract = {Using computational techniques, it is shown that pairing is a robust
	property of hole-doped antiferromagnetic insulators. In one dimension
	and for two-leg ladder systems, a BCS-like variational wave function
	with long-bond spin singlets and a Jastrow factor provides an accurate
	representation of the ground state of the t?J model, even though
	strong quantum fluctuations destroy the off-diagonal superconducting
	long-range order in this case. However, in two dimensions it is argued?and
	numerically confirmed using several techniques, especially quantum
	Monte Carlo?that quantum fluctuations are not strong enough to suppress
	superconductivity.},
  doi = {10.1103/PhysRevLett.88.117002},
  file = {APS Snapshot:files/304/Sorella et al. - 2002 - Superconductivity in the Two-Dimensional \$mathit .html:text/html;Full Text PDF:files/443/Sorella et al. - 2002 - Superconductivity in the Two-Dimensional \$mathit .pdf:application/pdf},
  owner = {vijay},
  timestamp = {2015.06.20},
  url = {http://link.aps.org/doi/10.1103/PhysRevLett.88.117002},
  urldate = {2015-04-21}
}

@ARTICLE{stewart_addendum_2002,
  author = {Stewart, G.},
  title = {Addendum to "{A} {Krylov}--{Schur} {Algorithm} for {Large} {Eigenproblems}"},
  journal = {SIAM Journal on Matrix Analysis and Applications},
  year = {2002},
  volume = {24},
  pages = {599--601},
  number = {2},
  month = jan,
  abstract = {In this addendum to an earlier paper by the author, it is shown how
	to compute a Krylov decomposition corresponding to an arbitrary Rayleigh
	quotient. This decomposition can be used to restart an Arnoldi process,
	with a selection of the Ritz vectors corresponding to that Rayleigh
	quotient., In this addendum to an earlier paper by the author, it
	is shown how to compute a Krylov decomposition corresponding to an
	arbitrary Rayleigh quotient. This decomposition can be used to restart
	an Arnoldi process, with a selection of the Ritz vectors corresponding
	to that Rayleigh quotient.},
  doi = {10.1137/S0895479802403150},
  file = {Full Text PDF:files/738/Stewart - 2002 - Addendum to A Krylov--Schur Algorithm for Large E.   pdf:application/pdf;Snapshot:files/740/S0895479802403150.html:text/html},
  issn = {0895-4798},
  url = {http://epubs.siam.org/doi/abs/10.1137/S0895479802403150},
  urldate = {2015-06-25}
}

@ARTICLE{stewart_krylov--schur_2002,
  author = {Stewart, G.},
  title = {A {Krylov}--{Schur} {Algorithm} for {Large} {Eigenproblems}},
  journal = {SIAM Journal on Matrix Analysis and Applications},
  year = {2002},
  volume = {23},
  pages = {601--614},
  number = {3},
  month = jan,
  abstract = {Sorensen's implicitly restarted Arnoldi algorithm is one of the most
	successful and flexible methods for finding a few eigenpairs of a
	large matrix. However, the need to preserve the structure of the
	Arnoldi decomposition on which the algorithm is based restricts the
	range of transformations that can be performed on the decomposition.
	In consequence, it is difficult to deflate converged Ritz vectors
	from the decomposition. Moreover, the potential forward instability
	of the implicit QR algorithm can cause unwanted Ritz vectors to persist
	in the computation. In this paper we introduce a general Krylov decomposition
	that solves both problems in a natural and efficient manner., Sorensen's
	implicitly restarted Arnoldi algorithm is one of the most successful
	and flexible methods for finding a few eigenpairs of a large matrix.
	However, the need to preserve the structure of the Arnoldi decomposition
	on which the algorithm is based restricts the range of transformations
	that can be performed on the decomposition. In consequence, it is
	difficult to deflate converged Ritz vectors from the decomposition.
	Moreover, the potential forward instability of the implicit QR algorithm
	can cause unwanted Ritz vectors to persist in the computation. In
	this paper we introduce a general Krylov decomposition that solves
	both problems in a natural and efficient manner.},
  doi = {10.1137/S0895479800371529},
  file = {Full Text PDF:files/736/Stewart - 2002 - A Krylov--Schur Algorithm for Large Eigenproblems.   pdf:application/pdf;Snapshot:files/739/S0895479800371529.html:text/html},
  issn = {0895-4798},
  url = {http://epubs.siam.org/doi/abs/10.1137/S0895479800371529},
  urldate = {2015-06-25}
}

@ARTICLE{stoyanova_hopping_2006,
  author = {Stoyanova, A. and Sousa, C. and De Graaf, C. and Broer, R.},
  title = {Hopping matrix elements from first-principles studies of overlapping
	fragments: {Double} exchange parameters in manganites},
  journal = {International Journal of Quantum Chemistry},
  year = {2006},
  volume = {106},
  pages = {2444--2457},
  number = {12},
  month = jan,
  note = {00009},
  abstract = {We recently developed a scheme for first-principles calculations of
	hopping matrix elements between localized states in extended systems.
	We apply the scheme to the determination of double exchange (DE)
	parameters in lightly hole-doped LaMnO3 and electron-doped CaMnO3.
	DE is one of the important factors for understanding the properties
	of doped manganites. The calculations are based on the construction
	of wave functions for localized hole states or localized electron
	states for large embedded clusters. The wave functions of these clusters
	are expressed in terms of localized orbitals, obtained from calculations
	on smaller units, or ?fragments,? centered around a transition metal
	ion. The starting point of electronic states expressed in terms of
	localized orbital sets is conceptually attractive. It also allows
	for a rigorous treatment of local electron correlation and electronic
	relaxation effects. In the present study, the fragments are embedded
	[MnO6] units. The large clusters contain either two or four Mn ions
	and all neighboring oxygen ligands. The results are compared with
	conventional embedded cluster calculations. In both compounds, the
	effective hopping matrix elements, or ?double exchange? (DE) parameters,
	in the ab planes (in the Pbnm space group) are larger than along
	the c axes. We found nearly perfect agreement with the Anderson?Hasegawa
	model for the spin dependence of the DE parameters. Nearest-neighbor
	parameters are more than one order of magnitude larger than next
	nearest-neighbor parameters. In LaMnO3 the DE in the ab planes is
	? ?0.26 eV. If there were no Jahn?Teller distortion present in the
	material, it would have been twice as large. In CaMnO3, the corresponding
	nearest-neighbor DE parameter for hopping of a doped electron in
	the ab planes is only ? ?0.17 eV, due to the antiparallel spin coupling.
	However, since this interaction is much larger than the exchange
	coupling, we suggest that it induces local ferromagnetic clusters
	around the doped electrons. © 2006 Wiley Periodicals, Inc. Int J
	Quantum Chem, 2006},
  copyright = {Copyright © 2006 Wiley Periodicals, Inc.},
  doi = {10.1002/qua.21040},
  file = {Full Text PDF:files/338/Stoyanova et al. - 2006 - Hopping matrix elements from first-principles stud.pdf:application/pdf;Snapshot:files/532/Stoyanova et al. - 2006 - Hopping matrix elements from first-principles stud.html:text/html},
  issn = {1097-461X},
  keywords = {CASCI, double exchange, manganites, state interaction},
  language = {en},
  owner = {vijay},
  shorttitle = {Hopping matrix elements from first-principles studies of overlapping
	fragments},
  timestamp = {2015.06.20},
  url = {http://onlinelibrary.wiley.com/doi/10.1002/qua.21040/abstract},
  urldate = {2014-07-18}
}

@ARTICLE{taratiel_refined_2004,
  author = {Taratiel, David and Guihéry, Nathalie},
  title = {A refined model of the double exchange phenomenon: {Test} on the
	stretched {N}2+ molecule},
  journal = {The Journal of Chemical Physics},
  year = {2004},
  volume = {121},
  pages = {7127--7135},
  number = {15},
  month = oct,
  note = {00009},
  abstract = {The N 2 + molecule is studied at different interatomic distances as
	a model molecule for the double exchange mechanism. The energy spectrum
	as well as the wave functions of the lowest states are analyzed and
	confronted both with the usual model of double exchange and with
	a recently proposed refined model. It is shown that the usual model
	fails to reproduce the energy spacings while the refined model is
	valid on a large domain of interatomic distances (in the magnetic
	regime). The study of a model molecule on a large domain of interatomic
	distances makes it possible to systematically investigate several
	regimes associated with different energetic state orderings. The
	perfect agreement between the refined model and the computed energies
	in the whole domain of stretched distances shows its applicability
	to a large number of real compounds. Finally, the respective contributions
	of dynamical and nondynamical correlations are analyzed.},
  doi = {10.1063/1.1786913},
  file = {Full Text PDF:files/573/Taratiel and Guihéry - 2004 - A refined model of the double exchange phenomenon.pdf:application/pdf;Snapshot:files/482/Taratiel and Guihéry - 2004 - A refined model of the double exchange phenomenon.html:text/html},
  issn = {0021-9606, 1089-7690},
  keywords = {Interatomic distances, Numerical modeling, Spectrum analysis, Wave
	functions},
  owner = {vijay},
  shorttitle = {A refined model of the double exchange phenomenon},
  timestamp = {2015.06.20},
  url = {http://scitation.aip.org/content/aip/journal/jcp/121/15/10.1063/1.1786913},
  urldate = {2014-06-20}
}

@ARTICLE{tasaki_nagaokas_1998,
  author = {Tasaki, Hal},
  title = {From {Nagaoka}'s {Ferromagnetism} to {Flat}-{Band} {Ferromagnetism}
	and {Beyond} {An} {Introduction} to {Ferromagnetism} in the {Hubbard}
	{Model}},
  journal = {Progress of Theoretical Physics},
  year = {1998},
  volume = {99},
  pages = {489--548},
  number = {4},
  month = apr,
  abstract = {It is believed that strong ferromagnetic interactions in some solids
	are generated by subtle interplay between quantum many-body effects
	and spin-independent Coulomb interactions between electrons. It is
	a challenging problem to verify this scenario in the Hubbard model,
	which is an idealized model for strongly interacting electrons in
	a solid. s ferromagnetism is a well-known rigorous example of ferromagnetism
	in the Hubbard model. It deals with the limiting situation in which
	there is one fewer electron than in the half-filling and the on-site
	Coulomb interaction is infinitely large. There are relatively new
	rigorous examples of ferromagnetism in Hubbard models called flat-band
	ferromagnetism. Flat-band ferromagnetism takes place in carefully
	prepared models in which the lowest bands (in the single-electron
	spectra) are ?flat.? Usually, these two approaches are regarded as
	two complimentary routes to ferromagnetism in the Hubbard model.
	resent paper we describe Nagaoka's ferromagnetism and flat-band ferromagnetism
	in detail, giving all the necessary background as well as complete
	(but elementary) mathematical proofs. By studying an intermediate
	model called the long-range hopping model, we also demonstrate that
	there is indeed a deep relation between these two seemingly different
	approaches to ferromagnetism. er discuss some attempts to go beyond
	these approaches. We briefly discuss recent rigorous example of ferromagnetism
	in the Hubbard model which has neither infinitely large parameters
	nor completely flat bands. We give preliminary discussion regarding
	possible experimental realizations of the (nearly-)flat- band ferromagnetism.
	Finally, we focus on some theoretical attempts to understand metallic
	ferromagnetism. We discuss three artificial one-dimensional models
	in which the existence of metallic ferromagnetism can be easily proved.
	tried to make the present paper as self-contained as possible, keeping
	in mind readers who are new to the field. Although the present paper
	is written as a review, it contains some material which appears for
	the first time.},
  doi = {10.1143/PTP.99.489},
  file = {Full Text PDF:files/69/Tasaki - 1998 - From Nagaoka's Ferromagnetism to Flat-Band Ferroma.  pdf:application/pdf;Snapshot:files/81/489.html:text/html},
  issn = {0033-068X, 1347-4081},
  language = {en},
  url = {http://ptp.oxfordjournals.org/content/99/4/489},
  urldate = {2015-07-23}
}

@ARTICLE{tasaki_extension_1989,
  author = {Tasaki, Hal},
  title = {Extension of {Nagaoka}'s theorem on the large- {\textbackslash}textit\{{U}\}
	{Hubbard} model},
  journal = {Physical Review B},
  year = {1989},
  volume = {40},
  pages = {9192--9193},
  number = {13},
  month = nov,
  abstract = {An extension is given of Nagaoka?s theorem on the existence of ferromagnetism
	in the large-U Hubbard model with precisely one hole. The present
	extension covers a large class of models with arbitrary non-negative
	hopping matrix elements and arbitrary spin-independent interactions.},
  doi = {10.1103/PhysRevB.40.9192},
  file = {APS Snapshot:files/113/PhysRevB.40.html:text/html;Full Text PDF:files/108/Tasaki - 1989 -    Extension of Nagaoka's theorem on the large-    t.pdf:application/pdf},
  url = {http://link.aps.org/doi/10.1103/PhysRevB.40.9192},
  urldate = {2015-07-23}
}

@ARTICLE{thouless_electrons_1974,
  author = {Thouless, D. J.},
  title = {Electrons in disordered systems and the theory of localization},
  journal = {Physics Reports},
  year = {1974},
  volume = {13},
  pages = {93--142},
  number = {3},
  month = oct,
  note = {01195},
  abstract = {This paper gives a review of the theory of noninteracting electrons
	in a static disordered lattice. The introductory section gives a
	brief survey of the main aspects of the problem and of its relevance
	to the physics of amorphous and disordered crystalline solids. The
	second section is concerned with the methods which can be used to
	find the density of states, both in the main part of the band, where
	the coherent potential approximation can be used, and in the tail
	of the band, where other methods must be used. The third section
	gives a survey of the theory of localization. There is a detailed
	discussion of the qualitative differences between localized and extended
	states which enable a sharp distinction to be made between them.
	There is a brief survey of the theory of one-dimensional systems
	and of the percolation problem, and then the Anderson model and its
	self-consistent modification are discussed. There is also a discussion
	of numerical work on the Anderson model and the use of path-integral
	methods. In the final section a tentative theory is proposed to combine
	various features of the problem which have been revealed by some
	of the different approaches.},
  doi = {10.1016/0370-1573(74)90029-5},
  file = {ScienceDirect Full Text PDF:files/445/Thouless - 1974 - Electrons in disordered systems and the theory of .pdf:application/pdf;ScienceDirect Snapshot:files/268/Thouless - 1974 - Electrons in disordered systems and the theory of .html:text/html},
  issn = {0370-1573},
  owner = {vijay},
  timestamp = {2015.06.20},
  url = {http://www.sciencedirect.com/science/article/pii/0370157374900295},
  urldate = {2014-06-17}
}

@INPROCEEDINGS{tranquada_spins_2013,
  author = {Tranquada, John M.},
  title = {Spins, stripes, and superconductivity in hole-doped cuprates},
  booktitle = {{AIP} {Conference} {Proceedings}},
  year = {2013},
  volume = {1550},
  pages = {114--187},
  month = aug,
  publisher = {AIP Publishing},
  note = {00001},
  abstract = {One of the major themes in correlated electron physics over the last
	quarter century has been the problem of high-temperature superconductivity
	in hole-doped copper-oxide compounds. Fundamental to this problem
	is the competition between antiferromagnetic spin correlations, a
	symptom of strong Coulomb interactions, and the kinetic energy of
	the doped carriers, which favors delocalization. After discussing
	some of the early challenges in the field, I describe the experimental
	picture provided by a variety of spectroscopic and transport techniques.
	Then I turn to the technique of neutron scattering, and discuss how
	it is used to determine spin correlations, especially in model systems
	of quantum magnetism. Neutron scattering and complementary techniques
	have determined the extent to which antiferromagnetic spin correlations
	survive in the cuprate superconductors. One experimental case involves
	the ordering of spin and charge stripes. I first consider related
	measurements on model compounds, such as La 2?x Sr x NiO 4+? , and
	then discuss the case of La 2?x Ba x CuO 4 . In the latter system,
	recent transport studies have demonstrated that quasi-two-dimensional
	superconductivity coexists with the stripe order, but with frustrated
	phase order between the layers. This has led to new concepts for
	the coexistence of spin order and superconductivity. While the relevance
	of stripe correlations to high-temperature superconductivity remains
	a subject of controversy, there is no question that stripes are an
	intriguing example of electron matter that results from strong correlations.},
  doi = {10.1063/1.4818402},
  file = {Full Text PDF:files/504/Tranquada - 2013 - Spins, stripes, and superconductivity in hole-dope.pdf:application/pdf;Snapshot:files/555/Tranquada - 2013 - Spins, stripes, and superconductivity in hole-dope.html:text/html},
  keywords = {Antiferromagnetism, Electron correlation calculations, High temperature
	superconductivity, Neutron scattering, Superconductivity},
  owner = {vijay},
  timestamp = {2015.06.20},
  url = {http://scitation.aip.org/content/aip/proceeding/aipcp/10.1063/1.4818402},
  urldate = {2014-07-17}
}

@ARTICLE{troyer_properties_1996,
  author = {Troyer, Matthias and Tsunetsugu, Hirokazu and Rice, T. M.},
  title = {Properties of lightly doped t-{J} two-leg ladders},
  journal = {Physical Review B},
  year = {1996},
  volume = {53},
  pages = {251--267},
  number = {1},
  month = jan,
  note = {00170},
  abstract = {We have numerically investigated the doped t-J ladder using exact
	diagonalization. We have studied both the limit of strong interchain
	coupling and isotropic coupling. The ladder scales to the Luther-Emery
	liquid regime in the strong interchain coupling limit. In this strong
	coupling limit there is a simple picture of the excitation spectrum
	that can be continued to explain the behavior at isotropic coupling.
	At J=0 we have indications of a ferromagnetic ground state. At a
	large J/t the ladder is phase separated into holes and a Heisenberg
	ladder. At intermediate coupling the ground state shows hole pairing
	with a modified d-wave symmetry. The excitation spectrum separates
	into a limited number of quasiparticles which carry charge +?e? and
	spin 1/2 and a triplet magnon mode. At half filling the former vanish
	but the latter evolves continuously into the magnon band of the spin
	liquid. At low doping the quasiparticles form a dilute Fermi gas
	with a strong attraction but simultaneously the Fermi wave vector,
	as would be measured in photoemission, is large. The dynamical structure
	factors are calculated and are found to be very similar to calculations
	on two-dimensional clusters. © 1996 The American Physical Society.},
  doi = {10.1103/PhysRevB.53.251},
  file = {APS Snapshot:files/344/Troyer et al. - 1996 - Properties of lightly doped t-J two-leg ladders.html:text/html;Full Text PDF:files/631/Troyer et al. - 1996 - Properties of lightly doped t-J two-leg ladders.pdf:application/pdf},
  owner = {vijay},
  timestamp = {2015.06.20},
  url = {http://link.aps.org/doi/10.1103/PhysRevB.53.251},
  urldate = {2014-05-30}
}

@ARTICLE{tsutsui_electronic_2000,
  author = {Tsutsui, Kenji and Koshibae, Wataru and Maekawa, Sadamichi},
  title = {Electronic structure and excitation spectra in doped nickelates},
  journal = {Physica B: Condensed Matter},
  year = {2000},
  volume = {284?288, Part 2},
  pages = {1471--1472},
  month = jul,
  abstract = {We examine the electronic structure and the excitation spectra in
	two-dimensional doped nickelates. The numerically exact diagonalization
	study on small clusters for the Hubbard model with degenerate eg
	orbital bands are performed. We find that the orbital dependence
	of the hopping matrix elements plays a crucial role in the electronic
	structure and the excitation spectra. The optical conductivity at
	the filling of one electron per site is shown to exhibit a characteristic
	feature.},
  doi = {10.1016/S0921-4526(99)02715-5},
  file = {ScienceDirect Full Text PDF:files/826/Tsutsui et al. - 2000 - Electronic structure and         excitation spectra in dop.pdf:application/pdf;ScienceDirect Snapshot:files/827/S0921452699027155.html:    text/html},
  issn = {0921-4526},
  keywords = {Nickelates, Numerically exact diagonalization, Optical conductivity,
	Strong correlation},
  url = {http://www.sciencedirect.com/science/article/pii/S0921452699027155},
  urldate = {2015-07-08}
}

@ARTICLE{tsutsui_electronic_2000,
  author = {Tsutsui, Kenji and Koshibae, Wataru and Maekawa, Sadamichi},
  title = {Electronic structure and excitation spectra in doped nickelates},
  journal = {Physica B: Condensed Matter},
  year = {2000},
  volume = {284?288, Part 2},
  pages = {1471--1472},
  month = jul,
  abstract = {We examine the electronic structure and the excitation spectra in
	two-dimensional doped nickelates. The numerically exact diagonalization
	study on small clusters for the Hubbard model with degenerate eg
	orbital bands are performed. We find that the orbital dependence
	of the hopping matrix elements plays a crucial role in the electronic
	structure and the excitation spectra. The optical conductivity at
	the filling of one electron per site is shown to exhibit a characteristic
	feature.},
  doi = {10.1016/S0921-4526(99)02715-5},
  file = {ScienceDirect Full Text PDF:files/826/Tsutsui et al. - 2000 - Electronic structure and         excitation spectra in dop.pdf:application/pdf;ScienceDirect Snapshot:files/827/S0921452699027155.html:    text/html},
  issn = {0921-4526},
  keywords = {Nickelates, Numerically exact diagonalization, Optical conductivity,
	Strong correlation},
  url = {http://www.sciencedirect.com/science/article/pii/S0921452699027155},
  urldate = {2015-07-08}
}

@ARTICLE{vallejo_double_2010,
  author = {Vallejo, E. and Navarro, O. and Avignon, M.},
  title = {Double and super-exchange model in one-dimensional systems},
  journal = {Journal of Magnetism and Magnetic Materials},
  year = {2010},
  volume = {322},
  pages = {1076--1078},
  number = {9?12},
  month = may,
  abstract = {We present an analytical and numerical study of the competition between
	double and super-exchange interactions in a one-dimensional model.
	For low super-exchange interaction energy we find phase separation
	between ferromagnetic and anti-ferromagnetic phases. When the super-exchange
	interaction energy gets larger, the conduction electrons are self-trapped
	within separate small magnetic polarons. These magnetic polarons
	contain a single electron inside two or three sites depending on
	the conduction electron density and form a Wigner crystallization.
	A new phase separation is found between these small polarons and
	the anti-ferromagnetic phase. Spin-glass behavior is obtained consistent
	with experimental results of the nickelate one-dimensional compound
	Y 2 - x Ca x BaNiO 5 .},
  doi = {10.1016/j.jmmm.2009.04.026},
  file = {ScienceDirect Full Text PDF:files/599/Vallejo et al. - 2010 - Double and super-exchange model in one-dimensional.pdf:application/pdf;ScienceDirect Snapshot:files/277/Vallejo et al. - 2010 - Double and super-exchange model in one-dimensional.html:text/html},
  issn = {0304-8853},
  keywords = {Classical spin model, Exchange and super-exchange interaction, Phase
	separation},
  owner = {vijay},
  series = {Proceedings of the {Joint} {European} {Magnetic} {Symposia}},
  timestamp = {2015.06.20},
  url = {http://www.sciencedirect.com/science/article/pii/S030488530900420X},
  urldate = {2015-06-11}
}

@ARTICLE{verstraete_matrix_2008,
  author = {Verstraete, F. and Murg, V. and Cirac, J. I.},
  title = {Matrix product states, projected entangled pair states, and variational
	renormalization group methods for quantum spin systems},
  journal = {Advances in Physics},
  year = {2008},
  volume = {57},
  pages = {143--224},
  number = {2},
  month = mar,
  abstract = {This article reviews recent developments in the theoretical understanding
	and the numerical implementation of variational renormalization group
	methods using matrix product states and projected entangled pair
	states.},
  doi = {10.1080/14789940801912366},
  file = {Snapshot:files/727/14789940801912366.html:text/html},
  issn = {0001-8732},
  url = {http://dx.doi.org/10.1080/14789940801912366},
  urldate = {2015-06-25}
}

@MISC{vijay_2015_20450,
  author = {vijay},
  title = {slepc_version: First Public Beta},
  month = jul,
  year = {2015},
  doi = {10.5281/zenodo.20450},
  url = {http://dx.doi.org/10.5281/zenodo.20450}
}

@INCOLLECTION{wang2014intel,
  author = {Wang, Endong and Zhang, Qing and Shen, Bo and Zhang, Guangyong and
	Lu, Xiaowei and Wu, Qing and Wang, Yajuan},
  title = {Intel Math Kernel Library},
  booktitle = {High-Performance Computing on the Intel{\textregistered} Xeon Phi?},
  publisher = {Springer},
  year = {2014},
  pages = {167--188}
}

@ARTICLE{weng_spin-charge_2001,
  author = {Weng, Z. Y. and Muthukumar, V. N. and Sheng, D. N. and Ting, C. S.},
  title = {Spin-charge separation in the single-hole-doped {Mott} antiferromagnet},
  journal = {Physical Review B},
  year = {2001},
  volume = {63},
  pages = {075102},
  number = {7},
  month = jan,
  abstract = {The motion of a single hole in a Mott antiferromagnet is investigated
	based on the t?J model. An exact expression of the energy spectrum
	is obtained, in which the irreparable phase string effect [Phys.
	Rev. Lett. 77, 5102 (1996)] is explicitly present. By identifying
	the phase string effect with spin backflow, we point out that spin-charge
	separation must exist in such a system: the doped hole has to decay
	into a neutral spinon and a spinless holon, together with the phase
	string. We show that while the spinon remains coherent, the holon
	motion is deterred by the phase string, resulting in its localization
	in space. We calculate the electron spectral function which explains
	the line shape of the spectral function as well as the ?quasiparticle?
	spectrum observed in angle-resolved photoemission experiments. Other
	analytic and numerical approaches are discussed based on the present
	framework.},
  doi = {10.1103/PhysRevB.63.075102},
  file = {APS Snapshot:files/459/Weng et al. - 2001 - Spin-charge separation in the single-hole-doped Mo.html:text/html;Full Text PDF:files/274/Weng et al. - 2001 - Spin-charge separation in the single-hole-doped Mo.pdf:application/pdf},
  owner = {vijay},
  timestamp = {2015.06.20},
  url = {http://link.aps.org/doi/10.1103/PhysRevB.63.075102},
  urldate = {2014-11-18}
}

@ARTICLE{white_friedel_2002,
  author = {White, Steven R. and Affleck, Ian and Scalapino, Douglas J.},
  title = {Friedel oscillations and charge density waves in chains and ladders},
  journal = {Physical Review B},
  year = {2002},
  volume = {65},
  pages = {165122},
  number = {16},
  month = apr,
  abstract = {The density matrix renormalization (DMRG) group method for ladders
	works much more efficiently with open boundary conditions. One consequence
	of these boundary conditions is ground-state charge density oscillations
	that often appear to be nearly constant in magnitude or to decay
	only slightly away from the boundaries. We analyze these using bosonization
	techniques, relating their detailed form to the correlation exponent
	and distinguishing boundary induced generalized Friedel oscillations
	from true charge density waves. We also discuss a different approach
	to extracting the correlation exponent from the finite size spectrum
	which uses exclusively open boundary conditions and can therefore
	take advantage of data for much larger system sizes. A general discussion
	of the Friedel oscillation wave vectors is given, and a convenient
	Fourier transform technique is used to determine it. DMRG results
	are analyzed on Hubbard and t?J chains and 2 leg t?J ladders. We
	present evidence for the existence of a long-ranged charge density
	wave state in the t?J ladder at a filling of n=0.75 and near J/ t?0.25.},
  doi = {10.1103/PhysRevB.65.165122},
  file = {APS Snapshot:files/1034/PhysRevB.65.html:text/html},
  url = {http://link.aps.org/doi/10.1103/PhysRevB.65.165122},
  urldate = {2015-09-02}
}

@ARTICLE{white_hole_1997,
  author = {White, Steven R. and Scalapino, D. J.},
  title = {Hole and pair structures in the t-{J} model},
  journal = {Physical Review B},
  year = {1997},
  volume = {55},
  pages = {6504--6517},
  number = {10},
  month = mar,
  abstract = {Using numerical results from density matrix renormalization group
	(DMRG) calculations for the t-J model, on systems as large as 10×7,
	we examine the structure of the one and two hole ground states in
	ladder systems and in two dimensional clusters. A simple theoretical
	framework is used to explain why holes bind in pairs in two-dimensional
	antiferromagnets. For the case J/t=0.5, which we have studied, the
	hole pairs reside predominantly on a 2×2 core plaquette with the
	probability that the holes are on diagonal sites greater than nearest-neighbor
	sites. There is a strong singlet bond connecting the spins on the
	two remaining sites of the plaquette. We find that a general characteristic
	of dynamic holes in an antiferromagnet is the presence of frustrating
	antiferromagnetic bonds connecting next-nearest-neighbor sites across
	the holes. Pairs of holes bind in order to share the frustrating
	bonds. At low doping, in addition to hole pairs, there are two additional
	low-energy structures which spontaneously form on certain finite
	systems. The first is an undoped L×2 spin-liquid region, or ladder.
	The second is a hole moving along a one dimensional chain of sites.
	At higher doping we expect that hole pairing is always favored.},
  doi = {10.1103/PhysRevB.55.6504},
  file = {APS Snapshot:files/375/White and Scalapino - 1997 - Hole and pair structures in the t-J model.html:text/html;Full Text PDF:files/626/White and Scalapino - 1997 - Hole and pair structures in the t-J model.pdf:application/pdf},
  owner = {vijay},
  timestamp = {2015.06.20},
  url = {http://link.aps.org/doi/10.1103/PhysRevB.55.6504},
  urldate = {2015-01-30}
}

@ARTICLE{wilson_renormalization_1975,
  author = {Wilson, Kenneth G.},
  title = {The renormalization group: {Critical} phenomena and the {Kondo} problem},
  journal = {Reviews of Modern Physics},
  year = {1975},
  volume = {47},
  pages = {773--840},
  number = {4},
  month = oct,
  note = {03333},
  abstract = {This review covers several topics involving renormalization group
	ideas. The solution of the s-wave Kondo Hamiltonian, describing a
	single magnetic impurity in a nonmagnetic metal, is explained in
	detail. See Secs. VII-IX. "Block spin" methods, applied to the two
	dimensional Ising model, are explained in Sec. VI. The first three
	sections give a relatively short review of basic renormalization
	group ideas, mainly in the context of critical phenomena. The relationship
	of the modern renormalization group to the older problems of divergences
	in statistical mechanics and field theory and field theoretic renormalization
	is discussed in Sec. IV. In Sec. V the special case of "marginal
	variables" is discussed in detail, along with the relationship of
	the modern renormalization group to its original formulation by Gell-Mann
	and Low and others.},
  doi = {10.1103/RevModPhys.47.773},
  file = {APS Snapshot:files/321/Wilson - 1975 - The renormalization group Critical phenomena and .html:text/html;Full Text PDF:files/543/Wilson - 1975 - The renormalization group Critical phenomena and .pdf:application/pdf},
  owner = {vijay},
  shorttitle = {The renormalization group},
  timestamp = {2015.06.20},
  url = {http://link.aps.org/doi/10.1103/RevModPhys.47.773},
  urldate = {2014-07-23}
}

@ARTICLE{xu_holes_2000-1,
  author = {Xu, Guangyong and Aeppli, G. and Bisher, M. E. and Broholm, C. and
	DiTusa, J. F. and Frost, C. D. and Ito, T. and Oka, K. and Paul,
	R. L. and Takagi, H. and Treacy, M. M. J.},
  title = {Holes in a {Quantum} {Spin} {Liquid}},
  journal = {Science},
  year = {2000},
  volume = {289},
  pages = {419--422},
  number = {5478},
  month = jul,
  note = {00059 PMID: 10903195},
  abstract = {Magnetic neutron scattering provides evidence for nucleation of antiferromagnetic
	droplets around impurities in a doped nickel oxide?based quantum
	magnet. The undoped parent compound contains a spin liquid with a
	cooperative singlet ground state and a gap in the magnetic excitation
	spectrum. Calcium doping creates excitations below the gap with an
	incommensurate structure factor. We show that weakly interacting
	antiferromagnetic droplets with a central phase shift of ? and a
	size controlled by the correlation length of the quantum liquid can
	account for the data. The experiment provides a quantitative impression
	of the magnetic polarization cloud associated with holes in a doped
	transition metal oxide.},
  doi = {10.1126/science.289.5478.419},
  file = {Full Text PDF:files/340/Xu et al. - 2000 - Holes in a Quantum Spin Liquid.pdf:application/pdf;Snapshot:files/431/Xu et al. - 2000 - Holes in a Quantum Spin Liquid.html:text/html},
  issn = {0036-8075, 1095-9203},
  language = {en},
  owner = {vijay},
  timestamp = {2015.06.20},
  url = {http://www.sciencemag.org/content/289/5478/419},
  urldate = {2014-07-18}
}

@INPROCEEDINGS{yamada_16.447_2005,
  author = {Yamada, S. and Imamura, T. and Machida, M.},
  title = {16.447 {TFlops} and 159-{Billion}-dimensional {Exact}-diagonalization
	for {Trapped} {Fermion}- {Hubbard} {Model} on the {Earth} {Simulator}},
  booktitle = {Supercomputing, 2005. {Proceedings} of the {ACM}/{IEEE} {SC} 2005
	{Conference}},
  year = {2005},
  pages = {44--44},
  month = nov,
  abstract = {In order to study a possibility of superfluidity in trapped atomic
	Fermi gases loaded on optical lattices, we implement an exact diagonalization
	code for the trapped Hubbard model on the Earth Simulator. Comparing
	two diagonalization algorithms, we find that the performance of the
	preconditioned conjugate gradient (PCG) method is 1.5 times superior
	to the conventional Lanczos one since the PCG method can conceal
	the communication overhead much more efficiently. Consequently, the
	PCG method shows 16.447 TFlops (50.2\% of the peak) on 512 nodes.
	On the other hand, we succeed in solving a 159-billion- dimensional
	matrix by using the conventional Lanczos method. To our knowledge,
	this dimension is a world- record. Numerical results reveal that
	an unconventional type of superfluidity specific to the confined
	system develops under repulsive interaction.},
  doi = {10.1109/SC.2005.1},
  file = {IEEE Xplore Abstract Record:files/879/cookiedetectresponse.html:text/html;IEEE Xplore Full Text  PDF:files/878/Yamada et al. - 2005 - 16.447 TFlops and 159-Billion-dimensional Exact-di.pdf:application/  pdf},
  keywords = {Atom optics, Charge carrier processes, Earth, Electrons, Gases, Lattices,
	Neutrons, Permission, Physics, Protons}
}

@ARTICLE{yamaguchi_n-band_1990,
  author = {Yamaguchi, Kizashi},
  title = {N-band {Hubbard} models for copper oxides and isoelectronic systems.
	{New} models for organic and organometallic magnetic conductors and
	superconductors},
  journal = {International Journal of Quantum Chemistry},
  year = {1990},
  volume = {37},
  pages = {167--196},
  number = {2},
  month = feb,
  note = {00046},
  abstract = {The electronic structures of undoped and doped copper oxides and other
	related oxides are investigated on the basis of the N-band Hubbard
	models. The Hubbard Hamiltonians for clusters of transition metal
	oxides are exactly diagonalized by the full valence-bond (VB) configuration
	interaction (CI) method in order to elucidate populations of doped
	holes, electronic excitation energies, etc. Possible mechanisms of
	the high-Tc superconductivity for oxide superconductors are discussed
	on the basis of the calculated results, together with available experiments.
	The analysis of correlation and spin correlation effects on doped
	copper oxides indicates theoretical possibilities of new models for
	organic and organometallic magnetic conductors and superconductors.
	Organic and organometallic analogs to copper oxides are therefore
	proposed on the basis of these results.},
  copyright = {Copyright © 1990 John Wiley \& Sons, Inc.},
  doi = {10.1002/qua.560370207},
  file = {Snapshot:files/270/Yamaguchi - 1990 - N-band Hubbard models for copper oxides and isoele.html:text/html},
  issn = {1097-461X},
  language = {en},
  owner = {vijay},
  timestamp = {2015.06.20},
  url = {http://onlinelibrary.wiley.com/doi/10.1002/qua.560370207/abstract},
  urldate = {2014-05-19}
}

@ARTICLE{yoshida_energy_2005-1,
  author = {Yoshida, M. and Shiraki, K. and Okubo, S. and Ohta, H. and Ito, T.
	and Takagi, H. and Kaburagi, M. and Ajiro, Y.},
  title = {Energy {Structure} of a {Finite} {Haldane} {Chain} in \$\{{\textbackslash}mathrm\{{Y}\}\}\_\{2\}\{{\textbackslash}mathrm\{{BaNi}\}\}\_\{0.96\}\{{\textbackslash}mathrm\{{Mg}\}\}\_\{0.04\}\{{\textbackslash}mathrm\{{O}\}\}\_\{5\}\$
	{Studied} by {High} {Field} {Electron} {Spin} {Resonance}},
  journal = {Physical Review Letters},
  year = {2005},
  volume = {95},
  pages = {117202},
  number = {11},
  month = sep,
  abstract = {This Letter presents the fine structure of energy levels for the edge
	states of a Haldane chain. In order to investigate the edge states,
	we have performed high field and multifrequency electron spin resonance
	(ESR) measurements of finite length S=1 antiferromagnetic chains
	in Y2BaNi0.96Mg0.04O5. Owing to the high spectral resolution by high
	fields and high frequencies, observed ESR signals can be separated
	into the contributions of the finite chains with various chain lengths.
	Our results clearly show that the edge spins actually interact with
	each other through the quantum spin chain and the interaction depends
	on the chain length N. This N dependence has been obtained experimentally
	for the first time, and shows that the correlation length ? in the
	real system is somewhat larger than that calculated by a simple Heisenberg
	model.},
  doi = {10.1103/PhysRevLett.95.117202},
  file = {APS Snapshot:files/466/Yoshida et al. - 2005 - Energy Structure of a Finite Haldane Chain in \$ m.html:text/html},
  owner = {vijay},
  timestamp = {2015.06.20},
  url = {http://link.aps.org/doi/10.1103/PhysRevLett.95.117202},
  urldate = {2015-04-11}
}

@ARTICLE{zener_interaction_1951,
  author = {Zener, Clarence},
  title = {Interaction between the d-{Shells} in the {Transition} {Metals}.
	{II}. {Ferromagnetic} {Compounds} of {Manganese} with {Perovskite}
	{Structure}},
  journal = {Physical Review},
  year = {1951},
  volume = {82},
  pages = {403--405},
  number = {3},
  month = may,
  abstract = {Recently, Jonker and Van Santen have found an empirical correlation
	between electrical conduction and ferromagnetism in certain compounds
	of manganese with perovskite structure. This observed correlation
	is herein interpreted in terms of those principles governing the
	interaction of the d-shells of the transition metals which were enunciated
	in the first paper of this series. Both electrical conduction and
	ferromagnetic coupling in these compounds are found to arise from
	a double exchange process, and a quantitative relation is developed
	between electrical conductivity and the ferromagnetic Curie temperature.},
  doi = {10.1103/PhysRev.82.403},
  file = {APS Snapshot:files/561/Zener - 1951 - Interaction between the d-Shells in the Transition.html:text/html;Full Text PDF:files/465/Zener - 1951 - Interaction between the d-Shells in the Transition.pdf:application/pdf},
  owner = {vijay},
  timestamp = {2015.06.20},
  url = {http://link.aps.org/doi/10.1103/PhysRev.82.403},
  urldate = {2014-11-25}
}

@ARTICLE{,
  owner = {vijay},
  timestamp = {2015.08.20}
}

@comment{jabref-meta: selector_publisher:}

@comment{jabref-meta: selector_author:}

@comment{jabref-meta: selector_journal:}

@comment{jabref-meta: selector_keywords:}

