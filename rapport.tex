\documentclass[ openright,titlepage,numbers=noenddot,headinclude,twoside,%
                footinclude=true,cleardoublepage=empty,abstractoff,%
                BCOR=5mm,paper=a4,fontsize=11pt,%
                ngerman,american,%lockflag%
]{scrreprt}
\usepackage{graphicx}
\usepackage{amsmath}
\usepackage[
backend=biber,
style=alphabetic,
sorting=ynt
]{biblatex}
\addbibresource{biblio.bib}
\usepackage{fancyhdr}
\usepackage{braket}
\usepackage{modiagram}
\usepackage{tabularx}

\graphicspath{ {./Figures/} }

\pagestyle{fancy}
\lhead{jose-david creme}
\rhead{TODO 1}
\lfoot{TODO 1}
\rfoot{TODO 1}
%\def\changemargin#1#2{\list{}{\rightmargin#2\leftmargin#1}\item[]}
%\let\endchangemargin=\endlist
\title{Modeling properties of the double exchange model hamiltonian}

\newcommand{\citepetsc}{\cite{petsc_web_page,petsc_user_ref,petsc_efficient}}
\newcommand{\citeslepc}{\cite{Hernandez_2003_SSL,slepc_users_manual}}


\begin{document}
\maketitle
\thispagestyle{fancy}
\chapter{Introduction}

Introduce the following topics
\begin{itemize}
  \item Colossal magnetic resistance
  \item Anomalous behavior of magnetic susceptibility in linear chain of Nickelates
\end{itemize}

\section{Experimental studies}

One dimensional transition metal chains have been synthesized and studied in
literature.~\cite{darriet_compound_1993,batlogg_haldane_1994} Experimental
studies on 1D Nickelates have shown that they exhibit a magnetic field
dependence in their resistivity upon doping.\cite{xu_holes_2000} Here, we focus
on the low temperature behavior of the 1D doped Nickelates close to the curie
temperature. Experimental studies of susceptibility of 1D doped Nickelates have
shown that near the Curie temperature, the materials show antiferromagnetic
behavior.\cite{Ramirez} This behavior is contrary to the usual double exchange
model where the holes are mobile throughout the chain. We develop arguments and
discuss some new ideas to explain this behavior of the magnetic susceptibility
in 1D doped Nickelates.

\section{Theoretical studies}

One dimensional transition metal chains are immencely useful models
for theoretical analysis due to the existance of powerful techniques
such as the density matrix renormalization group (DMRG) and lanczos
exact diagonalization techniques. Such methods can be applied to
high accuracy and have been used to study models similar to the
one studied here but with important differences.\cite{dagotto_correlated_1994,patel_emergence_2020}
Further work by Dagotto. et. al have shown details of the existance
of a ferromagnetic polaron and its extent in such 1D models.\cite{malvezzi_origin_2001} However, the parameter range
that we have chosen for the current study relies on those extracted
by \textit{ab initio} calculstions on Nickel dimers.\cite{bastardis_microscopic_2007,bastardis_isotropic_2008}
Therefore, the present study serves to augment the available literature
for parameter range extracted using \textit{ab initio} calculations.

\chapter{Double Exchange Hamiltonian}

The double exchange model is used to describe mixed lvalent transition
metal based molecules. The phenomenology of ferromagnetic and anti-ferromagnetic
behavior in such molecules is captured by the various parameters of the
double exchange model Hamiltonian (Eq:~\ref{eq:demodel}).

\begin{equation}
  \begin{split}
\hat{H} &= \sum_i 2K \hat{S}_{a_i}\cdot\hat{S}_{b_i} \\
        &+ \sum_{\braket{ij}} 2J_a \hat{S}_{a_i}\cdot\hat{S}_{a_j} \\
        &+ \sum_{\braket{ij}} t\left( \hat{c}^{\dagger}_{a_i}\cdot\hat{c}_{a_j} + \text{h.c.}\right ) \\
        &+ \sum_{i,i+1} V_{NN}\left ( \delta_{n_i,0}\ \delta_{n_{i+1},0} \right ) \\
        &+ \sum_{i,i+2} V_{NNN}\left ( \delta_{n_i,0}\ \delta_{n_{i+2},0} \right )
  \end{split}
\label{eq:demodel}
\end{equation}

The above Hamiltonian describes a model with two valence orbitals on each
atom (site) ideally of differnet symmetry (i.e. orthogonal, e.g. $a,b$). A schematic is shown in
Figure:~\ref{fig:deham}.

\begin{figure}[ht]
  \centering
\begin{modiagram}[names]
 \atom[$1$]{left}{
    1s = { 0; up} ,
    2s = { 1; up} ,
 }

 \atom[$2$]{right}{
    1s = { 0; up} ,
    2s = { 1;   } ,
 }
 \node[right,xshift=4mm] at (1sright) {$b$};
 \node[right,xshift=4mm] at (2sright) {$a$};
 \node[left,xshift=-4mm] at (1sleft) {$b$};
 \node[left,xshift=-4mm] at (2sleft) {$a$};

 \draw[<-,gray,semithick]   (2sright) edge[bend right] node [left] {} (2sleft);
 \draw[<->,gray,semithick] (1sright) edge[bend left] node [left] {} (1sleft);
 \draw[<->,gray,semithick] (1sleft) edge[bend left] node [left] {} (2sleft);

 \node[left,xshift=2.3cm, yshift=-9mm] at (1sleft) {$J$};
 \node[left,xshift= 4mm, yshift=5mm] at (1sleft) {$K$};
 \node[left,xshift=2.3cm, yshift=-0mm] at (2sleft) {$t$};

 \end{modiagram}
  \caption{\label{fig:deham} Orbital diagram representing the mail interactions of the double exchange model with two valence orbitals on each atom.}
\end{figure}

In Eq:~\ref{eq:demodel}, the spins of the electrons are represented by $S_a$,
where the subscript $a$ denotes the orbital on which the electron resides.  The
creation and annahilation operators of electrons are represented by
$\hat{c}^{\dagger}$ and $\hat{C}$ respectively. The parameters in the model
Hamiltonian are defined as follows:

\begin{itemize}

\item $K$ - The local exchange integral. This exchange integral represents
  Hund's rule and is always positive. The local exchange favors parallel
  coupling i.e. a local high spin determinant.

\item $J$ - The kinetic exchange interaction between orbitals of symmetry $a$
  on neighboring sites. This integral is responsible of the anti-parallel coupling.

\item $t$ - The kinetic energy integral or the hopping term. The
  transfer of electrons from one atom to the neighboring atom is represented
  by the hopping term. This also favors parallel coupling when taken along with
  the exchange integral $K$.

\item $V$ - The hole repulsion term. This takes into account the repulsion between
  holes which are on neighboring sites $V_{NN}$ or next-nearest neighboring sites $V_{NNN}$. The hole repulsion indirectly takes into account the repulsion between
  electrons occupying nearby sites.\cite{calzado_proposal_2001}

\end{itemize}

\section{Parameter Space}

All the parameters of the model are important for understanding the collective
properties of the double exchange hamiltonian. Here we give the order of magnitude
of all the parameters used keeping in mind that we target molecules based on
transition metal atoms.
Previous studies using \textit{ab initio} methods in order to extract model
parameters for various transition metal atoms such as Cu, Mn, and Ni have
provided realistic order of magnitudes of the values of the parameters.
In the present analysis, we use the full range of values in order to represent
the full range of transition metal atoms.

\begin{table}[h!]
\centering
\begin{tabular}{||c||}
 \hline
 Parameters  \\ [0.5ex]
 \hline\hline
 \\
    $ 0.01\lvert t \rvert \le J \le 0.15\lvert t \rvert $    \\ [1ex]
    $ 0.01\lvert t \rvert \le J \le 0.15\lvert t \rvert $    \\ [1ex]
    $ 0.4 \le K \le 0.8 $                                    \\ [1ex]
    $ V_{NN} = \frac{\alpha}{2r}\ ;\ 0.5 \le \alpha \le 0.8 $ \\ [1ex]
    $ V_{NNN} = \alpha V_{NN} $                                \\ [1ex]
 \hline
\end{tabular}
\label{tab:params}
\caption{Table with range of parameter values.}
\end{table}

Including the hole repulsion term significantly changes the low energy physics
of the model as shown by previous work\cite{calzado_proposal_2001}. Here we will
study the influence in the variation of all the above parameters.


\chapter{Methodology}

The double exchange hamiltonian is studied for one dimensional chain
of sites containing holes with a doping ratio of 1:3 i.e. one hole for every
three sites. This follows our previous study which shows that for
the physically meaningful range of parameter values (Table:~\ref{tab:params}),
a single hole aligns about three sites.\cite{crystals_chilkuri}

\section{Exact diagonalization}

An exact diagonalization method is used to obtain many low energy states of the
hamiltonian. Due to the exponential growth of the hilbert space with the number
of sites, exact diagonalization becomes challenging.Here we use our home made
code relying on PETSc\citepetsc and SLEPc\citeslepc libraries to perform exact diagonalization and
use a large number of high performance compute nodes for distributed storage of
the hamiltonian matrix and eigenvectors. We have used DEHam\cite{deham} to
perform exact diagonalization of very large number of sites up to 18 sites
containing 36 orbitals with 6 holes which corresponds to a hilbert space of
about $0.5 \times 10^9$ determinants.

\section{Model space}

\section{Model Hamiltonian}

\section{Projector}

\section{Effective Hamiltonian}

\section{Real Space Renormalization Group}

\chapter{Results}

\section{Hole repulsion}

%TODO

\subsection{Weight vs J}
\subsubsection{Variation with Nsites}
\subsubsection{Variation with Repulsion}

\begin{figure}[ht]
    \centering
    \includegraphics[scale=0.5]{12_4h_J_wmax_vs_xrep.pdf}
    \caption{\label{fig:}Variation of the value of $J$ for which we have a maximum in the weight vs the value of hole repulsion $V$. }
\end{figure}


\begin{figure}[ht]
    \centering
    \includegraphics[scale=0.5]{Wmax_vs_J_xrep525.png}
    \caption{\label{fig:}Maximal weight as a function of J for xrep 525. }
\end{figure}

\subsection{$J_{eff}$ vs $J$}
\subsubsection{Variation with Nsites}
\subsubsection{Variation with Repulsion}

\subsection{RSRG}

\chapter{Discussion}

\chapter{Conclusion}

\end{document}



%*************************************************************************
% Bibliographies
%*************************************************************************
\printbibliography
